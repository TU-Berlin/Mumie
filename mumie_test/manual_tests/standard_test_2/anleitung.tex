\documentclass[lang=de]{generic}

\begin{document}

\title{Standard-Test 2}
\subtitle{Anleitung f�r Tester}

\begin{authors}
  \author[rassy@math.tu-berlin.de]{Tilman Rassy}
\end{authors}

\version{$Id: anleitung.tex,v 1.2 2008/01/16 11:17:45 rassy Exp $}

\tableofcontents

\section{Allgemeines}

Bei diesem Test werden \emph{Punkteanzeige} und \emph{Bulk-Correction} gepr�ft. Der
Test sollte nach jeder nicht-trivialen �nderung an Codestellen durchgef�hrt
werden, die diese Funktionalit�ten betreffen.

F�r den Test wird ein bestimmter Kurs ausgew�hlt, im folgenden \emph{Test-Kurs}
genannt. Der Test erfolgt stichprobenhaft an (mindestens) einem Tutorium, im
folgenden \emph{Test-Tutorium} genannt. Das Test-Tutorium sollte mindestens
f�nf Studenten haben, im folgenden \emph{Student 1} bis \emph{Student 5}
genannt. Neben diesen Studenten werden folgende Benutzer gebraucht:

\begin{itemize}
\item Benutzer 1: Dozent des Test-Kurses
\item Benutzer 2: Dozent, aber nicht des Test-Kurses
\item Benutzer 3: Student und Mitglied des Test-Kurses
\end{itemize}

Es werden (mindestens) zwei Aufgabenbl�tter gebraucht, im folgenden \emph{Blatt
  1} und \emph{Blatt 2} genannt. Jedes Blatt sollte mindestens drei
elektronische und eine schriftliche Aufgabe haben.

\section{Vorbereitung: Erzeugung der Antworten}

Um den Test durchf�hren zu k�nnen, m�ssen Studenten-Antworten in hinreichender
Menge vorhanden sein. Wurden bereits bei einem fr�heren Test Studenten-Antworten
erzeugt, k�nnen diese mit einem speziellen Skript in den Server eingespielt
werden. Falls nicht, oder falls neue Daten gew�nscht sind, wie folgt
vorgehen:

Bearbeitungszeitr�ume von Blatt 1 und 2 so
einstellen, dass beide Bl�tter bearbeitbar sind, und als Student 1 bis Student
5 des Test-Tutoriums Aufgaben bearbeiten. Am Ende sollten f�r mindestens die
H�lfte aller Studenten/Aufgaben-Paare L�sungen gespeichert sein; die
schriftlichen Aufgaben nicht mitgerechnet. Umgekehrt sollte es mindestens eine
nicht bearbeitete Aufgabe geben, damit auch dieser Fall beim Test erfasst wird.

\section{Als Dozent des Test-Kurses}

\subsection{Punkteanzeige}

Auf \emph{Kurse} und dann unter dem Test-Kurs auf \emph{Tutorien: Anzeigen}
gehen. F�r das Test-Tutorium \emph{Punktestand} anklicken. Es wird die
Punktetabelle angezeigt.

Stichprobenhaft pr�fen, ob die Tabelle die richtigen Werte anzeigt,
insbesondere:

\begin{itemize}
\item Einzel-Punktzahlen (erreichte und erreichbare)
\item Gesamtpunktzahlen (erreichte und erreichbare)
\item Kennzeichnung von nicht bearbeiteten und nicht korrigierten Aufgaben
\end{itemize}

Falls noch nicht gen�gend Korrekturen vorliegen, diesen
Punkt auf nach der Bulk-Correction verschieben.

Layout pr�fen, insbesondere:

\begin{itemize}
\item Ist die Darstellung korrekt und gut lesbar?
\item Werden alle Zeichen richtig dargstellt?
\item Ist die Formatierung passend? (Links-, Rechtsb�ndigkeit, Blocksatz)
\item Sind Abst�nde und R�nder passend?
\item Ist die Schriftgr�sse passend?
\item Ist die Platzaufteilung gut?
\end{itemize}

\subsection{Bulk-Correction}

Sicherstellen, dass die Bearbeitungszeitr�ume f�r Blatt 1 und 2 abgelaufen
sind. In der Punkteanzeige f�r beide Bl�tter den \emph{Korrigieren}-Button
klicken. Pr�fen, ob die dadurch ausgel�ste Bulk-Correction funktioniert,
insbesondere:

\begin{itemize}
\item Wurden die Aufgaben tats�chlich korrigiert? (S. Anmerkungen unten)
\item Ist die Wartezeit akzeptabel?
\end{itemize}

Anmerkungen: (1) Es ist m�glich, dass eine Aufgabe nicht korrigiert werden
konnte, weil der Korrektor nicht funktioniert. Man erkennt das daran, dass im
Antwort-Fenster \emph{"Nicht alle Aufgaben von Aufgabenblatt X konnten
  korrigiert werden"} angezeigt wird (wobei X das Label des Aufgabenblatts
ist).  Das ist kein Fehler der Bulk-Correction (sondern des Korrektors), sollte
aber trotzdem gemeldet werden.

(2) M�glicherweise muss erst der Browser-Cache gel�scht werden, bevor die neuen
Punktzahlen angezeigt werden. Ist dies der Fall, sollte das als Bug gemeldet
werden.

\section{Als anderer Dozent}

Als Benutzer 2 (Dozent, aber nicht des Test-Kurses) auf \emph{Kurse} und dann
unter dem Test-Kurs auf \emph{Tutorien: Anzeigen} gehen. Es sollte die Meldung
\emph{"Zugriff verweigert"} erscheinen.

\section{Als Student}

Als Benutzer 3 (Student und Mitglied des Test-Kurses) auf \emph{Kurse} und dann
unter dem Test-Kurs auf \emph{Tutorien: Anzeigen} gehen. Es sollte die Meldung
\emph{"Zugriff verweigert"} erscheinen.

\section{Organisatorisches}

Alle Fehler sollen in unser Bug-Tracking-System eingetragen werden. URL:

\begin{preformatted}%
  \link{http://www3.math.tu-berlin.de/rt}{www3.math.tu-berlin.de/rt}
\end{preformatted}

Dabei bitte folgendes beachten:

\begin{itemize}
\item Fehler, die bereits gemeldet wurden und noch nicht behoben sind, nicht
  noch einmal melden. Falls sich durch diesem Test neue Erkenntnisse ergeben,
  diese bei dem alten Bug eintragen (als Comment oder Reply).
\item Korrekte Kategorie (Queue) angeben. Bei diesem Test ist die Katorie
  normalerweise \emph{japs}.
\item \emph{Owner} nicht setzen
\item F�r jede Durchf�hrung des Tests ein kurzes Protokoll
    anlegen und ins CVS einchecken, im selben Verzeichnis, in dem sich diese
    Anleitung befindet. Vorschlag f�r den Dateinamen:
\begin{preformatted}%
  protokoll\_\var{datum}.txt
\end{preformatted}
wobei \var{datum} das Datum des Tests im Format \var{YYYY-MM-dd} ist. Erstreckt
sich der Test �ber mehrere Tage, das Startdatum nehmen. Beginnen mehrere Tests
am selben Tag, Dateinamen in
\begin{preformatted}%
  protokoll\_\var{datum}_\var{n}.txt
\end{preformatted}
    ab�ndern, wobei \var{n} die Tests durchnumeriert, mit 1 beginnend.

    Das Protokol soll insbesondere folgende Angaben enthalten:

    \begin{itemize}
    \item Wer getestet hat
    \item Beginn und Ende des Tests
    \item Grund des Tests (z.B. die �nderungen im Mumie-Code).
    \item Welche (Pseudo-)Dokumente zum Test herangezogen wurden (Kurs,
      Tutorium, Aufgabenbl�tter, Benutzer)
    \item Welche Bugs gemeldet wurden (RT-Ticket-Ids), und bei welchen (Pseudo-)Dokumenten
      sie aufgetreten sind
    \end{itemize}

    Ein festes Format f�r das Protokoll gibt es nicht. Ausser den obigen
    Angaben kann das Protokoll weitere Beobachtungen und Bemerkungen
    enthalten.

\end{itemize}

\end{document}