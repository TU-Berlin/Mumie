\documentclass{japs.summary}

\begin{metainfo}
  \name{Test}
  \begin{description}
    Ein Test
  \end{description}
  \begin{changelog}
    Started
  \end{changelog}
  \begin{components}

  \end{components}
  \creategeneric
\end{metainfo}

\begin{content}

  \begin{ifstate}{work}
    Zustand des Arbeitsblatts = work
  \end{ifstate}

  \begin{ifstate}{feedback}

    Zustand des Arbeitsblatts = feedback

    \begin{ifpoints}{0 6}
      (A) Punktzahl zwischen 0 und 6
    \end{ifpoints}

    \begin{ifpoints}{6.1 10}
      (A) Punktzahl zwischen 6.1 und 10
    \end{ifpoints}

    \begin{ifpoints}{10.1 max}
      (A) Punktzahl zwischen 10.1 und maximaler Punktzahl
    \end{ifpoints}

    \begin{casepoints}
      \uptopoints{6}
        (B) Punktzahl zwischen 0 und 6
      \uptopoints{10}
        (B) Punktzahl zwischen 6.1 und 10
      \uptopoints{max}
        (B) Punktzahl zwischen 10.1 und maximaler Punktzahl
    \end{casepoints}

  \end{ifstate}

  \begin{iftimeframe}{inside}
    W"ahrend des Bearbeitungszeitraums
  \end{iftimeframe}

  \begin{iftimeframe}{after}
    Nach dem Bearbeitungszeitraum
  \end{iftimeframe}

\end{content}
