\documentclass{generic}


% ------------------------------------------------------------------------------
% Schriftgroessen
% ------------------------------------------------------------------------------

\renewcommand{\normalsize}{\fontsize{18}{22}\selectfont}
\renewcommand{\large}{\fontsize{20}{26}\selectfont}
\renewcommand{\Large}{\fontsize{22}{30}\selectfont}
\renewcommand{\small}{\fontsize{14}{16}\selectfont}
\newcommand{\smaller}{\fontsize{12}{14}\selectfont}
\newcommand{\verysmall}{\fontsize{6}{6}\selectfont}
\renewcommand{\footnotesize}{\fontsize{12}{14}\selectfont}
\newcommand{\tocsize}{\fontsize{12}{16}\selectfont}
\newcommand{\titlesize}{\fontsize{24}{36}\selectfont}

% ------------------------------------------------------------------------------
% Texthervorhebungen
% ------------------------------------------------------------------------------

%\definecolor{emphcolor}{rgb}{0.77,0.00,0.00}
%\renewcommand{\emph}[1]{\textbf{\textcolor{emphcolor}{#1}}}
\renewcommand{\emph}[1]{\textbf{#1}}

\newcommand{\term}[1]{\textit{#1}}
% \newcommand{\code}[1]{\texttt{#1}}
\newcommand{\var}[1]{\textit{#1}}

%\definecolor{headlinecolor}{rgb}{0.77,0.00, 0.00}
\definecolor{headlinecolor}{rgb}{0.67,0.09, 0.22}
\newcommand{\headline}[1]{\textbf{\textcolor{headlinecolor}{#1}}}

\definecolor{emphboxbordercolor}{rgb}{0.80,0.36,0.36}
\definecolor{emphboxcolor}{rgb}{0.98,0.98,0.77}
\newcommand{\emphbox}[2]{\fcolorbox{emphboxbordercolor}{emphboxcolor}{\parbox{#1}{#2}}}

\definecolor{labelcolor}{rgb}{0.35,0.35, 0.8}


% -------------------------------------------------------------------------------
% Pfeile
% -------------------------------------------------------------------------------

\newcommand{\myarrow}{{\Large\boldmath$\;\rightarrow\;$}}
\newcommand{\mylongarrow}{{\Large\boldmath$\;\longrightarrow\;$}}



% -------------------------------------------------------------------------------
% Befehle f�r Abbildungen 
% -------------------------------------------------------------------------------

\renewcommand{\thefigure}{\thesubsection.\arabic{figure}}

\newcommand{\mycaption}[2]{%
 \centerline{%
  \parbox{#1}{%
   \small \refstepcounter{figure} Figure \thefigure. #2 }}}


%---------------------------------------------------------------------------
% Gliederungsbefehle
%---------------------------------------------------------------------------

\makeatletter

\renewcommand{\section}{\@startsection
 {section}%                                   % Name
 {1}%                                         % Ebene
 {0pt}%                                       % Einzug
 {-0.0\baselineskip}%                        % Vorabstand 
 {0.85\baselineskip}%                         % Nachabstand
 {\fontsize{20}{24}\selectfont\itshape\bfseries}}%    % Stil

\makeatother

\newcommand{\mysection}[1]{%

\vspace*{-1.2\baselineskip}

{\color{headlinecolor}
\section{#1}}}

\newcounter{myappendix}

\newcommand{\myappendix}[2]{%
\hfill\makebox[0cm][l]{\tocref}

\vspace*{-1.2\baselineskip}

\refstepcounter{myappendix}
\section*{A\arabic{myappendix}\hspace{1em}#1}\hypertarget{#2}{}}



% -------------------------------------------------------------------------------
% Inhaltsverzeichnis 
% -------------------------------------------------------------------------------

\newcommand{\mytoctitle}[1]{\begin{center}\textbf{\hypertarget{inhalt}{#1}}\end{center}}

\newenvironment{mytoc}
  {\tocsize\textbf{\hypertarget{inhalt}{Inhalt}}\\\begin{enumerate}}
  {\end{enumerate}}

\newcommand{\mytocitem}[3]{\item[#1]\hyperlink{#3}{#2}}

\newcommand{\tocref}{\verysmall\hyperlink{inhalt}{TOC}\normalsize}

% -------------------------------------------------------------------------------
% Listen 
% -------------------------------------------------------------------------------

%\newcommand{\mylabelitemi}{\labelitemi}
\newcommand{\mylabelitemi}{\raisebox{0.25ex}{{\small $\blacktriangleright$}\hspace{0.1em}}}

%\newcommand{\mylabelitemii}{\labelitemii}
%\newcommand{\mylabelitemii}{\raisebox{0.25ex}{{\small $\triangleright$}\hspace{0.1em}}}
\newcommand{\mylabelitemii}{\raisebox{0.1ex}{\large\bfseries-}\hspace{0.075em}}

\newenvironment{mylist}[1][\mylabelitemi]
  {\begin{list}{{\color{labelcolor}#1}}
    {\setlength{\itemindent}{0cm}
     \setlength{\labelwidth}{1em}
     \setlength{\leftmargin}{0.5cm}
     \setlength{\rightmargin}{1cm}}}
  {\end{list}}

\newenvironment{myenum}
  {\begin{list}{\theenumi}
    {\usecounter{enumi}
     \setlength{\itemindent}{0cm}
     \setlength{\labelwidth}{1em}
     \setlength{\leftmargin}{0.5cm}
     \setlength{\rightmargin}{1cm}}}
  {\end{list}}

\newcommand{\pitem}{\pause\item}
\newcommand{\itemii}{\item[\mylabelitemii]}

\newcounter{romanlistcounter}
\newenvironment{romanlist}{\begin{list}%
 {(\roman{romanlistcounter})\hfill}{\usecounter{romanlistcounter}%
 \topsep0.05\baselineskip% 
 \partopsep0.85\baselineskip%
 \leftmargin2em%
 \labelwidth2em%
 \labelsep0pt%
 \itemindent0pt%
 \parsep0.15\baselineskip%
 \itemsep0.0\baselineskip}}%
 {\end{list}}

\newcounter{alphlistcounter}
\newenvironment{alphlist}{\begin{list}%
 {(\alph{alphlistcounter})\hfill}{\usecounter{alphlistcounter}%
 \topsep0.05\baselineskip% 
 \partopsep0.85\baselineskip%
 \leftmargin1.8em%
 \labelwidth1.8em%
 \labelsep0pt%
 \itemindent0pt%
 \parsep0.15\baselineskip%
 \itemsep0.0\baselineskip}}%
 {\end{list}}

\newcounter{arabiclistcounter}
\newenvironment{arabiclist}{\begin{list}%
 {(\arabic{arabiclistcounter})\hfill}{\usecounter{arabiclistcounter}%
 \topsep0.05\baselineskip% 
 \partopsep0.85\baselineskip%
 \leftmargin1.8em%
 \labelwidth1.8em%
 \labelsep0pt%               
 \itemindent0pt%
 \parsep0.15\baselineskip%
 \itemsep0.0\baselineskip}}%
 {\end{list}}





\begin{document}

\title{Aufl�sung generischer Dokumente}

\begin{authors}
  \author[rassy@math.tu-berlin.de]{Tilman Rassy}
\end{authors}

\version{$Id: aufl_gen_dok.tex,v 1.3 2007/09/24 12:57:20 rassy Exp $}

Unter der \emph{Aufl�sung eines generischen Dokuments} versteht man die
Aufgabe, zu einem gegebenem generischen Dokument, einem gegebenem Theme und
einer gegebenen Sprache das entsprechende "reale" Dokument aufzufinden. Diese
Spezifikation beschreibt die Hintergr�nde und Regeln der Aufl�sung generischer
Dokumente.

\tableofcontents

\section{Bezeichnungen}

In der Spezifikation werden durchg�ngig folgende Notationen verwendet:

\begin{table}[\cellaligns{cl}]
\head
Notation & Bedeutung
\body
$G$ & Menge aller generischen Dokumente \\
$T$ & Menge aller Themes \\
$L$ & Menge aller Sprachen \\
$R$ & Menge aller realen Dokumente \\
$t_0$ & Default-Theme \\
$\ell_0$ & Default-Sprache \\
$\ell_\ast$ & neutrale Sprache \\
$f$ & Aufl�sungs-Abbbildung (s. \ref{aufl_abb}) \\
$D$ & Definitionsbereich von $f$ (s. \ref{aufl_abb})
\end{table}

Die \emph{neutrale Sprache} kennzeichnet sprach-unspezifische Inhalte wie
z.B. Bilder oder CSS-Stylesheets. Ihr internationaler Sprach-Code ist
\code{zxx}.

\section{Die Aufl�sungs-Abbildung}\label{aufl_abb}

Sei $D$ die Menge aller Tripel $(g, t, \ell) \in G \times T \times L$, f�r die ein
reales Dokument existiert. Ferner sei $f$ die Abbildung $D \rightarrow R$, die
jedem $(g, t, \ell) \in D$ das entsprechende reale Dokument zuordnet. $f$ 
heisst \emph{Aufl�sungs-Abbildung}. Der Wert eines $(g, t, \ell) \in D$
bez�glich $f$ wird wie �blich mit $f(g, t, \ell)$ bezeichnet.

[Spachvereinbarung: Sei $(g, t, \ell) \in G \times T \times L$ beliebig. Wir
sagen, $f(g, t, \ell)$ \emph{existiert}, genau dann wenn $(g, t, \ell) \in D$.]

F�r jedes Paar von generischem und entsprechenden realen Dokumenttyp gibt es in
der Datenbank eine Tabelle \code{gdim_\var{doctype}}, die die
Aufl�sungs-Abbildung implementiert. \var{doctype} ist er Name des realen
Dokumenttyps. Die Tabelle hat folgende Spalten:

\begin{table}
\head
  Spalte & Beschreibung
\body
  \code{theme} & Id des Themes\\
  \code{language} & Id der Spache\\
  \code{generic_document} & Id des generischen Dokuments\\
  \code{document} & Id des realen Dokuments\\
\end{table}

\section{Aufl�sung eines generischen Dokuments}

Seien $t$ und $\ell$ das Theme bzw. die Sprache eines Benutzers. Angenommen,
der Benutzer fordert das generische Dokument $g$ an. Gesucht ist das reale
Dokument $r$, das der Japs ausliefern soll. Dabei ist zu bedenken, dass auch
dann ein geeignetes reales Dokument ausgeliefert werden soll, wenn $f(g, t,
\ell)$ nicht existiert.

Die Regel f�r das Auffinden von $r$ lautet: $r$ ist das erste Dokument in der
folgenden Reihe, das existiert:

\begin{enumerate}
\item $f(g, t,\ell)$
\item $f(g, t_0,\ell)$
\item $f(g, t,\ell_\ast)$
\item $f(g, t_0,\ell_\ast)$
\item $f(g, t,\ell_0)$
\item $f(g, t_0,\ell_0)$
\end{enumerate}

Die Regel ist offenbar so konzipiert, dass immer zuerst ein Dokument in der
Sprache des Benutzers gesucht wird. Erst wenn weder im Benutzer- noch im
Default-Theme ein Dokument in dieser Sprache existiert, wird auf die neutrale
und dann auf die Default-Sprache zur�ckgegriffen.

Damit $r$ immer existiert, wird folgendes gefordert:

\begin{enumerate}
\setcounter{enumi}{6}
\item F�r alle $g \in G$ existiert $f(g, t_0, \ell_\ast)$ oder $f(g, t_0, \ell_0)$.
\end{enumerate}

In Worten: F�r jedes generische Dokument existiert das entsprechende reale
Dokument im Default-Theme und der neutralen Sprache oder das entsprechende
reale Dokument im Default-Theme und der Default-Sprache.

In der Software wird obige Regel durch das Java-Interface

\begin{preformatted}%
  net.mumie.cocoon.util.GenericDocumentResolver
\end{preformatted}

repr�sentiert. Dieses Interface definiert folgende Methode:

\begin{preformatted}%
  public int resolve (int typeOfGeneric,
                      int idOfGeneric,
                      int languageId,
                      int themeId)
    throws GenericDocumentResolveException;
\end{preformatted}

Die Argumente haben folgende Bedeutungen:

\begin{table}
\head
  Argument & Bedeutung
\body
  \code{typeOfGeneric} & Typ des generischen Dokuments, als Zahlencode\\
  \code{idOfGeneric} & Id der des generischen Dokuments\\
  \code{languageId} & Id der Sprache\\
  \code{themeId} & Id des Themes
\end{table}

Die Methode liefert f�r das durch die Argumente gegebene Tripel aus generischem
Dokument, Theme und Sprache die Id des entsprechenden realen Dokuments zur�ck.

-- ENDE DER DATEI --

\end{document}