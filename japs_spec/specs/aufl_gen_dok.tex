\documentclass{generic}

% Author: Tilman Rassy <rassy@math.tu-berlin.de>
% $Id: macros.tex,v 1.5 2006/10/05 15:58:42 rassy Exp $

% XML code:
\newcommand{\element}[1]{\code[xml-element]{#1}}
\newcommand{\attrib}[1]{\code[xml-attrib]{#1}}

% Database code:
\newcommand{\dbtable}[1]{\code[db-table]{#1}}
\newcommand{\dbcol}[1]{\code[db-column]{#1}}
\newcommand{\sql}[1]{\code[sql]{#1}}

% TeX code:
\newcommand{\texcmd}[1]{\code[tex-cmd]{\backslash#1}}
\newcommand{\texenv}[1]{\code[tex-env]{#1}}

% General
\newcommand{\val}[1]{\code[value]{#1}}
\newcommand{\prm}[1]{\code[param]{#1}}

% Other:
\newcommand{\notimpl}[1]{\emph[not-impl]{#1}}
\newcommand{\new}[1]{\emph[new]{#1}}
\newcommand{\deprecated}[1]{\emph[deprecated]{#1}}
\newcommand{\wvar}[1]{\var[weak]{#1}}


\begin{document}

\title{Aufl�sung generischer Dokumente}

\begin{authors}
  \author[rassy@math.tu-berlin.de]{Tilman Rassy}
\end{authors}

\version{$Id: aufl_gen_dok.tex,v 1.3 2007/09/24 12:57:20 rassy Exp $}

Unter der \emph{Aufl�sung eines generischen Dokuments} versteht man die
Aufgabe, zu einem gegebenem generischen Dokument, einem gegebenem Theme und
einer gegebenen Sprache das entsprechende "reale" Dokument aufzufinden. Diese
Spezifikation beschreibt die Hintergr�nde und Regeln der Aufl�sung generischer
Dokumente.

\tableofcontents

\section{Bezeichnungen}

In der Spezifikation werden durchg�ngig folgende Notationen verwendet:

\begin{table}[\cellaligns{cl}]
\head
Notation & Bedeutung
\body
$G$ & Menge aller generischen Dokumente \\
$T$ & Menge aller Themes \\
$L$ & Menge aller Sprachen \\
$R$ & Menge aller realen Dokumente \\
$t_0$ & Default-Theme \\
$\ell_0$ & Default-Sprache \\
$\ell_\ast$ & neutrale Sprache \\
$f$ & Aufl�sungs-Abbbildung (s. \ref{aufl_abb}) \\
$D$ & Definitionsbereich von $f$ (s. \ref{aufl_abb})
\end{table}

Die \emph{neutrale Sprache} kennzeichnet sprach-unspezifische Inhalte wie
z.B. Bilder oder CSS-Stylesheets. Ihr internationaler Sprach-Code ist
\code{zxx}.

\section{Die Aufl�sungs-Abbildung}\label{aufl_abb}

Sei $D$ die Menge aller Tripel $(g, t, \ell) \in G \times T \times L$, f�r die ein
reales Dokument existiert. Ferner sei $f$ die Abbildung $D \rightarrow R$, die
jedem $(g, t, \ell) \in D$ das entsprechende reale Dokument zuordnet. $f$ 
heisst \emph{Aufl�sungs-Abbildung}. Der Wert eines $(g, t, \ell) \in D$
bez�glich $f$ wird wie �blich mit $f(g, t, \ell)$ bezeichnet.

[Spachvereinbarung: Sei $(g, t, \ell) \in G \times T \times L$ beliebig. Wir
sagen, $f(g, t, \ell)$ \emph{existiert}, genau dann wenn $(g, t, \ell) \in D$.]

F�r jedes Paar von generischem und entsprechenden realen Dokumenttyp gibt es in
der Datenbank eine Tabelle \code{gdim_\var{doctype}}, die die
Aufl�sungs-Abbildung implementiert. \var{doctype} ist er Name des realen
Dokumenttyps. Die Tabelle hat folgende Spalten:

\begin{table}
\head
  Spalte & Beschreibung
\body
  \code{theme} & Id des Themes\\
  \code{language} & Id der Spache\\
  \code{generic_document} & Id des generischen Dokuments\\
  \code{document} & Id des realen Dokuments\\
\end{table}

\section{Aufl�sung eines generischen Dokuments}

Seien $t$ und $\ell$ das Theme bzw. die Sprache eines Benutzers. Angenommen,
der Benutzer fordert das generische Dokument $g$ an. Gesucht ist das reale
Dokument $r$, das der Japs ausliefern soll. Dabei ist zu bedenken, dass auch
dann ein geeignetes reales Dokument ausgeliefert werden soll, wenn $f(g, t,
\ell)$ nicht existiert.

Die Regel f�r das Auffinden von $r$ lautet: $r$ ist das erste Dokument in der
folgenden Reihe, das existiert:

\begin{enumerate}
\item $f(g, t,\ell)$
\item $f(g, t_0,\ell)$
\item $f(g, t,\ell_\ast)$
\item $f(g, t_0,\ell_\ast)$
\item $f(g, t,\ell_0)$
\item $f(g, t_0,\ell_0)$
\end{enumerate}

Die Regel ist offenbar so konzipiert, dass immer zuerst ein Dokument in der
Sprache des Benutzers gesucht wird. Erst wenn weder im Benutzer- noch im
Default-Theme ein Dokument in dieser Sprache existiert, wird auf die neutrale
und dann auf die Default-Sprache zur�ckgegriffen.

Damit $r$ immer existiert, wird folgendes gefordert:

\begin{enumerate}
\setcounter{enumi}{6}
\item F�r alle $g \in G$ existiert $f(g, t_0, \ell_\ast)$ oder $f(g, t_0, \ell_0)$.
\end{enumerate}

In Worten: F�r jedes generische Dokument existiert das entsprechende reale
Dokument im Default-Theme und der neutralen Sprache oder das entsprechende
reale Dokument im Default-Theme und der Default-Sprache.

In der Software wird obige Regel durch das Java-Interface

\begin{preformatted}%
  net.mumie.cocoon.util.GenericDocumentResolver
\end{preformatted}

repr�sentiert. Dieses Interface definiert folgende Methode:

\begin{preformatted}%
  public int resolve (int typeOfGeneric,
                      int idOfGeneric,
                      int languageId,
                      int themeId)
    throws GenericDocumentResolveException;
\end{preformatted}

Die Argumente haben folgende Bedeutungen:

\begin{table}
\head
  Argument & Bedeutung
\body
  \code{typeOfGeneric} & Typ des generischen Dokuments, als Zahlencode\\
  \code{idOfGeneric} & Id der des generischen Dokuments\\
  \code{languageId} & Id der Sprache\\
  \code{themeId} & Id des Themes
\end{table}

Die Methode liefert f�r das durch die Argumente gegebene Tripel aus generischem
Dokument, Theme und Sprache die Id des entsprechenden realen Dokuments zur�ck.

-- ENDE DER DATEI --

\end{document}