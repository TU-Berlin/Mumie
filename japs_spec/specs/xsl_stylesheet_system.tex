\documentclass{generic}


% ------------------------------------------------------------------------------
% Schriftgroessen
% ------------------------------------------------------------------------------

\renewcommand{\normalsize}{\fontsize{18}{22}\selectfont}
\renewcommand{\large}{\fontsize{20}{26}\selectfont}
\renewcommand{\Large}{\fontsize{22}{30}\selectfont}
\renewcommand{\small}{\fontsize{14}{16}\selectfont}
\newcommand{\smaller}{\fontsize{12}{14}\selectfont}
\newcommand{\verysmall}{\fontsize{6}{6}\selectfont}
\renewcommand{\footnotesize}{\fontsize{12}{14}\selectfont}
\newcommand{\tocsize}{\fontsize{12}{16}\selectfont}
\newcommand{\titlesize}{\fontsize{24}{36}\selectfont}

% ------------------------------------------------------------------------------
% Texthervorhebungen
% ------------------------------------------------------------------------------

%\definecolor{emphcolor}{rgb}{0.77,0.00,0.00}
%\renewcommand{\emph}[1]{\textbf{\textcolor{emphcolor}{#1}}}
\renewcommand{\emph}[1]{\textbf{#1}}

\newcommand{\term}[1]{\textit{#1}}
% \newcommand{\code}[1]{\texttt{#1}}
\newcommand{\var}[1]{\textit{#1}}

%\definecolor{headlinecolor}{rgb}{0.77,0.00, 0.00}
\definecolor{headlinecolor}{rgb}{0.67,0.09, 0.22}
\newcommand{\headline}[1]{\textbf{\textcolor{headlinecolor}{#1}}}

\definecolor{emphboxbordercolor}{rgb}{0.80,0.36,0.36}
\definecolor{emphboxcolor}{rgb}{0.98,0.98,0.77}
\newcommand{\emphbox}[2]{\fcolorbox{emphboxbordercolor}{emphboxcolor}{\parbox{#1}{#2}}}

\definecolor{labelcolor}{rgb}{0.35,0.35, 0.8}


% -------------------------------------------------------------------------------
% Pfeile
% -------------------------------------------------------------------------------

\newcommand{\myarrow}{{\Large\boldmath$\;\rightarrow\;$}}
\newcommand{\mylongarrow}{{\Large\boldmath$\;\longrightarrow\;$}}



% -------------------------------------------------------------------------------
% Befehle f�r Abbildungen 
% -------------------------------------------------------------------------------

\renewcommand{\thefigure}{\thesubsection.\arabic{figure}}

\newcommand{\mycaption}[2]{%
 \centerline{%
  \parbox{#1}{%
   \small \refstepcounter{figure} Figure \thefigure. #2 }}}


%---------------------------------------------------------------------------
% Gliederungsbefehle
%---------------------------------------------------------------------------

\makeatletter

\renewcommand{\section}{\@startsection
 {section}%                                   % Name
 {1}%                                         % Ebene
 {0pt}%                                       % Einzug
 {-0.0\baselineskip}%                        % Vorabstand 
 {0.85\baselineskip}%                         % Nachabstand
 {\fontsize{20}{24}\selectfont\itshape\bfseries}}%    % Stil

\makeatother

\newcommand{\mysection}[1]{%

\vspace*{-1.2\baselineskip}

{\color{headlinecolor}
\section{#1}}}

\newcounter{myappendix}

\newcommand{\myappendix}[2]{%
\hfill\makebox[0cm][l]{\tocref}

\vspace*{-1.2\baselineskip}

\refstepcounter{myappendix}
\section*{A\arabic{myappendix}\hspace{1em}#1}\hypertarget{#2}{}}



% -------------------------------------------------------------------------------
% Inhaltsverzeichnis 
% -------------------------------------------------------------------------------

\newcommand{\mytoctitle}[1]{\begin{center}\textbf{\hypertarget{inhalt}{#1}}\end{center}}

\newenvironment{mytoc}
  {\tocsize\textbf{\hypertarget{inhalt}{Inhalt}}\\\begin{enumerate}}
  {\end{enumerate}}

\newcommand{\mytocitem}[3]{\item[#1]\hyperlink{#3}{#2}}

\newcommand{\tocref}{\verysmall\hyperlink{inhalt}{TOC}\normalsize}

% -------------------------------------------------------------------------------
% Listen 
% -------------------------------------------------------------------------------

%\newcommand{\mylabelitemi}{\labelitemi}
\newcommand{\mylabelitemi}{\raisebox{0.25ex}{{\small $\blacktriangleright$}\hspace{0.1em}}}

%\newcommand{\mylabelitemii}{\labelitemii}
%\newcommand{\mylabelitemii}{\raisebox{0.25ex}{{\small $\triangleright$}\hspace{0.1em}}}
\newcommand{\mylabelitemii}{\raisebox{0.1ex}{\large\bfseries-}\hspace{0.075em}}

\newenvironment{mylist}[1][\mylabelitemi]
  {\begin{list}{{\color{labelcolor}#1}}
    {\setlength{\itemindent}{0cm}
     \setlength{\labelwidth}{1em}
     \setlength{\leftmargin}{0.5cm}
     \setlength{\rightmargin}{1cm}}}
  {\end{list}}

\newenvironment{myenum}
  {\begin{list}{\theenumi}
    {\usecounter{enumi}
     \setlength{\itemindent}{0cm}
     \setlength{\labelwidth}{1em}
     \setlength{\leftmargin}{0.5cm}
     \setlength{\rightmargin}{1cm}}}
  {\end{list}}

\newcommand{\pitem}{\pause\item}
\newcommand{\itemii}{\item[\mylabelitemii]}

\newcounter{romanlistcounter}
\newenvironment{romanlist}{\begin{list}%
 {(\roman{romanlistcounter})\hfill}{\usecounter{romanlistcounter}%
 \topsep0.05\baselineskip% 
 \partopsep0.85\baselineskip%
 \leftmargin2em%
 \labelwidth2em%
 \labelsep0pt%
 \itemindent0pt%
 \parsep0.15\baselineskip%
 \itemsep0.0\baselineskip}}%
 {\end{list}}

\newcounter{alphlistcounter}
\newenvironment{alphlist}{\begin{list}%
 {(\alph{alphlistcounter})\hfill}{\usecounter{alphlistcounter}%
 \topsep0.05\baselineskip% 
 \partopsep0.85\baselineskip%
 \leftmargin1.8em%
 \labelwidth1.8em%
 \labelsep0pt%
 \itemindent0pt%
 \parsep0.15\baselineskip%
 \itemsep0.0\baselineskip}}%
 {\end{list}}

\newcounter{arabiclistcounter}
\newenvironment{arabiclist}{\begin{list}%
 {(\arabic{arabiclistcounter})\hfill}{\usecounter{arabiclistcounter}%
 \topsep0.05\baselineskip% 
 \partopsep0.85\baselineskip%
 \leftmargin1.8em%
 \labelwidth1.8em%
 \labelsep0pt%               
 \itemindent0pt%
 \parsep0.15\baselineskip%
 \itemsep0.0\baselineskip}}%
 {\end{list}}





\begin{document}

\title{XSL-Stylesheet-System}

\begin{authors}
  \author[rassy@math.tu-berlin.de]{Tilman Rassy}
\end{authors}

\version{$Id: xsl_stylesheet_system.tex,v 1.4 2006/10/13 13:00:42 rassy Exp $}

Diese Spezifikation beschreibt das XSL-Stylesheet-System des Japs. Sie erkl�rt
das sogenannte \emph{Root-Stylesheet} und gibt eine �bersicht �ber die sogenannten
\emph{Standard-XSL-Stylesheets}.

\tableofcontents

\section{Allgemeines}

"Reale" (d.h. nicht-generische) XSL-Stylesheets werden im Japs durch den
Dokument-Typ \val{xsl_stylesheet} dargestellt, generische durch den
Dokument-Typ \val{generic_xsl_stylesheet}. Der Content von realen
XSL-Stylesheets kann Mumie-spezifische \emph{XSL-Erweiterungselemente}
enthalten. Diese werden im Rahmen der dynamischen Seitenerzeugung "{}on-the-fly" in
normales XSL umgewandelt. Letzteres wird vom Root-Stylesheet erledigt (s.
\ref{root_stylesheet}).

\section{Das Root-Stylesheet}\label{root_stylesheet}

Das Root-Stylesheet transformiert den Content von XSL-Stylesheets in normales
XSL, indem es die Mumie-spezifischen Erweiterungselemente umwandelt.
Insbesondere f�gt es die sogenannte \href{japs_xsl_bibliothek.xhtml}{Japs-XSL-Bibliothek}
ein, sofern das transformierte Stylesheet dies verlangt.

Das Root-Stylesheet ist kein Dokument im Japs-Sinn. Es wird als Datei in das
Jar-Archiv \file{mumie-japs.jar} eingebunden, und zwar unter dem Pfad:

\begin{preformatted}[code]%
  net/mumie/cocoon/transformes/rootxsl.xsl
\end{preformatted}

Diese technische Sonderrolle ist notwendig, da es sonst bei der
"{}on-the-fly"-Generierung von XSL-Stylesheets zu unendlichen Schleifen kommen
w�rde.

\section{Standard-XSL-Stylesheets}

\emph{Standard-XSL-Stylesheets} sind solche, die auf jeden Fall voranden sein
m�ssen, damit der Japs funktioniert. Ihre Dateinamen folgen der entsprechenden
\href{dateinamen.xhtml}{Spezifikation}. Dateinamen von generischen
XSL-Stylesheets beginnen immer mit \file{g_xsl}, solche von von "realen" immer
mit \file{xsl}. Die meisten Standard-XSL-Stylesheets befinden sich in der Section
\file{system/common}. 

Im folgenden werden alle Standard-XSL-Stylesheets, nach Kategorien geordnet,
aufgelistet und kurz beschrieben. 



\subsection{Hilfsfunktionalit�ten}

Stylesheets, die n�tzliche Hilfsfunktionalit�ten bereitsstellen. Nicht
top-level verwendbar, m�ssen von anderen Stylesheets importiert oder inkludiert
werden.

\begin{table}
  \head
    Name & Beschreibung & Section
  \body
    \file{g_xsl_util} & Verschiedenes & \file{system/common} \\
    \file{xsl_util} & Default-Implementierung von \file{g_xsl_util} & \file{system/common} \\
    \file{xsl_math_signfix} & Korrektur von mathematischen Vorzeichen. Ersetzt
    z.B $4x^2--2x+-6$ durch $4x^2+2x-6$. & \file{system/common}
\end{table}

\subsection{Verarbeitung von MmTeX-Output}

Stylesheets, die MmTeX-Output verarbeiten und von anderen Stylesheets
importiert oder inkludiert werden.

\begin{table}
  \head
    Name & Beschreibung & Section
  \body
    \file{g_xsl_mmtex_stdlayout} &  Abs�tze, Listen, Tabellen, Inline-Markup
    (z.B. Hervorhebungen) & \file{system/common}\\
    \file{g_xsl_mmtex_media} & Bilder, Applets usw. & \file{system/common}\\
    \file{g_xsl_mmtex_math} & Mathematische Formeln & \file{system/common} \\
    \file{xsl_mmtex_stdlayout} & Default-Implementierung von
      \file{g_xsl_mmtex_stdlayout} & \file{system/common} \\
    \file{xsl_mmtex_media} & Default-Implementierung von
      \file{g_xsl_mmtex_media} & \file{system/common} \\
    \file{xsl_mmtex_math} & Default-Implementierung von
      \file{g_xsl_mmtex_math} & \file{system/common} \\
\end{table}

\subsection{Transformation von Dokumenten}

Stylesheets, die den Content von Dokumenten bestimmter Typen
transformieren. Alle "top-level".

\begin{table}
  \head
    Name & Beschreibung & Section
  \body
    \file{xsl_text}
    & Transformation XML-kodierter Plain-Text-Dokumente (z.B. JavaScript)
    & \file{system/misc}
    \\ 
    \file{xsl_css}
    & Transformation von XML-kodierten CSS-Stylesheets
    & \file{system/misc}
    \\
    \file{g_xsl_page}
    & Transformation von Dokumenten vom Typ \val{page}
    & \file{system/misc}
    \\
    \file{g_xsl_element}
    & Transformation von Dokumenten vom Typ \val{element} oder \val{subelement}
    & \file{system/element}
    \\
    \file{g_xsl_problem}
    & Transformation von Dokumenten vom Typ \val{problem}
    & \file{system/problem}
    \\
    \file{g_xsl_course_overview}
    & Transformation von Dokumenten vom Typ \val{course}, \val{course_section}
      oder \val{worksheet} zu einer �bersichtsseite
    & \file{system/course}
    \\
    \file{g_xsl_course_nav}
    & Transformation von Dokumenten vom Typ \val{course}, \val{course_section}
      oder \val{worksheet} zu einem Navigationsnetz
    & \file{system/course}
    \\
    \file{xsl_page}
    & Default-Implementierung von \file{g_xsl_page}
    & \file{system/misc}
    \\
    \file{xsl_element}
    & Default-Implementierung von \file{g_xsl_element}
    & \file{system/element}
    \\
    \file{xsl_problem}
    & Default-Implementierung von \file{g_xsl_problem}
    & \file{system/problem}
    \\
    \file{xsl_course_overview}
    & Default-Implementierung von \file{g_xsl_course_overview}
    & \file{system/course}
    \\ 
    \file{xsl_course_nav}
    & Default-Implementierung von \file{g_xsl_course_nav}
    & \file{system/course}
    \\ 
\end{table}

-- ENDE DER DATEI --

\end{document}