\documentclass{generic}

% Author: Tilman Rassy <rassy@math.tu-berlin.de>
% $Id: macros.tex,v 1.5 2006/10/05 15:58:42 rassy Exp $

% XML code:
\newcommand{\element}[1]{\code[xml-element]{#1}}
\newcommand{\attrib}[1]{\code[xml-attrib]{#1}}

% Database code:
\newcommand{\dbtable}[1]{\code[db-table]{#1}}
\newcommand{\dbcol}[1]{\code[db-column]{#1}}
\newcommand{\sql}[1]{\code[sql]{#1}}

% TeX code:
\newcommand{\texcmd}[1]{\code[tex-cmd]{\backslash#1}}
\newcommand{\texenv}[1]{\code[tex-env]{#1}}

% General
\newcommand{\val}[1]{\code[value]{#1}}
\newcommand{\prm}[1]{\code[param]{#1}}

% Other:
\newcommand{\notimpl}[1]{\emph[not-impl]{#1}}
\newcommand{\new}[1]{\emph[new]{#1}}
\newcommand{\deprecated}[1]{\emph[deprecated]{#1}}
\newcommand{\wvar}[1]{\var[weak]{#1}}


\begin{document}

\title{XSL-Stylesheet-System}

\begin{authors}
  \author[rassy@math.tu-berlin.de]{Tilman Rassy}
\end{authors}

\version{$Id: xsl_stylesheet_system.tex,v 1.4 2006/10/13 13:00:42 rassy Exp $}

Diese Spezifikation beschreibt das XSL-Stylesheet-System des Japs. Sie erkl�rt
das sogenannte \emph{Root-Stylesheet} und gibt eine �bersicht �ber die sogenannten
\emph{Standard-XSL-Stylesheets}.

\tableofcontents

\section{Allgemeines}

"Reale" (d.h. nicht-generische) XSL-Stylesheets werden im Japs durch den
Dokument-Typ \val{xsl_stylesheet} dargestellt, generische durch den
Dokument-Typ \val{generic_xsl_stylesheet}. Der Content von realen
XSL-Stylesheets kann Mumie-spezifische \emph{XSL-Erweiterungselemente}
enthalten. Diese werden im Rahmen der dynamischen Seitenerzeugung "{}on-the-fly" in
normales XSL umgewandelt. Letzteres wird vom Root-Stylesheet erledigt (s.
\ref{root_stylesheet}).

\section{Das Root-Stylesheet}\label{root_stylesheet}

Das Root-Stylesheet transformiert den Content von XSL-Stylesheets in normales
XSL, indem es die Mumie-spezifischen Erweiterungselemente umwandelt.
Insbesondere f�gt es die sogenannte \href{japs_xsl_bibliothek.xhtml}{Japs-XSL-Bibliothek}
ein, sofern das transformierte Stylesheet dies verlangt.

Das Root-Stylesheet ist kein Dokument im Japs-Sinn. Es wird als Datei in das
Jar-Archiv \file{mumie-japs.jar} eingebunden, und zwar unter dem Pfad:

\begin{preformatted}[code]%
  net/mumie/cocoon/transformes/rootxsl.xsl
\end{preformatted}

Diese technische Sonderrolle ist notwendig, da es sonst bei der
"{}on-the-fly"-Generierung von XSL-Stylesheets zu unendlichen Schleifen kommen
w�rde.

\section{Standard-XSL-Stylesheets}

\emph{Standard-XSL-Stylesheets} sind solche, die auf jeden Fall voranden sein
m�ssen, damit der Japs funktioniert. Ihre Dateinamen folgen der entsprechenden
\href{dateinamen.xhtml}{Spezifikation}. Dateinamen von generischen
XSL-Stylesheets beginnen immer mit \file{g_xsl}, solche von von "realen" immer
mit \file{xsl}. Die meisten Standard-XSL-Stylesheets befinden sich in der Section
\file{system/common}. 

Im folgenden werden alle Standard-XSL-Stylesheets, nach Kategorien geordnet,
aufgelistet und kurz beschrieben. 



\subsection{Hilfsfunktionalit�ten}

Stylesheets, die n�tzliche Hilfsfunktionalit�ten bereitsstellen. Nicht
top-level verwendbar, m�ssen von anderen Stylesheets importiert oder inkludiert
werden.

\begin{table}
  \head
    Name & Beschreibung & Section
  \body
    \file{g_xsl_util} & Verschiedenes & \file{system/common} \\
    \file{xsl_util} & Default-Implementierung von \file{g_xsl_util} & \file{system/common} \\
    \file{xsl_math_signfix} & Korrektur von mathematischen Vorzeichen. Ersetzt
    z.B $4x^2--2x+-6$ durch $4x^2+2x-6$. & \file{system/common}
\end{table}

\subsection{Verarbeitung von MmTeX-Output}

Stylesheets, die MmTeX-Output verarbeiten und von anderen Stylesheets
importiert oder inkludiert werden.

\begin{table}
  \head
    Name & Beschreibung & Section
  \body
    \file{g_xsl_mmtex_stdlayout} &  Abs�tze, Listen, Tabellen, Inline-Markup
    (z.B. Hervorhebungen) & \file{system/common}\\
    \file{g_xsl_mmtex_media} & Bilder, Applets usw. & \file{system/common}\\
    \file{g_xsl_mmtex_math} & Mathematische Formeln & \file{system/common} \\
    \file{xsl_mmtex_stdlayout} & Default-Implementierung von
      \file{g_xsl_mmtex_stdlayout} & \file{system/common} \\
    \file{xsl_mmtex_media} & Default-Implementierung von
      \file{g_xsl_mmtex_media} & \file{system/common} \\
    \file{xsl_mmtex_math} & Default-Implementierung von
      \file{g_xsl_mmtex_math} & \file{system/common} \\
\end{table}

\subsection{Transformation von Dokumenten}

Stylesheets, die den Content von Dokumenten bestimmter Typen
transformieren. Alle "top-level".

\begin{table}
  \head
    Name & Beschreibung & Section
  \body
    \file{xsl_text}
    & Transformation XML-kodierter Plain-Text-Dokumente (z.B. JavaScript)
    & \file{system/misc}
    \\ 
    \file{xsl_css}
    & Transformation von XML-kodierten CSS-Stylesheets
    & \file{system/misc}
    \\
    \file{g_xsl_page}
    & Transformation von Dokumenten vom Typ \val{page}
    & \file{system/misc}
    \\
    \file{g_xsl_element}
    & Transformation von Dokumenten vom Typ \val{element} oder \val{subelement}
    & \file{system/element}
    \\
    \file{g_xsl_problem}
    & Transformation von Dokumenten vom Typ \val{problem}
    & \file{system/problem}
    \\
    \file{g_xsl_course_overview}
    & Transformation von Dokumenten vom Typ \val{course}, \val{course_section}
      oder \val{worksheet} zu einer �bersichtsseite
    & \file{system/course}
    \\
    \file{g_xsl_course_nav}
    & Transformation von Dokumenten vom Typ \val{course}, \val{course_section}
      oder \val{worksheet} zu einem Navigationsnetz
    & \file{system/course}
    \\
    \file{xsl_page}
    & Default-Implementierung von \file{g_xsl_page}
    & \file{system/misc}
    \\
    \file{xsl_element}
    & Default-Implementierung von \file{g_xsl_element}
    & \file{system/element}
    \\
    \file{xsl_problem}
    & Default-Implementierung von \file{g_xsl_problem}
    & \file{system/problem}
    \\
    \file{xsl_course_overview}
    & Default-Implementierung von \file{g_xsl_course_overview}
    & \file{system/course}
    \\ 
    \file{xsl_course_nav}
    & Default-Implementierung von \file{g_xsl_course_nav}
    & \file{system/course}
    \\ 
\end{table}

-- ENDE DER DATEI --

\end{document}