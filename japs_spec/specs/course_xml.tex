\documentclass{generic}

% Author: Tilman Rassy <rassy@math.tu-berlin.de>
% $Id: macros.tex,v 1.5 2006/10/05 15:58:42 rassy Exp $

% XML code:
\newcommand{\element}[1]{\code[xml-element]{#1}}
\newcommand{\attrib}[1]{\code[xml-attrib]{#1}}

% Database code:
\newcommand{\dbtable}[1]{\code[db-table]{#1}}
\newcommand{\dbcol}[1]{\code[db-column]{#1}}
\newcommand{\sql}[1]{\code[sql]{#1}}

% TeX code:
\newcommand{\texcmd}[1]{\code[tex-cmd]{\backslash#1}}
\newcommand{\texenv}[1]{\code[tex-env]{#1}}

% General
\newcommand{\val}[1]{\code[value]{#1}}
\newcommand{\prm}[1]{\code[param]{#1}}

% Other:
\newcommand{\notimpl}[1]{\emph[not-impl]{#1}}
\newcommand{\new}[1]{\emph[new]{#1}}
\newcommand{\deprecated}[1]{\emph[deprecated]{#1}}
\newcommand{\wvar}[1]{\var[weak]{#1}}


\begin{document}

\title{Course-XML}

\begin{authors}
  \author[rassy@math.tu-berlin.de]{Tilman Rassy}
\end{authors}
\version{$Id: course_xml.tex,v 1.5 2006/08/11 16:18:22 rassy Exp $}

\tableofcontents

\section{Grunds�tzliche Struktur}

\begin{enumerate}
  
\item Namespace:
  \val{http://www.mumie.net/xml-namespace/document/content/course}

  �blicher Prefix: \val{crs}
  
\item Root-Element: \element{course}
  
\item Darin:

  \begin{enumerate}
  
  \item Ein optionales Element \element{abstract}; enth�lt eine Zusammenfassung
    des Kurses
  
  \item Ein Element \element{structure}; beschreibt die Netzstruktur des Kurses
  
  \item Ein Element \element{thread}; beschreibt den "roten Faden" des Kurses

  \end{enumerate}

\end{enumerate}



\section{Das abstract-Element}
  
Das \element{abstract}-Element darf Text mit den �blichen Formatierungen
(Abs�tze, Listen, Tabellen usw.) und mathematische Formeln enthalten, jedoch
vorerst keine Multimedia-Objekte (Bilder usw.). Die genaue Spezifikation
steht noch aus.



\section{Das structure-Element}
  
\begin{enumerate}
    
\item Das \element{structure}-Element selbst enth�lt beliebig viele
  \element{course_section}- und/oder \element{branches}-Elemente.
    
\item Jedes \element{branches}-Element enth�lt zwei oder mehr
  \element{branch}-Elemente.
    
\item Jedes \element{branch}-Element enth�lt beliebig viele
  \element{course_section}- und/oder \element{branches}-Elemente.
    
\item Jedes \element{course_section}-Element entspricht einer im Kurs
  enthaltenen course_section. Es hat die folgenden Attribute:

  \begin{enumerate}
    
  \item \attrib{lid} -- Binnen-ID der course_section
    
  \item \attrib{posx}, \attrib{posy} -- X- und Y-Koordinate der course_section
    in der grafischen Darstellung.

  \end{enumerate}
    
\item Jedes \element{course_section}-Element kann beliebig viele
  (einschliesslich 0) \element{course_subsection}-Kindelemente haben. Diese
  entsprechen den an die course_section "angeh�ngten" course_subsections.
  
\item Jedes \element{course_subsection}-Element ist leer, hat aber folgende
  Attribute:

  \begin{enumerate}
    
  \item \attrib{lid} -- Binnen-ID der course_subsection
    
  \item \attrib{align} -- Gibt an, an welcher Ecke der course_section die course_subsection
    angh�ngt ist. M�gliche Werte (selbsterkl�rend): \val{topleft}, \val{topright},
    \val{bottomleft}, \val{bottomright}
    
  \item \attrib{count} -- Gibt an, die wievielte course_subsection die betreffende
    course_subsection in der entsprechenden Ecke ist. Z�hlung von innen nach
    aussen, bei 1 beginnend.

    Beispiel:

    \image{course\_darstellung.png}

%    \begin{preformatted}%

%      +--------+
%      |        | <-- align="topleft" count="1"
%      |        |
%      |    +--------+
%      |    |        | <-- align="topleft" count="0"
%      +----|        |
%           |    +-------------+
%           |    |             |
%           + ---|             |
%                |             |
%                |             |
%                |             |
%                |             |----+
%                |             |    |
%                +-------------+    | <-- align="bottomright"
%                          |        |     count="0"
%                          |        |
%                          +--------+
%    \end{preformatted}

  \end{enumerate}
  
\item Jedes \element{branches}-Element entspricht einer Verzweigung im Netz;
  jedes \element{branch}-Kindelement entspricht dabei einem neuen Zweig. Das
  \element{branches}-Element hat die folgenden Attribute:

  \begin{enumerate}
    
  \item \attrib{type} -- Typ der Verzweigung. Entweder \val{and} oder \val{or}.
    
  \item \attrib{startposx}, \attrib{startposy} -- X- bzw. Y-Koordinate des
    �ffnenden (also oberen) Verzweigungpunkts.
    
  \item \attrib{endposx}, \attrib{endposy} X- bzw. Y-Koordinate des
    schlie�enden (also unteren) Verzweigungpunkts.

    \image{course\_verzweigungspunkte.png}

%    \begin{preformatted}%
%                    .
%                    .   
%                    . �ffnender Verzweigungspunkt
%                    |/
%                +---o---+
%                |       |
%                .       .
%                .       .
%                .       .
%                |       |
%                +---o---+
%                    |\verb'\'
%                    . schlie�ender Verzweigungspunkt
%                    .
%                    .
%    \end{preformatted}

  \item \attrib{has\_end} -- zeigt an, ob es sich um eine "offene"
  Verzweigung (i.e. eine Verzweigung ohne Endpunkt) handelt oder nicht.

  \end{enumerate}
  
  Das Attribut \attrib{endposx} \emph{mu�} vorhanden sein, wenn das
  Attribut \attrib{has\_end} auf \val{true} gesetzt ist; es \emph{darf
    nicht} vorhanden sein, wenn das Attribut \attrib{has\_end} auf
  \val{false} gesetzt ist.
  
\item Auch das \element{branch}-Element, kennt das Attribut
  \attrib{has\_end}; allerdings ist es hier, im Gegensatz zum
  \element{branches}-Element, optional. Hat dieses Attribut den Wert
  \val{false}, dann repr�sentiert das \element{branch}-Element einen
  Zweig, der \emph{nicht} in einem schlie�enden Verzweigungspunkt endet.
  Anderenfalls -- insbesondere, wenn dieses Attribut fehlt -- endet
  dieser Zweig in einem schlie�enden Verzweigungspunkt.

\end{enumerate}


\section{Das thread-Element}

\begin{enumerate}
    
\item Das \element{thread}-Element darf beliebig viele
  \element{course_section}- Kindelemente haben. Diese definieren, in der
  Reihenfolge ihres Auftretens, den roten Faden.
    
\item Jedes \element{course_section}-Kindelement ist leer, hat aber ein
  Attribut, n�mlich \attrib{lid}, die Binnen-ID der course_section.

\end{enumerate}


\section{XML-Elemente}

\begin{preformatted}%
<course>
  <!-- Content: abstract? structure thread -->
</course>
\end{preformatted}

\begin{preformatted}%
<abstract>
  <!-- Content: ABSTRACT_MISC -->
</abstract>
\end{preformatted}


\begin{preformatted}%
<structure>
  <!-- Content: course_section* branches* -->
</structure>
\end{preformatted}


\begin{preformatted}%
<thread>
  <!-- Content: course_section* -->
</thread>
\end{preformatted}

Als Nachfahre von \element{structure}:
\begin{preformatted}%
<course_section
  lid="LID"
  posx="NUMBER"
  posy="NUMBER">
  <!-- Content: course_subsection* -->
</course_section>
\end{preformatted}

Als Nachfahre von \element{thread}:
\begin{preformatted}%
<course_section
  lid="LID"/>
\end{preformatted}

\begin{preformatted}%
<course_subsection
  lid="LID"
  align="topleft|topright|bottomleft|bottomright"
  count="NUMBER"/>
\end{preformatted}

\begin{preformatted}%
<branches
  type="and|or"
  has_end="true|false"
  startposx="NUMBER"
  startposy="NUMBER"
  endposx="NUMBER"
  endposy="NUMBER">
  <!-- Content: branch+ -->
</branches>
\end{preformatted}

\begin{preformatted}%
<branch 
  has_end="true|false">
  <!-- Content: course_section+ -->
</branch>
\end{preformatted}

Abk�rzungen/Platzhalter:

\begin{description}[code-doc]
  
\item[ABSTRACT_MISC] Text mit den �blichen Formatierungen (Abs�tze, Listen,
  Tabellen usw.), mathematische Formeln, vorerst keine Multimedia-Objekte
  (Bilder usw.). Genaue Spezifikation steht noch aus.
  
\item[LID] Binnen-Id
  
\item[NUMBER] Ganze Zahl

\end{description}

\section{Beispiel}

\begin{preformatted}[code]%
<crs:course
  xmlns:crs="http://www.mumie.net/xml-namespace/document/content/course">
  <crs:abstract>
    <!-- Abstract -->
  </crs:abstract>
  <crs:structure>
    <crs:course_section lid="1" posx="72" posy="32"/>
    <crs:branches type="or" startposx="72" startposy="62" endx="72" endy="152">
      <crs:branch>
        <crs:course_section lid="3" posx="112" posy="102">
          <crs:course_subsection lid="5" align="bottomleft" count="0"/>
          <crs:course_subsection lid="6" align="bottomleft" count="1"/>
          <crs:course_subsection lid="7" align="bottomright" count="0"/>
        </crs:course_section>
      </crs:branch>
      <crs:branch>
        <crs:course_section lid="2" posx="32" posy="102">
          <crs:course_subsection lid="8" align="topleft" count="0"/>
        </crs:course_section>
      </crs:branch>
    </crs:branches>
    <crs:course_section lid="4" posx="72" posy="202"/>
  </crs:structure>
  <crs:thread>
    <crs:course_section lid="1"/>
    <crs:course_section lid="2"/>
    <crs:course_section lid="4"/>
  </crs:thread>
</crs:course>
\end{preformatted}


-- ENDE DER DATEI --

\end{document}
