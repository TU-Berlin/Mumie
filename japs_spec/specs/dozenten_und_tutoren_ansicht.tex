\documentclass{generic}

% Author: Tilman Rassy <rassy@math.tu-berlin.de>
% $Id: macros.tex,v 1.5 2006/10/05 15:58:42 rassy Exp $

% XML code:
\newcommand{\element}[1]{\code[xml-element]{#1}}
\newcommand{\attrib}[1]{\code[xml-attrib]{#1}}

% Database code:
\newcommand{\dbtable}[1]{\code[db-table]{#1}}
\newcommand{\dbcol}[1]{\code[db-column]{#1}}
\newcommand{\sql}[1]{\code[sql]{#1}}

% TeX code:
\newcommand{\texcmd}[1]{\code[tex-cmd]{\backslash#1}}
\newcommand{\texenv}[1]{\code[tex-env]{#1}}

% General
\newcommand{\val}[1]{\code[value]{#1}}
\newcommand{\prm}[1]{\code[param]{#1}}

% Other:
\newcommand{\notimpl}[1]{\emph[not-impl]{#1}}
\newcommand{\new}[1]{\emph[new]{#1}}
\newcommand{\deprecated}[1]{\emph[deprecated]{#1}}
\newcommand{\wvar}[1]{\var[weak]{#1}}


\begin{document}

\title{Dozenten- und Tutoren-Ansicht}

\begin{authors}
  \author[rassy@math.tu-berlin.de]{Tilman Rassy}
\end{authors}

\version{$Id: dozenten_und_tutoren_ansicht.tex,v 1.4 2006/08/11 16:18:22 rassy Exp $}

\tableofcontents

Diese Spezifikation beschreibt, was Dozenten und Tutoren zu sehen bekommen, wenn
sie in Kursen browsen, und welche speziellen Features ihnen dabei zur Verf�gung
stehen.

\section{Vorbemerkungen, Unterschiede zur Studenten-Ansicht}

Betrachtet ein Student eine Aufgabe im Browser, so gilt abh�ngig vom
Bearbeitungszeitraum des �bungsblatts folgendes:

\begin{itemize}
\item Hat der Bearbeitungszeitraum noch nicht begonnen, so sollte der Student
  die Aufgabe gar nicht sehen. (Versucht er es doch, bekommt er eine
  Fehlermeldung.)
\item L�uft der Bearbeitungszeitraum, so kann der Student die Aufgabe
  bearbeiten.
\item Ist der Bearbeitungszeitraum bereits abgelaufen, so kann der Student
  die Aufgabe nicht mehr bearbeiten, er bekommt aber die Musterl�sungen und
  eine korrigierte Fassung seiner L�sungen.
\end{itemize}

Es ist klar, dass dies f�r Dozenten und Tutoren anders sein muss.
Sinnvollerweise sollten Dozenten und Tutoren Aufgaben in zwei Modi betrachten
k�nnen: im ersten Modus werden, unabh�ngig vom Bearbeitungszeitraum, die
Musterl�sungen angezeigt, im zweiten die L�sungen eines bestimmten, vorher
ausgew�hlten Studenten. Dar�berhinnaus sollten Dozenten und Tutoren eine
�bersichtsfunktion haben, mit der sie die Leistungen der Studenten schnell
�berblicken k�nnen.

F�r Nicht-Aufgaben (Elemente mit Kategorie $\neq$ "problem", Subelemente,
Kurse, Course-Sections und -Subsections) sollte die Ansicht f�r Studenten,
Dozenten und Tutoren gleich sein.

\section{Solution und Correction View}

Alle Dokumente eines Kurses k�nnen Dozenten und Tutoren in zwei Modi betrachten
(vgl. o.):

\begin{enumerate}
\item \emph{Solution View:} Bei Aufgaben werden die Musterl�sungen angezeigt,
  ausserdem ein Button zum Wechseln in den Correction View. Bei Nicht-Aufgaben
  normale Ansicht bis auf den Button zum Wechseln in den Correction View.
\item \emph{Correction View:} Bei Aufgaben werden die Antworten eines zuvor
  ausgew�hlten Studenten angezeigt, ausserdem Name und Id des Studenten, ein
  Button zum Wechseln in den Solution View und ein Button zum Wechseln des
  Studenten. Bei Nicht-Aufgaben normale Ansicht bis auf Namen und Id des
  Studenten und den Buttons zum Wechseln in den Solution View bzw. Wechseln des
  Studenten.
\end{enumerate}

Default ist der Solution View. Der Correction View steht erst nach dem Ende der
Bearbeitungszeit zur Verf�gung. Dann kann zwischen Solution und Correction View
sowie zwischen Correction Views zu verschiedenen Studenten beliebig gewechselt
werden, wie durch folgendes Schema veranschaulicht:

\image{solution\_view\_correction\_view.png}

Der Wechsel vollzieht sich immer zum selben Dokument im selben Kurs,
d.h. befindet sich der Benutzer in einem Kurs C und dort (z.B.) in einer
Aufgabe A und wechselt er vom Solution View zum Correction View f�r einen
Studenten S, so betrachtet er anschliessend Aufgabe A in Kurs C f�r Student S.

\section{Studenten-Auswahl-Dialog}

Wird vom Solution in den Correction View oder innerhalb des Correction View zu
einem anderen Studenten gewechselt, so erscheint ein Dialog zur Auswahl des
Studenten. Dieser enth�lt eine Liste mit den Ids, Nachnamen, Vornamen und
Account-Namen der Studenten. Wird ein Listenelement angeklickt, gilt der
entsprechende Student als ausgew�hlt und der Dialog wird geschlossen.

Der Dialog soll folgende Optionen anbieten:

\begin{enumerate}
\item Ordnen der Liste wahlweise nach: Ids, Nachnamen, Vornamen oder
  Account-Namen.
\item �ffnen der neuen Seite im aktuellen oder einem neuen Browser-Fenster (die
  neue Seite ist diejenige, die nach dem �bergang vom Solution in den
  Correction View bzw. nach dem Wechsel zu einem anderen Studenten zu sehen
  sein soll).
\item Auflistung aller Studenten oder nur derjenigen eines bestimmten
  Tutoriums.
\end{enumerate}

Das Dialogfenster k�nnte also z.B. wie folgt aussehen:

\image{studenten\_auswahl\_dialog.png}

\section{Ansichten}
Spezifikation TutorView
\begin{description}
\item [\var{X}\label{single_scores}] Eine (Matrix-)Tabelle f�r jedes Tutorium
  \element{tutorial} und jedes �bungsblatt \element{course_subsection} mit:
\begin{itemize}
\item den verschiedenen Aufgaben \element{\$problem} als Spalten und 
\item den verschiedenen Teilnehmern \element{\$user} als Zeilen.
\item In den Zellen dieser Tabelle stehen die erreichten Punkte \element{$x$} im Verh�ltnis (als
Bruch $\frac{x}{y}$) zur - �ber die Spalte konstanten - Anzahl \element{$y$} der
erreichbaren Punkte.
\end{itemize}
Die �bersicht k�nnte also z.B. wie folgt aussehen:

\image{studentenleistungen\_uebersicht.png}

\item [\var{Y}\label{summed_scores}] Eine (Listen-)Tabelle f�r jedes Tutorium
  \element{tutorial} (und Summen �ber gewisse, z.B. alle einer Lehrveranstaltung
  \element{class}, s.u.)) mit:\\
\begin{itemize}
\item den �bungsbl�ttern \element{\$course_subsection} als (Kopf-)Zeilen,
\item den (optional ausblendbaren) Aufgaben \element{\$problem} als (Unter-)Zeilen
\item und verschiedenen Attributen \element{\$attr} als Spalten.\\
\end{itemize}
In den Zellen stehen die zugeh"origen Attributwerte:
\begin{itemize}
\item Name
\item Lid
\item Summe $\sum$ �ber alle Studenten des Tutoriums (gewisser Tutorien) der erreichten Punkte.
\end{itemize}
Achtung: Die Attribute von �bungsbl�ttern und Aufgaben m�ssen
identifizierbar sein. Das heisst: in einer Spalte dieser Tabelle stehen
untereinander Lids von �bungsbl�ttern und Aufgaben. Dem sollte bei der
Vereinbarung von Konventionen bei der Lid-Vergabe im CourseCreator Rechnung
getragen werden.
\end{description}

-- ENDE DER DATEI --

\end{document}
