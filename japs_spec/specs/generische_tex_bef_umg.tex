\documentclass{generic}

% Author: Tilman Rassy <rassy@math.tu-berlin.de>
% $Id: macros.tex,v 1.5 2006/10/05 15:58:42 rassy Exp $

% XML code:
\newcommand{\element}[1]{\code[xml-element]{#1}}
\newcommand{\attrib}[1]{\code[xml-attrib]{#1}}

% Database code:
\newcommand{\dbtable}[1]{\code[db-table]{#1}}
\newcommand{\dbcol}[1]{\code[db-column]{#1}}
\newcommand{\sql}[1]{\code[sql]{#1}}

% TeX code:
\newcommand{\texcmd}[1]{\code[tex-cmd]{\backslash#1}}
\newcommand{\texenv}[1]{\code[tex-env]{#1}}

% General
\newcommand{\val}[1]{\code[value]{#1}}
\newcommand{\prm}[1]{\code[param]{#1}}

% Other:
\newcommand{\notimpl}[1]{\emph[not-impl]{#1}}
\newcommand{\new}[1]{\emph[new]{#1}}
\newcommand{\deprecated}[1]{\emph[deprecated]{#1}}
\newcommand{\wvar}[1]{\var[weak]{#1}}


\newcommand{\b}{\backslash}
\newcommand{\c}[1]{\backslash#1}
\newcommand{\cmdref}[1]{\href{\#cmd_#1}{\backslash\/#1}}
\newcommand{\envref}[1]{\href{\#env_#1}{#1}}

\begin{document}

\title{Generische TeX-Befehle und -Umgebungen}

\begin{authors}
  \author[rassy@math.tu-berlin.de]{Tilman Rassy}
\end{authors}

\version{$Id: generische_tex_bef_umg.tex,v 1.4 2007/09/24 12:57:20 rassy Exp $}

Diese Spezifikation beschreibt die "generischen" TeX-Befehle und -Umgebungen;
das sind solche, die nicht f�r einen bestimmten Dokumnettyp oder eine bestimmte
Kategorie spezifisch sind. Es handelt sich dabei um Texthervorhebungen, Listen,
Tabellen u.�.

\tableofcontents

\section{Befehle}

\subsection{\c{align}}\label{cmd_align}

Spezifiziert die horizontale Ausrichtung einer Tabelle. Prototyp:

\begin{preformatted}[code]%
  \c{align}\{\var{align}\}
\end{preformatted}

\var{align} darf die Werte \code{r}, \code{c} und \code{l} annehmen. Deren
Bedeutung ist "rechts" bzw. "zentriert" bzw. "links". Der Befehl ist nur im
optionalen Argument \var{style} der \envref{table}-Umgebung erlaubt.

Beispiele:

\begin{preformatted}[code]%
  \c{begin}\{table\}[\c{align}\{l\}]
    L11 \& L12  \b\b
    L21 \& L22
  \c{end}\{table\}

  \c{begin}\{table\}[\c{align}\{c\}]
    C11 \& C12  \b\b
    C21 \& C22
  \c{end}\{table\}

  \c{begin}\{table\}[\c{align}\{r\}]
    R11 \& R12  \b\b
    R21 \& R22
  \c{end}\{table\}
\end{preformatted}

Die Beispiele w�rden in etwa so dargestellt:

\begin{table}[\align{l}]
  L11 & L12 \\
  L21 & L22
\end{table}

\begin{table}[\align{c}]
  C11 & C12 \\
  C21 & C22
\end{table}

\begin{table}[\align{r}]
  R11 & R12 \\
  R21 & R22
\end{table}

Das genaue Aussehen h�ngt nat�rlich von den XSL- und CSS-Stylesheets ab.


\subsection{\c{emph}}

Hebt ein Textst�ck hervor. Prototyp:

\begin{preformatted}[code]%
  \c{emph}\{\var{text}\}
\end{preformatted}

Hierbei ist \var{text} das hervorzuhebende Textst�ck. Es wird �blicherweise
fett und/oder kursiv gesetzt.

\subsection{\c{mark}}

Markiert ein Textst�ck. Prototyp:

\begin{preformatted}[code]%
  \c{mark}[\var{num}]\{\var{text}\}
\end{preformatted}

Hierbei ist \var{text} das zu markierende Textst�ck. Es stehen insgesamt zehn
verschiedene Markierungsarten zur Verf�gung. Mit dem optionalen Argument
\var{num} kann eine davon ausgew�hlt werden. \var{num} darf die Werte 0,
\ldots, 9 annehmen; Default ist 0.


\subsection{\c{valign}}\label{cmd_valign}

[Z.z. noch nicht vollst�ndig implementiert.]


\section{Umgebungen}

\subsection{table}\label{env_table}

Erzeugt eine Tabelle. Prototypen:

\begin{preformatted}[code]%
  \c{begin}\{table\}[\var{class}][\var{style}]
    \meta{[}\cmdref{head}
      \var{h11} \& \var{h12} \& \meta{...} \b\b
       \meta{.     .}
       \meta{.     .}
       \meta{.     .}
      \var{hN1} \& \var{hN2} \& \meta{... ]} 
    \meta{[}\cmdref{body}
      \var{b11} \& \var{b12} \& \meta{...} \b\b
       \meta{.     .}
       \meta{.     .}
       \meta{.     .}
      \var{bN1} \& \var{bN2} \& \meta{... ]} 
    \meta{[}\cmdref{foot}
      \var{f11} \& \var{f12} \& \meta{...} \b\b
       \meta{.     .}
       \meta{.     .}
       \meta{.     .}
      \var{fN1} \& \var{fN2} \& \meta{... ]} 
  \c{end}\{table\}

  \c{begin}\{table\}[\var{class}][\var{style}]
    \var{b11} \& \var{b12} \& \meta{...} \b\b
     \meta{.     .}
     \meta{.     .}
     \meta{.     .}
    \var{bN1} \& \var{bN2} \& \meta{...} 
  \c{end}\{table\}
\end{preformatted}

Das optionale Argument \var{class} weist der Tabelle eine Klasse zu. Mit dem
zweiten optionalen Argument, \var{style}, kann der Tabellenstil beeinflusst
werden. \var{style} darf die Befehle \cmdref{align}, \cmdref{valign},
\cmdref{cellclasses}, \cmdref{cellaligns}, \cmdref{cellvaligns} enthalten.

\var{h11}, \var{b11}, \var{f11} usw. sind Tabellenzellen.

Mit Hilfe der Befehle \cmdref{head}, \cmdref{body} und \cmdref{foot} l�sst sich
die Tabelle in einen Kopf-, Haupt- und Fussbereich unterteilen. Jeder dieser
Bereiche darf auch weggelassen werden.

Beispiel:

\begin{preformatted}[code]%
  \c{begin}\{table\}
    \c{head}
      Name \& Vorname \& Id
    \c{body}
      M�ller \& Hans \& 123 \b\b
      Meier \& Marta \& 456 \b\b
      Schmidt \& Erna \& 789
  \c{end}\{table\}
\end{preformatted}

Das Beispiel w�rde in etwa so dargestellt:

\begin{table}
  \head
    Name & Vorname & Id
  \body
    M�ller & Hans & 123 \\
    Meier & Marta & 456 \\
    Schmidt & Erna & 789
\end{table}

Das genaue Aussehen h�ngt nat�rlich von den XSL- und CSS-Stylesheets ab.

-- ENDE DER DATEI --

\end{document}