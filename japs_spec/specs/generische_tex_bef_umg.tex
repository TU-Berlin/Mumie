\documentclass{generic}


% ------------------------------------------------------------------------------
% Schriftgroessen
% ------------------------------------------------------------------------------

\renewcommand{\normalsize}{\fontsize{18}{22}\selectfont}
\renewcommand{\large}{\fontsize{20}{26}\selectfont}
\renewcommand{\Large}{\fontsize{22}{30}\selectfont}
\renewcommand{\small}{\fontsize{14}{16}\selectfont}
\newcommand{\smaller}{\fontsize{12}{14}\selectfont}
\newcommand{\verysmall}{\fontsize{6}{6}\selectfont}
\renewcommand{\footnotesize}{\fontsize{12}{14}\selectfont}
\newcommand{\tocsize}{\fontsize{12}{16}\selectfont}
\newcommand{\titlesize}{\fontsize{24}{36}\selectfont}

% ------------------------------------------------------------------------------
% Texthervorhebungen
% ------------------------------------------------------------------------------

%\definecolor{emphcolor}{rgb}{0.77,0.00,0.00}
%\renewcommand{\emph}[1]{\textbf{\textcolor{emphcolor}{#1}}}
\renewcommand{\emph}[1]{\textbf{#1}}

\newcommand{\term}[1]{\textit{#1}}
% \newcommand{\code}[1]{\texttt{#1}}
\newcommand{\var}[1]{\textit{#1}}

%\definecolor{headlinecolor}{rgb}{0.77,0.00, 0.00}
\definecolor{headlinecolor}{rgb}{0.67,0.09, 0.22}
\newcommand{\headline}[1]{\textbf{\textcolor{headlinecolor}{#1}}}

\definecolor{emphboxbordercolor}{rgb}{0.80,0.36,0.36}
\definecolor{emphboxcolor}{rgb}{0.98,0.98,0.77}
\newcommand{\emphbox}[2]{\fcolorbox{emphboxbordercolor}{emphboxcolor}{\parbox{#1}{#2}}}

\definecolor{labelcolor}{rgb}{0.35,0.35, 0.8}


% -------------------------------------------------------------------------------
% Pfeile
% -------------------------------------------------------------------------------

\newcommand{\myarrow}{{\Large\boldmath$\;\rightarrow\;$}}
\newcommand{\mylongarrow}{{\Large\boldmath$\;\longrightarrow\;$}}



% -------------------------------------------------------------------------------
% Befehle f�r Abbildungen 
% -------------------------------------------------------------------------------

\renewcommand{\thefigure}{\thesubsection.\arabic{figure}}

\newcommand{\mycaption}[2]{%
 \centerline{%
  \parbox{#1}{%
   \small \refstepcounter{figure} Figure \thefigure. #2 }}}


%---------------------------------------------------------------------------
% Gliederungsbefehle
%---------------------------------------------------------------------------

\makeatletter

\renewcommand{\section}{\@startsection
 {section}%                                   % Name
 {1}%                                         % Ebene
 {0pt}%                                       % Einzug
 {-0.0\baselineskip}%                        % Vorabstand 
 {0.85\baselineskip}%                         % Nachabstand
 {\fontsize{20}{24}\selectfont\itshape\bfseries}}%    % Stil

\makeatother

\newcommand{\mysection}[1]{%

\vspace*{-1.2\baselineskip}

{\color{headlinecolor}
\section{#1}}}

\newcounter{myappendix}

\newcommand{\myappendix}[2]{%
\hfill\makebox[0cm][l]{\tocref}

\vspace*{-1.2\baselineskip}

\refstepcounter{myappendix}
\section*{A\arabic{myappendix}\hspace{1em}#1}\hypertarget{#2}{}}



% -------------------------------------------------------------------------------
% Inhaltsverzeichnis 
% -------------------------------------------------------------------------------

\newcommand{\mytoctitle}[1]{\begin{center}\textbf{\hypertarget{inhalt}{#1}}\end{center}}

\newenvironment{mytoc}
  {\tocsize\textbf{\hypertarget{inhalt}{Inhalt}}\\\begin{enumerate}}
  {\end{enumerate}}

\newcommand{\mytocitem}[3]{\item[#1]\hyperlink{#3}{#2}}

\newcommand{\tocref}{\verysmall\hyperlink{inhalt}{TOC}\normalsize}

% -------------------------------------------------------------------------------
% Listen 
% -------------------------------------------------------------------------------

%\newcommand{\mylabelitemi}{\labelitemi}
\newcommand{\mylabelitemi}{\raisebox{0.25ex}{{\small $\blacktriangleright$}\hspace{0.1em}}}

%\newcommand{\mylabelitemii}{\labelitemii}
%\newcommand{\mylabelitemii}{\raisebox{0.25ex}{{\small $\triangleright$}\hspace{0.1em}}}
\newcommand{\mylabelitemii}{\raisebox{0.1ex}{\large\bfseries-}\hspace{0.075em}}

\newenvironment{mylist}[1][\mylabelitemi]
  {\begin{list}{{\color{labelcolor}#1}}
    {\setlength{\itemindent}{0cm}
     \setlength{\labelwidth}{1em}
     \setlength{\leftmargin}{0.5cm}
     \setlength{\rightmargin}{1cm}}}
  {\end{list}}

\newenvironment{myenum}
  {\begin{list}{\theenumi}
    {\usecounter{enumi}
     \setlength{\itemindent}{0cm}
     \setlength{\labelwidth}{1em}
     \setlength{\leftmargin}{0.5cm}
     \setlength{\rightmargin}{1cm}}}
  {\end{list}}

\newcommand{\pitem}{\pause\item}
\newcommand{\itemii}{\item[\mylabelitemii]}

\newcounter{romanlistcounter}
\newenvironment{romanlist}{\begin{list}%
 {(\roman{romanlistcounter})\hfill}{\usecounter{romanlistcounter}%
 \topsep0.05\baselineskip% 
 \partopsep0.85\baselineskip%
 \leftmargin2em%
 \labelwidth2em%
 \labelsep0pt%
 \itemindent0pt%
 \parsep0.15\baselineskip%
 \itemsep0.0\baselineskip}}%
 {\end{list}}

\newcounter{alphlistcounter}
\newenvironment{alphlist}{\begin{list}%
 {(\alph{alphlistcounter})\hfill}{\usecounter{alphlistcounter}%
 \topsep0.05\baselineskip% 
 \partopsep0.85\baselineskip%
 \leftmargin1.8em%
 \labelwidth1.8em%
 \labelsep0pt%
 \itemindent0pt%
 \parsep0.15\baselineskip%
 \itemsep0.0\baselineskip}}%
 {\end{list}}

\newcounter{arabiclistcounter}
\newenvironment{arabiclist}{\begin{list}%
 {(\arabic{arabiclistcounter})\hfill}{\usecounter{arabiclistcounter}%
 \topsep0.05\baselineskip% 
 \partopsep0.85\baselineskip%
 \leftmargin1.8em%
 \labelwidth1.8em%
 \labelsep0pt%               
 \itemindent0pt%
 \parsep0.15\baselineskip%
 \itemsep0.0\baselineskip}}%
 {\end{list}}





\newcommand{\b}{\backslash}
\newcommand{\c}[1]{\backslash#1}
\newcommand{\cmdref}[1]{\href{\#cmd_#1}{\backslash\/#1}}
\newcommand{\envref}[1]{\href{\#env_#1}{#1}}

\begin{document}

\title{Generische TeX-Befehle und -Umgebungen}

\begin{authors}
  \author[rassy@math.tu-berlin.de]{Tilman Rassy}
\end{authors}

\version{$Id: generische_tex_bef_umg.tex,v 1.4 2007/09/24 12:57:20 rassy Exp $}

Diese Spezifikation beschreibt die "generischen" TeX-Befehle und -Umgebungen;
das sind solche, die nicht f�r einen bestimmten Dokumnettyp oder eine bestimmte
Kategorie spezifisch sind. Es handelt sich dabei um Texthervorhebungen, Listen,
Tabellen u.�.

\tableofcontents

\section{Befehle}

\subsection{\c{align}}\label{cmd_align}

Spezifiziert die horizontale Ausrichtung einer Tabelle. Prototyp:

\begin{preformatted}[code]%
  \c{align}\{\var{align}\}
\end{preformatted}

\var{align} darf die Werte \code{r}, \code{c} und \code{l} annehmen. Deren
Bedeutung ist "rechts" bzw. "zentriert" bzw. "links". Der Befehl ist nur im
optionalen Argument \var{style} der \envref{table}-Umgebung erlaubt.

Beispiele:

\begin{preformatted}[code]%
  \c{begin}\{table\}[\c{align}\{l\}]
    L11 \& L12  \b\b
    L21 \& L22
  \c{end}\{table\}

  \c{begin}\{table\}[\c{align}\{c\}]
    C11 \& C12  \b\b
    C21 \& C22
  \c{end}\{table\}

  \c{begin}\{table\}[\c{align}\{r\}]
    R11 \& R12  \b\b
    R21 \& R22
  \c{end}\{table\}
\end{preformatted}

Die Beispiele w�rden in etwa so dargestellt:

\begin{table}[\align{l}]
  L11 & L12 \\
  L21 & L22
\end{table}

\begin{table}[\align{c}]
  C11 & C12 \\
  C21 & C22
\end{table}

\begin{table}[\align{r}]
  R11 & R12 \\
  R21 & R22
\end{table}

Das genaue Aussehen h�ngt nat�rlich von den XSL- und CSS-Stylesheets ab.


\subsection{\c{emph}}

Hebt ein Textst�ck hervor. Prototyp:

\begin{preformatted}[code]%
  \c{emph}\{\var{text}\}
\end{preformatted}

Hierbei ist \var{text} das hervorzuhebende Textst�ck. Es wird �blicherweise
fett und/oder kursiv gesetzt.

\subsection{\c{mark}}

Markiert ein Textst�ck. Prototyp:

\begin{preformatted}[code]%
  \c{mark}[\var{num}]\{\var{text}\}
\end{preformatted}

Hierbei ist \var{text} das zu markierende Textst�ck. Es stehen insgesamt zehn
verschiedene Markierungsarten zur Verf�gung. Mit dem optionalen Argument
\var{num} kann eine davon ausgew�hlt werden. \var{num} darf die Werte 0,
\ldots, 9 annehmen; Default ist 0.


\subsection{\c{valign}}\label{cmd_valign}

[Z.z. noch nicht vollst�ndig implementiert.]


\section{Umgebungen}

\subsection{table}\label{env_table}

Erzeugt eine Tabelle. Prototypen:

\begin{preformatted}[code]%
  \c{begin}\{table\}[\var{class}][\var{style}]
    \meta{[}\cmdref{head}
      \var{h11} \& \var{h12} \& \meta{...} \b\b
       \meta{.     .}
       \meta{.     .}
       \meta{.     .}
      \var{hN1} \& \var{hN2} \& \meta{... ]} 
    \meta{[}\cmdref{body}
      \var{b11} \& \var{b12} \& \meta{...} \b\b
       \meta{.     .}
       \meta{.     .}
       \meta{.     .}
      \var{bN1} \& \var{bN2} \& \meta{... ]} 
    \meta{[}\cmdref{foot}
      \var{f11} \& \var{f12} \& \meta{...} \b\b
       \meta{.     .}
       \meta{.     .}
       \meta{.     .}
      \var{fN1} \& \var{fN2} \& \meta{... ]} 
  \c{end}\{table\}

  \c{begin}\{table\}[\var{class}][\var{style}]
    \var{b11} \& \var{b12} \& \meta{...} \b\b
     \meta{.     .}
     \meta{.     .}
     \meta{.     .}
    \var{bN1} \& \var{bN2} \& \meta{...} 
  \c{end}\{table\}
\end{preformatted}

Das optionale Argument \var{class} weist der Tabelle eine Klasse zu. Mit dem
zweiten optionalen Argument, \var{style}, kann der Tabellenstil beeinflusst
werden. \var{style} darf die Befehle \cmdref{align}, \cmdref{valign},
\cmdref{cellclasses}, \cmdref{cellaligns}, \cmdref{cellvaligns} enthalten.

\var{h11}, \var{b11}, \var{f11} usw. sind Tabellenzellen.

Mit Hilfe der Befehle \cmdref{head}, \cmdref{body} und \cmdref{foot} l�sst sich
die Tabelle in einen Kopf-, Haupt- und Fussbereich unterteilen. Jeder dieser
Bereiche darf auch weggelassen werden.

Beispiel:

\begin{preformatted}[code]%
  \c{begin}\{table\}
    \c{head}
      Name \& Vorname \& Id
    \c{body}
      M�ller \& Hans \& 123 \b\b
      Meier \& Marta \& 456 \b\b
      Schmidt \& Erna \& 789
  \c{end}\{table\}
\end{preformatted}

Das Beispiel w�rde in etwa so dargestellt:

\begin{table}
  \head
    Name & Vorname & Id
  \body
    M�ller & Hans & 123 \\
    Meier & Marta & 456 \\
    Schmidt & Erna & 789
\end{table}

Das genaue Aussehen h�ngt nat�rlich von den XSL- und CSS-Stylesheets ab.

-- ENDE DER DATEI --

\end{document}