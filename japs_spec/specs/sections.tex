\documentclass{generic}


% ------------------------------------------------------------------------------
% Schriftgroessen
% ------------------------------------------------------------------------------

\renewcommand{\normalsize}{\fontsize{18}{22}\selectfont}
\renewcommand{\large}{\fontsize{20}{26}\selectfont}
\renewcommand{\Large}{\fontsize{22}{30}\selectfont}
\renewcommand{\small}{\fontsize{14}{16}\selectfont}
\newcommand{\smaller}{\fontsize{12}{14}\selectfont}
\newcommand{\verysmall}{\fontsize{6}{6}\selectfont}
\renewcommand{\footnotesize}{\fontsize{12}{14}\selectfont}
\newcommand{\tocsize}{\fontsize{12}{16}\selectfont}
\newcommand{\titlesize}{\fontsize{24}{36}\selectfont}

% ------------------------------------------------------------------------------
% Texthervorhebungen
% ------------------------------------------------------------------------------

%\definecolor{emphcolor}{rgb}{0.77,0.00,0.00}
%\renewcommand{\emph}[1]{\textbf{\textcolor{emphcolor}{#1}}}
\renewcommand{\emph}[1]{\textbf{#1}}

\newcommand{\term}[1]{\textit{#1}}
% \newcommand{\code}[1]{\texttt{#1}}
\newcommand{\var}[1]{\textit{#1}}

%\definecolor{headlinecolor}{rgb}{0.77,0.00, 0.00}
\definecolor{headlinecolor}{rgb}{0.67,0.09, 0.22}
\newcommand{\headline}[1]{\textbf{\textcolor{headlinecolor}{#1}}}

\definecolor{emphboxbordercolor}{rgb}{0.80,0.36,0.36}
\definecolor{emphboxcolor}{rgb}{0.98,0.98,0.77}
\newcommand{\emphbox}[2]{\fcolorbox{emphboxbordercolor}{emphboxcolor}{\parbox{#1}{#2}}}

\definecolor{labelcolor}{rgb}{0.35,0.35, 0.8}


% -------------------------------------------------------------------------------
% Pfeile
% -------------------------------------------------------------------------------

\newcommand{\myarrow}{{\Large\boldmath$\;\rightarrow\;$}}
\newcommand{\mylongarrow}{{\Large\boldmath$\;\longrightarrow\;$}}



% -------------------------------------------------------------------------------
% Befehle f�r Abbildungen 
% -------------------------------------------------------------------------------

\renewcommand{\thefigure}{\thesubsection.\arabic{figure}}

\newcommand{\mycaption}[2]{%
 \centerline{%
  \parbox{#1}{%
   \small \refstepcounter{figure} Figure \thefigure. #2 }}}


%---------------------------------------------------------------------------
% Gliederungsbefehle
%---------------------------------------------------------------------------

\makeatletter

\renewcommand{\section}{\@startsection
 {section}%                                   % Name
 {1}%                                         % Ebene
 {0pt}%                                       % Einzug
 {-0.0\baselineskip}%                        % Vorabstand 
 {0.85\baselineskip}%                         % Nachabstand
 {\fontsize{20}{24}\selectfont\itshape\bfseries}}%    % Stil

\makeatother

\newcommand{\mysection}[1]{%

\vspace*{-1.2\baselineskip}

{\color{headlinecolor}
\section{#1}}}

\newcounter{myappendix}

\newcommand{\myappendix}[2]{%
\hfill\makebox[0cm][l]{\tocref}

\vspace*{-1.2\baselineskip}

\refstepcounter{myappendix}
\section*{A\arabic{myappendix}\hspace{1em}#1}\hypertarget{#2}{}}



% -------------------------------------------------------------------------------
% Inhaltsverzeichnis 
% -------------------------------------------------------------------------------

\newcommand{\mytoctitle}[1]{\begin{center}\textbf{\hypertarget{inhalt}{#1}}\end{center}}

\newenvironment{mytoc}
  {\tocsize\textbf{\hypertarget{inhalt}{Inhalt}}\\\begin{enumerate}}
  {\end{enumerate}}

\newcommand{\mytocitem}[3]{\item[#1]\hyperlink{#3}{#2}}

\newcommand{\tocref}{\verysmall\hyperlink{inhalt}{TOC}\normalsize}

% -------------------------------------------------------------------------------
% Listen 
% -------------------------------------------------------------------------------

%\newcommand{\mylabelitemi}{\labelitemi}
\newcommand{\mylabelitemi}{\raisebox{0.25ex}{{\small $\blacktriangleright$}\hspace{0.1em}}}

%\newcommand{\mylabelitemii}{\labelitemii}
%\newcommand{\mylabelitemii}{\raisebox{0.25ex}{{\small $\triangleright$}\hspace{0.1em}}}
\newcommand{\mylabelitemii}{\raisebox{0.1ex}{\large\bfseries-}\hspace{0.075em}}

\newenvironment{mylist}[1][\mylabelitemi]
  {\begin{list}{{\color{labelcolor}#1}}
    {\setlength{\itemindent}{0cm}
     \setlength{\labelwidth}{1em}
     \setlength{\leftmargin}{0.5cm}
     \setlength{\rightmargin}{1cm}}}
  {\end{list}}

\newenvironment{myenum}
  {\begin{list}{\theenumi}
    {\usecounter{enumi}
     \setlength{\itemindent}{0cm}
     \setlength{\labelwidth}{1em}
     \setlength{\leftmargin}{0.5cm}
     \setlength{\rightmargin}{1cm}}}
  {\end{list}}

\newcommand{\pitem}{\pause\item}
\newcommand{\itemii}{\item[\mylabelitemii]}

\newcounter{romanlistcounter}
\newenvironment{romanlist}{\begin{list}%
 {(\roman{romanlistcounter})\hfill}{\usecounter{romanlistcounter}%
 \topsep0.05\baselineskip% 
 \partopsep0.85\baselineskip%
 \leftmargin2em%
 \labelwidth2em%
 \labelsep0pt%
 \itemindent0pt%
 \parsep0.15\baselineskip%
 \itemsep0.0\baselineskip}}%
 {\end{list}}

\newcounter{alphlistcounter}
\newenvironment{alphlist}{\begin{list}%
 {(\alph{alphlistcounter})\hfill}{\usecounter{alphlistcounter}%
 \topsep0.05\baselineskip% 
 \partopsep0.85\baselineskip%
 \leftmargin1.8em%
 \labelwidth1.8em%
 \labelsep0pt%
 \itemindent0pt%
 \parsep0.15\baselineskip%
 \itemsep0.0\baselineskip}}%
 {\end{list}}

\newcounter{arabiclistcounter}
\newenvironment{arabiclist}{\begin{list}%
 {(\arabic{arabiclistcounter})\hfill}{\usecounter{arabiclistcounter}%
 \topsep0.05\baselineskip% 
 \partopsep0.85\baselineskip%
 \leftmargin1.8em%
 \labelwidth1.8em%
 \labelsep0pt%               
 \itemindent0pt%
 \parsep0.15\baselineskip%
 \itemsep0.0\baselineskip}}%
 {\end{list}}





\begin{document}

\title{Sections}

\begin{authors}
  \author[rassy@math.tu-berlin.de]{Tilman Rassy}
  \author[lehmannf@math.tu-berlin.de]{Fritz Lehmann-Grube}
\end{authors}

\version{$Id: sections.tex,v 1.9 2006/10/13 13:00:42 rassy Exp $}

Der Japs beinhaltet eine hierarchische Gliederunsstruktur, in der alle
Dokumente und Pseudo-Dokumente eingeordnet sind. Diese Spezifikation beschreibt
die Gliederunsstruktur selbst sowie ihre Implementierung.

\tableofcontents

\section{Sections-Konzept}

F�r den Japs wird eine abstrakte hierarchische Gliederungsstruktur
definiert. Diese hat eine Implementation in der Japs-DB einerseits und im
lokalen Dateisystem des Benutzers andererseits.

Die Gliederungsstruktur wird durch sog. "Sections" gebildet. Es gelten
folgende Prinzipien:

\begin{enumerate}

\item Sections sind Pseudo-Dokumente.
  
\item Jedes Dokument, Pseudo-Dokument und jede Dokumenten-Quelle ist in genau
  einer Section enthalten.
  
\item Damit ist insbesondere jede Section in genau einer Section enthalten.
  Dadurch entsteht die hierarchische Struktur.
  
\item\label{rootsection} Die Struktur hat eine eindeutige Wurzel, die sog. "Rootsection". Die
  Rootsection ist in sich selbst enthalten. Ansonsten sind Zyklen verboten.
  
\item Jede Section darf Dokumente, Pseudo-Dokumente und Dokumenten-Quellen
  enthalten.

\end{enumerate}


\section{Implementation in der DB}

Sections werden durch die Tabelle \dbtable{sections} dargestellt. Diese hat
mindestens die folgenden Spalten:

\begin{enumerate}

\item \dbcol{id} -- Prim�rschl�ssel
  
\item \dbcol{filename}\ref{filename} -- Lokaler Verzeichnisname bei Implemetierung im
  Dateisystem. 
  
\item \dbcol{contained_in}\label{contained_in} -- id der Section, in der diese
  Section enthalten ist.

\end{enumerate}
  
Die Spalten \ref{filename} und \ref{contained_in} (\dbcol{filename}
bzw. \dbcol{contained_in}) sind auch in den Tabellen der Dokumente, �brigen
Pseudo-Dokumente und Quellen enthalten. \dbcol{filename} enth�lt dabei den
sog. Rumpfnamen, d.h. den Dateinamen ohne Pfad und Endungen
(s. \ref{rumpfname}).



\section{Implementation auf dem lokalen Dateisystem}

Die Gliederungsstruktur wird auf eine Verzeichnisstruktur abgebildet. Dabei
gelten folgende Prinzipien:

\begin{enumerate}
  
\item\label{section_verz} Jede Section entspricht genau einem Verzeichnis. Solche Verzeichnisse
  sollen "Section-Verzeichnisse" genannt werden. Die Master-Datei einer Section
  liegt in ihrem Section-Verzeichnis.  Die Master-Datei einer Section heisst
  immer \file{.meta.xml}.
  
\item Die "ist-enthalten-in"-Beziehung der Sections wird durch die
  "ist-Unterverzeichnis-von"-Beziehung der Section-Verzeichnisse dargestellt.
  
\item F�r alle (Pseudo-)Dokumenttypen \var{typ} ausser \code{section} gilt:
  Enth�lt eine Section (Pseudo-)Dokumente des Typs \var{typ}, so enth�lt das
  entsprechende Section-Verzeichnis die Master- und ggf. Content-Dateien der
  entsprechenden (Pseudo-)Dokumente.
  
\item Enth�lt eine Section Text- und/oder Bin�r-Quellen, so enth�lt das
  entsprechende Section-Verzeichnis die entsprechenden Quelldateien.
  
\item Der "Rootsection" (siehe \ref{rootsection}) entspricht das sogenannte
  Checkin- Verzeichnis. Es ist das Wurzelverzeichnis f�r alle Dokumente und
  Pseudo-Dokumente in lokalen Dateisystemen.

\item Es sollten im gesamten Checkin-Verzeichnis nur Unterverzeichnisse
  vorkommen, die Section-Verzeichnisse sind (s. \ref{section_verz}).

\end{enumerate}



\section{Dateien}

Zu einem nicht-generischen Dokument geh�ren folgende Dateien:

\begin{enumerate}
  
\item \emph{Master-Datei:} XML-Datei mit den Meta-Informationen.
  
\item \emph{Content-Datei:} Datei mit dem eigentlichen Inhalt; bei
  Text-Dokumenten: XML; bei Bin�r-Dokumenten: entsprechendes Bin�r-Format.
  
\item \emph{(Evtl.:) Quelldatei(en):} Datei(en) mit den Quellen des Dokuments;
  bei Text-Quellen: entsprechendes Text-Format; bei Bin�r-Quellen:
  entsprechendes Bin�r-Format.

\end{enumerate}

Ein Dokument kann eine, mehrere oder auch gar keine Quelldatei haben.

Master- und Content-Datei geh�ren eindeutig zu einem Dokument. Quelldateien
k�nnen auch zu mehreren Dokumenten geh�ren.

Master- und Content-Datei eines Dokuments m�ssen immer im selben Verzeichnis
liegen.

Generische Dokumente und Pseudo-Dokumente haben lediglich eine Master-Datei.


\section{Dateinamen}

-- Siehe Spezifikation \href{dateinamen.xhtml}{Dateinamen} --



\section{Standard-Gliederungsstruktur}

Die h�heren (Rootsection-nahen) Teile der Gliederungsstruktur unterliegen einem
Standard, der im folgenden erl�utert wird. Die tieferen Teile bleiben von
diesem Standard unber�hrt. F�r den \file{content}-Zweig (s.u.) k�nnen aber an
anderer Stelle Standards eingef�hrt werden. Diese sind typischerweise
fachlich-mathematisch motiviert.

Der Standard definiert folgende Struktur:

\newcommand{\secref}[1]{\href{\##1}{#1}}

\begin{preformatted}%
   \href{\#ROOT}{ROOT}
    |-\href{\#system}{system}
    |  |-\href{\#themes}{themes}
    |  |-\href{\#languages}{languages}
    |  |-\href{\#util}{util}
    |  |-\href{\#system_commom}{commom}
    |  |-\href{\#misc}{misc}
    |  |-\href{\#element}{element}
    |  |-\href{\#problem}{problem}
    |  |-\href{\#start}{start}
    |  |-\href{\#admin}{admin}
    |  .
    |  .
    |  .
    |-\href{\#org}{org}
    |  |-\href{\#users}{users}
    |  |-\href{\#user}{user}_\href{\#groups}{groups}
    |  |-\href{\#uni}{uni}
    |  |  |-\href{\#semester}{<semester1>}
    |  |  |  |-\href{\#classes}{classes}
    |  |  |  |-\href{\#courses}{courses}
    |  |  |  |-\href{\#tutorials}{tutorials}
    |  |  |  .
    |  |  |  .
    |  |  |  .
    |  |  |-\href{\#semester}{<semester2>}
    |  |  |  |-\href{\#classes}{classes}
    |  |  |  |-\href{\#courses}{courses}
    |  |  |  |-\href{\#tutorials}{tutorials}
    |  |  |  .
    |  |  |  .
    |  |  |  .
    |-\href{\#content}{content}
       .
       .
       .
\end{preformatted}

\subsection{ROOT}\label{ROOT}

Die Rootsection

\subsection{ROOT/system}\label{system}

(Pseudo-)Dokumente technischer Natur (Themes, Sprachen, XSL- und
CSS-Stylesheets, Grafiken f�r das Web-Layout usw.) und solche f�r die Bedienung
des Systems (Startseite, Admin-Seite usw.).

\subsection{ROOT/system/themes}\label{themes}

Themes.

\subsection{ROOT/system/languages}\label{languages}

Sprachen.

\subsection{ROOT/system/util}\label{util}

Hilfsmittel (z.B. der Japs-Client).

\subsection{ROOT/system/common}\label{system_common}

Gemeinsame Resourcen verschiedener Dokumente (XSL- und CSS-Stylesheets,
Grafiken usw.)

\subsection{ROOT/system/misc}\label{misc}

Resourcen (XSL- und CSS-Stylesheets, Grafiken usw.) einzelner
(Pseudo-)Dokument-Typen oder einzelner (Pseudo-)Dokumente.

\subsection{ROOT/system/element}\label{element}

Resourcen f�r Dokumente der Typen "{}element" und "{}subelement'.

\subsection{ROOT/system/problem}\label{problem}

Resourcen f�r Dokumente des Typs "{}problem".

\subsection{ROOT/system/start}\label{start}

Resourcen f�r die Startseite (XSL- und CSS-Stylesheets,
Grafiken usw., evtl. \val{page}-Dokumente)

\subsection{ROOT/system/admin}\label{admin}

Resourcen f�r die Administrator-Seite (XSL- und CSS-Stylesheets, Grafiken usw.,
evtl. \val{page}-Dokumente)

\subsection{ROOT/org}\label{org}

(Pseudo-)Dokumente organisatorischer Natur (Benutzer, Benutzer-Gruppen,
Lehrveranstaltungen usw.)

\subsection{ROOT/org/users}\label{users}

Benutzer

\subsection{ROOT/org/user_groups}\label{user_groups}

Benutzer-Gruppen


\subsection{ROOT/org/uni}\label{uni}

Alles, was mit den Lehrveranstaltungen zu tun hat. Der Name dieser Section kann
an die Institution, an der die MUMIE eingestezt wird, angeglichen werden,
z.B. \code{tu-berlin} f�r \emph{Technische Universit�t Berlin}. Stellt die
MUMIE-Instanz Material f�r mehrere Institutionen bereit, so gibt es f�r jede
Instanz eine solche Section.

Enth�lt f�r jedes Semester eine Unter-Section, in der die (Pseudo-)Dokumente zu
diesem Semester liegen. 

\subsection{ROOT/org/uni/<semestes>}\label{semester}

Enth�lt (Pseudo-)Dokumente zu einem bestimmten Semester. \code{<semester>} ist
hierbei durch eine Bezeichnung f�r das Semester zu ersetzten,
z.B. \code{sesem_06} f�r Sommersemester 2006.

Enth�lt insbesondere das Pseudo-Dokument vom Typ \code{semester}, das dieses
Semester repr�sentiert.

\subsection{ROOT/org/uni/<semester>/classes}\label{classes}

Lehrveranstaltungen

\subsection{ROOT/org/uni/<semester>/tutorials}\label{tutorials}

Tutorien

\subsection{ROOT/org/uni/<semester>/courses}\label{courses}

Kurse, auch Kurs-Abschnitte und -Unterabschnitte

\subsection{ROOT/content}\label{content}

Der eigentliche mathematische Inhalt (Elemente, Subelemente, Aufgaben, Applets,
Bilder usw.). Die entsprechenden Quellen, soweit vorhanden, werden ebenfalls
hier abgelegt.

%\section{Fachlogische Gliederung des Checkin-Verzeichnisses}

%(Insbesondere) folgende Sections sollten existieren:

%\begin{enumerate}

%\item \file{unsorted} -- Vor�bergehender Aufbewahrungsort f�r Dokumente
%  (Pseudo-Dokumente, Quellen), die erst noch einsortiert werden m�ssen.

%\item \file{system} --  Dokumente technischer Natur. Gliederung nach
%  Verwendungszweck.

%  \begin{enumerate}

%  \item \file{etc} --  Themes, User, Usergruppen usw.

%  \item \file{common} -- Gemeinsame Resourcen verschiedener Dokumente.

%    \begin{enumerate}

%    \item \file{generic} -- Generische Dokumente
        
%    \item \file{default} -- Dokumente des Default-Themes

%    \end{enumerate}

%  \item \file{course_viewer} -- Dokumente zur Darstellung von Kursen

%    \begin{enumerate}

%    \item \file{generic} -- Generische Dokumente
        
%    \item \file{default} -- Dokumente des Default-Themes

%    \end{enumerate}

%  \item \file{desktop} -- Dokumente zur Darstellung des Desktops

%    \begin{enumerate}

%    \item \file{generic} -- Generische Dokumente
        
%    \item \file{default} -- Dokumente des Default-Themes

%    \end{enumerate}

%  \item \file{error_messages} -- Fehlerseiten

%    \begin{enumerate}

%    \item \file{generic} -- Generische Dokumente
        
%    \item \file{default} -- Dokumente des Default-Themes

%    \end{enumerate}

%  \item  \file{upload} -- Upload-Seite

%    \begin{enumerate}

%    \item \file{generic} -- Generische Dokumente
        
%    \item \file{default} -- Dokumente des Default-Themes

%    \end{enumerate}

%  \end{enumerate}

%\item \file{content} -- Lehrinhalte
     
%  \begin{enumerate}
    
%  \item \file{common} -- Gemeinsame Resourcen

%    \begin{enumerate}

%    \item \file{generic} -- Generische Dokumente
        
%    \item \file{default} -- Dokumente des Default-Themes

%    \end{enumerate}

%  \item \file{lineare_algebra}

%  \item \file{analysis}

%  \item \file{usw.}
          
%  \end{enumerate}

%\end{enumerate}

%In der folgenden Abbildung ist dies skizziert:

%\image{sections.png}
          

-- ENDE DER DATEI --

\end{document}