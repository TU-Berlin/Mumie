\documentclass{generic}

% Author: Tilman Rassy <rassy@math.tu-berlin.de>
% $Id: macros.tex,v 1.5 2006/10/05 15:58:42 rassy Exp $

% XML code:
\newcommand{\element}[1]{\code[xml-element]{#1}}
\newcommand{\attrib}[1]{\code[xml-attrib]{#1}}

% Database code:
\newcommand{\dbtable}[1]{\code[db-table]{#1}}
\newcommand{\dbcol}[1]{\code[db-column]{#1}}
\newcommand{\sql}[1]{\code[sql]{#1}}

% TeX code:
\newcommand{\texcmd}[1]{\code[tex-cmd]{\backslash#1}}
\newcommand{\texenv}[1]{\code[tex-env]{#1}}

% General
\newcommand{\val}[1]{\code[value]{#1}}
\newcommand{\prm}[1]{\code[param]{#1}}

% Other:
\newcommand{\notimpl}[1]{\emph[not-impl]{#1}}
\newcommand{\new}[1]{\emph[new]{#1}}
\newcommand{\deprecated}[1]{\emph[deprecated]{#1}}
\newcommand{\wvar}[1]{\var[weak]{#1}}


\begin{document}

\title{Entwicklung von Dokumenten mit Java-Quellen}

\begin{authors}
  \author[rassy@math.tu-berlin.de]{Tilman Rassy}
\end{authors}

\version{$Id: entw_dok_mit_java_quellen.tex,v 1.4 2008/08/11 11:57:33 rassy Exp $}

Diese Spezifikation beschreibt die Besonderheiten bei der Entwicklung von
Dokumenten mit Java-Quellen. Das sind z.Z. Dokumente der Typen "{}applet" und
"java_class".

\tableofcontents

\section{Einleitung}

Dokumente der Typen "{}applet" und "java_class" liegen zun�chst als Java-Quellen
vor. Die Quelldateien haben Namen der Form \var{pure_name}\file{.src.java}, wobei
\var{pure_name} der Rumpfname (\emph{pure name}) des Dokuments ist. Hieraus
m�ssen jeweils eine Master-Datei \var{pure_name}\file{.meta.xml} und eine
Content-Datei generiert werden. Im Fall von Applets ist die Content-Datei ein
Jar-Archiv, im Fall von "java_class" Dokumenten eine Datei mit
Java-Bytecode. Die Content-Dateien heissen entsprechend
\var{pure_name}\file{.content.jar} bzw. \var{pure_name}\file{.content.class}.

�hnlich wie bei TeX-Quellen werden die Metainformationen im Quelltext angeben.
Dazu dienen Javadoc-Tags, und zwar \code{@author} und zus�tzliche,
Mumie-spezifische Tags (s. Abschnitt \ref{javadoc_tags}).

Das Erzeugen der Master- und Content-Datei erfolgt durch den Mmcdk-Befehl
\file{mmjava}. Im Fall von Applets wird dabei die Content-Datei, wenn n�tig,
gleich signiert.


\section{Entwicklungsverzeichnis vs. Checkin-Verzeichnis}\label{entw_checkin_verz}

Als Mumie-Dokumente m�ssen Applets und Java-Klassen in eine
\href{sections.xhtml}{Section} und damit in ein Verzeichnis des Checkin-Baums
einsortiert werden. Andererseits ist es in der Java-Entwicklung �blich (und bei
einigen IDE's sogar unumg�nglich), dass die Quellen in einer
Verzeichnisstruktur liegen, die sich aus der Java-Paketstruktur ergibt.
Ausserdem muessen die Dateinamen der Quellen die Form
\var{pure_name}\file{.java} und nicht \var{pure_name}\file{.src.java} haben.

Um das Problem zu l�sen, wird folgende Vorgehensweise empfohlen: Die Quellen
werden in Verzeichnissen abgelegt, die sich aus der Paketstruktur ergeben. Im
Checkin-Baum werden Softlinks zu den Quellen gemacht. Die Softlinks haben
Dateinamen der Form \var{pure_name}\file{.src.java}, die entsprechenden Quellen
Dateinamen der Form \var{pure_name}\file{.java}. Damit diese Links automatisch
erstellt werden k�nnen, muss der Pfad der Section, in dem das Java-Dokument
liegen soll, als spezieller Javadoc-Tag \code{@mm.section} angegeben werden (s.
auch Abschnitt \ref{javadoc_tags}).


\section{Zus�tzliche Javadoc-Tags}\label{javadoc_tags}

Dieser Abschnitt listet alle Mumie-Javadoc-Tags auf. Die Tags sind s�mtlich vom
Block-Typ und nur im Javadoc-Kommentar zur Klasse erlaubt (also nicht in
Methoden- und Variablen-Kommentaren) (vgl. Javadoc-Manual von Sun).

\begin{enumerate}

\item \code{@mm.buildClasspath}

  Zus�tzlicher Classpath-Eintrag f�r die Kompilierung. Der Wert wird als
  Pfad relativ zu \var{prefix}\file{/lib/java} interpretiert, wobei
  \var{prefix} der gemeinsame Installations-Prefix aller Mumie-Pakete ist. Das
  Tag darf mehrfach vorkommen oder auch ganz fehlen.

  Das Tag ist dazu gedacht, Bibliotheken und Klassen anzugeben, die zur
  Kompilierzeit ben�tigt werden, sich aber nicht aus einem
  \code{@mm.requireApplet}- oder \code{@mm.requireJar}-Tag ergeben. Beispiel:
  Ein Dokument vom Typ "java_class" ben�tigt gewisse Bibliotheken aus dem
  \file{WEB-INF/lib}-Verzeichnis des Cocoon.

\item \code{@mm.category}
  
  Kategorie, als String. Nur beim Dokument-Typ "java_class" erlaubt, dort
  zwingend.

\item \code{@mm.changelog}

  Werdegang des Dokuments (f�r Entwickler). Optional

\item \code{@mm.copyright}

  Copyright-Angabe. Optional.

\item \code{@mm.description}

  Beschreibung des Dokuments (f�r Benutzer). Optional

\item \code{@mm.height}

  Nur beim Dokument-Typ "{}applet" erlaubt. Gibt die empfohlene H�he in Pixeln
  an. Optional, Default ist 400.

  Die Angabe bezieht sich auf den Bereich, der dem Applet vom Browser auf der
  XHTML-Seite zur Verf�gung gettsellt wird. Bei Applets, die zun�chst als
  Button starten und sp�ter in einem eigenen Fenster laufen, ist dies
  \emph{nicht} die H�he des Fensters, sondern die H�he des Buttons.

\item \code{@mm.section}

  Checkin-Pfad der Section, in die das Dokuments einsortiert werden
  soll. Dieses Tag legt somit auch fest, in wo im Checkin-Verzeichnis der
  Softlink zur Quelle dieses Dokuments gemacht wird (s. Abschnitt
  \ref{entw_checkin_verz}). Zwingend

\item \code{@mm.sign}

  Nur beim Dokument-Typ "{}applet" erlaubt. Gibt an, ob das Applet signiert
  werden soll oder nicht. M�gliche Werte (Bedeutung selbsterkl�rend): \code{true},
  \code{false}, \code{yes}, \code{no}. Optional, Default ist \code{no}.

\item \code{@mm.infoPage}

  Checkin-Pfad der Info-Seite des Dokuments. Die Info-Seite muss vom Typ
  \code{generic_page} sein. Optional

\item \code{@mm.thumbnail}

  Checkin-Pfad des Thumbnails des Dokuments. Das Thumbnail muss vom Typ
  \code{generic_image} sein. Optional

\item \code{@mm.status}

  \emph{[VERALTET]} Status des Dokuments, als String. Optional

\item \code{@mm.width}

  Nur beim Dokument-Typ "{}applet" erlaubt. Gibt die empfohlene Breite in Pixeln
  an. Optional, Default ist 400.

  Die Angabe bezieht sich auf den Bereich, der dem Applet vom Browser auf der
  XHTML-Seite zur Verf�gung getstellt wird. Bei Applets, die zun�chst als
  Button starten und sp�ter in einem eigenen Fenster laufen, ist dies
  \emph{nicht} die Breite des Fensters, sondern die Breite des Buttons.

\item \code{@mm.type}

  Dokument-Typ, als String. Zwingend

\item \code{@mm.requireApplet}

  Nur beim Dokument-Typ "{}applet" erlaubt. Gibt ein ben�tigtes anderes
  Appletan. Der Wert muss ein Checkin-Pfad zur Master-Datei eines Dokuments vom
  Typ "{}applet" sein. Das Tag darf mehrfach vorkommen oder auch ganz fehlen.

\item \code{@mm.requireJar}

  Nur beim Dokument-Typ "{}applet" erlaubt. Gibt ein ben�tigtes Jar an. Der
  Wert muss ein Checkin-Pfad zur Master-Datei eines Dokuments vom Typ "{}jar"
  sein. Das Tag darf mehrfach vorkommen oder auch ganz fehlen.
  
\end{enumerate}


\section{Angabe von Autoren}

Autoren werden durch das Standard-Javadoc-Tag \code{@author} angegeben. Mmcdk
und Japs ben�tigen den Autor als Checkin-Pfad der entsprechenden
Master-Datei. Erzeugt man aus der Java-Quelle mit dem \file{javadoc}-Tool eine
API-Dokumentation in HTML-Format, so wird das Tag ebenfalls ausgewertet. Hier
w�nscht man sich eine mehr menschen-lesbare Form, z.B.:

\begin{preformatted}%
  \var{Vorname} \var{Nachname} <\var{E-Mail}>
\end{preformatted}

Um beiden Anforderungen zu gen�gen, erwartet Mmcdk, dass der Wert des
\code{@author}-Tags den Checkin-Pfad der Master-Datei des Autors in runden
Klammern enth�lt. Ausserhalb der Klammern k�nnen die menschen-lesbaren Angaben
stehen. Zul�ssige Formate w�ren z.B.

\begin{preformatted}%
  \var{Vorname} \var{Nachname} <\var{E-Mail}> (\var{pfad})
\end{preformatted}

und

\begin{preformatted}%
  \var{Nachname} (\var{pfad})
\end{preformatted}

Hierbei ist \var{pfad} der Checkin-Pfad der Master-Datei des Autors. Konkretes Beispiel:

\begin{preformatted}%
  Markus Gronau <gronau@math.tu-berlin.de> (org/users/usr_gronau.meta.xml)
\end{preformatted}

-- ENDE DER DATEI --

\end{document}