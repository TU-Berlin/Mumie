\documentclass{generic}

% Author: Tilman Rassy <rassy@math.tu-berlin.de>
% $Id: macros.tex,v 1.5 2006/10/05 15:58:42 rassy Exp $

% XML code:
\newcommand{\element}[1]{\code[xml-element]{#1}}
\newcommand{\attrib}[1]{\code[xml-attrib]{#1}}

% Database code:
\newcommand{\dbtable}[1]{\code[db-table]{#1}}
\newcommand{\dbcol}[1]{\code[db-column]{#1}}
\newcommand{\sql}[1]{\code[sql]{#1}}

% TeX code:
\newcommand{\texcmd}[1]{\code[tex-cmd]{\backslash#1}}
\newcommand{\texenv}[1]{\code[tex-env]{#1}}

% General
\newcommand{\val}[1]{\code[value]{#1}}
\newcommand{\prm}[1]{\code[param]{#1}}

% Other:
\newcommand{\notimpl}[1]{\emph[not-impl]{#1}}
\newcommand{\new}[1]{\emph[new]{#1}}
\newcommand{\deprecated}[1]{\emph[deprecated]{#1}}
\newcommand{\wvar}[1]{\var[weak]{#1}}


\begin{document}

\title{Kursmodell}

\begin{authors}
  \author[rassy@math.tu-berlin.de]{Tilman Rassy}
\end{authors}

\version{$Id: kursmodell.tex,v 1.3 2006/08/11 16:18:22 rassy Exp $}

\section{Grunds�tzliches}

\begin{enumerate}
  
\item Im Zusammenhang mit Kursen gibt es drei Dokumnenttypen (statt wie bisher
  nur einen):

  \begin{enumerate}
    
  \item \emph{course} -- Entspricht einem ganzen Kurs (bisher: toplevel course
    section)
    
  \item \emph{course_section} -- Zur Gliederung von Kursen (bisher: nicht-toplevel
    course section)
    
  \item \emph{course_subsection} -- An eine course_section angeh�ngte
    Zusatzeinheit, z.B. Prelearning, Training.

  \end{enumerate}
    
\item course_sections und course_subsections stehen zueinander wie Elemente und
  Subelemente. Sie sollen auch analog grafisch dargestellt werden, also:
       
  \begin{preformatted}
    +--------+
    |        | <-- course_subsection
    |        |
    |    +-------------+
    |    |             |
    + ---|             |
         |             |
         |             | <-- course_section
         |             |
         |             |
         |             |
         +-------------+
  \end{preformatted}
  
\item F�r course_subsections, vielleicht auch f�r courses und course_sections,
  werden Kategorien eingef�hrt (analog wie bei Elementen und Subelementen). Es
  soll zumindest die folgenden Kategorien geben:

  \begin{enumerate}

  \item \emph{prelearn} --  Die course_subsection ist ein "Prelearning-Arbeitsblatt" 

  \item \emph{training} --  Die course_subsection ist ein "Training-Arbeitsblatt"

  \end{enumerate}
  
\item Kurse (courses) sind einem Semester zugeordnet. Dazu wird der Begriff
  "Semester" in der DB durch eine eigene Tabelle modelliert (s.u.). Die Tabelle
  \dbtable{courses} enth�lt eine Spalte \dbcol{semester} mit Fremdschl�ssel in der
  \dbtable{semesters}-Tabelle.
  
  In config.xml muss dazu das Attribut \attrib{has-semester} (default: no) eingf�hrt
  werden.
   
\item Semester werden in der DB durch eine Tabelle \dbtable{semesters} dargestellt.
  Spalten:

  \begin{enumerate}

  \item \dbcol{id} -- Id (Prim�rschl�ssel)
    
  \item \dbcol{name} -- Name, z.B. "WS 2004/2005"
    
  \item \dbcol{start_time}  -- Semesterbeginn als Timestamp oder Datum
    
  \item \dbcol{end_time}    -- Semesterende als Timestamp oder Datum

  \end{enumerate}

  (vgl. prelearning/sketch.txt 7.2.)
  
\item course_subsections k�nnen einen "Bearbeitungszeitraum" haben. Es ist noch
  zu kl�ren, ob dieser bei den Metainfos oder im Inhalt der course_secions
  abgespeichert wird.

\item course_sections k�nnen nicht wie bisher geschachtelt werden. Damit ergibt
   sich folgende Hierarchie:

   \begin{preformatted}
     courses
        \verb'^'
        |  enthalten in
        +---------------- course_sections
                                 \verb'^'
                                 | angeh�ngt an
                                 +---------------- course_subsections
   \end{preformatted}

\end{enumerate}

-- ENDE DER DATEI --

\end{document}