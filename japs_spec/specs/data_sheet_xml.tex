\documentclass{generic}

% Author: Tilman Rassy <rassy@math.tu-berlin.de>
% $Id: macros.tex,v 1.5 2006/10/05 15:58:42 rassy Exp $

% XML code:
\newcommand{\element}[1]{\code[xml-element]{#1}}
\newcommand{\attrib}[1]{\code[xml-attrib]{#1}}

% Database code:
\newcommand{\dbtable}[1]{\code[db-table]{#1}}
\newcommand{\dbcol}[1]{\code[db-column]{#1}}
\newcommand{\sql}[1]{\code[sql]{#1}}

% TeX code:
\newcommand{\texcmd}[1]{\code[tex-cmd]{\backslash#1}}
\newcommand{\texenv}[1]{\code[tex-env]{#1}}

% General
\newcommand{\val}[1]{\code[value]{#1}}
\newcommand{\prm}[1]{\code[param]{#1}}

% Other:
\newcommand{\notimpl}[1]{\emph[not-impl]{#1}}
\newcommand{\new}[1]{\emph[new]{#1}}
\newcommand{\deprecated}[1]{\emph[deprecated]{#1}}
\newcommand{\wvar}[1]{\var[weak]{#1}}


\begin{document}

\title{Data-Sheet-XML}

\begin{authors}
  \author[rassy@math.tu-berlin.de]{Tilman Rassy}
\end{authors}

\version{$Id: data_sheet_xml.tex,v 1.4 2006/08/11 16:18:22 rassy Exp $}

  Das "Data Sheet XML" ist ein generisches XML-Format zur Beschreibung von
  Konfigurationsdaten u.�. mit einer einfachen hierarchischen Struktur. Es
  soll insbesondere zur Versorgung der Trainings-Applets mit Eingabedaten
  verwendet werden.

\tableofcontents

\section{Grunds�tzliches}

\begin{enumerate}
  
\item Namespace: \val{http://www.mumie.net/xml-namespace/data-sheet}
  
\item Root-Element: \element{data_sheet}
  
\item Darin: Beliebig viele \element{unit}- und/oder \element{data}-Elemente.
  
\item In einem \element{unit}-Element: Beliebig viele \element{unit}- und/oder
  \element{data}-Elemente.
  
\item Die \element{data}-Elemente dienen zur Aufnahme der eigentlichen Daten.
  Jedes solche Element entspricht einem Datum. Der Inhalt eines
  \element{data}-Elements darf beliebiges XML sein, dass jedoch entweder nicht
  zum Data-Sheet-Namespace (http://www.mumie.net/xml-namespace/data-sheet)
  geh�rt oder ein reiner Text-Knoten ist.
  
\item Die \element{unit}-Elemente dienen zur Gliederung und Strukturierung der
  Daten. Sie k�nnen beliebig geschachtelt werden. Dadurch l�sst sich eine
  baumartige Anordnung der Daten erreichen.
  
\item \label{name} Jedes \element{unit}- und \element{data}-Element hat ein
  \attrib{name}-Attribut und damit einen Namen. Der Name darf aus Buchstaben,
  Zahlen, und den Zeichen "-", "_", "." bestehen und muss mit einem Buchstaben
  beginnen. Unter den Kindelementen eines \element{unit}-Elements muss der Name
  eindeutig sein.

\end{enumerate}



\section{XML-Elemente}

\begin{preformatted}%
<data_sheet>
  <!-- Content: unit* data* -->
</data_sheet>
\end{preformatted}

\begin{preformatted}%
<unit
  name="NAME">
  <!-- Content: unit* data* -->
</unit>
\end{preformatted}

\begin{preformatted}%
<data
  name="NAME">
  <!-- PCDATA ANY_EXCEPT_DS_ELEMENT* -->
</data>
\end{preformatted}

Abk�rzungen/Platzhalter:

\begin{description}[code-doc]
  
\item[NAME] Name eines unit- oder data-Elements. S. \ref{name}.
  
\item[PCDATA] Text
  
\item[ANY_EXCEPT_DS_ELEMENT] Element ausserhalb des Data-Sheet-Namespaces.

\end{description}



\section{Beispiel}

\begin{preformatted}[code]%
<data_sheet xmlns="http://www.mumie.net/xml-namespace/data-sheet">
  <unit name="matrix_pair">
    <data name="left_matrix">
      <mtable xmlns="http://www.w3.org/1998/Math/MathML">
        <!-- MathML-Code  -->
      </mtable>
    </data>
    <data name="right_matrix">
      <mtable xmlns="http://www.w3.org/1998/Math/MathML">
        <!-- MathML-Code -->
      </mtable>
    </data>
  </unit>
  <data name="vectors"/>
    <mrow xmlns="http://www.w3.org/1998/Math/MathML">
      <!-- MathML-Code -->
    </mrow>
    <mrow xmlns="http://www.w3.org/1998/Math/MathML">
      <!-- MathML-Code -->
    </mrow>
  </data>
</data_sheet>
\end{preformatted}



\section{Adressierung}\label{adressierung}

Jedes \element{data}-Element l�sst sich durch einen XPath-artigen String
adressieren. Dieser wird aus den Namen der Eltern-Units und seinem eigenen
Namen, mit "/"-Zeichen getrennt, gebildet:

\begin{preformatted}%
  \var{UNIT_NAME}/\var{KIND_UNIT_NAME}/.../\var{DATA_NAME}
\end{preformatted}

Z.B. h�tte die zweite Matrix in obigen Beispiel die Adresse

\begin{preformatted}%
  matrix_pair/right_matrix
\end{preformatted}

Statt "Adresse" wird auch (h�ufiger) die Bezeichnung "Pfad" ("path")
verwendet.



\section{Extraction-XML}\label{extraction_xml}

\begin{enumerate}
  
\item Das sogenannnte Extraction-XML bietet die M�glichkeit, Knoten in
  beliebigen XML-Dokumenten als Data-Sheet-Daten zu markieren. Diese k�nnen
  dann extrahiert und zu einem Data-Sheet zusammengefasst werden.
  
\item Namespace des Extraction-XMLs:
  \val{http://www.mumie.net/xml-namespace/data-sheet/extract}

  �blicher Prefix: \val{dsx}

  Extraction-XML-Elemente werden immer mit Prefix geschrieben.
  
\item Das Extraction-XML umfasst zwei Elemente: \element{dsx:data} und
  \element{dsx:datalabel}.
  
\item Das \element{dsx:data}-Element zeichnet seine Nachfahren als den Inhalt
  eines \element{data}-Elements des Data-Sheets aus. Es hat ein oder zwei
  Attribute:

  \begin{enumerate}
    
  \item \attrib{path} -- \label{path} Der Pfad des \element{data}-Elements im
    Data-Sheet.
    
  \item \attrib{clickable} -- \label{clickable} Optional. M�gliche Werte:
    "yes|no". Gibt an, ob die Daten in der gerenderten Form des Quelldokuments
    anklickbar sein sollen. Was beim Klick passiert, ist Sache der
    Transformation, die das Quelldokument nach (X)HTML (oder ein anderers
    Endformat) �bersetzt.

  \end{enumerate}
  
\item Das \element{dsx:datalabel}-Element zeichnet sein Elternelement als den
  Inhalt eines \element{data}-Elements des Data-Sheets aus. Leer, ein oder zwei
  Attribute:

  \begin{enumerate}
    
  \item \attrib{path} -- S. \ref{path}.
    
  \item \attrib{clickable} -- S. \ref{clickable}.

  \end{enumerate}

\end{enumerate}

-- ENDE DER DATEI --

\end{document}