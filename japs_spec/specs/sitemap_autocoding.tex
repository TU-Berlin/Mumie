\documentclass{generic}


% ------------------------------------------------------------------------------
% Schriftgroessen
% ------------------------------------------------------------------------------

\renewcommand{\normalsize}{\fontsize{18}{22}\selectfont}
\renewcommand{\large}{\fontsize{20}{26}\selectfont}
\renewcommand{\Large}{\fontsize{22}{30}\selectfont}
\renewcommand{\small}{\fontsize{14}{16}\selectfont}
\newcommand{\smaller}{\fontsize{12}{14}\selectfont}
\newcommand{\verysmall}{\fontsize{6}{6}\selectfont}
\renewcommand{\footnotesize}{\fontsize{12}{14}\selectfont}
\newcommand{\tocsize}{\fontsize{12}{16}\selectfont}
\newcommand{\titlesize}{\fontsize{24}{36}\selectfont}

% ------------------------------------------------------------------------------
% Texthervorhebungen
% ------------------------------------------------------------------------------

%\definecolor{emphcolor}{rgb}{0.77,0.00,0.00}
%\renewcommand{\emph}[1]{\textbf{\textcolor{emphcolor}{#1}}}
\renewcommand{\emph}[1]{\textbf{#1}}

\newcommand{\term}[1]{\textit{#1}}
% \newcommand{\code}[1]{\texttt{#1}}
\newcommand{\var}[1]{\textit{#1}}

%\definecolor{headlinecolor}{rgb}{0.77,0.00, 0.00}
\definecolor{headlinecolor}{rgb}{0.67,0.09, 0.22}
\newcommand{\headline}[1]{\textbf{\textcolor{headlinecolor}{#1}}}

\definecolor{emphboxbordercolor}{rgb}{0.80,0.36,0.36}
\definecolor{emphboxcolor}{rgb}{0.98,0.98,0.77}
\newcommand{\emphbox}[2]{\fcolorbox{emphboxbordercolor}{emphboxcolor}{\parbox{#1}{#2}}}

\definecolor{labelcolor}{rgb}{0.35,0.35, 0.8}


% -------------------------------------------------------------------------------
% Pfeile
% -------------------------------------------------------------------------------

\newcommand{\myarrow}{{\Large\boldmath$\;\rightarrow\;$}}
\newcommand{\mylongarrow}{{\Large\boldmath$\;\longrightarrow\;$}}



% -------------------------------------------------------------------------------
% Befehle f�r Abbildungen 
% -------------------------------------------------------------------------------

\renewcommand{\thefigure}{\thesubsection.\arabic{figure}}

\newcommand{\mycaption}[2]{%
 \centerline{%
  \parbox{#1}{%
   \small \refstepcounter{figure} Figure \thefigure. #2 }}}


%---------------------------------------------------------------------------
% Gliederungsbefehle
%---------------------------------------------------------------------------

\makeatletter

\renewcommand{\section}{\@startsection
 {section}%                                   % Name
 {1}%                                         % Ebene
 {0pt}%                                       % Einzug
 {-0.0\baselineskip}%                        % Vorabstand 
 {0.85\baselineskip}%                         % Nachabstand
 {\fontsize{20}{24}\selectfont\itshape\bfseries}}%    % Stil

\makeatother

\newcommand{\mysection}[1]{%

\vspace*{-1.2\baselineskip}

{\color{headlinecolor}
\section{#1}}}

\newcounter{myappendix}

\newcommand{\myappendix}[2]{%
\hfill\makebox[0cm][l]{\tocref}

\vspace*{-1.2\baselineskip}

\refstepcounter{myappendix}
\section*{A\arabic{myappendix}\hspace{1em}#1}\hypertarget{#2}{}}



% -------------------------------------------------------------------------------
% Inhaltsverzeichnis 
% -------------------------------------------------------------------------------

\newcommand{\mytoctitle}[1]{\begin{center}\textbf{\hypertarget{inhalt}{#1}}\end{center}}

\newenvironment{mytoc}
  {\tocsize\textbf{\hypertarget{inhalt}{Inhalt}}\\\begin{enumerate}}
  {\end{enumerate}}

\newcommand{\mytocitem}[3]{\item[#1]\hyperlink{#3}{#2}}

\newcommand{\tocref}{\verysmall\hyperlink{inhalt}{TOC}\normalsize}

% -------------------------------------------------------------------------------
% Listen 
% -------------------------------------------------------------------------------

%\newcommand{\mylabelitemi}{\labelitemi}
\newcommand{\mylabelitemi}{\raisebox{0.25ex}{{\small $\blacktriangleright$}\hspace{0.1em}}}

%\newcommand{\mylabelitemii}{\labelitemii}
%\newcommand{\mylabelitemii}{\raisebox{0.25ex}{{\small $\triangleright$}\hspace{0.1em}}}
\newcommand{\mylabelitemii}{\raisebox{0.1ex}{\large\bfseries-}\hspace{0.075em}}

\newenvironment{mylist}[1][\mylabelitemi]
  {\begin{list}{{\color{labelcolor}#1}}
    {\setlength{\itemindent}{0cm}
     \setlength{\labelwidth}{1em}
     \setlength{\leftmargin}{0.5cm}
     \setlength{\rightmargin}{1cm}}}
  {\end{list}}

\newenvironment{myenum}
  {\begin{list}{\theenumi}
    {\usecounter{enumi}
     \setlength{\itemindent}{0cm}
     \setlength{\labelwidth}{1em}
     \setlength{\leftmargin}{0.5cm}
     \setlength{\rightmargin}{1cm}}}
  {\end{list}}

\newcommand{\pitem}{\pause\item}
\newcommand{\itemii}{\item[\mylabelitemii]}

\newcounter{romanlistcounter}
\newenvironment{romanlist}{\begin{list}%
 {(\roman{romanlistcounter})\hfill}{\usecounter{romanlistcounter}%
 \topsep0.05\baselineskip% 
 \partopsep0.85\baselineskip%
 \leftmargin2em%
 \labelwidth2em%
 \labelsep0pt%
 \itemindent0pt%
 \parsep0.15\baselineskip%
 \itemsep0.0\baselineskip}}%
 {\end{list}}

\newcounter{alphlistcounter}
\newenvironment{alphlist}{\begin{list}%
 {(\alph{alphlistcounter})\hfill}{\usecounter{alphlistcounter}%
 \topsep0.05\baselineskip% 
 \partopsep0.85\baselineskip%
 \leftmargin1.8em%
 \labelwidth1.8em%
 \labelsep0pt%
 \itemindent0pt%
 \parsep0.15\baselineskip%
 \itemsep0.0\baselineskip}}%
 {\end{list}}

\newcounter{arabiclistcounter}
\newenvironment{arabiclist}{\begin{list}%
 {(\arabic{arabiclistcounter})\hfill}{\usecounter{arabiclistcounter}%
 \topsep0.05\baselineskip% 
 \partopsep0.85\baselineskip%
 \leftmargin1.8em%
 \labelwidth1.8em%
 \labelsep0pt%               
 \itemindent0pt%
 \parsep0.15\baselineskip%
 \itemsep0.0\baselineskip}}%
 {\end{list}}





\begin{document}

\title{Sitemap-Autocoding}

\begin{authors}
  \author[rassy@math.tu-berlin.de]{Tilman Rassy}
\end{authors}

\version{$Id: sitemap_autocoding.tex,v 1.6 2007/05/28 23:39:26 rassy Exp $}

Diese Spezifikation beschreibt die Erzeugung der Sitemap. Sie umfasst sowohl die
Abl�ufe als auch das Format des dabei verwendeten XMLs.

\tableofcontents

\section{�berblick}\label{ueberblick}

Die Sitemap \file{sitemap.xmap} wird aus einer XML-Quelle
\file{sitemap.xmap.src} automatisch erzeugt. Dies geschieht w�hrend des
Build-Prozesses oder auch im laufenden Betrieb. Letzteres kann notwendig sein,
wenn sich Dokumente ge�ndert haben, die in der Sitemap referenziert werden. Bei
der �bersetzung von \file{sitemap.xmap.src} nach \file{sitemap.xmap} werden
Checkin-Pfade und Namen zu Datenbank-Ids aufgel�st. Die �bersetzung
bewerkstelligt ein Cocoon-Transformer,

\begin{preformatted}%
  net.mumie.cocoon.transformers.SitemapTransformer
\end{preformatted}

und setzt daher einen laufenden JAPS voraus.


\section{XML-Format}

Die Quelle \file{sitemap.xmap.src} besteht zum gr��ten Teil aus "normalem"
Sitemap-XML, also aus XML des entsprechenden vom Apache-Projekt definierten
Formats. Dieses XML hat den Namensraum

\begin{preformatted}%
  \val{http://apache.org/cocoon/sitemap/1.0}
\end{preformatted}

und den Prefix \val{map}. Dar�berhinaus enth�lt \file{sitemap.xmap.src}
MUMIE-spezifisches XML. Dieses hat den Namensraum

\begin{preformatted}%
  \val{http://www.mumie.net/xml-namespace/sitemap-autocoding}
\end{preformatted}

und den Prefix \val{smac}. Es wird \emph{SmacXML} genannt. Z.z. gibt es in
SmacXML nur Attribute, keine Elemente. Sie werden bei der Konvertierung in
gleichnamige "normale" Attribute �bersetzt. Beispiel:

\begin{preformatted}[code]%
  <map: parameter
    name="{}user-groups"
    smac:value="{}ugr-names-to-ids(admins, lecturers, tutors)"/>
\end{preformatted}

Wird �bersetzt in:

\begin{preformatted}[code]%
  <map: parameter
    name="{}user-groups"
    value="0, 3, 6"/>
\end{preformatted}

Die Werte von SmacXML-Attributen sind sogenannte \emph{Smac-Funktionen}. Sie
werden bei der �bersetzung ausgef�hrt und geben jeweils einen String zur�ck.
Dieser String ist der neue Wert des Attributs.

Abschnitt \ref{smac_funktionen} enth�lt eine Liste aller Smac-Funktionen.

Ein Smac-Funktionsaufruf hat die Form

\begin{preformatted}[code]%
  \var{name}(\var{param1}, \meta{...}, \var{paramN})
\end{preformatted}

Hierbei sind \var{name} der Name der Funktion und \var{param1}, \ldots,
\var{paramN} die ihr �bergebenden Parameter. Parameter k�nnen in einfache oder
doppelte Anf�hrungszeichen (\code{'} bzw. \code{"}) eingeschlossen werden,
m�ssen aber nicht. Als Parameter-Angabe sind also

\begin{preformatted}%
  foo  "foo"  'foo'
\end{preformatted}

�quivalent. Werden keine Anf�hrungszeichen verwendet, m�ssen die Zeichen

\begin{preformatted}%
  \{'"(),
\end{preformatted}

sowie Whitespaces durch einen Backslash gesch�tzt werden. Innerhalb eines
in einfache Anf�hrungszeichen eingeschlossenen Parameters muss nur das
\code{'}-Zeichen durch einen Backslash gesch�tzt werden. Innerhalb eines
in doppelte Anf�hrungszeichen eingeschlossenen Parameters muss nur das
\code{"}-Zeichen durch einen Backslash gesch�tzt werden.

\section{Liste aller Smac-Funktionen}\label{smac_funktionen}

\subsection{add-url-prefix}

F�gt den Japs-URL-Prefix zu einer relativen Japs-URL hinzu. Prototyp:

\begin{preformatted}[code]%
  add-url-prefix(\var{rel_url})
\end{preformatted}

\var{rel_url} ist die relative Japs-URL. 

Beispiel:

\begin{preformatted}[code]%
  add-url-prefix(auth/login)
\end{preformatted}

Wird �bersetzt in:

\begin{preformatted}[code]%
  https://japs.mumie.net/cocoon/auth/login
\end{preformatted}


\subsection{resolve-path}

L�st den Checkin-Pfad eines Dokuments zu der entsprechenden Datenbank-Id auf,
f�gt diese in ein URL-Template ein und gibt die URL zur�ck. Prototyp:

\begin{preformatted}[code]%
  resolve-path(\var{url_template}, \var{path}, \var{doctype})
\end{preformatted}

Hierbei sind \var{url_template} das URL-Template, \var{path} der Checkin-Pfad und
\var{doctype} der Dokument-Typ. \var{url_template} darf folgende folgende
Platzhalter enthalten:

\begin{enumerate}
\item % \code{\%\{prefix\}}

Wird durch den Japs-URL-Prefix ersetzt. Dieser hat normalerweise die Form

\begin{preformatted}%
    https://\var{host}\optional{:\var{port}}/cocoon
\end{preformatted}

wobei \var{host} und \var{port} der Hostname bzw. die Portnummer der Servers
sind. Der Prefix wird dem Cocoon-Transformer, der die Sitemap erzeugt, als
Konfigurationseintrag \code{url-prefix} �bergeben (s. \ref{ueberblick} und
API-Dokumentation des Transformers).

\item % \code{\%\{id\}}

Wird durch die Id des Dokuments ersetzt, das durch \var{doctype} und \var{path}
bestimmt ist.

\end{enumerate}

\emph{Bemerkung:} Zwar ist das Dokument bereits durch \var{path} allein
bestimmt, ohne den Dokumenttyp w�re die Suche in der Datenbank aber etwas
aufw�ndig. Daher muss der Dokumenttyp (\var{doctype}) mit angegeben werden.

Beispiel:

\begin{preformatted}[code]%
  resolve-path
    (\%\{prefix\}/protected/view/document/type-name/page/id/\%\{id\},
     checkin/system/misc/pge_layout_sample.meta.xml,
     page)
\end{preformatted}

Wird �bersetzt in:

\begin{preformatted}[code]%
  https://japs.mumie.net/cocoon/protected/view/document/\meta{\backslash}
  type-name/page/id/56
\end{preformatted}

[Der Backslash am Ende der ersten Zeile ist nicht Teil der URL, sondern zeigt
nur an, dass die URL in Wirklichkeit aus \emph{einer} Zeile besteht, die aus
Platzgr�nden in zwei Zeilen aufgespalten wurde.] Der URL-Prefix und die Id sind
nat�rlich nur Beispiele und k�nnten auch anders lauten.

\subsection{ugr-names-to-ids}

L�st eine Liste von Benutzergruppen-Namen zu den entsprechenden Datenbank-Ids
auf und gibt diese zur�ck. Prototyp:

\begin{preformatted}[code]%
  ugr-names-to-ids(\var{name1}, \meta{...}, \var{nameN})
\end{preformatted}

Hierbei sind \var{name1}, \ldots, \var{nameN} die Namen der
Benutzergruppen. Der zur�ckgegebene String ist

\begin{preformatted}[code]%
  \var{id1}, \meta{...}, \var{idN}
\end{preformatted}
 
wobei \var{id1} die Id der Benutzergruppe \var{name1} ist, \var{id2} die von
\var{name2} usw.

Beispiel:

\begin{preformatted}[code]%
  ugr-names-to-ids(admins, lecturers, tutors)
\end{preformatted}

Wird �bersetzt in:

\begin{preformatted}[code]%
  0, 3, 6
\end{preformatted}

Die Ids sind nat�rlich nur Beispiele und k�nnten auch anders lauten.


\subsection{use-mode-name-to-code}

Wandelt einen Use-Mode-Namen in den entsprechenden Zahlencode um. Prototyp:

\begin{preformatted}[code]%
  use-mode-name-to-code(\var{name})
\end{preformatted}

\var{name} ist der Use-Mode-Name

Beispiel:

\begin{preformatted}[code]%
  use-mode-name-to-code(info)
\end{preformatted}

Wird �bersetzt in:

\begin{preformatted}[code]%
  \var{code}
\end{preformatted}

wobei \var{code} der Zahlencode des Use-Modes \val{info} ist.

-- WIRD FORTGESETZT --

\end{document}