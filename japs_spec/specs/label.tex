\documentclass{generic}

% Author: Tilman Rassy <rassy@math.tu-berlin.de>
% $Id: macros.tex,v 1.5 2006/10/05 15:58:42 rassy Exp $

% XML code:
\newcommand{\element}[1]{\code[xml-element]{#1}}
\newcommand{\attrib}[1]{\code[xml-attrib]{#1}}

% Database code:
\newcommand{\dbtable}[1]{\code[db-table]{#1}}
\newcommand{\dbcol}[1]{\code[db-column]{#1}}
\newcommand{\sql}[1]{\code[sql]{#1}}

% TeX code:
\newcommand{\texcmd}[1]{\code[tex-cmd]{\backslash#1}}
\newcommand{\texenv}[1]{\code[tex-env]{#1}}

% General
\newcommand{\val}[1]{\code[value]{#1}}
\newcommand{\prm}[1]{\code[param]{#1}}

% Other:
\newcommand{\notimpl}[1]{\emph[not-impl]{#1}}
\newcommand{\new}[1]{\emph[new]{#1}}
\newcommand{\deprecated}[1]{\emph[deprecated]{#1}}
\newcommand{\wvar}[1]{\var[weak]{#1}}


\begin{document}

\title{Labels}

\begin{authors}
  \author[vrichter@math.tu-berlin.de]{Verena Richter}
\end{authors}

\tableofcontents

\section{Spezifikation der Labels in der Datenbank}
\subsection{Japs}
In den Tabellen
\begin{itemize}
\item refs\_course\_subsection\_problem, 
\item refs\_course\_course\_section,
\item refs\_course\_course\_subsection, 
\item refs\_course\_section\_element,
\item refs\_course\_section\_subelement 
\end{itemize}
gibt es jeweils die Spalte ''label'' vom Typ Text.

\subsection{Schreiben der Labels in die Datenbank}
Die Labels werden mittels der theoriegraph*.meta.xml -Dateien in die Datenbank
geschrieben. Sie sind ref\_attribute, somit (wie die points von problem) Kinder von
\begin{itemize}
\item course\_section
\item course\_subsection
\item element
\item subelement
\item problem
\end{itemize}
und haben die Attribute name=''label'' und value='' ''.\\
Somit muss der Abschnitt der theoriegraph*.meta.xml-Datei folgendermassen
aussehen:
\begin{preformatted}[code]%
    <mumie:components>
      <mumie:course_section lid="1" url="theoriegraph_0.meta.xml">
        <mumie:ref_attribute name="label" value="SectionLabel"/>
      </mumie:course_section>
      <mumie:course_subsection lid="2" url="theoriegraph_0_sub_0.meta.xml">
        <mumie:ref_attribute name="label" value="subSectionLabel"/>
      </mumie:course_subsection>
    </mumie:components>
\end{preformatted}
\section{Labels im CourseCreator}
Die Labels sind Membervariablen der Klassen 
\begin{itemize}
\item SubElementProperties
\item ElementProperties
\item ProblemProperties
\item SectionProperties
\item SubSectionProperties
\end{itemize}
welche im Package net.mumie.coursecreator.graph.cells zu finden sind. Die
Labels sind vom Typ String. Initialisiert werden die Labels beim Aufruf eines Konstruktors und werden dann auf
den Defaultwert gesetzt. Der Defaultwert eines Labels ist der Inhalt der Konstante DEFAULT\_LABEL der
Klasse Constants im Package net.mumie.coursecreator.
\subsection{�ndern der Labels im Coursecreator}
Die Klassen ElementMarqueeHandler und SectionMarqueeHandler aus dem Package
net.mumie.coursecreator.graph besitzen die Funktion createPopupMenu(), welche
den Menupunkt ''Label setzen'' erzeugt.\\
Die Klasse ElementMarqueeHandler ist der mouseHandler f�r 
\begin{itemize}
\item ElementCell
\item ProblemCell
\item SubElementCell
\end{itemize}
SectionMarqueeHandler f�r
\begin{itemize}
\item SectionCell
\item SubSectionCell
\end{itemize}

Die Klasse CCController besitzt die Funktion assignLabelDialog(Object cell) welche
den Dialog f�r das Setzen der Labels von ''cell'' erzeugt. Die Funktion wird
von den MarqueeHandlern aufgerufen und
\begin{itemize}
\item erzeugt eine Fehlermeldung (und bricht dann ab), falls kein Label (leerer String) eingegeben
wurde,
\item bricht ab, falls das Label nicht ver�ndert wurde, oder
\item setzt das neue Label, indem die jeweilige setLabel(value) Funktion der Klasse
\begin{itemize}
\item SubElementProperties
\item ElementProperties
\item ProblemProperties
\item SectionProperties
\item SubSectionProperties
\end{itemize}
aufgerufen wird.
\end{itemize}

\subsection{Laden und Speichern}
\subsubsection{lokal Laden und Speichern}
Zum lokal speichern hat jede der Klassen 
\begin{itemize}
\item ElementCell
\item ProblemCell
\item SubElementCell
\item SectionCell
\item SubSectionCell
\end{itemize}
eine Funktion getUnfinishedXML(..) in der alle Attribute, also auch die Labels des jeweiligen
Types gesetzt werden. Die Funktionen werden jeweils von der Funktion
graph2xml(..) der Klasse UnfinishedGraphSaver aus dem Package
net.mumie.coursecreator.io aufgerufen.

Die Labels werden als Attribute der Knoten
\begin{itemize}
\item Section
\item SubSection
\item Element
\item SubElement
\item Problem
\end{itemize}
gespeichert.

F�r das lokale Laden wird durch die parse(..)-Funktion der Klasse
UnfinishedGraphHandler aus dem Package net.mumie.coursecreator.io.helper die
Funktion setComponents(..) der gleichen Klasse auf. Diese liest das Label aus
und erzeugt entsprechend des Types eine neue *Properties Instanz.
Beispiel:
\begin{preformatted}[code]%
String label = getSingleNode(act,"./@label").getNodeValue();
AbstractMainComponentCellProperties prop = new
SectionProperties(lid,id,label,first);
SectionCell sc =
((SectionGraph)ng).insertSection(p,(SectionProperties)prop,!lid.equals(""));
\end{preformatted}
\subsubsection{Speichern in die Datenbank}

Die Sections und Elements werden unterschiedlich behandelt.\\

Erzeugt werden die Sectionlabels in der Funktion buildComponents(..) der Klasse GraphSaver
des Packages net.mumie.coursecreator.io.\\

F�r die Elements hat jede der Klassen
\begin{itemize}
\item ElementCell
\item ProblemCell
\item SubElementCell
\end{itemize} aus dem Package net.mumie.coursecreator.graphs.cells eine
Funktion getXML(..), die die Labels in die XML-Datei schreibt. Diese
getXML(..)-Funktion wird jedoch ebenfalls vom GraphSaver aufgerufen.

Die Labels werden mittels der theoriegraph*.meta.xml Dateien in die Datenbank
geschrieben. Die Labels sind ref\_attribute, somit Kinder von
\begin{itemize}
\item course\_section
\item course\_subsection
\item element
\item subelement
\item problem
\end{itemize}
und haben die Attribute name=''label'' und value='' ''.

Beispiel:
\begin{preformatted}[code]%
Element labelElem = document[0].createElement(XMLConstants.PREFIX_META + "ref_attribute");
labelElem.setAttribute("name", "label");
labelElem.setAttribute("value", prop.getLabel());
xmlElem.appendChild(labelElem);
\end{preformatted}

wobei xmlElem ebenfalls vom Typ Element ist und zB. die course\_section darstellt und
prop die SectionProperties der zugeh�rigen SectionCell ist.\\

\subsubsection{Laden aus der Datenbank}
Wie beim Speichern werden die Elements und die Sections in unterschiedlichen
Funktionen behandelt, die Verarbeitungsweise ist aber gleich.\\

Die Sectionlabels werden in der Funktion setCourseComponents(..), die
Elementlabels in der Funktion setSectionComponents(..) der Klasse
GraphJaxenHandler aus dem package net.mumie.coursecreator.io.xml
ausgelesen und ja nach Typ wird eine neue *Properties Instanz erzeugt. Beide
Funktionen (setCourseComponents und setSectionComponents) werden von von der parse(..)-Funktion der gleichen
Klasse aufgerufen.\\

Sollte es kein Label geben, wird das Defaultlabel DEFAULT\_LABEL der
Klasse Constants im Package net.mumie.coursecreator gesetzt.\\

Beispiel:\\
\begin{preformatted}[code]%
Node labelNode = this.getSingleNode(root,metaSource+"/"+MUMIE+"ref_attribute[@name=$\backslash$"label$\backslash$"]/@value");
String label;
if (labelNode== null) label = Constants.DEFAULT_LABEL;
else label = labelNode.getNodeValue();
SectionProperties prop = new SectionProperties(lid,id,label,false);
\end{preformatted}
\subsection{Beschr�nkungen des Labels}
Keine, es darf nur nicht leer sein (das sollte aber nicht m�glich sein, da es beim
Erstellen einen Wert erh�lt, und beim Setzen kein Leerstring eingegeben werden kann).
\end{document}