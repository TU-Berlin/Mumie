\documentclass{generic}


% ------------------------------------------------------------------------------
% Schriftgroessen
% ------------------------------------------------------------------------------

\renewcommand{\normalsize}{\fontsize{18}{22}\selectfont}
\renewcommand{\large}{\fontsize{20}{26}\selectfont}
\renewcommand{\Large}{\fontsize{22}{30}\selectfont}
\renewcommand{\small}{\fontsize{14}{16}\selectfont}
\newcommand{\smaller}{\fontsize{12}{14}\selectfont}
\newcommand{\verysmall}{\fontsize{6}{6}\selectfont}
\renewcommand{\footnotesize}{\fontsize{12}{14}\selectfont}
\newcommand{\tocsize}{\fontsize{12}{16}\selectfont}
\newcommand{\titlesize}{\fontsize{24}{36}\selectfont}

% ------------------------------------------------------------------------------
% Texthervorhebungen
% ------------------------------------------------------------------------------

%\definecolor{emphcolor}{rgb}{0.77,0.00,0.00}
%\renewcommand{\emph}[1]{\textbf{\textcolor{emphcolor}{#1}}}
\renewcommand{\emph}[1]{\textbf{#1}}

\newcommand{\term}[1]{\textit{#1}}
% \newcommand{\code}[1]{\texttt{#1}}
\newcommand{\var}[1]{\textit{#1}}

%\definecolor{headlinecolor}{rgb}{0.77,0.00, 0.00}
\definecolor{headlinecolor}{rgb}{0.67,0.09, 0.22}
\newcommand{\headline}[1]{\textbf{\textcolor{headlinecolor}{#1}}}

\definecolor{emphboxbordercolor}{rgb}{0.80,0.36,0.36}
\definecolor{emphboxcolor}{rgb}{0.98,0.98,0.77}
\newcommand{\emphbox}[2]{\fcolorbox{emphboxbordercolor}{emphboxcolor}{\parbox{#1}{#2}}}

\definecolor{labelcolor}{rgb}{0.35,0.35, 0.8}


% -------------------------------------------------------------------------------
% Pfeile
% -------------------------------------------------------------------------------

\newcommand{\myarrow}{{\Large\boldmath$\;\rightarrow\;$}}
\newcommand{\mylongarrow}{{\Large\boldmath$\;\longrightarrow\;$}}



% -------------------------------------------------------------------------------
% Befehle f�r Abbildungen 
% -------------------------------------------------------------------------------

\renewcommand{\thefigure}{\thesubsection.\arabic{figure}}

\newcommand{\mycaption}[2]{%
 \centerline{%
  \parbox{#1}{%
   \small \refstepcounter{figure} Figure \thefigure. #2 }}}


%---------------------------------------------------------------------------
% Gliederungsbefehle
%---------------------------------------------------------------------------

\makeatletter

\renewcommand{\section}{\@startsection
 {section}%                                   % Name
 {1}%                                         % Ebene
 {0pt}%                                       % Einzug
 {-0.0\baselineskip}%                        % Vorabstand 
 {0.85\baselineskip}%                         % Nachabstand
 {\fontsize{20}{24}\selectfont\itshape\bfseries}}%    % Stil

\makeatother

\newcommand{\mysection}[1]{%

\vspace*{-1.2\baselineskip}

{\color{headlinecolor}
\section{#1}}}

\newcounter{myappendix}

\newcommand{\myappendix}[2]{%
\hfill\makebox[0cm][l]{\tocref}

\vspace*{-1.2\baselineskip}

\refstepcounter{myappendix}
\section*{A\arabic{myappendix}\hspace{1em}#1}\hypertarget{#2}{}}



% -------------------------------------------------------------------------------
% Inhaltsverzeichnis 
% -------------------------------------------------------------------------------

\newcommand{\mytoctitle}[1]{\begin{center}\textbf{\hypertarget{inhalt}{#1}}\end{center}}

\newenvironment{mytoc}
  {\tocsize\textbf{\hypertarget{inhalt}{Inhalt}}\\\begin{enumerate}}
  {\end{enumerate}}

\newcommand{\mytocitem}[3]{\item[#1]\hyperlink{#3}{#2}}

\newcommand{\tocref}{\verysmall\hyperlink{inhalt}{TOC}\normalsize}

% -------------------------------------------------------------------------------
% Listen 
% -------------------------------------------------------------------------------

%\newcommand{\mylabelitemi}{\labelitemi}
\newcommand{\mylabelitemi}{\raisebox{0.25ex}{{\small $\blacktriangleright$}\hspace{0.1em}}}

%\newcommand{\mylabelitemii}{\labelitemii}
%\newcommand{\mylabelitemii}{\raisebox{0.25ex}{{\small $\triangleright$}\hspace{0.1em}}}
\newcommand{\mylabelitemii}{\raisebox{0.1ex}{\large\bfseries-}\hspace{0.075em}}

\newenvironment{mylist}[1][\mylabelitemi]
  {\begin{list}{{\color{labelcolor}#1}}
    {\setlength{\itemindent}{0cm}
     \setlength{\labelwidth}{1em}
     \setlength{\leftmargin}{0.5cm}
     \setlength{\rightmargin}{1cm}}}
  {\end{list}}

\newenvironment{myenum}
  {\begin{list}{\theenumi}
    {\usecounter{enumi}
     \setlength{\itemindent}{0cm}
     \setlength{\labelwidth}{1em}
     \setlength{\leftmargin}{0.5cm}
     \setlength{\rightmargin}{1cm}}}
  {\end{list}}

\newcommand{\pitem}{\pause\item}
\newcommand{\itemii}{\item[\mylabelitemii]}

\newcounter{romanlistcounter}
\newenvironment{romanlist}{\begin{list}%
 {(\roman{romanlistcounter})\hfill}{\usecounter{romanlistcounter}%
 \topsep0.05\baselineskip% 
 \partopsep0.85\baselineskip%
 \leftmargin2em%
 \labelwidth2em%
 \labelsep0pt%
 \itemindent0pt%
 \parsep0.15\baselineskip%
 \itemsep0.0\baselineskip}}%
 {\end{list}}

\newcounter{alphlistcounter}
\newenvironment{alphlist}{\begin{list}%
 {(\alph{alphlistcounter})\hfill}{\usecounter{alphlistcounter}%
 \topsep0.05\baselineskip% 
 \partopsep0.85\baselineskip%
 \leftmargin1.8em%
 \labelwidth1.8em%
 \labelsep0pt%
 \itemindent0pt%
 \parsep0.15\baselineskip%
 \itemsep0.0\baselineskip}}%
 {\end{list}}

\newcounter{arabiclistcounter}
\newenvironment{arabiclist}{\begin{list}%
 {(\arabic{arabiclistcounter})\hfill}{\usecounter{arabiclistcounter}%
 \topsep0.05\baselineskip% 
 \partopsep0.85\baselineskip%
 \leftmargin1.8em%
 \labelwidth1.8em%
 \labelsep0pt%               
 \itemindent0pt%
 \parsep0.15\baselineskip%
 \itemsep0.0\baselineskip}}%
 {\end{list}}





\begin{document}

\title{Kurs als Netz}

\begin{authors}
  \author[lehmannf@math.tu-berlin.de]{Fritz Lehmann-Grube}
  \author[rassy@math.tu-berlin.de]{Tilman Rassy}
  \author[seiler@math.tu-berlin.de]{Ruedi Seiler}
\end{authors}

\version{$Id: kurs_als_netz.tex,v 1.5 2006/08/11 16:18:22 rassy Exp $}

\tableofcontents
\section{Pr"aambel}\label{preambel}
Didaktisches Ziel: Vermittlung nicht nur von Konzepten, Resultaten und Anwendungen, sondern
vor allem von \emph{(logischen) Zusammenh�ngen}. Dies ist wichtige
Voraussetzung f�r die F�higkeit, Probleme zu l�sen, d.h. Verbindungen vom
Bekanntem zum Unbekannten herzustellen.

Typische Zusammenh�nge, die es zu vermitteln gilt, sind:
\begin{enumerate}
\item Die \emph{R�ckw�rtssicht} (''was-wird-gebraucht?''): Welche Begriffe,
  Definitionen usw. werden gebraucht, um ein gegebenes Resultat (z.B.  ein
  Theorem) formulieren zu k�nnen? Welche Resultate werden gebraucht, um eine
  gebene Anwendungen ausf�hren zu k�nnen? Welche Begriffe m�ssen bekannt sein,
  um einen gegebenen Begriff zu verstehen?
\item Die \emph{Vorw�rtssicht} (''wof�r-hat-das-Bedeutung?''): Welche
  Definitionen, Theoreme usw. setzen eine gebene Definition oder ein gebenenes
  Theorem voraus? Was h�ngt alles von einem gegebenem Begriff ab?
\end{enumerate}

Die o.g. Zusammenh�nge k�nnen nat�rlich sprachlich ausgedr�ckt werden.
Tats�chlich ist dies die Methode, die in Lehrb�chern normalerweise angewandt
wird. Dieser Spezifikation liegt jedoch die �berzeugung zu Grunde, dass sich
die Zusammenh�nge �bersichtlicher und einpr�gsamer durch \emph{Graphen}
vermitteln lassen. ''Graph'' ist hierbei durchaus im Sinne der Graphentheorie
zu verstehen: Eine Menge $G$ von \emph{Knoten} mit einer Menge $K\subset G
\times G$ von \emph{Kanten}. $G$ besteht in diesem Fall aus Definitionen,
Theoremen usw. sowie aus logischen ''und''- und ''oder''-Verkn�pfungen (N�heres
s. \ref{struct_graph}).

In der \emph{R�ckw�rtssicht} sind z.B. alle Definitionen, von einem gegebenem Theorem gebraucht
werden, durch einen gerichteten Pfad im Graphen mit einem logischen
''und''-Knoten verbunden und dieser wiederum mit dem Theorem:

\image{rueckwaertssicht.png}

Falls es mehrere Gruppierungen von Definitionen gibt, die jeweils gen�gen, um
das Theorem zu formulieren, kann dies mit einem logischen ''oder''-Knoten
ausgedr�ckt werden:

\image{rueckwaertssicht\_02.png}

Anstelle von Definitionen k�nnen nat�rlich auch andere Theoreme stehen, die von
dem gegebenen Theorem gebraucht werden.

In der \emph{Vorw�rtssicht} zeigt z.B. ein Theorem auf alle von ihm
beeinflussten Theoreme, Anwendungen usw. Zeigen heisst: Es gibt einen
vorw�rtsgerichteten Pfad von dem Theorem auf eine davon beeinflussten Theoreme, Anwendungen usw.


\section{Struktur des Graphen}\label{struct_graph}

Graphen der eben skizzierten Art gibt es f�r Dokumente des Typs \val{course}
(Kurs), \val{course_section} (Kursabschnitt) und \val{course_subsection}
(Kurs-Unterabschnitt, z.B. �bungsblatt). Die Graphen bestehen aus Knoten und
Kanten wie oben beschrieben. 

\subsection{Knoten}

Es gibt folgende Typen von Knoten:
  
\begin{enumerate}
\item\label{elem_knoten} \emph{Element-Knoten} -- stellen Elemente dar, also
  Dokumente vom Typ \val{element}
\item\label{aufg_knoten} \emph{Aufgaben-Knoten} -- stellen Aufgaben dar, also
  Dokumente vom Typ \val{problem}
\item\label{kursabschn_knoten} \emph{Ein Kursabschnitts-Knoten} -- stellen
  Kursabschnitte dar, also Dokumente vom Typ \val{course_section}
\item \emph{''Und''-Knoten}
\item \emph{''Oder''-Knoten}
\item \emph{Neutrale Verzweigungsknoten}
\item\label{aux_knoten} \emph{St�tzknoten} -- unsichtbare Hilfsknoten, um die Linien in
  bestimmter Weise zu f�hren
\end{enumerate}

\ref{elem_knoten}, \ref{aufg_knoten} und \ref{kursabschn_knoten} werden auch
zusammenfassend \emph{Dokument-Knoten} genannt. F�r ihr Auftreten gelten
folgende Einschr�nkungen:

\label{einschr_dokumentknoten}
\begin{enumerate}
  \setcounter{enumi}{7}
  \item \emph{Element-Knoten} gibt es nur in Kurse-Abschnitten,
  \item \emph{Aufgaben-Knoten} gibt es nur in Kurse-Unterabschnitten [gilt
    zumindest f�r die beiden derzeit einzigen Kurs-Unterabschnitt-Kategorien
    \val{homework} und \val{pretest}],
  \item \emph{Kursabschnitts-Knoten} gibt es nur in Kursen.
\end{enumerate}

Jeder Knoten hat eine eindeutige \emph{Knoten-Id}, mit der er
ggf. identifiziert werden kann.

An Dokumentknoten k�nnen mehrere ''Subdokumentknoten'' angeh�ngt sein,
z.B. Subelemente an Elemente, diese werden aber nicht als Teil des Graphen
aufgefasst.

\subsection{Kanten}

Eine Kante ist ein geordnetes Paar von Knoten. Sie kann mit dem geordneten
Paar der entsprechenden Knoten-Ids identifiziert werden. Die Interpretation der
Richtung der Kante ist wie folgt: Ist $(g_1, g_2)$ eine Kante, so ist der
Knoten $g_1$ logische Voraussetzung von $g_2$.


\section{Darstellung des Graphen}

\subsection{Knoten}

F�r die visuelle Darstellung erhalten Knoten eine x- und eine y-Koordinate in
Pixeln. Die Knoten selbst werden durch Icons dargestellt. Die an einen
Dokumentknoten angeh�ngten ''Subdokumentknoten'' werden ebenfalls durch Icons
dargestellt. Die m�glichen Positionen der ''Subdokumentknoten''-Icons relativ
zum Dokumentknoten-Icon sind \emph{oben-links}, \emph{oben-rechts},
\emph{unten-rechts}, \emph{unten-links}. Siehe auch folgende Zeichnung: 

\image{course\_darstellung\_ohne\_text.png}

\subsection{Kanten}

Eine Kante wird durch eine Linie dargestellt, die das Icon des Startknotens mit
dem Icon des Endknotens verbindet. Die Linie besteht aus einem oder mehreren
\emph{Segmenten}. Jedes Segment ist entweder senkrecht oder waagerecht. Die
Punkte, an denen sich die Segmente ber�hren, werden \emph{Zwischenpunkte}
genannt. Die Zwischenpunkte werden mit $1$ beginnend durchnumeriert. Die
Numerierung beginnt beim Startknoten.

Die folgende Abbildung zeigt vier Beispiele von Kanten unterschiedlicher
Gestalt. $P_1$, $P_2$, $\ldots$ bezeichnen die Zwischenpunkte.

\image{kanten\_schritte.png}

\end{document}
