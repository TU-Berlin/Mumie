\documentclass{generic}

% Author: Tilman Rassy <rassy@math.tu-berlin.de>
% $Id: macros.tex,v 1.5 2006/10/05 15:58:42 rassy Exp $

% XML code:
\newcommand{\element}[1]{\code[xml-element]{#1}}
\newcommand{\attrib}[1]{\code[xml-attrib]{#1}}

% Database code:
\newcommand{\dbtable}[1]{\code[db-table]{#1}}
\newcommand{\dbcol}[1]{\code[db-column]{#1}}
\newcommand{\sql}[1]{\code[sql]{#1}}

% TeX code:
\newcommand{\texcmd}[1]{\code[tex-cmd]{\backslash#1}}
\newcommand{\texenv}[1]{\code[tex-env]{#1}}

% General
\newcommand{\val}[1]{\code[value]{#1}}
\newcommand{\prm}[1]{\code[param]{#1}}

% Other:
\newcommand{\notimpl}[1]{\emph[not-impl]{#1}}
\newcommand{\new}[1]{\emph[new]{#1}}
\newcommand{\deprecated}[1]{\emph[deprecated]{#1}}
\newcommand{\wvar}[1]{\var[weak]{#1}}


\begin{document}

\title{Multiple-Choice-XML}

\begin{authors}
  \author[rassy@math.tu-berlin.de]{Tilman Rassy}
\end{authors}
\version{$Id: multiple_choice_xml.tex,v 1.3 2006/08/11 16:18:22 rassy Exp $}

\deprecated{Veraltet: Wird ersetzt durch \href{problem_xml.xhtml}{Problem-XML}.}

\tableofcontents

\section{Grunds�tzliche Struktur}

\begin{enumerate}
  
\item Ein Part, ein oder mehrere Subparts
  
\item Part vom subtype \val{mchoiceproblems}

\item \label{subparts} Jedem Subpart entspricht eine Multiple-Choice-Aufgabe.
  
\item Der Subpart darf Text mit den �blichen Formatierungen (Abs�tze, Listen,
  Tabellen usw.) sowie mathematische Formeln und Multimedia (Bilder usw.)
  enthalten.
  
\item Der Subpart muss ein oder mehrere Elemente \element{choices} enthalten.
  Jedes definiert eine Gruppe m�glicher Antworten. Empfehlung: In jedem Subpart
  nur ein \element{choices}-Element, von Ausnahmef�llen abgesehen.
  
\item \label{choices_type} \element{choices} hat ein \attrib{type}-Attribut mit
  folgenden m�glichen Werten:

  \begin{enumerate}
    
  \item \val{unique} Genau eine Antwort kann angekreuzt werden.
    
  \item \val{multiple} Eine oder mehrere Antworten k�nnen angekreuzt werden.
    
  \item \val{yesno} F�r jede Antwort muss "ja" oder "nein" angekreuzt werden.

  \end{enumerate}
  
\item \element{choices} hat ein oder mehrere \element{choice}-Kindelemente.
  Jedes entspricht einer Antwort.
  
\item Jedes \element{choice}-Element hat folgende Kinder:

  \begin{enumerate}
    
  \item Genau ein \element{assertion}-Element. Enth�lt den Wortlaut der zur
    Wahl stehenden Antwort. Darf Text mit den �blichen Formatierungen sowie
    mathematische Formeln und Multimedia enthalten.
    
  \item Ein \element{solution}-Element. Gibt die L�sung an (d.h. ob die Antwort
    richtig oder falsch ist). Kein Inhalt, ein Attribut: \attrib{value};
    m�gliche Werte: "true" (Antwort ist richtig) und "false" (Antwort ist
    falsch). S. \ref{subparts}.  Braucht bei \element{choices} mit
    type="unique" nur bei der richtigen Antwort zu stehen. Sonst zwingend.
    
  \item Ein optionales \element{explanation}-Element. Enth�lt eine ausf�hrliche
    Begr�ndung der L�sung. Darf Text mit den �blichen Formatierungen sowie
    mathematische Formeln und Multimedia enthalten.

  \end{enumerate}
  
\item Im Anschluss an die \element{choice}-Elemente kann \element{choices} ein
  \element{commonexpl}-Kindelement haben. Enth�lt eine ausf�hrliche Begr�ndung
  der L�sung. Darf Text mit den �blichen Formatierungen sowie mathematische
  Formeln und Multimedia enthalten. Empfehlung: Existiert dieses Element, so
  sollten die \element{chioce}-Elemente keine \element{explanation}-Kinder
  haben.
  
\item \label{choices_key} Jedes \element{choices}-Element hat ein Attribut
  \attrib{key}, das dieses Element eindeutig identifiziert. Der Wert ist eine
  ganze Zahl, welche die Nummer des \element{choices}-Elements angibt. Dazu
  denke man sich die \element{chioces}-Elemente von 1 beginnend im XML-Code
  fortlaufend numeriert.
  
\item \label{choice_key} Jedes \element{chioce}-Element hat ein Attribut
  \attrib{key}, das dieses Element innerhalb des umschliessenden
  \element{choices}-Elements eindeutig indentifiziert. Der Wert ist eine ganze
  Zahl, welche die Nummer des \element{choice}-Elements angibt. Dazu denke man
  sich die \element{choice}-Kindelemente des umschliessenden
  \element{choices}-Elements von 1 beginnend fortlaufend numeriert.

\end{enumerate}


\section{XML-Elemente}

Root-Element:
\begin{preformatted}%
<element>
  <!-- Content: title answers part -->
</element>
\end{preformatted}

\begin{preformatted}%
<title>
  <!-- Content: TITLE_MISC -->
</title>
\end{preformatted}

\begin{preformatted}%
<part
  subtype="mchoiceproblems">
  <!-- Content: subpart+ -->
</part>
\end{preformatted}

\begin{preformatted}%
<subpart>
  <!-- Content: MISC choices -->
</subpart>
\end{preformatted}

\begin{preformatted}%
<choices
  type="unique|multiple|yesno"
  key="KEY">
  <!-- Content: choice+ commonexpl?-->
</choices>
\end{preformatted}

\begin{preformatted}%
<choice
  key="KEY">
  <!-- Content: assertion solution explanation?-->
</choice>
\end{preformatted}

\begin{preformatted}%
<assertion>
  <!-- Content: MISC -->
</assertion>
\end{preformatted}

\begin{preformatted}%
<solution
  value="true|false">
  <!-- Content: MISC -->
</solution>
\end{preformatted}

\begin{preformatted}%
<explanation>
  <!-- Content: MISC -->
</explanation>
\end{preformatted}

\begin{preformatted}%
<commonexpl>
  <!-- Content: MISC -->
</commonexpl>
\end{preformatted}

Abk�rzungen/Platzhalter:

\begin{description}[code-doc]
  
\item[KEY] Ganze Zahl, verwendet zur Identifizierung von \element{choice}- und
  \element{choices}- Elementen (s. \ref{choices_key}).
  
\item[MISC] Text mit den �blichen Formatierungen (Abs�tze, Listen, Tabellen
  usw.), mathematische Formeln, Multimedia-Objekte (Bilder usw.).
  
\item[TITLE_MISC] Text mit den in Titeln �blichen Formatierungen.

\end{description}


\section{Beispiel}

\begin{preformatted}[code]%
<element xmlns="http://www.mumie.net/xml-namespace/document/content/element">
  <title>Der Titel</title>
  <part subtype="mchoiceproblems">
    <subpart>
      <par>
        Gegeben sei folgende Matrix:
      </par>
      <!--
        die Matrix
      -->
      <par>
        Entscheiden Sie, ob die folgenden Aussagen richtig oder falsch sind:
      </par>
      <choices key="1" type="yesno">
        <choice key="1">
          <assertion>
            Die Matrix ist symmetrisch
          </assertion>
          <solution value="true"/>
          <explanation>
            <!-- Naehere Erklaerung -->
          </explanation>
        </choice>
        <choice key="2">
          <assertion>
            Die Matrix ist eine Dreiecksmatrix
          </assertion>
          <solution value="false"/>
          <explanation>
            <!-- Naehere Erklaerung -->
          </explanation>
        </choice>
      </choices>
    </subpart>
  </part>
</element>
\end{preformatted}


\section{XHTML-Formularelemente}\label{xhtml_formularelemente}

Werden die Multiple-Choice-Fragen auf Client-Seite durch XHTML-Formularelemente
dargestellt, so sollen folgende Regeln gelten:

\begin{enumerate}
  
\item Alle Request-Parameter, die einem Formular-Element zu einer m�glichen
  Antwort entsprechen, haben den gemeinsamen Prefix \val{answer.}.
  
\item \label{choices_unique} '\element{choices}
  \attrib{type}="\val{unique}"'-Elemente werden durch '\element{input}
  \attrib{type}="\val{radio}"'-Elemente dargestellt. F�r jedes
  \element{choice}-Kindelement gibt es genau ein \element{input}-Element.
  Dessen \attrib{name}- und \attrib{value}-Attribute sind gegeben durch
       
  \begin{preformatted}%
    name="answer.unique.CHOICES-KEY"  value="CHOICE-KEY"
  \end{preformatted}

  Hierbei ist \var{CHOICES-KEY} der Wert des \attrib{key}-Attributs des umschliessenden
  \element{choices}-Elements und \var{CHOICE-KEY} der Wert des \attrib{key}-Attributs des
  \element{choice}-Elements. (Es haben also alle \element{input}-Elemente, die zum selben
  \element{choices}-Element geh�ren, dasselbe \attrib{name}-Attribut.)
  
  Beispiel:

  \begin{preformatted}[code]%
    <choices key="5" type="unique">
      <!-- Drei 'choice'-Kindelemente -->
    <choices>
  \end{preformatted}

  entspricht
    
  \begin{preformatted}[code]%
    <input type="radio" name="answer.unique.5" value="1"/>
    <input type="radio" name="answer.unique.5" value="2"/>
    <input type="radio" name="answer.unique.5" value="3"/>
  \end{preformatted}
    
\item '\element{choices} \attrib{type}="\val{multiple}"'-Elemente werden durch
  '\element{input} \attrib{type}="\val{checkbox}"'- Elemente dargestellt. F�r
  jedes \element{choice}-Kindelement gibt es genau ein \element{input}-Element.
  Dessen \attrib{name}- und \attrib{value}-Attribute sind gegeben durch

  \begin{preformatted}%
    name="answer.multiple.CHOICES-KEY.CHOICE-KEY"  value="selected"
  \end{preformatted}

  Hierbei sind \var{CHOICES-KEY} und \var{CHOICE-KEY} wie in \ref{choices_unique} definiert.

  Beispiel:

  \begin{preformatted}[code]%
    <choices key="5" type="multiple">
      <!-- Drei 'choice'-Kindelemente -->
    <choices>
  \end{preformatted}

  entspricht

  \begin{preformatted}[code]%
    <input type="checkbox" name="answer.multiple.5.1" value="selected"/>
    <input type="checkbox" name="answer.multiple.5.2" value="selected"/>
    <input type="checkbox" name="answer.multiple.5.3" value="selected"/>
  \end{preformatted}
  
\item '\element{choices} \attrib{type}="\val{yesno}"'-Elemente werden durch
  Paare von '\element{input} \attrib{type}="\val{radio}"'-Elementen
  dargestellt. F�r jedes \element{choice}-Kindelement gibt es genau zwei
  \element{input}-Elemente, eins f�r die "Ja"-Antwort und eins f�r die
  "Nein"-Antwort. Deren \attrib{name}- und \attrib{value}-Attribute sind
  gegeben durch

  \begin{preformatted}%
    name="answer.yesno.CHOICES-KEY.CHOICE-KEY"  value="yes"    ("Ja"-Antwort)
    name="answer.yesno.CHOICES-KEY.CHOICE-KEY"  value="no"     ("Nein"-Antwort)
  \end{preformatted}
    
  Hierbei sind \var{CHOICES-KEY} und \var{CHOICE-KEY} wie in \ref{choices_unique} definiert.
    
  Beispiel:
    
  \begin{preformatted}[code]%
    <choices key="5" type="yesno">
      <!-- Drei 'choice'-Kindelemente -->
    <choices>
  \end{preformatted}
    
  entspricht
    
  \begin{preformatted}[code]%
    <input type="radio" name="answer.yesno.5.1" value="yes"/>
    <input type="radio" name="answer.yesno.5.1" value="no"/>
    <input type="radio" name="answer.yesno.5.2" value="yes"/>
    <input type="radio" name="answer.yesno.5.2" value="no"/>
    <input type="radio" name="answer.yesno.5.3" value="yes"/>
    <input type="radio" name="answer.yesno.5.3" value="no"/>
  \end{preformatted}

\end{enumerate}



\section{Antwort-XML}

[Das sog. "Antwort-XML" h�lt fest, welche Antworten der Studierende angekreuzt
hat.]

\begin{enumerate}
  
\item Namespace: \val{http://www.mumie.net/xml-namespace/multiple-choice-answers}
  
\item Root-Element: \element{answers}.
  
\item Darin: Beliebig viele (einschliesslich 0) \element{answer}-Kindelemente.
  
\item Jedes \element{answer}-Kindelement entspricht genau einer vom Studierenden
  beantworteten Frage des Multiple-Choice-Tests; umgekehrt muss es zu jeder
  beantworteten Frage ein solches Element geben. (Daraus folgt insbesondere,
  dass es zu nicht beanteworteten Fragen kein \element{answer}-Element gibt.)
  
\item Jedes \element{answer}-Element hat zwei Attribute: \attrib{choices} und
  \attrib{type}. Ersteres enth�lt den Key des entsprechenden
  \element{choices}-Elements (s. \ref{choices_key}), letzteres den Typ (s.
  \ref{choices_type}).
  
\item Jedes \element{answer}-Element hat ein oder mehrere
  \element{selected}-Kindelemente. Jedes entspricht genau einer angekreutzten
  m�glichen Antwort auf die Frage; genauer gilt:
  
\item Im Fall \attrib{type}="\val{unique}": Es gibt genau ein
  \element{selected}-Kindelement. Dieses hat ein \attrib{choice}-Attribut, dass
  den Key des \element{choice}-Elements (s. \ref{choice_key}) der angekreutzten
  Antwort enth�lt.
  
\item Im Fall \attrib{type}="\val{multiple}": Es gibt zu jeder angekreuzten
  Antwort ein \element{selected}-Kindelement. Dieses hat ein
  \attrib{choice}-Attribut, dass den Key des entsprechenden
  \element{choice}-Elements (s. \ref{choice_key}) enth�lt.
  
\item Im Fall \attrib{type}="\val{yesno}": Es gibt zu jeder Antwort, bei der
  "ja" oder "nein" angekreuzt wurde, ein \element{selected}-Kindelement. Dieses
  hat zwei Attribute: \attrib{choice} und \attrib{value}. Ersteres enth�lt den
  Key des entsprechenden \element{choice}-Elements (s. \ref{choice_key}),
  letzteres den Wert "yes" oder "no", je nachdem, ob "ja" oder "nein"
  angekreuzt wurde.
  
\item F�r jede Antwort, die der Studierende angekreuzt hat, muss ein
  \element{answer}-Element vorhanden sein;
  
\item Jedes \element{answer}-Element hat die Form

\end{enumerate}

Beispiel:

\begin{preformatted}[code]%
<answers xmlns="http://www.mumie.net/xml-namespace/multiple-choice-answers">
  <answer choices="1" type="yesno">
    <selected choice="1" value="no"/>
    <selected choice="2" value="yes"/>
    <selected choice="4" value="yes"/>
  </answer>
  <answer choices="2" type="unique">
    <selected choice="2"/>
  </answer>
  <answer choices="3" type="multiple">
    <selected choice="1"/>
    <selected choice="2"/>
    <selected choice="5"/>
  </answer>
</answers>
\end{preformatted}


\section{Daten im 'mumie:dynamic_data'-Element}

Das Metainfo-Element 'mumie:dynamic_data' hat die folgenden Kindelemente:

\begin{enumerate}
  
\item Ein optionales Element 'answers'. Dieses enth�lt das Antwort-XML (s.
  \ref{xhtml_formularelemente}) der letzten Sitzung, falls vorhanden.
  
\item Eine Reihe von Elementen der Form:

  \begin{preformatted}%
    <mumie:param name="NAME" value="VALUE"/>
  \end{preformatted}

  Im einzelnen:

  \begin{enumerate}
  
  \item \var{NAME} = \val{is_course_member}, \var{VALUE} = \val{yes|no}. Gibt
    an, ob der Benutzer, von dem der Request stammt, Teilnehmer des Kurses ist.
  
  \item \var{NAME} = \val{timeframe_relation}, \var{VALUE} =
    \val{before|inside|after}. Gibt an, ob der gegenw�rtige Zeitpunkt vor, in,
    oder nach dem Bearbeitungszeitraum des �bungsblatts liegt.
  
  \item \var{NAME} = \val{stored_answers}, \var{VALUE} = \var{yes|no}. Gibt an,
    ob diese Seite die Response auf einen Upload der Antworten ist. Optional,
    Default ist "no".

  \end{enumerate}

\end{enumerate}



-- ENDE DER DATEI --

\end{document}
