\documentclass{generic}


% ------------------------------------------------------------------------------
% Schriftgroessen
% ------------------------------------------------------------------------------

\renewcommand{\normalsize}{\fontsize{18}{22}\selectfont}
\renewcommand{\large}{\fontsize{20}{26}\selectfont}
\renewcommand{\Large}{\fontsize{22}{30}\selectfont}
\renewcommand{\small}{\fontsize{14}{16}\selectfont}
\newcommand{\smaller}{\fontsize{12}{14}\selectfont}
\newcommand{\verysmall}{\fontsize{6}{6}\selectfont}
\renewcommand{\footnotesize}{\fontsize{12}{14}\selectfont}
\newcommand{\tocsize}{\fontsize{12}{16}\selectfont}
\newcommand{\titlesize}{\fontsize{24}{36}\selectfont}

% ------------------------------------------------------------------------------
% Texthervorhebungen
% ------------------------------------------------------------------------------

%\definecolor{emphcolor}{rgb}{0.77,0.00,0.00}
%\renewcommand{\emph}[1]{\textbf{\textcolor{emphcolor}{#1}}}
\renewcommand{\emph}[1]{\textbf{#1}}

\newcommand{\term}[1]{\textit{#1}}
% \newcommand{\code}[1]{\texttt{#1}}
\newcommand{\var}[1]{\textit{#1}}

%\definecolor{headlinecolor}{rgb}{0.77,0.00, 0.00}
\definecolor{headlinecolor}{rgb}{0.67,0.09, 0.22}
\newcommand{\headline}[1]{\textbf{\textcolor{headlinecolor}{#1}}}

\definecolor{emphboxbordercolor}{rgb}{0.80,0.36,0.36}
\definecolor{emphboxcolor}{rgb}{0.98,0.98,0.77}
\newcommand{\emphbox}[2]{\fcolorbox{emphboxbordercolor}{emphboxcolor}{\parbox{#1}{#2}}}

\definecolor{labelcolor}{rgb}{0.35,0.35, 0.8}


% -------------------------------------------------------------------------------
% Pfeile
% -------------------------------------------------------------------------------

\newcommand{\myarrow}{{\Large\boldmath$\;\rightarrow\;$}}
\newcommand{\mylongarrow}{{\Large\boldmath$\;\longrightarrow\;$}}



% -------------------------------------------------------------------------------
% Befehle f�r Abbildungen 
% -------------------------------------------------------------------------------

\renewcommand{\thefigure}{\thesubsection.\arabic{figure}}

\newcommand{\mycaption}[2]{%
 \centerline{%
  \parbox{#1}{%
   \small \refstepcounter{figure} Figure \thefigure. #2 }}}


%---------------------------------------------------------------------------
% Gliederungsbefehle
%---------------------------------------------------------------------------

\makeatletter

\renewcommand{\section}{\@startsection
 {section}%                                   % Name
 {1}%                                         % Ebene
 {0pt}%                                       % Einzug
 {-0.0\baselineskip}%                        % Vorabstand 
 {0.85\baselineskip}%                         % Nachabstand
 {\fontsize{20}{24}\selectfont\itshape\bfseries}}%    % Stil

\makeatother

\newcommand{\mysection}[1]{%

\vspace*{-1.2\baselineskip}

{\color{headlinecolor}
\section{#1}}}

\newcounter{myappendix}

\newcommand{\myappendix}[2]{%
\hfill\makebox[0cm][l]{\tocref}

\vspace*{-1.2\baselineskip}

\refstepcounter{myappendix}
\section*{A\arabic{myappendix}\hspace{1em}#1}\hypertarget{#2}{}}



% -------------------------------------------------------------------------------
% Inhaltsverzeichnis 
% -------------------------------------------------------------------------------

\newcommand{\mytoctitle}[1]{\begin{center}\textbf{\hypertarget{inhalt}{#1}}\end{center}}

\newenvironment{mytoc}
  {\tocsize\textbf{\hypertarget{inhalt}{Inhalt}}\\\begin{enumerate}}
  {\end{enumerate}}

\newcommand{\mytocitem}[3]{\item[#1]\hyperlink{#3}{#2}}

\newcommand{\tocref}{\verysmall\hyperlink{inhalt}{TOC}\normalsize}

% -------------------------------------------------------------------------------
% Listen 
% -------------------------------------------------------------------------------

%\newcommand{\mylabelitemi}{\labelitemi}
\newcommand{\mylabelitemi}{\raisebox{0.25ex}{{\small $\blacktriangleright$}\hspace{0.1em}}}

%\newcommand{\mylabelitemii}{\labelitemii}
%\newcommand{\mylabelitemii}{\raisebox{0.25ex}{{\small $\triangleright$}\hspace{0.1em}}}
\newcommand{\mylabelitemii}{\raisebox{0.1ex}{\large\bfseries-}\hspace{0.075em}}

\newenvironment{mylist}[1][\mylabelitemi]
  {\begin{list}{{\color{labelcolor}#1}}
    {\setlength{\itemindent}{0cm}
     \setlength{\labelwidth}{1em}
     \setlength{\leftmargin}{0.5cm}
     \setlength{\rightmargin}{1cm}}}
  {\end{list}}

\newenvironment{myenum}
  {\begin{list}{\theenumi}
    {\usecounter{enumi}
     \setlength{\itemindent}{0cm}
     \setlength{\labelwidth}{1em}
     \setlength{\leftmargin}{0.5cm}
     \setlength{\rightmargin}{1cm}}}
  {\end{list}}

\newcommand{\pitem}{\pause\item}
\newcommand{\itemii}{\item[\mylabelitemii]}

\newcounter{romanlistcounter}
\newenvironment{romanlist}{\begin{list}%
 {(\roman{romanlistcounter})\hfill}{\usecounter{romanlistcounter}%
 \topsep0.05\baselineskip% 
 \partopsep0.85\baselineskip%
 \leftmargin2em%
 \labelwidth2em%
 \labelsep0pt%
 \itemindent0pt%
 \parsep0.15\baselineskip%
 \itemsep0.0\baselineskip}}%
 {\end{list}}

\newcounter{alphlistcounter}
\newenvironment{alphlist}{\begin{list}%
 {(\alph{alphlistcounter})\hfill}{\usecounter{alphlistcounter}%
 \topsep0.05\baselineskip% 
 \partopsep0.85\baselineskip%
 \leftmargin1.8em%
 \labelwidth1.8em%
 \labelsep0pt%
 \itemindent0pt%
 \parsep0.15\baselineskip%
 \itemsep0.0\baselineskip}}%
 {\end{list}}

\newcounter{arabiclistcounter}
\newenvironment{arabiclist}{\begin{list}%
 {(\arabic{arabiclistcounter})\hfill}{\usecounter{arabiclistcounter}%
 \topsep0.05\baselineskip% 
 \partopsep0.85\baselineskip%
 \leftmargin1.8em%
 \labelwidth1.8em%
 \labelsep0pt%               
 \itemindent0pt%
 \parsep0.15\baselineskip%
 \itemsep0.0\baselineskip}}%
 {\end{list}}





\begin{document}

\title{Document-Member-Index-XML}

\begin{authors}
  \author[rassy@math.tu-berlin.de]{Tilman Rassy}
\end{authors}

\version{$Id: document_member_index_xml.tex,v 1.2 2006/08/11 16:18:22 rassy Exp $}

\tableofcontents

Ein \emph{Document-Member-Index} listet die Mitglieder eines Dokuments. Dies
ist nur m�glich bei Dokumenttypen, die Mitglieder zulassen, was im Moment nur
f�r den Dokumenttyp \code{course} der Fall ist. Diese Spezifikation beschreibt
das XML, in dem Document-Member-Indexes geschrieben werden.

\section{Grunds�tzliche Struktur}

\begin{enumerate}
  
\item Namensraum ist der des Metainfo-XMLs, also:\\
  \val{http://www.mumie.net/xml-namespace/document/metainfo}.

  �blicher Prefix: \val{mumie}.

  Das Document-Member-Index-XML wird normalerweise mit Prefixes geschrieben.

\item Root-Element: \element{mumie:document_member_index}. Keine Attribute.

\item Darunter: Genau ein \element{mumie:document} und genau ein
  \element{mumie:members}-Kindelement.
  
\item \label{document_element} Das \element{mumie:document}-Element enth�lt
  genau ein Kindelement, und zwar das Dokument, zu dem der Index geh�rt, in
  seiner Standard-XML-Darstellung (einschliesslich Root-Element) in einem
  Use-Mode, der frei gew�hlt werden kann. \label{document_use_mode} Dieser
  Use-Mode heisst \emph{Document-Use-Mode}. Sein Default ist \val{link}.
  
\item \label{members_element} Das \element{mumie:members}-Element enth�lt die
  Eintr�ge des Index in Form von \element{mumie:user}-Kindelementen. Jedes
  solche Element entspricht einem Benutzer in seiner Standard-XML-Darstellung
  in einem Use-Mode, der frei gew�hlt werden kann, aber f�r alle Index-Eintr�ge
  gleich sein muss. \label{member_use_mode} Dieser Use-Mode heisst
  \emph{Member-Use-Mode}. Sein Default ist \val{component}.

\end{enumerate}

\section{Default-Struktur beim Dokumenttyp \code{course}}

Ist der Dokumenttyp \code{course} und haben der Document- und Member-Use-Mode
ihre Default-Werte \val{link} bzw. \val{component} (s. \ref{document_use_mode}
und \ref{member_use_mode}), so sieht der Member-Index wie folgt aus:

\begin{enumerate}

\item Grunds�tzliche Struktur wie oben.

\item Das \element{mumie:document}-Element (s. \ref{document_element}) enth�lt
  ein \element{mumie:course}-Kindelement. Dieses hat keinen Inhalt, aber zwei
  Attribute: \attrib{id}, die Id des Documents, und \attrib{use_mode}, den
  Document-Use-Mode als numerischen Code.

\item Die \element{mumie:user}-Kindelemente des
  \element{mumie:members}-Elements haben ein \attrib{id}-Attribut, das die Id
  des Benutzers enth�lt, und folgende Kindelemente:

  \begin{enumerate}

    \item \element{mumie:login_name}: Enth�lt den Login-Namen des Benutzees als
      Text-Knoten. Keine Attribute, keine Kindelemente.

    \item \element{mumie:first_name}: Enth�lt den Vornamen des Benutzees als
      Text-Knoten. Keine Attribute, keine Kindelemente.

    \item \element{mumie:surname}: Enth�lt den Nachnamen des Benutzees als
      Text-Knoten. Keine Attribute, keine Kindelemente.
  
    \item \element{email}: Enth�lt die E-Mail_Adddresse des Benutzees als
      Text-Knoten. Keine Attribute, keine Kindelemente.
      
    \item \element{mumie:theme}: Gibt das Theme des Benutzers an. Ein Attribut:
      \attrib{id}, die Id des Themes. Keine weiteren Attribute, kein Inhalt.

    \item \element{mumie:role}: Gibt die \emph{User-Role} des Benutzers als
      Mitglied des Kurses an. Zwei Attribute: \attrib{id}, die Id der
      User-Role, und \attrib{name}, der Name der User-Role.

  \end{enumerate}

\end{enumerate}

\section{Beispiel}

\begin{preformatted}[code]%
<mumie:document_member_index
  xmlns:mumie="http://www.mumie.net/xml-namespace/document/metainfo">
  <mumie:document>
    <mumie:course use_mode="2" id="12"/>
  </mumie:document>
  <mumie:members>
    <mumie:user id="41">
      <mumie:login_name>muellerm</mumie:login_name>
      <mumie:first_name>Marion</mumie:first_name>
      <mumie:surname>M\&\#252;ller</mumie:surname>
      <mumie:email>marion_mueller@provider.net</mumie:email>
      <mumie:theme id="0"/>
      <mumie:role id="3" name="attendee"/>
    </mumie:user>
    <mumie:user id="45">
      <mumie:login_name>schmidts</mumie:login_name>
      <mumie:first_name>Stephan</mumie:first_name>
      <mumie:surname>Schmidt</mumie:surname>
      <mumie:email>stephan23@foo.com</mumie:email>
      <mumie:theme id="0"/>
      <mumie:role id="3" name="attendee"/>
    </mumie:user>
  </mumie:members>
</mumie:document_member_index>
\end{preformatted}


-- ENDE DER DATEI --

\end{document}