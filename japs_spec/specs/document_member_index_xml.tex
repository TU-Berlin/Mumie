\documentclass{generic}

% Author: Tilman Rassy <rassy@math.tu-berlin.de>
% $Id: macros.tex,v 1.5 2006/10/05 15:58:42 rassy Exp $

% XML code:
\newcommand{\element}[1]{\code[xml-element]{#1}}
\newcommand{\attrib}[1]{\code[xml-attrib]{#1}}

% Database code:
\newcommand{\dbtable}[1]{\code[db-table]{#1}}
\newcommand{\dbcol}[1]{\code[db-column]{#1}}
\newcommand{\sql}[1]{\code[sql]{#1}}

% TeX code:
\newcommand{\texcmd}[1]{\code[tex-cmd]{\backslash#1}}
\newcommand{\texenv}[1]{\code[tex-env]{#1}}

% General
\newcommand{\val}[1]{\code[value]{#1}}
\newcommand{\prm}[1]{\code[param]{#1}}

% Other:
\newcommand{\notimpl}[1]{\emph[not-impl]{#1}}
\newcommand{\new}[1]{\emph[new]{#1}}
\newcommand{\deprecated}[1]{\emph[deprecated]{#1}}
\newcommand{\wvar}[1]{\var[weak]{#1}}


\begin{document}

\title{Document-Member-Index-XML}

\begin{authors}
  \author[rassy@math.tu-berlin.de]{Tilman Rassy}
\end{authors}

\version{$Id: document_member_index_xml.tex,v 1.2 2006/08/11 16:18:22 rassy Exp $}

\tableofcontents

Ein \emph{Document-Member-Index} listet die Mitglieder eines Dokuments. Dies
ist nur m�glich bei Dokumenttypen, die Mitglieder zulassen, was im Moment nur
f�r den Dokumenttyp \code{course} der Fall ist. Diese Spezifikation beschreibt
das XML, in dem Document-Member-Indexes geschrieben werden.

\section{Grunds�tzliche Struktur}

\begin{enumerate}
  
\item Namensraum ist der des Metainfo-XMLs, also:\\
  \val{http://www.mumie.net/xml-namespace/document/metainfo}.

  �blicher Prefix: \val{mumie}.

  Das Document-Member-Index-XML wird normalerweise mit Prefixes geschrieben.

\item Root-Element: \element{mumie:document_member_index}. Keine Attribute.

\item Darunter: Genau ein \element{mumie:document} und genau ein
  \element{mumie:members}-Kindelement.
  
\item \label{document_element} Das \element{mumie:document}-Element enth�lt
  genau ein Kindelement, und zwar das Dokument, zu dem der Index geh�rt, in
  seiner Standard-XML-Darstellung (einschliesslich Root-Element) in einem
  Use-Mode, der frei gew�hlt werden kann. \label{document_use_mode} Dieser
  Use-Mode heisst \emph{Document-Use-Mode}. Sein Default ist \val{link}.
  
\item \label{members_element} Das \element{mumie:members}-Element enth�lt die
  Eintr�ge des Index in Form von \element{mumie:user}-Kindelementen. Jedes
  solche Element entspricht einem Benutzer in seiner Standard-XML-Darstellung
  in einem Use-Mode, der frei gew�hlt werden kann, aber f�r alle Index-Eintr�ge
  gleich sein muss. \label{member_use_mode} Dieser Use-Mode heisst
  \emph{Member-Use-Mode}. Sein Default ist \val{component}.

\end{enumerate}

\section{Default-Struktur beim Dokumenttyp \code{course}}

Ist der Dokumenttyp \code{course} und haben der Document- und Member-Use-Mode
ihre Default-Werte \val{link} bzw. \val{component} (s. \ref{document_use_mode}
und \ref{member_use_mode}), so sieht der Member-Index wie folgt aus:

\begin{enumerate}

\item Grunds�tzliche Struktur wie oben.

\item Das \element{mumie:document}-Element (s. \ref{document_element}) enth�lt
  ein \element{mumie:course}-Kindelement. Dieses hat keinen Inhalt, aber zwei
  Attribute: \attrib{id}, die Id des Documents, und \attrib{use_mode}, den
  Document-Use-Mode als numerischen Code.

\item Die \element{mumie:user}-Kindelemente des
  \element{mumie:members}-Elements haben ein \attrib{id}-Attribut, das die Id
  des Benutzers enth�lt, und folgende Kindelemente:

  \begin{enumerate}

    \item \element{mumie:login_name}: Enth�lt den Login-Namen des Benutzees als
      Text-Knoten. Keine Attribute, keine Kindelemente.

    \item \element{mumie:first_name}: Enth�lt den Vornamen des Benutzees als
      Text-Knoten. Keine Attribute, keine Kindelemente.

    \item \element{mumie:surname}: Enth�lt den Nachnamen des Benutzees als
      Text-Knoten. Keine Attribute, keine Kindelemente.
  
    \item \element{email}: Enth�lt die E-Mail_Adddresse des Benutzees als
      Text-Knoten. Keine Attribute, keine Kindelemente.
      
    \item \element{mumie:theme}: Gibt das Theme des Benutzers an. Ein Attribut:
      \attrib{id}, die Id des Themes. Keine weiteren Attribute, kein Inhalt.

    \item \element{mumie:role}: Gibt die \emph{User-Role} des Benutzers als
      Mitglied des Kurses an. Zwei Attribute: \attrib{id}, die Id der
      User-Role, und \attrib{name}, der Name der User-Role.

  \end{enumerate}

\end{enumerate}

\section{Beispiel}

\begin{preformatted}[code]%
<mumie:document_member_index
  xmlns:mumie="http://www.mumie.net/xml-namespace/document/metainfo">
  <mumie:document>
    <mumie:course use_mode="2" id="12"/>
  </mumie:document>
  <mumie:members>
    <mumie:user id="41">
      <mumie:login_name>muellerm</mumie:login_name>
      <mumie:first_name>Marion</mumie:first_name>
      <mumie:surname>M\&\#252;ller</mumie:surname>
      <mumie:email>marion_mueller@provider.net</mumie:email>
      <mumie:theme id="0"/>
      <mumie:role id="3" name="attendee"/>
    </mumie:user>
    <mumie:user id="45">
      <mumie:login_name>schmidts</mumie:login_name>
      <mumie:first_name>Stephan</mumie:first_name>
      <mumie:surname>Schmidt</mumie:surname>
      <mumie:email>stephan23@foo.com</mumie:email>
      <mumie:theme id="0"/>
      <mumie:role id="3" name="attendee"/>
    </mumie:user>
  </mumie:members>
</mumie:document_member_index>
\end{preformatted}


-- ENDE DER DATEI --

\end{document}