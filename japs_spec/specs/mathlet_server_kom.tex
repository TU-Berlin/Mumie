\documentclass{generic}

% Author: Tilman Rassy <rassy@math.tu-berlin.de>
% $Id: macros.tex,v 1.5 2006/10/05 15:58:42 rassy Exp $

% XML code:
\newcommand{\element}[1]{\code[xml-element]{#1}}
\newcommand{\attrib}[1]{\code[xml-attrib]{#1}}

% Database code:
\newcommand{\dbtable}[1]{\code[db-table]{#1}}
\newcommand{\dbcol}[1]{\code[db-column]{#1}}
\newcommand{\sql}[1]{\code[sql]{#1}}

% TeX code:
\newcommand{\texcmd}[1]{\code[tex-cmd]{\backslash#1}}
\newcommand{\texenv}[1]{\code[tex-env]{#1}}

% General
\newcommand{\val}[1]{\code[value]{#1}}
\newcommand{\prm}[1]{\code[param]{#1}}

% Other:
\newcommand{\notimpl}[1]{\emph[not-impl]{#1}}
\newcommand{\new}[1]{\emph[new]{#1}}
\newcommand{\deprecated}[1]{\emph[deprecated]{#1}}
\newcommand{\wvar}[1]{\var[weak]{#1}}


\begin{document}

\title{Mathlet-Server-Kommunikation}

\begin{authors}
  \author[rassy@math.tu-berlin.de]{Tilman Rassy}
  \author[gronau@math.tu-berlin.de]{Markus Gronau}
\end{authors}

\version{$Id: mathlet_server_kom.tex,v 1.2 2006/08/11 16:18:22 rassy Exp $}

Dieses Dokument beschreibt die Kommunikation zwischen Mathlets und dem
Japs-Server.

\tableofcontents

\section{Applet-Parameter}

Applet-seitig wird die Kommunikation durch eine Reihe von Parametern
gesteuert:
\begin{enumerate}
\item\label{param_homeworkMode} \prm{homeworkMode} - \val{true} oder \val{false}. Falls
  \val{true}, wird der Hausaufgabenmodus des Applets aktiviert und das Applet in
  diesem Modus gestartet. Default ist \val{false}.

  F�r Applets, mit denen Aufgaben bearbeitet werden sollen, muss also dieser
  Parameter immer gesetzt und gleich \val{true} sein.
\item\label{param_urlPrefix} \prm{urlPrefix} - Der URL-Prefix des Japs, z.B.
  \code{http://japs.mumie.net/cocoon}.
\item\label{param_problemRef} \prm{problemRef} - Eine Datenbank-Id. Bezeichnet
  die Referenz vom �bergeordneten �bungssblatt auf diese Aufgabe. Wird f�r die
  korrekte Zuordnung der Anfragen beim Japs ben�tigt.
\end{enumerate}

Folgende Parameter k�nnen nur im Debug-Modus (s. \ref{param_debug} )benutzt werden:
\begin{enumerate}
\setcounter{enumi}{3}
\item\label{param_debug} \prm{debug} - \val{true} oder \val{false}. Falls
  \val{true}, wird der Debug-Modus des Applets aktiviert. Auf diese Weise
  k�nnen Funktionalit�ten, f�r die normalerweise Serververbindung ben�tigt
  wird, ohne Serververbindung getestet werden. Es wird dann im Applet ein
  "Select"-Button angezeigt.  Default ist \val{false}.
\item\label{param_correctorClass} \prm{correctorClass} - Der vollst�ndige
  Klassenname der Corrector-Klasse in Java-Notation (Packages durch Punkte
  getrennt, keine \code{.java}-Endung), z.B.
  \code{net.mumie.corrector.BasisInR3}.
\item\label{param_debugSheetURL} \prm{debugSheetURL} - Enth�lt die URL zum
  Input-Data-Sheet; setzt die Parameter \ref{param_urlPrefix},
  \ref{param_problemRef} und \ref{param_debugSheetFile} ausser Kraft.
\item\label{param_debugSheetFile} \prm{debugSheetFile} - Enth�lt den Datei-Pfad
  zum Input-Data-Sheet; setzt die Parameter \ref{param_urlPrefix},
  \ref{param_problemRef} und \ref{param_debugSheetURL} ausser Kraft.
\end{enumerate}

\section{Download des Input-Data-Sheets}\label{upload}

Die URL, unter der das Input-Data-Sheet heruntergeladen werden kann, wird aus
dem URL-Prefix (s. \ref{param_urlPrefix}) und dem Pfad

\begin{preformatted}[code]%
  protected/homework/problem-data
\end{preformatted}

gebildet. F�r letzteren steht in Java die statische Konstante

\begin{preformatted}[code]%
  JapsPath.PROBLEM_DATA
\end{preformatted}

zur Verf�gung. Die Klasse \code{JapsPath} befindet sich im Package
\code{net.mumie.japs.datasheet}. Die URL muss mit folgendem Request-Parameter
versehen werden:

\begin{enumerate}
\item\label{req_param_ref} \prm{ref} - Wert des Applet-Parameters
  \prm{problemRef} (s.  \ref{param_problemRef})
\end{enumerate}

F�r den Namen des Request-Parameters steht in Java die statische Konstante

\begin{preformatted}[code]%
  Japs.RequestParam.PROBLEM_REF
\end{preformatted}

zur Verf�gung. Die Klasse \code{JapsRequestParam} befindet sich im Package
\code{net.mumie.japs.datasheet}.

\section{Upload des Antwort-Data-Sheets}

BEMERKUNG: Technisch gesehen handelt es sich nicht um einen Upload im �blichen
Sinne, da der Content-Type nicht "multipart/form-data", sondern
"application/x-www-form-urlencoded" ist, die Daten also wie HTML-Formlulardaten
�bertragen werden. Die Bezeichnung ist aber gerechtfertigt, da ein
Datei-�hnliches Gebilde auf den Server hochgeladen wird.

Die URL, unter der das Antwort-Data-Sheet an den Japs geschickt werden muss,
wird aus dem URL-Prefix (s. \ref{param_urlPrefix}) und dem Pfad

\begin{preformatted}[code]%
  protected/homework/store-problem-answers
\end{preformatted}

gebildet. F�r letzteren steht in Java die statische Konstante

\begin{preformatted}[code]%
  JapsPath.STORE_PROBLEM_ANSWERS
\end{preformatted}

zur Verf�gung. F�r die Klasse \code{JapsPath} s. \ref{upload}. Es m�ssen folgende
Request-Parameter gesetzt werden:

\begin{enumerate}
\item \prm{ref} - S. \ref{req_param_ref}
\item \prm{content} - XML-Code des Data-Sheets
\end{enumerate}

Auch f�r den Namen des Request-Parameters \prm{content} steht in Java eine
statische Konstante zur Verf�gung:

\begin{preformatted}[code]%
  JapsRequestParam.CONTENT
\end{preformatted}

F�r die Klasse \code{JapsRequestParam} s. \ref{upload}.

Die Request-Methode sollte POST und der Content-Type
"application/x-www-form-urlencoded" sein (die Daten werden also wie bei einem
HTML-Formular �bertragen). Den Content-Type setzt der JapsClient automatisch
richtig (korrekte Anwendung vorausgesetzt, vgl. \ref{beispiele}).

Die Response des Servers auf den Upload enth�lt einen zus�tzlichen Header mit
dem Namen

\begin{preformatted}[code]%
  X-Mumie-Status
\end{preformatted}

Bei der Verarbeitung der Response sollte zuerst dieser Header gelesen werden
(nachdem durch Pr�fen des Server-Response-Codes sichergestellt ist, dass kein
Http-Error aufgetreten ist; aber der \code{JapsClient} tut dies automatisch).
Die m�glichen Header-Werte sind:

\begin{enumerate}
\item\label{header_ok_issued_receipt} \val{"OK. Issued receipt"} - Das Data-Sheet
  wurde erfolgreich in der Datenbank gespeichert. Der Japs hat �ber den Empfang
  eine elektronisch signierte Quittung ausgestellt und zur�ckgeschickt.
\item\label{header_ok} \val{"OK"} - Das Data-Sheet wurde erfolgreich in der
  Datenbank gespeichert. Der Japs hat keine Quittung ausgestellt.
\item\label{header_error_ncm} \val{"ERROR: Not a course member"} - Die Daten
  konnten nicht gespeichert werden, da der Benutzer nicht Teilnehmer des Kurses
  ist.
\item\label{header_error_btf} \val{"ERROR: Before timeframe"} - Die Daten konnten
  nicht gespeichert werden, da der Bearbeitungszeitraum noch nicht begonnen
  hat.
\item\label{header_error_atf} \val{"ERROR: After timeframe"} - Die Daten konnten
  nicht gespeichert werden, da der Bearbeitungszeitraum bereits verstrichen
  ist.
%\item\label{header_error_http} \val{"ERROR: Http error: \var{code} \var{descr}"}
%  - Ein Http-Fehler ist aufgetreten. \var{code} ist der Response-Code, \var{descr}
%  eine Beschreibung des Fehlers (BEMERKUNG: Ob ein solcher Fehler aufgetreten
%  ist, kann auch durch direktes Abfragen des Server Response Codes festgestellt
%  werden).
\end{enumerate}

Im ersten Fall (\ref{header_ok_issued_receipt}) enth�lt der Response-Body die
Quittung als Zip-Datei, in allen anderen einen String, der mit dem Header-Wert
identisch ist.

\section{Beispiele (Java-Code)}\label{beispiele}

\subsection{Download des Input-Data-Sheets}

\begin{preformatted}[code]%
  // Initializing japs client:
  JapsClient client = new AppletJapsClient(urlPrefix, this);

  // Providing a document builder:
  DocumentBuilderFactory factory = DocumentBuilderFactory.newInstance();
  factory.setNamespaceAware(true);
  DocumentBuilder documentBuilder = factory.newDocumentBuilder();

  // Loading data sheet:
  DataSheet inputDataSheet = null;
  Map params = new Hashtable();
  params.put(JapsRequestParam.PROBLEM_REF, problemRef);
  HttpURLConnection connection =
    client.get(JapsPath.PROBLEM_DATA, params);
  if ( connection != null )
    \{
      Document document = documentBuilder.parse(connection.getInputStream());
      inputDataSheet = new DataSheet(document);
      System.out.println("Loaded input data sheet from server");
    \}
  else
    System.err.println("User has canceled login dialog");
\end{preformatted}

\subsection{Upload des Antwort-Data-Sheets}

\begin{preformatted}[code]%
  // Initializing japs client:
  JapsClient client = new AppletJapsClient(urlPrefix, this);

  // Sending data sheet:
  Map params = new Hashtable();
  params.put(JapsRequestParam.PROBLEM_REF, problemRef);
  params.put(JapsRequestParam.CONTENT, answerDataSheet.toXMLCode());
  HttpURLConnection connection =
    client.post(JapsPath.STORE_PROBLEM_ANSWERS, params);
  if ( connection != null )
    \{
      System.out.println("Sent answer data sheet to server");

      // Check the X-Mumie-Status header:
      String header = connection.getHeaderField(JapsResponseHeader.STATUS);
      if ( header.equals("OK. Issued receipt") )
        \{
          // Handle that case here
        \}
      else if ( header.equals("OK") )
        \{
          // Handle that case here
        \}
      else if ( header.startsWith("ERROR: ") )
        \{
          // Handle that case here
        \}
      else
        \{
          // Should not happen
        \}
    \}
  else
    System.err.println("User has canceled login dialog");
\end{preformatted}

Um im ersten Fall die Quittung in einer Datei zu speichern, l�sst man sich am besten
vom \code{connection}-Object einen \code{InputStream} geben:

\begin{preformatted}[code]%
  InputStream in = connection.getInputStream();
  FileOutputStream out = new FileOutputStream(filename);
  byte[] buffer = new byte[1024];
  int length;
  while ( (length = in.read(buffer)) != -1 )
    out.write(buffer, 0, length);
\end{preformatted}


-- ENDE DER DATEI --

\end{document}