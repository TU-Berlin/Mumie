\documentclass{generic}

% Author: Tilman Rassy <rassy@math.tu-berlin.de>
% $Id: macros.tex,v 1.5 2006/10/05 15:58:42 rassy Exp $

% XML code:
\newcommand{\element}[1]{\code[xml-element]{#1}}
\newcommand{\attrib}[1]{\code[xml-attrib]{#1}}

% Database code:
\newcommand{\dbtable}[1]{\code[db-table]{#1}}
\newcommand{\dbcol}[1]{\code[db-column]{#1}}
\newcommand{\sql}[1]{\code[sql]{#1}}

% TeX code:
\newcommand{\texcmd}[1]{\code[tex-cmd]{\backslash#1}}
\newcommand{\texenv}[1]{\code[tex-env]{#1}}

% General
\newcommand{\val}[1]{\code[value]{#1}}
\newcommand{\prm}[1]{\code[param]{#1}}

% Other:
\newcommand{\notimpl}[1]{\emph[not-impl]{#1}}
\newcommand{\new}[1]{\emph[new]{#1}}
\newcommand{\deprecated}[1]{\emph[deprecated]{#1}}
\newcommand{\wvar}[1]{\var[weak]{#1}}


\begin{document}

\title{Dokumente}

\begin{authors}
  \author[rassy@math.tu-berlin.de]{Tilman Rassy}
\end{authors}

\version{$Id: dokumente.tex,v 1.1 2008/12/12 13:27:40 rassy Exp $}

\section{\lang{de}{Dokumente und Pseudo-Dokumente}\lang{en}{Documents and pseudo-documents}}

\lang{de}{\emph{Dokumente} sind in Mumie Einheiten mit textuellem oder bin�rem
  Inhalt, z.B. (mathematische) S�tze, Definitionen, Bilder oder Java-Applets.
  Neben Dokumenten gibt es in Mumie auch sogenannte \emph{Pseudo-Dokumente}.
  Das sind Einheiten, die sich formal wie Dokumente behandeln lassen, aber der
  anschaulichen Bedeutung des Begriffes nach keine wirklichen Dokumente sind
  und i.d.R. auch keinen Inhalt haben. Beispiele: Benutzer, Benutzergruppen,
  Semester.  }

\end{document}