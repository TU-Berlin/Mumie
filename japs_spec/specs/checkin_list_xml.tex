\documentclass{generic}


% ------------------------------------------------------------------------------
% Schriftgroessen
% ------------------------------------------------------------------------------

\renewcommand{\normalsize}{\fontsize{18}{22}\selectfont}
\renewcommand{\large}{\fontsize{20}{26}\selectfont}
\renewcommand{\Large}{\fontsize{22}{30}\selectfont}
\renewcommand{\small}{\fontsize{14}{16}\selectfont}
\newcommand{\smaller}{\fontsize{12}{14}\selectfont}
\newcommand{\verysmall}{\fontsize{6}{6}\selectfont}
\renewcommand{\footnotesize}{\fontsize{12}{14}\selectfont}
\newcommand{\tocsize}{\fontsize{12}{16}\selectfont}
\newcommand{\titlesize}{\fontsize{24}{36}\selectfont}

% ------------------------------------------------------------------------------
% Texthervorhebungen
% ------------------------------------------------------------------------------

%\definecolor{emphcolor}{rgb}{0.77,0.00,0.00}
%\renewcommand{\emph}[1]{\textbf{\textcolor{emphcolor}{#1}}}
\renewcommand{\emph}[1]{\textbf{#1}}

\newcommand{\term}[1]{\textit{#1}}
% \newcommand{\code}[1]{\texttt{#1}}
\newcommand{\var}[1]{\textit{#1}}

%\definecolor{headlinecolor}{rgb}{0.77,0.00, 0.00}
\definecolor{headlinecolor}{rgb}{0.67,0.09, 0.22}
\newcommand{\headline}[1]{\textbf{\textcolor{headlinecolor}{#1}}}

\definecolor{emphboxbordercolor}{rgb}{0.80,0.36,0.36}
\definecolor{emphboxcolor}{rgb}{0.98,0.98,0.77}
\newcommand{\emphbox}[2]{\fcolorbox{emphboxbordercolor}{emphboxcolor}{\parbox{#1}{#2}}}

\definecolor{labelcolor}{rgb}{0.35,0.35, 0.8}


% -------------------------------------------------------------------------------
% Pfeile
% -------------------------------------------------------------------------------

\newcommand{\myarrow}{{\Large\boldmath$\;\rightarrow\;$}}
\newcommand{\mylongarrow}{{\Large\boldmath$\;\longrightarrow\;$}}



% -------------------------------------------------------------------------------
% Befehle f�r Abbildungen 
% -------------------------------------------------------------------------------

\renewcommand{\thefigure}{\thesubsection.\arabic{figure}}

\newcommand{\mycaption}[2]{%
 \centerline{%
  \parbox{#1}{%
   \small \refstepcounter{figure} Figure \thefigure. #2 }}}


%---------------------------------------------------------------------------
% Gliederungsbefehle
%---------------------------------------------------------------------------

\makeatletter

\renewcommand{\section}{\@startsection
 {section}%                                   % Name
 {1}%                                         % Ebene
 {0pt}%                                       % Einzug
 {-0.0\baselineskip}%                        % Vorabstand 
 {0.85\baselineskip}%                         % Nachabstand
 {\fontsize{20}{24}\selectfont\itshape\bfseries}}%    % Stil

\makeatother

\newcommand{\mysection}[1]{%

\vspace*{-1.2\baselineskip}

{\color{headlinecolor}
\section{#1}}}

\newcounter{myappendix}

\newcommand{\myappendix}[2]{%
\hfill\makebox[0cm][l]{\tocref}

\vspace*{-1.2\baselineskip}

\refstepcounter{myappendix}
\section*{A\arabic{myappendix}\hspace{1em}#1}\hypertarget{#2}{}}



% -------------------------------------------------------------------------------
% Inhaltsverzeichnis 
% -------------------------------------------------------------------------------

\newcommand{\mytoctitle}[1]{\begin{center}\textbf{\hypertarget{inhalt}{#1}}\end{center}}

\newenvironment{mytoc}
  {\tocsize\textbf{\hypertarget{inhalt}{Inhalt}}\\\begin{enumerate}}
  {\end{enumerate}}

\newcommand{\mytocitem}[3]{\item[#1]\hyperlink{#3}{#2}}

\newcommand{\tocref}{\verysmall\hyperlink{inhalt}{TOC}\normalsize}

% -------------------------------------------------------------------------------
% Listen 
% -------------------------------------------------------------------------------

%\newcommand{\mylabelitemi}{\labelitemi}
\newcommand{\mylabelitemi}{\raisebox{0.25ex}{{\small $\blacktriangleright$}\hspace{0.1em}}}

%\newcommand{\mylabelitemii}{\labelitemii}
%\newcommand{\mylabelitemii}{\raisebox{0.25ex}{{\small $\triangleright$}\hspace{0.1em}}}
\newcommand{\mylabelitemii}{\raisebox{0.1ex}{\large\bfseries-}\hspace{0.075em}}

\newenvironment{mylist}[1][\mylabelitemi]
  {\begin{list}{{\color{labelcolor}#1}}
    {\setlength{\itemindent}{0cm}
     \setlength{\labelwidth}{1em}
     \setlength{\leftmargin}{0.5cm}
     \setlength{\rightmargin}{1cm}}}
  {\end{list}}

\newenvironment{myenum}
  {\begin{list}{\theenumi}
    {\usecounter{enumi}
     \setlength{\itemindent}{0cm}
     \setlength{\labelwidth}{1em}
     \setlength{\leftmargin}{0.5cm}
     \setlength{\rightmargin}{1cm}}}
  {\end{list}}

\newcommand{\pitem}{\pause\item}
\newcommand{\itemii}{\item[\mylabelitemii]}

\newcounter{romanlistcounter}
\newenvironment{romanlist}{\begin{list}%
 {(\roman{romanlistcounter})\hfill}{\usecounter{romanlistcounter}%
 \topsep0.05\baselineskip% 
 \partopsep0.85\baselineskip%
 \leftmargin2em%
 \labelwidth2em%
 \labelsep0pt%
 \itemindent0pt%
 \parsep0.15\baselineskip%
 \itemsep0.0\baselineskip}}%
 {\end{list}}

\newcounter{alphlistcounter}
\newenvironment{alphlist}{\begin{list}%
 {(\alph{alphlistcounter})\hfill}{\usecounter{alphlistcounter}%
 \topsep0.05\baselineskip% 
 \partopsep0.85\baselineskip%
 \leftmargin1.8em%
 \labelwidth1.8em%
 \labelsep0pt%
 \itemindent0pt%
 \parsep0.15\baselineskip%
 \itemsep0.0\baselineskip}}%
 {\end{list}}

\newcounter{arabiclistcounter}
\newenvironment{arabiclist}{\begin{list}%
 {(\arabic{arabiclistcounter})\hfill}{\usecounter{arabiclistcounter}%
 \topsep0.05\baselineskip% 
 \partopsep0.85\baselineskip%
 \leftmargin1.8em%
 \labelwidth1.8em%
 \labelsep0pt%               
 \itemindent0pt%
 \parsep0.15\baselineskip%
 \itemsep0.0\baselineskip}}%
 {\end{list}}




\newcommand{\oci}{Online-Checkin}


\begin{document}

\title{Checkin List XML}

\begin{authors}
  \author[sinha@math.tu-berlin.de]{Uwe Sinha}
\end{authors}

\version{$Id: checkin_list_xml.tex,v 1.2 2006/08/11 16:18:22 rassy Exp $}

Es erscheint sinnvoll, f�r alle Checkin-Aktionen ein einheitliches
R�ckgabeformat zu definieren. Auf diese Weise kann -- unabh�ngig davon,
ob online oder zur Build-Zeit, ob per Web-Interface oder aus einem
MUMIE-Tool (z.B. mmcdk, CourseCreator) heraus, eingecheckt wurde -- die
Information �ber Erfolg oder Mi�erfolg des Checkin-Versuchs von jedem
Tool in gleicher Weise gelesen werden.

\tableofcontents

\section{XML-Struktur}

\subsection{Wurzelelement}

Das Wurzelelement hei�t \element{checkin}. Es kennt keine Attribute. 

\subsection{Kinder des Wurzelelements}

Beim Build-Checkin ist das einzig zul�ssige Kindelement unterhalb des
Wurzelelements das Element \element{documents}. Auch dieses Element
kennt keine Attribute. M�gliche Kindelemente haben jeweils den Namen
eines MUMIE-Dokumenttypen, also \element{element}, \element{image},
\element{generic_page} o.�.  (siehe 1.3).

Beim Online-Checkin kann zus�tzlich noch ein Kindelement \element{error}
vorhanden sein (siehe 1.4). Es dient zur Herausstellung von
Fehlermeldungen des JAPS. Auch dieses Element kennt keine Attribute.


\subsection{Hochgeladene Dokumente}

Als Elementnamen sind hier s�mtliche Namen von MUMIE-Dokumenttypen
zugelassen, also z.B. \element{element}, \element{image} usw. Ebenfalls
zugelassen sind die Namen generischer Dokumenttypen,
z.B. \element{generic_image}, sowie \element{course_section}.

Ggf. k�nnen die Elementnamen mit dem Namespace \val{mumie:} versehen sein
(vgl. 2.3).


\subsubsection{Attribute}

Die in 1.3 definierten Elemente kennen als einzige -- abgesehen von
ihren Kindelementen, vgl. 1.3.2 -- Elemente des Checkin List XML
Attribute. Zul�ssig sind die Attribute \attrib{id} und \attrib{url}.

\begin{itemize}
\item \attrib{id}: der Wert dieses Attributs ist die Datenbank-ID, die das
  zugeh�rige MUMIE-Dokument beim Checkin erhalten hat. Ein Wert von -1,
  zeigt zus�tzlich zum \element{error}-Element an, da� der Checkin
  fehlgeschlagen ist. 

\item \attrib{url}: der Wert dieses Attributs ist eine Pfad innerhalb eines
  Filesystems, von dem aus das Dokument eingecheckt wurde (bzw., im
  Falle eines Fehlschlags: werden sollte). Dieser Pfad bezeichnet immer
  die Master-Datei des Dokuments, die die Metainformationen enth�lt, und
  deren Name auf \code{.meta.xml} endet (vgl. 1.3.2). Der Pfad ist beim
  Build-Checkin relativ zur Wurzel des Checkin-Verzeichnisses, beim
  Online-Checkin relativ zur Wurzel des Filesystems innerhalb des
  hochgeladenen ZIP-Archivs.  
\end{itemize}

\subsubsection{Kindelemente}

Als Kindelemente eines der in 1.3 definierten "Dokument-Elemente" sind
\element{content} und \element{source} zul�ssig. 

\element{content} bezeichnet die Datei, aus der der Inhalt des Dokuments
eingelesen wurde. Zu jedem einzucheckenden MUMIE-Dokument gibt es
maximal eine Inhaltsdatei, daher darf dieses Element auch nur h�chstens
einmal vorkommen. 

\element{source} bezeichnet eine Datei, aus der die im \element{content}-Element
aufgef�hrte Datei erzeugt worden ist. Dies kann z.B. die TeX-Quelle
eines MUMIE-Elements sein, die Java-Quelle eines Applets, o.�. Da
Dokumente aus mehreren Quellen zusammengesetzt sein k�nnen, darf das
\element{source}-Element mehrfach auftreten. Dies wird v.a. bei Applets
und/oder JAR-Archiven sehr h�ufig der Fall sein.

\element{content} und \element{source} kennen beide das Attribut
\attrib{url}, als dessen Wert der Pfad zur Inhaltsdatei (bei
\element{content}) bzw. zur Quelldatei (bei \element{source}) angegeben
wird. F�r den Pfad gilt dasselbe, wie unter 1.3.1 gesagt.


\subsection{Fehlermeldungen beim Online-Checkin}

Das \element{error}-Element hat mindestens zwei und h�chstens drei Kinder:

\begin{itemize}
\item \element{exception-class}: Das hierunter befindliche Text-Element bezeichnet
  die Java-Klasse, zu der die aufgetretene Exception geh�rt; also
  \code{java.lang.Exception} bzw. eine Unterklasse davon. 

\item \element{message}: Das hierunter befindliche Text-Element enth�lt die in
  der Exception enthaltene Fehlermeldung. Dies ist genau der String, den
  man durch Aufruf der Methode \code{getMessage()} der Exception erh�lt. 

\item \element{explanation}: Eine zus�tzliche Erkl�rung der Umst�nde, unter denen
  der Fehler aufgetreten ist. Dieses Element ist optional.  
\end{itemize}

\section{Sonstiges}
%
\subsection{\code{CheckInGenerator} zur Ausgabe einer Checkin List bewegen}
%
Der \code{CheckInGenerator} kennt ein Sitemap-Parameter
\val{respond-checkin-list}; setzt man es auf den Wert \val{true}, so
wird eine Checkin List, wie in dieser Spezifikation beschrieben,
ausgegeben. Setzt man es auf \val{false} (oder l��t es ganz weg), so
wird das "propriet�re" XML, wie Thomas D�hring es Mitte(?) 2003
definiert hat, ausgegeben.

\subsection{Beispiele}
%
\subsubsection{\oci}
%
Das folgende Beispiel illustriert die XML-Antwort auf einen Checkin-Versuch, der
wegen fehlender Berechtigung zum Anlegen neuer Bilder fehlgeschlagen
ist. Bei einem erfolgreichen Versuch h�tten die \attrib{id}-Attribute einen
Wert gr��er als -1.

\begin{preformatted}
<checkin>
  <documents>
    <image id="-1" url="fritz.jpg.meta.xml">
      <content url="fritz.jpg.content"/>
    </image>
    <page id="-1" url="page_ref_fritz.meta.xml">
      <content url="page_ref_fritz.content"/>
    </page>
  </documents>
  <error>
    <exception-class>
      net.mumie.cocoon.documents.checkin.CheckInException
    </exception-class>
    <message>
      The files you uploaded have not been checked in. There was an
      exception: No permission to check in a new document of type "image"
    </message>
    <explanation>
      No permission to check in a new document of type "image"
    </explanation>
  </error>
</checkin>
\end{preformatted}

\subsubsection{Build-Checkin} 


Als Beispiel empfiehlt es sich, im checkin-Verzeichnis die Datei
checkin.xml anzusehen. Letztendlich sieht eine vom Build-Checkin
erzeugte Checkin-Liste strukturell nicht anders aus als das Beispiel aus
Abschnitt 2.1.1 ohne das \element{error}-Element und dessen Kinder.


\subsection{Namespace}

Wegen der gro�en �hnlichkeit zum Metainfo-XML kann derselbe Namespace
benutzt werden -- also \val{mumie:}. Tats�chlich wird bisher noch kein
Namespace benutzt, dies kann aber ggf. noch eingebaut werden. 

--- ENDE DER DATEI ---
\end{document}
