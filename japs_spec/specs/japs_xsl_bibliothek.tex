\documentclass{generic}

% Author: Tilman Rassy <rassy@math.tu-berlin.de>
% $Id: macros.tex,v 1.5 2006/10/05 15:58:42 rassy Exp $

% XML code:
\newcommand{\element}[1]{\code[xml-element]{#1}}
\newcommand{\attrib}[1]{\code[xml-attrib]{#1}}

% Database code:
\newcommand{\dbtable}[1]{\code[db-table]{#1}}
\newcommand{\dbcol}[1]{\code[db-column]{#1}}
\newcommand{\sql}[1]{\code[sql]{#1}}

% TeX code:
\newcommand{\texcmd}[1]{\code[tex-cmd]{\backslash#1}}
\newcommand{\texenv}[1]{\code[tex-env]{#1}}

% General
\newcommand{\val}[1]{\code[value]{#1}}
\newcommand{\prm}[1]{\code[param]{#1}}

% Other:
\newcommand{\notimpl}[1]{\emph[not-impl]{#1}}
\newcommand{\new}[1]{\emph[new]{#1}}
\newcommand{\deprecated}[1]{\emph[deprecated]{#1}}
\newcommand{\wvar}[1]{\var[weak]{#1}}


\begin{document}

\title{Japs-XSL-Bibliothek}

\begin{authors}
  \author[rassy@math.tu-berlin.de]{Tilman Rassy}
\end{authors}

\version{$Id: japs_xsl_bibliothek.tex,v 1.8 2006/08/11 16:18:22 rassy Exp $}

\tableofcontents

Die \emph{Japs-XSL-Bibliothek} enth�lt XSL-Templates, -Parameter und
-Variablen, die von beinahe allen XSL-Stylesheets gebraucht werden. Die
bereitgestellten Funktionalit�ten betreffen vorwiegend das Aufl�sen von
Binnen-Ids (LIDs) und die Konstruktion von URLs.

\section{Einbindung}

Soll die Japs-XSL-Bibliothek in einem Stylesheet verwendet werden, so muss in
dessen Content das Element
\begin{preformatted}%
  <mumie:insert-japs-xsl-lib/>
\end{preformatted}
stehen. Dieses Element wird beim �bersetzen des Stylesheets (Transformation
XSL+Mumie-Erg�nzungs-Elemente $\longrightarrow$ XSL durch \file{xsl_online.xsl}
bzw. \file{xsl_mmcdk.xsl}) durch die Templates, Parameter und Variablen der
Japs-XSL-Bibliothek ersetzt.

\section{Parameter}

\subsection{xsl.url-prefix}\label{xsl_url_prefix}

Der Prefix f�r alle Japs-URLs. Hat normalerweise die Form
\begin{preformatted}%
  http://\var{host}\optional{:\var{port}}/cocoon
\end{preformatted}
Der Wert wird zur Build-Zeit festgelegt und kann mit einer Ant-Property
eingestellt werden.

\subsection{xsl.url-prefix.cocoon-internal}

Der Prefix des Cocoon-Pseudoprotokolls f�r URLs relativ zum
Cocoon-Basisverzeichnis. Der Wert ist
\begin{preformatted}%
  cocoon:
\end{preformatted}
Er wird in \file{xsl_online.src.xsl} gesetzt.

\section{Templates}

\subsection{xsl.document-url}

Erzeugt die URL eines Dokuments. Parameter:
\begin{description}[code-doc]
  \item[id] Die ID des Dokuments. Default ist das \attrib{id}-Attribut des
    aktuellen Elements.
  \item[qualified-name] Der Qualified Name des Dokuments. Default ist das
    \attrib{qualified-name}-Attribut des aktuellen Elements. \emph{Deprecated:
      Qualified Names identifizieren Dokumente nicht notwendigerweise
      eindeutig. Dieser Parameter wird daher in Zukunft abgeschafft.}
  \item[document-type.name] Der Typ des Dokuments, als Name. Default ist der
    Local Name des aktuellen Elements.
  \item[context] Der Kontext (vgl. \href{url_design.txt}{url_design}). Default ist das
    \attrib{context}-Attribut des aktuellen Elements, falls ein solches Attribut
    existiert, sonst \val{"view"}.
  \item[internal] Falls \val{"yes"}, beginnt die URL mit dem
    \code{cocoon:}-Pseudoprotokoll anstatt mit
    \href{\#xsl_url_prefix}{xsl.url-prefix}. Default ist das
    \attrib{internal}-Attribut des aktuellen Elements, oder \val{"no"} wenn
    dieses Attribut nicht existiert.
\end{description}

\subsection{xsl.error}

Beendet die Transformation mit einer Fehlermeldung. Parameter:
\begin{description}[code-doc]
  \item[message] Der Wortlaut der Fehlermeldung. 
\end{description}

\subsection{xsl.get-param-from-dynamic-data}

Sucht einen Parameter im \element{mumie:dynamic_data}-Abschnitt und gibt dessen
Wert zur�ck. Wird der Parameter nicht gefunden, ist der R�ckgabewert
leer.

Parameter (des Templates):

\begin{description}[code-doc]
  \item[name] Der Name des zu suchenden Parameters.
  \item[default] Ein Default-Wert, falls der Parameter nicht gefunden wird.
\end{description}

\subsection{xsl.resolve-lid}\label{xsl_resolve_lid}

L�st eine Binnen-ID (LID) auf und gibt die URL des entsprechenden Dokuments
zur�ck. Parameter:
\begin{description}[code-doc]
  \item[lid] Die Binnen-ID. Default ist das \attrib{lid}-Attribut des aktuellen
    Elements.
  \item[context] Der Kontext (vgl \href{url_design.txt}{url_design}). Default ist das
    \attrib{context}-Attribut des aktuellen Elements, falls ein solches Attribut
    existiert, sonst \val{"view"}.
  \item[internal] Falls \val{"yes"}, beginnt die URL mit dem
    \code{cocoon:}-Pseudoprotokoll anstatt mit
    \href{\#xsl_url_prefix}{xsl.url-prefix}. Default ist das
    \attrib{internal}-Attribut des aktuellen Elements, oder \val{"no"} wenn
    dieses Attribut nicht existiert.
  \item[parameters] Request-Parameter, als String in URL-Syntax, also:
    \val{?\var{KEY1}=\var{VALUE1}\&\var{KEY2}=\var{VALUE2}...} Per Default leer.
\end{description}

\subsection{xsl.set-applet-object-src-attributes}

Spezielles Template zum Einbinden von Applets mit dem
(X)HTML-\element{object}-Element. Setzt die Attribute \attrib{type},
\attrib{classid}, \attrib{archive}, \attrib{width} und \attrib{height}.
Parameter:

\begin{description}[code-doc]
  \item[lid] Die Binnen-ID (LID) des Applets (Dokumenttyp:
    \code{applet}). Default: \attrib{lid}-Attribut des aktuellen Elements.
\end{description}

Die Werte der Attribute werden im Element mit dem XPath
\begin{preformatted}%
  /*/mumie:components/mumie:applet[@lid=\$lid]
\end{preformatted}
nachgeschlagen (es sollte nur ein solches Element geben). Derzeit bekommt das
\attrib{type}-Attribut den Wert
\begin{preformatted}%
  application/x-java-applet
\end{preformatted}
und das \attrib{classid}-Attribut den Wert
\begin{preformatted}%
  java:\var{qualified_name}
\end{preformatted}
wobei \var{qualified_name} f�r den Fully Qualified Name der Applet-Klasse
steht. Das ist insbesondere f�r den Mozilla geeignet.

\subsection{xsl.set-applet-src-attributes}

Spezielles Templates f�r (X)HTML-\element{applet}-Elemente. Setzt die
Attribute \attrib{code}, \attrib{codebase}, \attrib{archive}, \attrib{width} und
\attrib{height}. Parameter:

\begin{description}[code-doc]
  \item[lid] Die Binnen-ID (LID) des Applets (Dokumenttyp:
    \code{applet}). Default: \attrib{lid}-Attribut des aktuellen Elements.
\end{description}

Die Werte der Attribute werden im Element mit dem XPath
\begin{preformatted}%
  /*/mumie:components/mumie:applet[@lid=\$lid]
\end{preformatted}
nachgeschlagen (es sollte nur ein solches Element geben).

\subsection{xsl.set-image-src-attributes}

Spezielles Template f�r (X)HTML-\element{img}-Elemente (u.�.). Setzt die
Attribute \attrib{src}, \attrib{width}, und \attrib{height}. Parameter:

\begin{description}[code-doc]
  \item[request-param-element-name] Local Name der Kindelemente des aktuellen Elements, die
    Request-Parameter angeben (s.u.). Default: \element{request_param}
\end{description}

Die Werte der drei Attribute werden wie folgt ermittelt:
Hat das aktuelle Element ein \attrib{lid}-Attribut, so wird f�r \attrib{src} das Template
\link{\#xsl_resolve_lid}{xsl.resolve-lid} aufgerufen und \attrib{width} und
\attrib{height} werden im Element mit dem XPath
\begin{preformatted}%
  /*/mumie:components/mumie:image[@lid=current()/@lid]
\end{preformatted}
nachgeschlagen (es sollte h�chstens ein solches Element geben). Hat das
aktuelle Element ein \attrib{params_prefix}-Attribut, so wird f�r \attrib{src} das Template
\link{\#xsl_url_from_params}{xsl.url-from-params} aufgerufen  und \attrib{width} und
\attrib{height} werden aus den Parametern 
\begin{preformatted}%
  \var{params_prefix}.width  \box{bzw.}  \var{params_prefix}.height
\end{preformatted}
ausgelesen, dabei ist \var{params_prefix} der Wert von \attrib{params_prefix}.

Der Wert des \attrib{src}-Attributs kann mit Request-Parametern versehen
werden. Dazu wird das Template
\link{\#xsl_set_request_param}{xsl.set-request-param} auf alle diejenigen
Kindelemente des aktuellen Elements angewandt, deren Local Name gleich dem Wert
von \code{request-param-element-name} ist.

L�sst sich der Werte von \attrib{width} bzw. \attrib{height} nicht
ermitteln, wird das Atribut nicht gesetzt.


\subsection{xsl.set-ref-attribute}

Setzt ein Hyperlink-Attribut. Parameter:

\begin{description}[code-doc]
  \item[name] Name des Attributs. Default: \attrib{href}.
  \item[default] Default-Wert.
  \item[request-param-element-name] Local Name der Kindelemente des aktuellen Elements, die
    Request-Parameter angeben (s.u.). Default: \element{request_param}
  \item[use-params-if-no-lid] Falls \val{yes}, wird die URL beim Fehlen eines
    \attrib{lid}-Attributs aus Dokument-Parametern gebildet, sofern ein
    \attrib{params_prefix}-Attribut existiert (s.u.). Default: \val{no}.
\end{description}

Der Wert des Attributs ergibt sich wie folgt:
\begin{itemize}
\item Hat das aktuelle Element ein \attrib{lid}-Attribut, so ist der Wert die
  URL des Dokuments, das dieser Binnen-Id entspricht. Konstruiert wird dieser
  Wert durch das Template \link{\#xsl_resolve_lid}{xsl.resolve-lid}.
\item Hat das aktuelle Element kein \attrib{lid}-Attribut, aber ein
  \attrib{params_prefix}-Attribut und ist \code{use-params-if-no-lid} gleich
  \val{yes}, so wird der Wert aus Parametern des Dynamic-Data-Abschnitts
  gebildet. Dazu wird das Template
  \link{\#xsl_url_from_params}{xsl.url-from-params} aufgerufen.
\item Andernfalls wird der Default-Wert eingesetzt.
\end{itemize}

In allen F�llen wird anschlie�end das Template
\link{\#xsl_set_request_param}{xsl.set-request-param} auf alle diejenigen
Kindelemente des aktuellen Elements angewandt, deren Local Name gleich dem Wert
von \code{request-param-element-name} ist. Dies ergibt den String der
Request-Parameter.


\subsection{xsl.set-request-param}\label{xsl_set_request_param}

Konstruiert einen String, der einen Request-Parameter in einer URL
setzt. Parameter:

\begin{description}[code-doc]
  \item[name] Name des Request-Parameters. Default: \attrib{name}-Attribut des
    aktuellen Elements.
  \item[value] Wert des Request-Parameters. Der Default ergibt sich wie folgt:

    Hat das aktuelle Element ein \attrib{value}-Attribut, so ist der
    Default der Wert dieses Attributs.
      
    Hat das aktuelle Element ein \attrib{value_of_param}-Attribut, so
    ist der Default der Wert desjenigen Parameters im
    \element{mumie:dynamic_data}-Abschnitt, dessen Name gleich dem Wert des
    \attrib{value_of_param}-Attributs ist.
    
    Hat das aktuelle Element ein \attrib{value_of_lid}-Attribut, so wird das
    \link{\#xsl_resolve_lid}{xsl.resolve-lid}-Template aufgerufen, wobei dessen
    \code{lid}-Parameter gleich dem Wert des \attrib{value_of_lid}-Attributs
    gesetzt wird. Der R�ckgabewert ist der Default.

    Sind mehrere dieser Attribute vorhanden, so werden diese mit den
    Priorit�ten \attrib{value} > \attrib{value_of_param} >
    \attrib{value_of_lid} ausgewertet.

  \item[separator]
    Das Zeichen, mit dem dieser Request-Parameter von seinem Vorg�nger bzw. dem
    Teil der URL vor den Request-Parametern getrennt werden soll. Default:
    \code{?}, wenn die XPath-Funktion \code{position()} 1 ergibt, sonst
    \code{\&}.
\end{description}

\subsection{xsl.url-from-params}\label{xsl_url_from_params}

Erzeugt eine Dokument-URL aus Parametern aus dem
\element{mumie:dynamic_data}-Abschnitt. Parameter (des Templates):
\begin{description}[code-doc]
  \item[prefix] Der Prefix der Parameter, die zur Erzeugung der URL
    herangezogen werden.
\end{description}
Die Daten f�r die URL bezieht das Template aus folgenden Parametern des
\element{mumie:dynamic_data}-Abschnitts:

\begin{table}
  \head
    Name & Beschreibung & Zwingend \\
  \body
    \var{prefix}\code{.id} & Die Id des Dokuments & Ja \\
    \var{prefix}\code{.type-name} & Der Dokumenttyp, als Name & Ja \\
    \var{prefix}\code{.context} & Der Kontext der URL & Nein. Default ist
      \val{"view"} \\
    \var{prefix}\code{.internal} & Falls \val{"yes"}, beginnt die URL mit dem
      \code{cocoon:}-Pseudoprotokoll anstatt mit
      \href{\#xsl_url_prefix}{xsl.url-prefix} & Nein. Default ist \val{"no"}
\end{table}

\subsection{xsl.url-of-self}

Gibt die URL des gerade transformierten Dokuments zur�ck. Parameter:
\begin{description}[code-doc]
  \item[context] Der Kontext (vgl \href{url_design.txt}{url_design}). Default ist das
    \attrib{context}-Attribut des aktuellen Elements, falls ein solches Attribut
    existiert, sonst \val{"view"}.
  \item[internal] Falls \val{"yes"}, beginnt die URL mit dem
    \code{cocoon:}-Pseudoprotokoll anstatt mit
    \href{\#xsl_url_prefix}{xsl.url-prefix}. Default ist das
    \attrib{internal}-Attribut des aktuellen Elements, oder \val{"no"} wenn
    dieses Attribut nicht existiert.
\end{description}

-- ENDE DER DATEI --

\end{document}