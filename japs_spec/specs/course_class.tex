\documentclass{generic}

%
% ------------------------------------------------------------------------------
% Schriftgroessen
% ------------------------------------------------------------------------------

\renewcommand{\normalsize}{\fontsize{18}{22}\selectfont}
\renewcommand{\large}{\fontsize{20}{26}\selectfont}
\renewcommand{\Large}{\fontsize{22}{30}\selectfont}
\renewcommand{\small}{\fontsize{14}{16}\selectfont}
\newcommand{\smaller}{\fontsize{12}{14}\selectfont}
\newcommand{\verysmall}{\fontsize{6}{6}\selectfont}
\renewcommand{\footnotesize}{\fontsize{12}{14}\selectfont}
\newcommand{\tocsize}{\fontsize{12}{16}\selectfont}
\newcommand{\titlesize}{\fontsize{24}{36}\selectfont}

% ------------------------------------------------------------------------------
% Texthervorhebungen
% ------------------------------------------------------------------------------

%\definecolor{emphcolor}{rgb}{0.77,0.00,0.00}
%\renewcommand{\emph}[1]{\textbf{\textcolor{emphcolor}{#1}}}
\renewcommand{\emph}[1]{\textbf{#1}}

\newcommand{\term}[1]{\textit{#1}}
% \newcommand{\code}[1]{\texttt{#1}}
\newcommand{\var}[1]{\textit{#1}}

%\definecolor{headlinecolor}{rgb}{0.77,0.00, 0.00}
\definecolor{headlinecolor}{rgb}{0.67,0.09, 0.22}
\newcommand{\headline}[1]{\textbf{\textcolor{headlinecolor}{#1}}}

\definecolor{emphboxbordercolor}{rgb}{0.80,0.36,0.36}
\definecolor{emphboxcolor}{rgb}{0.98,0.98,0.77}
\newcommand{\emphbox}[2]{\fcolorbox{emphboxbordercolor}{emphboxcolor}{\parbox{#1}{#2}}}

\definecolor{labelcolor}{rgb}{0.35,0.35, 0.8}


% -------------------------------------------------------------------------------
% Pfeile
% -------------------------------------------------------------------------------

\newcommand{\myarrow}{{\Large\boldmath$\;\rightarrow\;$}}
\newcommand{\mylongarrow}{{\Large\boldmath$\;\longrightarrow\;$}}



% -------------------------------------------------------------------------------
% Befehle f�r Abbildungen 
% -------------------------------------------------------------------------------

\renewcommand{\thefigure}{\thesubsection.\arabic{figure}}

\newcommand{\mycaption}[2]{%
 \centerline{%
  \parbox{#1}{%
   \small \refstepcounter{figure} Figure \thefigure. #2 }}}


%---------------------------------------------------------------------------
% Gliederungsbefehle
%---------------------------------------------------------------------------

\makeatletter

\renewcommand{\section}{\@startsection
 {section}%                                   % Name
 {1}%                                         % Ebene
 {0pt}%                                       % Einzug
 {-0.0\baselineskip}%                        % Vorabstand 
 {0.85\baselineskip}%                         % Nachabstand
 {\fontsize{20}{24}\selectfont\itshape\bfseries}}%    % Stil

\makeatother

\newcommand{\mysection}[1]{%

\vspace*{-1.2\baselineskip}

{\color{headlinecolor}
\section{#1}}}

\newcounter{myappendix}

\newcommand{\myappendix}[2]{%
\hfill\makebox[0cm][l]{\tocref}

\vspace*{-1.2\baselineskip}

\refstepcounter{myappendix}
\section*{A\arabic{myappendix}\hspace{1em}#1}\hypertarget{#2}{}}



% -------------------------------------------------------------------------------
% Inhaltsverzeichnis 
% -------------------------------------------------------------------------------

\newcommand{\mytoctitle}[1]{\begin{center}\textbf{\hypertarget{inhalt}{#1}}\end{center}}

\newenvironment{mytoc}
  {\tocsize\textbf{\hypertarget{inhalt}{Inhalt}}\\\begin{enumerate}}
  {\end{enumerate}}

\newcommand{\mytocitem}[3]{\item[#1]\hyperlink{#3}{#2}}

\newcommand{\tocref}{\verysmall\hyperlink{inhalt}{TOC}\normalsize}

% -------------------------------------------------------------------------------
% Listen 
% -------------------------------------------------------------------------------

%\newcommand{\mylabelitemi}{\labelitemi}
\newcommand{\mylabelitemi}{\raisebox{0.25ex}{{\small $\blacktriangleright$}\hspace{0.1em}}}

%\newcommand{\mylabelitemii}{\labelitemii}
%\newcommand{\mylabelitemii}{\raisebox{0.25ex}{{\small $\triangleright$}\hspace{0.1em}}}
\newcommand{\mylabelitemii}{\raisebox{0.1ex}{\large\bfseries-}\hspace{0.075em}}

\newenvironment{mylist}[1][\mylabelitemi]
  {\begin{list}{{\color{labelcolor}#1}}
    {\setlength{\itemindent}{0cm}
     \setlength{\labelwidth}{1em}
     \setlength{\leftmargin}{0.5cm}
     \setlength{\rightmargin}{1cm}}}
  {\end{list}}

\newenvironment{myenum}
  {\begin{list}{\theenumi}
    {\usecounter{enumi}
     \setlength{\itemindent}{0cm}
     \setlength{\labelwidth}{1em}
     \setlength{\leftmargin}{0.5cm}
     \setlength{\rightmargin}{1cm}}}
  {\end{list}}

\newcommand{\pitem}{\pause\item}
\newcommand{\itemii}{\item[\mylabelitemii]}

\newcounter{romanlistcounter}
\newenvironment{romanlist}{\begin{list}%
 {(\roman{romanlistcounter})\hfill}{\usecounter{romanlistcounter}%
 \topsep0.05\baselineskip% 
 \partopsep0.85\baselineskip%
 \leftmargin2em%
 \labelwidth2em%
 \labelsep0pt%
 \itemindent0pt%
 \parsep0.15\baselineskip%
 \itemsep0.0\baselineskip}}%
 {\end{list}}

\newcounter{alphlistcounter}
\newenvironment{alphlist}{\begin{list}%
 {(\alph{alphlistcounter})\hfill}{\usecounter{alphlistcounter}%
 \topsep0.05\baselineskip% 
 \partopsep0.85\baselineskip%
 \leftmargin1.8em%
 \labelwidth1.8em%
 \labelsep0pt%
 \itemindent0pt%
 \parsep0.15\baselineskip%
 \itemsep0.0\baselineskip}}%
 {\end{list}}

\newcounter{arabiclistcounter}
\newenvironment{arabiclist}{\begin{list}%
 {(\arabic{arabiclistcounter})\hfill}{\usecounter{arabiclistcounter}%
 \topsep0.05\baselineskip% 
 \partopsep0.85\baselineskip%
 \leftmargin1.8em%
 \labelwidth1.8em%
 \labelsep0pt%               
 \itemindent0pt%
 \parsep0.15\baselineskip%
 \itemsep0.0\baselineskip}}%
 {\end{list}}





\begin{document}

\title{Repository der Kursdokumente}

\begin{authors}
  \author[lehmannf@math.tu-berlin.de]{Fritz Lehmann-Grube}
\end{authors}
\version{$Id: course_class.tex,v 1.1 2007/09/04 17:18:01 lehmannf Exp $}

\tableofcontents

\section{Allgemeines}

Nachdem eine Mumie installiert und die Inhaltsdokumente (z.B. aus dem CVS Repository
  "linear_algebra_content", ehemals "japs_content") eingecheckt sind, sind zwei Dinge zu
  tun, um einen Kurs durchf�hren zu k�nnen:
\begin{itemize}
\item Die Lehrveranstaltung (Semester, Tutorien) installieren. Siehe dazu die
Spezifikation (\href{moses_sync.xhtml}{"Synchronisation mit MOSES"}). Dieser
Schritt kann ausserdem vorbereitet sein durch ein sogenanntes
\href{\#sync_tarball}{"Sync Tarball"}
\item Die Kursdokumente installieren
\end{itemize}
Beides einzeln ist noch nicht besonders gut beschrieben, jedoch f�r den nicht mehr ganz
  blutigen JAPS-Anf�nger Heinz unproblematisch.
Leider sind die beiden Punkte nicht ganz unabh�ngig.

Es ist wichtig, folgendes zu verstehen:

\subsection{Vom Repository zum Checkin} Die CVS Repositorys sind nicht identisch mit dem Checkin-Verzeichnis, sondern
 werden da "hinein ge-mountet".
 %Das sollte man nochmal genauer erkl�ren, aber erstmal gen�gt es vielleicht, dass
 %du, Heinz, das wohl schon kennst.

\subsection{Inhalt des Repositorys} Das CVS Repository "linear_algebra_courses" enth�lt genau die Dokumente
  fuer den Kurs und seine Komponenten, also die Kursabschnitte und Arbeitsbl�tter
  sowie die Summarys.
\subsection{"Sync" Tarball} Dieses CVS Repository "linear_algebra_courses" enthaelt aber _nicht_ die Pseudodokumente zur Lehrveranstaltung,
  Semester etc., denn diese sollen ja gar nicht als Masterdateien (*.meta.xml)
  aus dem CVS kommen, sondern per "sync"-Befehl vom LMS, resp. vom Sync-User erzeugt
  werden.\\
\subsection{Referenz vom Kurs zur Lehrveranstaltung} Der Kurs wird sp�ter - im CVS ist er (inzwischen) ohne diese Referenz - einer
  Lehrveranstaltung zugeordnet. 
\subsection{Ordnung in der Datenbank} Dennoch sollen die Kurs-Dokumente und die Lehrveranstaltung in der Ordnung
  der Mumie Datenbank "beieinander" liegen. Siehe dazu die Spezifikation
  (\href{sections.xhtml\#_DEFAULT_ID_20}{"Section-Struktur", Abschnitt 6 "Standard-Gliederungsstruktur"}).
  Da der Kurs eine Referenz auf die Lehrveranstaltung enthalten wird, m�sste eigentlich diese
  zuerst da sein. Dass es in der Praxis nicht so ist, bringt uns ein gewisses logisches
  Problem:
  Beim Bau der Verzeichnisstruktur des CVS Repositorys "linear_algebra_courses" musste
  _antizipiert_ werden, wie bzw. wohin ins Checkin-Verzeichnis sp�ter der Sync-User
  die Lehrveranstaltung etc. erzeugen wird. Im vorliegenden Fall "linear_algebra_courses"
  wurde durch die Wahl
  "org/uni/ws_07_08/courses/crs_..."
  insbesondere folgendes antizipiert:
\begin{itemize}
\item A: Das sync_home des Sync Users, der das LMS vertritt (in Berlin: Moses, in Z�rich:
  Lemuren resp. ac) ist "org/uni"
\item B: Das vom Sync User anzulegende Semester hat einen Namen, aus dem von vom zust�ndigen
  Programmst�ck im JAPS (NewSemesterSyncCommand.java %oder so �hnlich%) der Section-pure_name
  "ws_07_08" erzeugt wird.\\\\
  Ausserdem muss
\item C: Im Kursdokument die Referenz auf die Lehrveranstaltung erzeugt werden.
\end{itemize}
\subsection{ToDo f�r den Admin} Diese Punkte A, B und C ziehen gewisse "ToDo"s f�r den Administrator nach sich, die
  �ber das einfache 
sync-tarball aufrufen sowie cvs-auschecken >> mounten >> japs-einchecken
  hinausgehen. Darum geht es in der Bedienungsanleitung.\\
 Man muss also entweder den sync-tarball an das CVS Repository anpassen oder \href{\#cvs-anpassung}{umgekehrt}.


\section{"Sync" Tarball}\label{sync_tarball}
  Zun�chst muss vom Administrator des JAPS ein Sync-Benutzer und sein Sync-Home
  eingecheckt werden. Er braucht dazu zwei Masterdateien
\begin{preformatted}[code]%
<MM_CHECKIN_ROOT/>org/\var{sync_home}/.meta.xml
<MM_CHECKIN_ROOT/>org/users/usr_\var{sync}.meta.xml
\end{preformatted}, wobei zu beachten sit, dass letztere eine Referenz auf erstere der Form
\begin{preformatted}[code]%
  <mumie:sync_home path="org/\var{sync}/.meta.xml"/>
\end{preformatted}
enth�lt.
 Dieser so erzeugte Sync-Benutzer soll benutzt werden um die Teilnehmer
 (Studenten), die Tutorien, die Lehrveranstaltung und die Semester zu verwalten.
 F�r ihn ein MMCDK-\var{alias} erzeugt. Schreibe dazu eine entsprechende Zeile
 in \$HOME/.mmjvmd/mmjvmd.init
 Dann wird zun�chst das Semester erzeugt. Dabei entsteht implizit eine neue
 Section, in welche das Semester und sp�ter die zugeh�rigen Lehrveranstaltungen
 und Tutorien einsortiert werden
 
\begin{preformatted}[code]%
  mmsrv -p sync-id=\var{semester-sync-id} -p name=\var{"Semester Name"} -a \var{alias} protected/sync/new-semester
\end{preformatted}
 
tbc.

\section{(empfohlen) Bedienungsanleitung zur Anpassung des CVS Repository "linear_algebra_courses"
    an den tarball}\label{cvs-anpassung}
\begin{itemize}
\item Zu A: Wenn die sync_home Section nicht "uni" heisst, sondern z.B. "sync", so wird
    _nach_ dem cvs checkout, aber _vor_ dem mounten "./build.sh mount-checkin" das entsprechende
    Verzeichnis umbenannt: z.B. in der shell:
\begin{preformatted}[code]%
cd <cvs/>linear_algebra_courses/checkin/org/
mv uni sync
\end{preformatted}
    Ausserdem muss dann das build-Script angepasst werden. Editiere dazu die Datei
\begin{preformatted}[code]%
<cvs/>linear_algebra_courses/build.sh
\end{preformatted}
    Etwa um Zeile 49 �ndere
\begin{preformatted}[code]%
\# Subdirectories of the 'org' checkin directory:
org_dirs="
  uni"
\end{preformatted}
    zu
\begin{preformatted}[code]%
\# Subdirectories of the 'org' checkin directory:
org_dirs="
  sync"
\end{preformatted}
Nun m�ssen noch in alle Dokumenten die Pfade der Referenzen ausgetauscht
werden. Dazu steht der MMCDK-Befehl "mmrpl" bereit, der regul�re Ausdr�cke in
Dateien umwandelt. Er muss auf alle Masterdateien angewandt werden:
\begin{preformatted}[code]%
mmrpl -h
mmrpl 'path=\backslash"org\backslash/uni\backslash/' 'path="org\backslash/sync\backslash/' *.meta.xml
\end{preformatted}

\item Zu B: Wenn das Semester nicht "WS 07/08" heisst, sondern z.B. "Herbstsemester 07", so sollte
    beim ausf�hren des tarballs eine Section "/org/<sync_home>/herbstsemester_07" entstanden sein.
    Diesen Verzeichnisnamen muss man nun �bernehmen!!
\begin{preformatted}[code]%
cd <cvs/>linear_algebra_courses/checkin/org/<sync_home>/
mv ws_07_08 herbstsemester_07
\end{preformatted}
Wieder m�ssen in alle Dokumenten die Pfade der Referenzen ausgetauscht
werden.
\begin{preformatted}[code]%
mmrpl 'path=\backslash"org\backslash/sync\backslash/ws_07_08\backslash/' 'path="org\backslash/sync\backslash/herbstsemester_07\backslash/' *.meta.xml
\end{preformatted}

    Hiernach muss nicht mehr das build-Script ge�ndert werden.

Jetzt kann ge-mountet werden:
\begin{preformatted}[code]%
cd <cvs/>linear_algebra_courses/
./build.sh mount-checkin
\end{preformatted}

\item Zu C: Wenn es schnell gehen soll, kann man im Kursdokument
\begin{preformatted}[code]%
<cvs/>linear_algebra_courses/checkin/org/<sync_home>/<herbstsemester_07>/courses/crs_lineare_algebra.meta.xml
\end{preformatted}
einfach von Hand die Referenz auf die Lehrveranstaltung einf�gen, z.B.
\begin{preformatted}[code]%
<mumie:course ...>
 <mumie:class path="org/sync/herbstsemester_07/cls_XX.meta.xml"/>
\end{preformatted}
Eigentlich aber soll der Kurs mit dem CourseCreator bearbeitet werden, und das empfehle ich auch, da der
 Umgang damit ohnehin gelernt werden sollte.
\end{itemize}

-- ENDE DER DATEI --
\end{document}
