\documentclass{generic}

%% Author: Tilman Rassy <rassy@math.tu-berlin.de>
% $Id: macros.tex,v 1.5 2006/10/05 15:58:42 rassy Exp $

% XML code:
\newcommand{\element}[1]{\code[xml-element]{#1}}
\newcommand{\attrib}[1]{\code[xml-attrib]{#1}}

% Database code:
\newcommand{\dbtable}[1]{\code[db-table]{#1}}
\newcommand{\dbcol}[1]{\code[db-column]{#1}}
\newcommand{\sql}[1]{\code[sql]{#1}}

% TeX code:
\newcommand{\texcmd}[1]{\code[tex-cmd]{\backslash#1}}
\newcommand{\texenv}[1]{\code[tex-env]{#1}}

% General
\newcommand{\val}[1]{\code[value]{#1}}
\newcommand{\prm}[1]{\code[param]{#1}}

% Other:
\newcommand{\notimpl}[1]{\emph[not-impl]{#1}}
\newcommand{\new}[1]{\emph[new]{#1}}
\newcommand{\deprecated}[1]{\emph[deprecated]{#1}}
\newcommand{\wvar}[1]{\var[weak]{#1}}


\begin{document}

\title{Repository der Kursdokumente}

\begin{authors}
  \author[lehmannf@math.tu-berlin.de]{Fritz Lehmann-Grube}
\end{authors}
\version{$Id: course_class.tex,v 1.1 2007/09/04 17:18:01 lehmannf Exp $}

\tableofcontents

\section{Allgemeines}

Nachdem eine Mumie installiert und die Inhaltsdokumente (z.B. aus dem CVS Repository
  "linear_algebra_content", ehemals "japs_content") eingecheckt sind, sind zwei Dinge zu
  tun, um einen Kurs durchf�hren zu k�nnen:
\begin{itemize}
\item Die Lehrveranstaltung (Semester, Tutorien) installieren. Siehe dazu die
Spezifikation (\href{moses_sync.xhtml}{"Synchronisation mit MOSES"}). Dieser
Schritt kann ausserdem vorbereitet sein durch ein sogenanntes
\href{\#sync_tarball}{"Sync Tarball"}
\item Die Kursdokumente installieren
\end{itemize}
Beides einzeln ist noch nicht besonders gut beschrieben, jedoch f�r den nicht mehr ganz
  blutigen JAPS-Anf�nger Heinz unproblematisch.
Leider sind die beiden Punkte nicht ganz unabh�ngig.

Es ist wichtig, folgendes zu verstehen:

\subsection{Vom Repository zum Checkin} Die CVS Repositorys sind nicht identisch mit dem Checkin-Verzeichnis, sondern
 werden da "hinein ge-mountet".
 %Das sollte man nochmal genauer erkl�ren, aber erstmal gen�gt es vielleicht, dass
 %du, Heinz, das wohl schon kennst.

\subsection{Inhalt des Repositorys} Das CVS Repository "linear_algebra_courses" enth�lt genau die Dokumente
  fuer den Kurs und seine Komponenten, also die Kursabschnitte und Arbeitsbl�tter
  sowie die Summarys.
\subsection{"Sync" Tarball} Dieses CVS Repository "linear_algebra_courses" enthaelt aber _nicht_ die Pseudodokumente zur Lehrveranstaltung,
  Semester etc., denn diese sollen ja gar nicht als Masterdateien (*.meta.xml)
  aus dem CVS kommen, sondern per "sync"-Befehl vom LMS, resp. vom Sync-User erzeugt
  werden.\\
\subsection{Referenz vom Kurs zur Lehrveranstaltung} Der Kurs wird sp�ter - im CVS ist er (inzwischen) ohne diese Referenz - einer
  Lehrveranstaltung zugeordnet. 
\subsection{Ordnung in der Datenbank} Dennoch sollen die Kurs-Dokumente und die Lehrveranstaltung in der Ordnung
  der Mumie Datenbank "beieinander" liegen. Siehe dazu die Spezifikation
  (\href{sections.xhtml\#_DEFAULT_ID_20}{"Section-Struktur", Abschnitt 6 "Standard-Gliederungsstruktur"}).
  Da der Kurs eine Referenz auf die Lehrveranstaltung enthalten wird, m�sste eigentlich diese
  zuerst da sein. Dass es in der Praxis nicht so ist, bringt uns ein gewisses logisches
  Problem:
  Beim Bau der Verzeichnisstruktur des CVS Repositorys "linear_algebra_courses" musste
  _antizipiert_ werden, wie bzw. wohin ins Checkin-Verzeichnis sp�ter der Sync-User
  die Lehrveranstaltung etc. erzeugen wird. Im vorliegenden Fall "linear_algebra_courses"
  wurde durch die Wahl
  "org/uni/ws_07_08/courses/crs_..."
  insbesondere folgendes antizipiert:
\begin{itemize}
\item A: Das sync_home des Sync Users, der das LMS vertritt (in Berlin: Moses, in Z�rich:
  Lemuren resp. ac) ist "org/uni"
\item B: Das vom Sync User anzulegende Semester hat einen Namen, aus dem von vom zust�ndigen
  Programmst�ck im JAPS (NewSemesterSyncCommand.java %oder so �hnlich%) der Section-pure_name
  "ws_07_08" erzeugt wird.\\\\
  Ausserdem muss
\item C: Im Kursdokument die Referenz auf die Lehrveranstaltung erzeugt werden.
\end{itemize}
\subsection{ToDo f�r den Admin} Diese Punkte A, B und C ziehen gewisse "ToDo"s f�r den Administrator nach sich, die
  �ber das einfache 
sync-tarball aufrufen sowie cvs-auschecken >> mounten >> japs-einchecken
  hinausgehen. Darum geht es in der Bedienungsanleitung.\\
 Man muss also entweder den sync-tarball an das CVS Repository anpassen oder \href{\#cvs-anpassung}{umgekehrt}.


\section{"Sync" Tarball}\label{sync_tarball}
  Zun�chst muss vom Administrator des JAPS ein Sync-Benutzer und sein Sync-Home
  eingecheckt werden. Er braucht dazu zwei Masterdateien
\begin{preformatted}[code]%
<MM_CHECKIN_ROOT/>org/\var{sync_home}/.meta.xml
<MM_CHECKIN_ROOT/>org/users/usr_\var{sync}.meta.xml
\end{preformatted}, wobei zu beachten sit, dass letztere eine Referenz auf erstere der Form
\begin{preformatted}[code]%
  <mumie:sync_home path="org/\var{sync}/.meta.xml"/>
\end{preformatted}
enth�lt.
 Dieser so erzeugte Sync-Benutzer soll benutzt werden um die Teilnehmer
 (Studenten), die Tutorien, die Lehrveranstaltung und die Semester zu verwalten.
 F�r ihn ein MMCDK-\var{alias} erzeugt. Schreibe dazu eine entsprechende Zeile
 in \$HOME/.mmjvmd/mmjvmd.init
 Dann wird zun�chst das Semester erzeugt. Dabei entsteht implizit eine neue
 Section, in welche das Semester und sp�ter die zugeh�rigen Lehrveranstaltungen
 und Tutorien einsortiert werden
 
\begin{preformatted}[code]%
  mmsrv -p sync-id=\var{semester-sync-id} -p name=\var{"Semester Name"} -a \var{alias} protected/sync/new-semester
\end{preformatted}
 
tbc.

\section{(empfohlen) Bedienungsanleitung zur Anpassung des CVS Repository "linear_algebra_courses"
    an den tarball}\label{cvs-anpassung}
\begin{itemize}
\item Zu A: Wenn die sync_home Section nicht "uni" heisst, sondern z.B. "sync", so wird
    _nach_ dem cvs checkout, aber _vor_ dem mounten "./build.sh mount-checkin" das entsprechende
    Verzeichnis umbenannt: z.B. in der shell:
\begin{preformatted}[code]%
cd <cvs/>linear_algebra_courses/checkin/org/
mv uni sync
\end{preformatted}
    Ausserdem muss dann das build-Script angepasst werden. Editiere dazu die Datei
\begin{preformatted}[code]%
<cvs/>linear_algebra_courses/build.sh
\end{preformatted}
    Etwa um Zeile 49 �ndere
\begin{preformatted}[code]%
\# Subdirectories of the 'org' checkin directory:
org_dirs="
  uni"
\end{preformatted}
    zu
\begin{preformatted}[code]%
\# Subdirectories of the 'org' checkin directory:
org_dirs="
  sync"
\end{preformatted}
Nun m�ssen noch in alle Dokumenten die Pfade der Referenzen ausgetauscht
werden. Dazu steht der MMCDK-Befehl "mmrpl" bereit, der regul�re Ausdr�cke in
Dateien umwandelt. Er muss auf alle Masterdateien angewandt werden:
\begin{preformatted}[code]%
mmrpl -h
mmrpl 'path=\backslash"org\backslash/uni\backslash/' 'path="org\backslash/sync\backslash/' *.meta.xml
\end{preformatted}

\item Zu B: Wenn das Semester nicht "WS 07/08" heisst, sondern z.B. "Herbstsemester 07", so sollte
    beim ausf�hren des tarballs eine Section "/org/<sync_home>/herbstsemester_07" entstanden sein.
    Diesen Verzeichnisnamen muss man nun �bernehmen!!
\begin{preformatted}[code]%
cd <cvs/>linear_algebra_courses/checkin/org/<sync_home>/
mv ws_07_08 herbstsemester_07
\end{preformatted}
Wieder m�ssen in alle Dokumenten die Pfade der Referenzen ausgetauscht
werden.
\begin{preformatted}[code]%
mmrpl 'path=\backslash"org\backslash/sync\backslash/ws_07_08\backslash/' 'path="org\backslash/sync\backslash/herbstsemester_07\backslash/' *.meta.xml
\end{preformatted}

    Hiernach muss nicht mehr das build-Script ge�ndert werden.

Jetzt kann ge-mountet werden:
\begin{preformatted}[code]%
cd <cvs/>linear_algebra_courses/
./build.sh mount-checkin
\end{preformatted}

\item Zu C: Wenn es schnell gehen soll, kann man im Kursdokument
\begin{preformatted}[code]%
<cvs/>linear_algebra_courses/checkin/org/<sync_home>/<herbstsemester_07>/courses/crs_lineare_algebra.meta.xml
\end{preformatted}
einfach von Hand die Referenz auf die Lehrveranstaltung einf�gen, z.B.
\begin{preformatted}[code]%
<mumie:course ...>
 <mumie:class path="org/sync/herbstsemester_07/cls_XX.meta.xml"/>
\end{preformatted}
Eigentlich aber soll der Kurs mit dem CourseCreator bearbeitet werden, und das empfehle ich auch, da der
 Umgang damit ohnehin gelernt werden sollte.
\end{itemize}

-- ENDE DER DATEI --
\end{document}
