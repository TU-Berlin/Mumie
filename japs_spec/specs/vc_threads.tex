\documentclass{generic}


% ------------------------------------------------------------------------------
% Schriftgroessen
% ------------------------------------------------------------------------------

\renewcommand{\normalsize}{\fontsize{18}{22}\selectfont}
\renewcommand{\large}{\fontsize{20}{26}\selectfont}
\renewcommand{\Large}{\fontsize{22}{30}\selectfont}
\renewcommand{\small}{\fontsize{14}{16}\selectfont}
\newcommand{\smaller}{\fontsize{12}{14}\selectfont}
\newcommand{\verysmall}{\fontsize{6}{6}\selectfont}
\renewcommand{\footnotesize}{\fontsize{12}{14}\selectfont}
\newcommand{\tocsize}{\fontsize{12}{16}\selectfont}
\newcommand{\titlesize}{\fontsize{24}{36}\selectfont}

% ------------------------------------------------------------------------------
% Texthervorhebungen
% ------------------------------------------------------------------------------

%\definecolor{emphcolor}{rgb}{0.77,0.00,0.00}
%\renewcommand{\emph}[1]{\textbf{\textcolor{emphcolor}{#1}}}
\renewcommand{\emph}[1]{\textbf{#1}}

\newcommand{\term}[1]{\textit{#1}}
% \newcommand{\code}[1]{\texttt{#1}}
\newcommand{\var}[1]{\textit{#1}}

%\definecolor{headlinecolor}{rgb}{0.77,0.00, 0.00}
\definecolor{headlinecolor}{rgb}{0.67,0.09, 0.22}
\newcommand{\headline}[1]{\textbf{\textcolor{headlinecolor}{#1}}}

\definecolor{emphboxbordercolor}{rgb}{0.80,0.36,0.36}
\definecolor{emphboxcolor}{rgb}{0.98,0.98,0.77}
\newcommand{\emphbox}[2]{\fcolorbox{emphboxbordercolor}{emphboxcolor}{\parbox{#1}{#2}}}

\definecolor{labelcolor}{rgb}{0.35,0.35, 0.8}


% -------------------------------------------------------------------------------
% Pfeile
% -------------------------------------------------------------------------------

\newcommand{\myarrow}{{\Large\boldmath$\;\rightarrow\;$}}
\newcommand{\mylongarrow}{{\Large\boldmath$\;\longrightarrow\;$}}



% -------------------------------------------------------------------------------
% Befehle f�r Abbildungen 
% -------------------------------------------------------------------------------

\renewcommand{\thefigure}{\thesubsection.\arabic{figure}}

\newcommand{\mycaption}[2]{%
 \centerline{%
  \parbox{#1}{%
   \small \refstepcounter{figure} Figure \thefigure. #2 }}}


%---------------------------------------------------------------------------
% Gliederungsbefehle
%---------------------------------------------------------------------------

\makeatletter

\renewcommand{\section}{\@startsection
 {section}%                                   % Name
 {1}%                                         % Ebene
 {0pt}%                                       % Einzug
 {-0.0\baselineskip}%                        % Vorabstand 
 {0.85\baselineskip}%                         % Nachabstand
 {\fontsize{20}{24}\selectfont\itshape\bfseries}}%    % Stil

\makeatother

\newcommand{\mysection}[1]{%

\vspace*{-1.2\baselineskip}

{\color{headlinecolor}
\section{#1}}}

\newcounter{myappendix}

\newcommand{\myappendix}[2]{%
\hfill\makebox[0cm][l]{\tocref}

\vspace*{-1.2\baselineskip}

\refstepcounter{myappendix}
\section*{A\arabic{myappendix}\hspace{1em}#1}\hypertarget{#2}{}}



% -------------------------------------------------------------------------------
% Inhaltsverzeichnis 
% -------------------------------------------------------------------------------

\newcommand{\mytoctitle}[1]{\begin{center}\textbf{\hypertarget{inhalt}{#1}}\end{center}}

\newenvironment{mytoc}
  {\tocsize\textbf{\hypertarget{inhalt}{Inhalt}}\\\begin{enumerate}}
  {\end{enumerate}}

\newcommand{\mytocitem}[3]{\item[#1]\hyperlink{#3}{#2}}

\newcommand{\tocref}{\verysmall\hyperlink{inhalt}{TOC}\normalsize}

% -------------------------------------------------------------------------------
% Listen 
% -------------------------------------------------------------------------------

%\newcommand{\mylabelitemi}{\labelitemi}
\newcommand{\mylabelitemi}{\raisebox{0.25ex}{{\small $\blacktriangleright$}\hspace{0.1em}}}

%\newcommand{\mylabelitemii}{\labelitemii}
%\newcommand{\mylabelitemii}{\raisebox{0.25ex}{{\small $\triangleright$}\hspace{0.1em}}}
\newcommand{\mylabelitemii}{\raisebox{0.1ex}{\large\bfseries-}\hspace{0.075em}}

\newenvironment{mylist}[1][\mylabelitemi]
  {\begin{list}{{\color{labelcolor}#1}}
    {\setlength{\itemindent}{0cm}
     \setlength{\labelwidth}{1em}
     \setlength{\leftmargin}{0.5cm}
     \setlength{\rightmargin}{1cm}}}
  {\end{list}}

\newenvironment{myenum}
  {\begin{list}{\theenumi}
    {\usecounter{enumi}
     \setlength{\itemindent}{0cm}
     \setlength{\labelwidth}{1em}
     \setlength{\leftmargin}{0.5cm}
     \setlength{\rightmargin}{1cm}}}
  {\end{list}}

\newcommand{\pitem}{\pause\item}
\newcommand{\itemii}{\item[\mylabelitemii]}

\newcounter{romanlistcounter}
\newenvironment{romanlist}{\begin{list}%
 {(\roman{romanlistcounter})\hfill}{\usecounter{romanlistcounter}%
 \topsep0.05\baselineskip% 
 \partopsep0.85\baselineskip%
 \leftmargin2em%
 \labelwidth2em%
 \labelsep0pt%
 \itemindent0pt%
 \parsep0.15\baselineskip%
 \itemsep0.0\baselineskip}}%
 {\end{list}}

\newcounter{alphlistcounter}
\newenvironment{alphlist}{\begin{list}%
 {(\alph{alphlistcounter})\hfill}{\usecounter{alphlistcounter}%
 \topsep0.05\baselineskip% 
 \partopsep0.85\baselineskip%
 \leftmargin1.8em%
 \labelwidth1.8em%
 \labelsep0pt%
 \itemindent0pt%
 \parsep0.15\baselineskip%
 \itemsep0.0\baselineskip}}%
 {\end{list}}

\newcounter{arabiclistcounter}
\newenvironment{arabiclist}{\begin{list}%
 {(\arabic{arabiclistcounter})\hfill}{\usecounter{arabiclistcounter}%
 \topsep0.05\baselineskip% 
 \partopsep0.85\baselineskip%
 \leftmargin1.8em%
 \labelwidth1.8em%
 \labelsep0pt%               
 \itemindent0pt%
 \parsep0.15\baselineskip%
 \itemsep0.0\baselineskip}}%
 {\end{list}}





\begin{document}

\title{Versionskontrolle (VCThreads)}

\begin{authors}
  \author[lehmannf@math.tu-berlin.de]{Fritz Lehmann-Grube}
\end{authors}

\version{$Id: vc_threads.tex,v 1.3 2006/08/11 16:18:23 rassy Exp $}

\section{Einleitung}

F�r unser Dokumentenverwaltungssystem JAPS brauchen wir ein Konzept von
Versionen von Dokumenten.
Wir nehmen an, dass es sich bei Texten nicht lohnt und bei Bin�rdaten
fehleranf�llig ist, ein 'merge'-Konzept, wie es aus CVS bekannt ist, zu
adaptieren.
Die Versionskontrolle soll eine \emph{unabh�ngige} Funktionalit�t
sein. Das entspricht unserer Vorstellung von einem Pool von Inhalten.

\section{Zusammenfassung}

Wir speichern verschiedene Versionen als selbst�ndige, komplette
Dokumente. Wird eine neue Version zugef�gt, bleiben �ltere Versionen davon
unber�hrt.

Insbesondere werden wir ein Dokument im allgemeinen direkt �ber seine id
ansprechen, und nur, wenn es im Kontext der Versionskontrolle betrachtet wird, 
als bestimmte Version eines Threads.

Wird eine neue Version eines Dokuments der Datenbank hinzugef�gt, so werden in
allen anderen Dokumenten, welche dieses als Komponente enthalten, die neue
Version eingef�gt.

W�hrend Leserechte sich auf einzelne Dokumente beziehen, werden Schreibrechte
auf VCThreads definiert. Das Schreibrecht auf einem VCThread bedeutet also das
Recht, neue Versionen hinzuzuf�gen.

VCThreads sind Dokumenttypweise definiert, das hei�t, Dokumente eines
vc\_threads m�ssen alle denselben Typ haben.

Generische Dokumente haben keine Versionskontrolle.

\section{Modell}

Ein vern�nftiges Versionskontrollsystem f�r Dokumente, die in einer
relationalen Datenbank gespeichert sind, l�sst sich auf folgender einfachen
Struktur aufbauen:

Wir f�hren (f�r jeden nicht-generischen Dokumenttyp) die Entit�t 'Version
Control Thread', kurz VCThread, mit den Attributen 'name', 'description' und
'created' ein, sowie die 1:n Relation 'ist Version von' zwischen Dokumenten und
VCThreads. Diese Relation erh�lt noch das eindeutige Attribut 'Versionsnummer'.
%\image{ER_VCT}
\begin{preformatted}
   -------------                     ------------------
   |           |    version (nr)     |                | ---name
   |  Page     |  n -------------> 1 |  VCThread Page | ---description
   |           |                     |                | ---created
   -------------                     ------------------
\end{preformatted}

Jedes nicht-generische Dokument ist damit eine Version genau eines
VCThreads. Jedes Dokument hat also ein Paar (vc_thread,version) von Attributen,
wobei 'vc_thread' eine Referenz (Fremdschl�ssel)  auf einen VCThread und
'version' eine Nummer ist. Jedes solche Paar kommt unter den Dokumenten eines
Typs h�chstens einmal vor.
Beispiel:
\begin{preformatted}

|----------------------
|  vc_threads_page    |
|--------------------------------------------------------------------------------
| id |              name               | description |          created         |
|----+---------------------------------+-------------+---------------------------
|  1 | Thread Blank Page               | 'BLA'       | xxx
|  2 | Thread Course Browser           | 'BLUBB'     | xxy

|------------
|  pages    |
|-----------------------------------------------------------------
| id |              name               | vc_thread | version |....
|----+---------------------------------+-----------+--------------
|  0 | Blank Page                      |         1 |       1 |
|  1 | Course Browser                  |         2 |       1 |
|  2 | Blank Page 2                    |         1 |       2 |

\end{preformatted}


\end{document}
