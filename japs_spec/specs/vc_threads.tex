\documentclass{generic}

% Author: Tilman Rassy <rassy@math.tu-berlin.de>
% $Id: macros.tex,v 1.5 2006/10/05 15:58:42 rassy Exp $

% XML code:
\newcommand{\element}[1]{\code[xml-element]{#1}}
\newcommand{\attrib}[1]{\code[xml-attrib]{#1}}

% Database code:
\newcommand{\dbtable}[1]{\code[db-table]{#1}}
\newcommand{\dbcol}[1]{\code[db-column]{#1}}
\newcommand{\sql}[1]{\code[sql]{#1}}

% TeX code:
\newcommand{\texcmd}[1]{\code[tex-cmd]{\backslash#1}}
\newcommand{\texenv}[1]{\code[tex-env]{#1}}

% General
\newcommand{\val}[1]{\code[value]{#1}}
\newcommand{\prm}[1]{\code[param]{#1}}

% Other:
\newcommand{\notimpl}[1]{\emph[not-impl]{#1}}
\newcommand{\new}[1]{\emph[new]{#1}}
\newcommand{\deprecated}[1]{\emph[deprecated]{#1}}
\newcommand{\wvar}[1]{\var[weak]{#1}}


\begin{document}

\title{Versionskontrolle (VCThreads)}

\begin{authors}
  \author[lehmannf@math.tu-berlin.de]{Fritz Lehmann-Grube}
\end{authors}

\version{$Id: vc_threads.tex,v 1.3 2006/08/11 16:18:23 rassy Exp $}

\section{Einleitung}

F�r unser Dokumentenverwaltungssystem JAPS brauchen wir ein Konzept von
Versionen von Dokumenten.
Wir nehmen an, dass es sich bei Texten nicht lohnt und bei Bin�rdaten
fehleranf�llig ist, ein 'merge'-Konzept, wie es aus CVS bekannt ist, zu
adaptieren.
Die Versionskontrolle soll eine \emph{unabh�ngige} Funktionalit�t
sein. Das entspricht unserer Vorstellung von einem Pool von Inhalten.

\section{Zusammenfassung}

Wir speichern verschiedene Versionen als selbst�ndige, komplette
Dokumente. Wird eine neue Version zugef�gt, bleiben �ltere Versionen davon
unber�hrt.

Insbesondere werden wir ein Dokument im allgemeinen direkt �ber seine id
ansprechen, und nur, wenn es im Kontext der Versionskontrolle betrachtet wird, 
als bestimmte Version eines Threads.

Wird eine neue Version eines Dokuments der Datenbank hinzugef�gt, so werden in
allen anderen Dokumenten, welche dieses als Komponente enthalten, die neue
Version eingef�gt.

W�hrend Leserechte sich auf einzelne Dokumente beziehen, werden Schreibrechte
auf VCThreads definiert. Das Schreibrecht auf einem VCThread bedeutet also das
Recht, neue Versionen hinzuzuf�gen.

VCThreads sind Dokumenttypweise definiert, das hei�t, Dokumente eines
vc\_threads m�ssen alle denselben Typ haben.

Generische Dokumente haben keine Versionskontrolle.

\section{Modell}

Ein vern�nftiges Versionskontrollsystem f�r Dokumente, die in einer
relationalen Datenbank gespeichert sind, l�sst sich auf folgender einfachen
Struktur aufbauen:

Wir f�hren (f�r jeden nicht-generischen Dokumenttyp) die Entit�t 'Version
Control Thread', kurz VCThread, mit den Attributen 'name', 'description' und
'created' ein, sowie die 1:n Relation 'ist Version von' zwischen Dokumenten und
VCThreads. Diese Relation erh�lt noch das eindeutige Attribut 'Versionsnummer'.
%\image{ER_VCT}
\begin{preformatted}
   -------------                     ------------------
   |           |    version (nr)     |                | ---name
   |  Page     |  n -------------> 1 |  VCThread Page | ---description
   |           |                     |                | ---created
   -------------                     ------------------
\end{preformatted}

Jedes nicht-generische Dokument ist damit eine Version genau eines
VCThreads. Jedes Dokument hat also ein Paar (vc_thread,version) von Attributen,
wobei 'vc_thread' eine Referenz (Fremdschl�ssel)  auf einen VCThread und
'version' eine Nummer ist. Jedes solche Paar kommt unter den Dokumenten eines
Typs h�chstens einmal vor.
Beispiel:
\begin{preformatted}

|----------------------
|  vc_threads_page    |
|--------------------------------------------------------------------------------
| id |              name               | description |          created         |
|----+---------------------------------+-------------+---------------------------
|  1 | Thread Blank Page               | 'BLA'       | xxx
|  2 | Thread Course Browser           | 'BLUBB'     | xxy

|------------
|  pages    |
|-----------------------------------------------------------------
| id |              name               | vc_thread | version |....
|----+---------------------------------+-----------+--------------
|  0 | Blank Page                      |         1 |       1 |
|  1 | Course Browser                  |         2 |       1 |
|  2 | Blank Page 2                    |         1 |       2 |

\end{preformatted}


\end{document}
