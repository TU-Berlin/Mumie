\documentclass{generic}


% ------------------------------------------------------------------------------
% Schriftgroessen
% ------------------------------------------------------------------------------

\renewcommand{\normalsize}{\fontsize{18}{22}\selectfont}
\renewcommand{\large}{\fontsize{20}{26}\selectfont}
\renewcommand{\Large}{\fontsize{22}{30}\selectfont}
\renewcommand{\small}{\fontsize{14}{16}\selectfont}
\newcommand{\smaller}{\fontsize{12}{14}\selectfont}
\newcommand{\verysmall}{\fontsize{6}{6}\selectfont}
\renewcommand{\footnotesize}{\fontsize{12}{14}\selectfont}
\newcommand{\tocsize}{\fontsize{12}{16}\selectfont}
\newcommand{\titlesize}{\fontsize{24}{36}\selectfont}

% ------------------------------------------------------------------------------
% Texthervorhebungen
% ------------------------------------------------------------------------------

%\definecolor{emphcolor}{rgb}{0.77,0.00,0.00}
%\renewcommand{\emph}[1]{\textbf{\textcolor{emphcolor}{#1}}}
\renewcommand{\emph}[1]{\textbf{#1}}

\newcommand{\term}[1]{\textit{#1}}
% \newcommand{\code}[1]{\texttt{#1}}
\newcommand{\var}[1]{\textit{#1}}

%\definecolor{headlinecolor}{rgb}{0.77,0.00, 0.00}
\definecolor{headlinecolor}{rgb}{0.67,0.09, 0.22}
\newcommand{\headline}[1]{\textbf{\textcolor{headlinecolor}{#1}}}

\definecolor{emphboxbordercolor}{rgb}{0.80,0.36,0.36}
\definecolor{emphboxcolor}{rgb}{0.98,0.98,0.77}
\newcommand{\emphbox}[2]{\fcolorbox{emphboxbordercolor}{emphboxcolor}{\parbox{#1}{#2}}}

\definecolor{labelcolor}{rgb}{0.35,0.35, 0.8}


% -------------------------------------------------------------------------------
% Pfeile
% -------------------------------------------------------------------------------

\newcommand{\myarrow}{{\Large\boldmath$\;\rightarrow\;$}}
\newcommand{\mylongarrow}{{\Large\boldmath$\;\longrightarrow\;$}}



% -------------------------------------------------------------------------------
% Befehle f�r Abbildungen 
% -------------------------------------------------------------------------------

\renewcommand{\thefigure}{\thesubsection.\arabic{figure}}

\newcommand{\mycaption}[2]{%
 \centerline{%
  \parbox{#1}{%
   \small \refstepcounter{figure} Figure \thefigure. #2 }}}


%---------------------------------------------------------------------------
% Gliederungsbefehle
%---------------------------------------------------------------------------

\makeatletter

\renewcommand{\section}{\@startsection
 {section}%                                   % Name
 {1}%                                         % Ebene
 {0pt}%                                       % Einzug
 {-0.0\baselineskip}%                        % Vorabstand 
 {0.85\baselineskip}%                         % Nachabstand
 {\fontsize{20}{24}\selectfont\itshape\bfseries}}%    % Stil

\makeatother

\newcommand{\mysection}[1]{%

\vspace*{-1.2\baselineskip}

{\color{headlinecolor}
\section{#1}}}

\newcounter{myappendix}

\newcommand{\myappendix}[2]{%
\hfill\makebox[0cm][l]{\tocref}

\vspace*{-1.2\baselineskip}

\refstepcounter{myappendix}
\section*{A\arabic{myappendix}\hspace{1em}#1}\hypertarget{#2}{}}



% -------------------------------------------------------------------------------
% Inhaltsverzeichnis 
% -------------------------------------------------------------------------------

\newcommand{\mytoctitle}[1]{\begin{center}\textbf{\hypertarget{inhalt}{#1}}\end{center}}

\newenvironment{mytoc}
  {\tocsize\textbf{\hypertarget{inhalt}{Inhalt}}\\\begin{enumerate}}
  {\end{enumerate}}

\newcommand{\mytocitem}[3]{\item[#1]\hyperlink{#3}{#2}}

\newcommand{\tocref}{\verysmall\hyperlink{inhalt}{TOC}\normalsize}

% -------------------------------------------------------------------------------
% Listen 
% -------------------------------------------------------------------------------

%\newcommand{\mylabelitemi}{\labelitemi}
\newcommand{\mylabelitemi}{\raisebox{0.25ex}{{\small $\blacktriangleright$}\hspace{0.1em}}}

%\newcommand{\mylabelitemii}{\labelitemii}
%\newcommand{\mylabelitemii}{\raisebox{0.25ex}{{\small $\triangleright$}\hspace{0.1em}}}
\newcommand{\mylabelitemii}{\raisebox{0.1ex}{\large\bfseries-}\hspace{0.075em}}

\newenvironment{mylist}[1][\mylabelitemi]
  {\begin{list}{{\color{labelcolor}#1}}
    {\setlength{\itemindent}{0cm}
     \setlength{\labelwidth}{1em}
     \setlength{\leftmargin}{0.5cm}
     \setlength{\rightmargin}{1cm}}}
  {\end{list}}

\newenvironment{myenum}
  {\begin{list}{\theenumi}
    {\usecounter{enumi}
     \setlength{\itemindent}{0cm}
     \setlength{\labelwidth}{1em}
     \setlength{\leftmargin}{0.5cm}
     \setlength{\rightmargin}{1cm}}}
  {\end{list}}

\newcommand{\pitem}{\pause\item}
\newcommand{\itemii}{\item[\mylabelitemii]}

\newcounter{romanlistcounter}
\newenvironment{romanlist}{\begin{list}%
 {(\roman{romanlistcounter})\hfill}{\usecounter{romanlistcounter}%
 \topsep0.05\baselineskip% 
 \partopsep0.85\baselineskip%
 \leftmargin2em%
 \labelwidth2em%
 \labelsep0pt%
 \itemindent0pt%
 \parsep0.15\baselineskip%
 \itemsep0.0\baselineskip}}%
 {\end{list}}

\newcounter{alphlistcounter}
\newenvironment{alphlist}{\begin{list}%
 {(\alph{alphlistcounter})\hfill}{\usecounter{alphlistcounter}%
 \topsep0.05\baselineskip% 
 \partopsep0.85\baselineskip%
 \leftmargin1.8em%
 \labelwidth1.8em%
 \labelsep0pt%
 \itemindent0pt%
 \parsep0.15\baselineskip%
 \itemsep0.0\baselineskip}}%
 {\end{list}}

\newcounter{arabiclistcounter}
\newenvironment{arabiclist}{\begin{list}%
 {(\arabic{arabiclistcounter})\hfill}{\usecounter{arabiclistcounter}%
 \topsep0.05\baselineskip% 
 \partopsep0.85\baselineskip%
 \leftmargin1.8em%
 \labelwidth1.8em%
 \labelsep0pt%               
 \itemindent0pt%
 \parsep0.15\baselineskip%
 \itemsep0.0\baselineskip}}%
 {\end{list}}





\begin{document}

\title{Quality Feedback}
\subtitle{Halbautomatische Fehlerbericht-Erzeugung}

\begin{authors}
  \author[rassy@math.tu-berlin.de]{Tilman Rassy}
\end{authors}

\version{$Id: applet_quality_feedback.tex,v 1.4 2006/08/11 16:18:22 rassy Exp $}

\tableofcontents

\section{Einleitung}

Tritt zur Laufzeit des Applets ein Fehler auf, so soll dieser abgefangen und
dem Benutzer gemeldet werden. Muss von einen Bug im Applet ausgegangen werden,
so soll automatisch ein Fehlerbericht in XML-Form erstellt werden, der vom
Benutzer erg�nzt und an den Server geschickt werden kann. [Liegt mit grosser
Wahrscheinlichkeit kein Bug vor, sondern z.B. eine falsche Benutzerangabe oder
eine fehlerhafte Software-Umgebung, so soll der Fehlerbericht unterdr�ckt
werden.] 

Unter einem Fehler ist immer ein geworfenes \code{Throwable}, also eine
\code{Exception} oder ein \code{Error}, zu verstehen.

\section{Angaben im Fehlerbericht}

Der Fehlerbericht enth�lt folgende Daten:

\begin{enumerate}
  \item Zeitpunkt des Fehlers,
  \item Benutzer (gegeben durch Id)
  \item Aufgabe und �bungsblatt (gegeben durch
    Course-Subsection-Aufgaben-Referenz-Id),
  \item Applet (Klassenname, ggf. weitere Angaben),
  \item Java (Version und Hersteller laut System Properties java.vm.version
    und java.vendor),
  \item Betriebssystem (Architektur, Name und Version laut System Properties
    os.arch, os.name, os.version),
  \item Throwable (Klassenname, Meldung, Stacktrace),
  \item Kommentar des Benutzers.
\end{enumerate}

\section{Benutzerdialog}

Nachdem der Fehler aufgetreten ist, soll der Benutzer sofort durch einen
Popup-Dialog informiert werden. Dieser Dialog sollte insbesondere enthalten:

\begin{enumerate}
  \item Die Aussage, dass ein Fehler aufgetreten ist.
  \item Die Aussage, dass ein Fehlerbericht erstellt wurde.
  \item Die M�glichkeit, den Fehlerbericht durch eine Notiz zu erg�nzen.
  \item Die M�glichkeit, den Fehlerbericht einzusehen
    (s. \ref{fehlerbericht_einsehen}).
  \item Die M�glichkeit, den Fehlerbericht abzuschicken. Der Dialog wird danach
    beendet.
  \item Die M�glichkeit, den Dialog zu beenden, ohne den Fehlerbericht
    abzuschicken.
  \item Ggf. die Empfehlung, vor dem Weiterarbeiten den Browser zu schliessen
    und neu zu starten.
\end{enumerate}

Angaben wie das geworfene \code{Throwable}, dessen Meldung (\emph{message})
oder der Stacktrace sind f�r den Benutzer meist wenig aufschlussreich und
brauchen im Dialog nicht aufzutreten. [Wenn der Benutzer doch an diesen Angaben
interessiert ist, kann er den Fehlerbericht einsehen. Dort sind die Angaben
enthalten. Ausserdem findet er sie in der Java-Konsole.]

Demnach k�nnte der Dialog beispielsweise wie folgt aussehen:

\image[][c]{applet\_quality\_feedback\_dialog.png}

\section{Fehlerbericht einsehen}\label{fehlerbericht_einsehen}

Die Einsicht in den Fehlerbericht wird am Besten durch ein weiteres
Popup-Fenster realisiert. Sie soll folgendes umfassen:

\begin{enumerate}
  \item Eine Liste, welche Arten von Daten im Bericht vorkommen. Die konkreten
    Werte der Daten brauchen nicht aufgef�hrt zu werden.
  \item Der XML-Code des Fehlerberichts.
\end{enumerate}

Damit k�nnte das Fenster beispielsweise so aussehen:

\image[][c]{applet\_quality\_feedback\_einsicht.png}

\section{Fehlerbericht abschicken}

Der Fehlerbericht wird an folgende URL geschickt:

\begin{preformatted}[code]%
  \var{prefix}/protected/qf/applet
\end{preformatted}

wobei \var{prefix} der gemeinsame Prefix aller JAPS-URLs ist, also
z.B. \val{http://japs.mumie.net/cocoon}.

Der URL werden zwei Request-Parameter hinzugef�gt:

\begin{enumerate}
\item \val{report} -- Der Fehlerbericht,
\item \val{subject} -- eine Kurzbezeichnung des Fehlers. 
\end{enumerate}

Konventionsvorschlag f�r die Kurzbezeichnung:

\begin{preformatted}[code]%
  "\var{ref\_id}/\var{applet\_class}: \var{throwable\_class}: \var{message}"
\end{preformatted}

Dabei ist \var{ref\_id} die Id der Course-Subsection-Problem-Referenz,
\var{applet\_class} die Applet-Klasse, \var{throwable\_class} die Throwable-Klasse
und \var{message} die Meldung des Throwables. Beispiel:

\begin{preformatted}[code]%
  45/net.mumie.mathlet.linalg.Foo: java.lang.NullPointerException: Cannot add null
\end{preformatted}

Der Fehlerbericht wird am besten mit der Http-POST-Methode geschickt.

Der Server antwortet mit einem kurzen String. Dieser ist \code{"OK"}, wenn der
Fehlerbericht korrekt empfangen und verarbeitet wurde. Andernfalls enth�hlt er
eine Fehlermeldung. Der Benutzer soll mit Hilfe eines Popup-Fensters �ber den
Erfolg bzw. Misserfolg des Vorgangs informiert werden.

\section{XML-Struktur}

\subsection{Beispiel}

\begin{preformatted}[code]
<applet_quality_feedback
   xmlns="http://www.mumie.net/xml-namespace/applet-quality-feedback">

  <!-- Timestamp: Formatierter Wert, Format (s. SimpleDateFormat), Wert als
    Millisekunden seit 1.1.1970 (raw) -->
  <time value="2005-12-18 19:24:12 354" format="yyyy-MM-dd HH:mm:ss S"
        raw="167233672"/>

  <!-- Benutzer, gegeben durch Id: -->
  <user id="34"/>

  <!-- Aufgabe, gegeben durch Ref-Id: -->
  <problem ref_id="45"/>

  <!-- Applet, Klassenname, ggf. weitere Angaben: -->
  <applet class="net.mumie.mathlet.linalg.Foo"/>

  <!-- Java: Version und Hersteller, laut System Properties java.vm.version
    und java.vendor -->
  <java version="1.4.2_08-b03" vendor="Sun Microsystems Inc."/>

  <!-- Rechner/Betriebssystem: Architektur, Name, Version, laut System Properties
     os.arch, os.name, os.version -->
  <system arch="i386" name="Linux" version="2.4.21-297-athlon"/>

  <!-- Geworfenes Throwable: Klassenname, Fehlermeldung, Stacktrace -->
  <throwable class="java.lang.NullPointerException">
    <message>Cannot add null!</message>
    <stacktrace>
      java.lang.NullPointerException: Cannot add null!
      net.mumie.mathletfactory.test.markus.TestApplet.init ( line 33 )
      net.mumie.mathletfactory.test.markus.TestApplet.main ( line 177 )
    </stacktrace>
    <cause>
      <throwable class="java.lang.IllegalArgumentException">
        <message>Parameter is null!</message>
        <stacktrace>
          java.lang.IllegalArgumentException: Parameter is null!
          net.mumie.mathletfactory.test.markus.TestApplet.init ( line 32 )
          net.mumie.mathletfactory.test.markus.TestApplet.main ( line 177 )
        </stacktrace>
      </throwable>
    </cause>
  </throwable>

  <!-- Vom Benutzer verfasster Kommentar: -->
  <user_comment>
    ...
  </user_comment>

</applet_quality_feedback>
\end{preformatted}

-- ENDE DER DATEI --

\end{document}