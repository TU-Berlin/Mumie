\documentclass{generic}

% Author: Tilman Rassy <rassy@math.tu-berlin.de>
% $Id: macros.tex,v 1.5 2006/10/05 15:58:42 rassy Exp $

% XML code:
\newcommand{\element}[1]{\code[xml-element]{#1}}
\newcommand{\attrib}[1]{\code[xml-attrib]{#1}}

% Database code:
\newcommand{\dbtable}[1]{\code[db-table]{#1}}
\newcommand{\dbcol}[1]{\code[db-column]{#1}}
\newcommand{\sql}[1]{\code[sql]{#1}}

% TeX code:
\newcommand{\texcmd}[1]{\code[tex-cmd]{\backslash#1}}
\newcommand{\texenv}[1]{\code[tex-env]{#1}}

% General
\newcommand{\val}[1]{\code[value]{#1}}
\newcommand{\prm}[1]{\code[param]{#1}}

% Other:
\newcommand{\notimpl}[1]{\emph[not-impl]{#1}}
\newcommand{\new}[1]{\emph[new]{#1}}
\newcommand{\deprecated}[1]{\emph[deprecated]{#1}}
\newcommand{\wvar}[1]{\var[weak]{#1}}


\begin{document}

\title{Personalisierte-Aufgaben-XML}

\begin{authors}
  \author[rassy@math.tu-berlin.de]{Tilman Rassy}
\end{authors}

\version{$Id: ppd_xml.tex,v 1.4 2006/08/11 16:18:23 rassy Exp $}

Personalisierte Aufgaben entstehen dadurch, dass Teile der Daten einer Aufgabe
(z.B. die Koeffizienten einer Linearkombination) f�r jeden Teilnehmer anders
gew�hlt werden. Dies l�sst sich z.B. durch Zufallszahlen oder Zahlen, die aus
der Matrikelnummer generiert werden, realisieren.

Das Aufgaben-XML enth�lt Platzhalter f�r die personalisierten Daten. Diese
geh�ren einem speziellen XML, dem \emph{Personalized Probelm Data XML} oder kurz
\emph{PPD-XML} an. Es ist Gegenstand dieser Spezifikation.

\tableofcontents

\section{Namensraum}

Der Namensraum des PPD-XMLs ist
\val{http://www.mumie.net/xml-namespace/personalized-problem-data}. Der �bliche
Prefix lautet \val{ppd}. Das PPD-XML wird normalerweise mit Prefixes
geschrieben.

\section{Datasheet-Pfade}

Normalerweise werden die Werte, die die personalisierten Daten f�r einen
bestimmten Studenten annehmen, nach ihrer Erzeugung in einem Datasheet
abgelegt. Deshalb m�ssen die Daten im PPD-XML mit Pfaden versehen werden.  S.
hierzu \href{data_sheet_xml\#adressierung}{Adressierung von
  Datasheet-Eintr�gen}.

\section{Elemente}\label{ppd_xml}

Im Folgenden werden alle PPD-XML-Elemente aufgelistet.

\subsection{ppd:copy}\label{element_copy}

Erzeugt eine Kopie eines anderen PPD-Ausdrucks. Prototyp:

\begin{preformatted}[code]%
  <ppd:copy path="\var{path}"/>
\end{preformatted}

\var{path} gibt den Pfad an, unter dem der zu kopierende PPD-Ausdruck im
Datasheet eingeordnet ist.

Beispiel:

\begin{preformatted}[code]%
  <ppd:random_integer path="user/problem/value1" min="1" max="4"/>

  <!-- ... -->

  <ppd:copy path="user/problem/value1"/>
\end{preformatted}

Das erste PPD-Element erzeugt eine zuf�llige ganze Zahl zwischen 1 und 4 (s.
\ref{element_random_integer}), das zweite eine Kopie davon.

\subsection{ppd:random_integer}\label{element_random_integer}

Erzeugt eine zuf�llige ganze Zahl. Prototyp:

\begin{preformatted}[code]%
  <ppd:random_integer path="\var{path}" min="\var{min}" max="\var{max}" \optional{non_zero="\alt{{yes}{no}}"}/>
\end{preformatted}

\var{min} und \var{max} bestimmen den Bereich, in dem die Zufallszahl $x$
liegen soll. Es m�ssen ganze Zahlen mit $\,\mbox{\var{min}} \le
\mbox{\var{max}}$ sein.  Mit \attrib{non_zero} kann festgelegt werden, ob $x$
von Null verschieden sein muss (\val{yes}) oder nicht (\val{no}). Default ist
\val{no}. \var{path} gibt den Pfad an, unter dem $x$ in das Datasheet
eingeordnet wird.

[Genauer gilt: Der Bereich $\Omega$, aus dem $x$ kommt, ist gleich $ \{
\mbox{\var{min}}, \ldots, \mbox{\var{max}} \}\setminus\{0\}$, falls \attrib{non_zero} den
Wert \val{yes} hat, andernfalls $\{ \mbox{\var{min}}, \ldots, \mbox{\var{max}}
\}$. Innerhalb von $\Omega$ sind die Zufallszahlen gleichverteilt.]

\subsection{ppd:random_rational}\label{element_random_rational}

Erzeugt eine zuf�llige rationale Zahl. Prototyp:

\begin{preformatted}[code]%
  <ppd:random_rational path="\var{path}"
                       numerator_min="\var{num_min}" numerator_max="\var{num_max}"
                       denominator_min="\var{den_min}" denominator_max="\var{den_max}"
                       \optional{non_zero="\alt{{yes}{no}}"} \optional{reduce="\alt{{yes}{no}}"}/>
\end{preformatted}

\var{num_min} und \var{num_max} bestimmen den Bereich, in dem der Z�hler der
Zufallszahl $x$ liegen soll, \var{den_min}, \var{den_max} den entsprechenden
Bereich f�r den Nenner. Es m�ssen ganze Zahlen mit $\,\mbox{\var{num_min}} \le
\mbox{\var{num_max}}$ und $\,\mbox{\var{den_min}} \le \mbox{\var{den_max}}$
sein. Mit \attrib{non_zero} kann festgelegt werden, ob $x$ von Null verschieden
sein muss (\val{yes}) oder nicht (\val{no}). Default ist \val{no}. Hat
\attrib{reduce} den Wert \val{yes}, so wird $x$ so weit wie m�glich gek�rzt,
andernfalls nicht gek�rzt.  Default f�r \attrib{reduce} ist \val{yes}.
\var{path} gibt den Pfad an, unter dem $x$ in das Datasheet eingeordnet wird.

[Genauer gilt: Z�hler und Nenner werden zuf�llig und unabh�ngig aus Bereichen
$\Omega_Z$ bzw. $\Omega_N$ gezogen, die wie folgt definiert sind: Hat
\attrib{non_zero} den Wert \val{yes}, so ist $\Omega_Z = \{
\mbox{\var{num_min}}, \ldots, \mbox{\var{num_max}} \}\setminus\{0\}$,
andernfalls $\Omega_Z = \{ \mbox{\var{num_min}}, \ldots, \mbox{\var{num_max}}
\}$. Unabh�ngig von \attrib{non_zero} ist $\Omega_N = \{ \mbox{\var{den_min}},
\ldots, \mbox{\var{den_max}} \}\setminus\{0\}$. Z�hler und Nenner sind in
$\Omega_Z$ bzw. $\Omega_N$ gleichverteilt.]

\subsection{ppd:random_real}\label{element_random_real}

Erzeugt eine zuf�llige reelle Zahl (tats�chlich eine zuf�llige
\code{double}-Zahl im IT-Sinne). Prototyp:

\begin{preformatted}[code]%
  <ppd:random_real path="\var{path}" min="\var{min}" max="\var{max}"/>
\end{preformatted}

\var{min} und \var{max} bestimmen den Bereich, in dem die Zufallszahl $x$
liegen soll. Es m�ssen reelle Zahlen mit $\,\mbox{\var{min}} \le
\mbox{\var{max}}$ sein. \var{path} gibt den Pfad an, unter dem $x$ in das
Datasheet eingeordnet wird.

[Genauer gilt: Der Bereich $\Omega$, aus dem $x$ kommt, ist gleich
$[\mbox{\var{min}}, \mbox{\var{max}}]$. Innerhalb von $\Omega$ sind die
Zufallszahlen gleichverteilt.]



-- ENDE DER DATEI --

\end{document}