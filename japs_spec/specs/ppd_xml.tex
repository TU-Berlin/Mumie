\documentclass{generic}


% ------------------------------------------------------------------------------
% Schriftgroessen
% ------------------------------------------------------------------------------

\renewcommand{\normalsize}{\fontsize{18}{22}\selectfont}
\renewcommand{\large}{\fontsize{20}{26}\selectfont}
\renewcommand{\Large}{\fontsize{22}{30}\selectfont}
\renewcommand{\small}{\fontsize{14}{16}\selectfont}
\newcommand{\smaller}{\fontsize{12}{14}\selectfont}
\newcommand{\verysmall}{\fontsize{6}{6}\selectfont}
\renewcommand{\footnotesize}{\fontsize{12}{14}\selectfont}
\newcommand{\tocsize}{\fontsize{12}{16}\selectfont}
\newcommand{\titlesize}{\fontsize{24}{36}\selectfont}

% ------------------------------------------------------------------------------
% Texthervorhebungen
% ------------------------------------------------------------------------------

%\definecolor{emphcolor}{rgb}{0.77,0.00,0.00}
%\renewcommand{\emph}[1]{\textbf{\textcolor{emphcolor}{#1}}}
\renewcommand{\emph}[1]{\textbf{#1}}

\newcommand{\term}[1]{\textit{#1}}
% \newcommand{\code}[1]{\texttt{#1}}
\newcommand{\var}[1]{\textit{#1}}

%\definecolor{headlinecolor}{rgb}{0.77,0.00, 0.00}
\definecolor{headlinecolor}{rgb}{0.67,0.09, 0.22}
\newcommand{\headline}[1]{\textbf{\textcolor{headlinecolor}{#1}}}

\definecolor{emphboxbordercolor}{rgb}{0.80,0.36,0.36}
\definecolor{emphboxcolor}{rgb}{0.98,0.98,0.77}
\newcommand{\emphbox}[2]{\fcolorbox{emphboxbordercolor}{emphboxcolor}{\parbox{#1}{#2}}}

\definecolor{labelcolor}{rgb}{0.35,0.35, 0.8}


% -------------------------------------------------------------------------------
% Pfeile
% -------------------------------------------------------------------------------

\newcommand{\myarrow}{{\Large\boldmath$\;\rightarrow\;$}}
\newcommand{\mylongarrow}{{\Large\boldmath$\;\longrightarrow\;$}}



% -------------------------------------------------------------------------------
% Befehle f�r Abbildungen 
% -------------------------------------------------------------------------------

\renewcommand{\thefigure}{\thesubsection.\arabic{figure}}

\newcommand{\mycaption}[2]{%
 \centerline{%
  \parbox{#1}{%
   \small \refstepcounter{figure} Figure \thefigure. #2 }}}


%---------------------------------------------------------------------------
% Gliederungsbefehle
%---------------------------------------------------------------------------

\makeatletter

\renewcommand{\section}{\@startsection
 {section}%                                   % Name
 {1}%                                         % Ebene
 {0pt}%                                       % Einzug
 {-0.0\baselineskip}%                        % Vorabstand 
 {0.85\baselineskip}%                         % Nachabstand
 {\fontsize{20}{24}\selectfont\itshape\bfseries}}%    % Stil

\makeatother

\newcommand{\mysection}[1]{%

\vspace*{-1.2\baselineskip}

{\color{headlinecolor}
\section{#1}}}

\newcounter{myappendix}

\newcommand{\myappendix}[2]{%
\hfill\makebox[0cm][l]{\tocref}

\vspace*{-1.2\baselineskip}

\refstepcounter{myappendix}
\section*{A\arabic{myappendix}\hspace{1em}#1}\hypertarget{#2}{}}



% -------------------------------------------------------------------------------
% Inhaltsverzeichnis 
% -------------------------------------------------------------------------------

\newcommand{\mytoctitle}[1]{\begin{center}\textbf{\hypertarget{inhalt}{#1}}\end{center}}

\newenvironment{mytoc}
  {\tocsize\textbf{\hypertarget{inhalt}{Inhalt}}\\\begin{enumerate}}
  {\end{enumerate}}

\newcommand{\mytocitem}[3]{\item[#1]\hyperlink{#3}{#2}}

\newcommand{\tocref}{\verysmall\hyperlink{inhalt}{TOC}\normalsize}

% -------------------------------------------------------------------------------
% Listen 
% -------------------------------------------------------------------------------

%\newcommand{\mylabelitemi}{\labelitemi}
\newcommand{\mylabelitemi}{\raisebox{0.25ex}{{\small $\blacktriangleright$}\hspace{0.1em}}}

%\newcommand{\mylabelitemii}{\labelitemii}
%\newcommand{\mylabelitemii}{\raisebox{0.25ex}{{\small $\triangleright$}\hspace{0.1em}}}
\newcommand{\mylabelitemii}{\raisebox{0.1ex}{\large\bfseries-}\hspace{0.075em}}

\newenvironment{mylist}[1][\mylabelitemi]
  {\begin{list}{{\color{labelcolor}#1}}
    {\setlength{\itemindent}{0cm}
     \setlength{\labelwidth}{1em}
     \setlength{\leftmargin}{0.5cm}
     \setlength{\rightmargin}{1cm}}}
  {\end{list}}

\newenvironment{myenum}
  {\begin{list}{\theenumi}
    {\usecounter{enumi}
     \setlength{\itemindent}{0cm}
     \setlength{\labelwidth}{1em}
     \setlength{\leftmargin}{0.5cm}
     \setlength{\rightmargin}{1cm}}}
  {\end{list}}

\newcommand{\pitem}{\pause\item}
\newcommand{\itemii}{\item[\mylabelitemii]}

\newcounter{romanlistcounter}
\newenvironment{romanlist}{\begin{list}%
 {(\roman{romanlistcounter})\hfill}{\usecounter{romanlistcounter}%
 \topsep0.05\baselineskip% 
 \partopsep0.85\baselineskip%
 \leftmargin2em%
 \labelwidth2em%
 \labelsep0pt%
 \itemindent0pt%
 \parsep0.15\baselineskip%
 \itemsep0.0\baselineskip}}%
 {\end{list}}

\newcounter{alphlistcounter}
\newenvironment{alphlist}{\begin{list}%
 {(\alph{alphlistcounter})\hfill}{\usecounter{alphlistcounter}%
 \topsep0.05\baselineskip% 
 \partopsep0.85\baselineskip%
 \leftmargin1.8em%
 \labelwidth1.8em%
 \labelsep0pt%
 \itemindent0pt%
 \parsep0.15\baselineskip%
 \itemsep0.0\baselineskip}}%
 {\end{list}}

\newcounter{arabiclistcounter}
\newenvironment{arabiclist}{\begin{list}%
 {(\arabic{arabiclistcounter})\hfill}{\usecounter{arabiclistcounter}%
 \topsep0.05\baselineskip% 
 \partopsep0.85\baselineskip%
 \leftmargin1.8em%
 \labelwidth1.8em%
 \labelsep0pt%               
 \itemindent0pt%
 \parsep0.15\baselineskip%
 \itemsep0.0\baselineskip}}%
 {\end{list}}





\begin{document}

\title{Personalisierte-Aufgaben-XML}

\begin{authors}
  \author[rassy@math.tu-berlin.de]{Tilman Rassy}
\end{authors}

\version{$Id: ppd_xml.tex,v 1.4 2006/08/11 16:18:23 rassy Exp $}

Personalisierte Aufgaben entstehen dadurch, dass Teile der Daten einer Aufgabe
(z.B. die Koeffizienten einer Linearkombination) f�r jeden Teilnehmer anders
gew�hlt werden. Dies l�sst sich z.B. durch Zufallszahlen oder Zahlen, die aus
der Matrikelnummer generiert werden, realisieren.

Das Aufgaben-XML enth�lt Platzhalter f�r die personalisierten Daten. Diese
geh�ren einem speziellen XML, dem \emph{Personalized Probelm Data XML} oder kurz
\emph{PPD-XML} an. Es ist Gegenstand dieser Spezifikation.

\tableofcontents

\section{Namensraum}

Der Namensraum des PPD-XMLs ist
\val{http://www.mumie.net/xml-namespace/personalized-problem-data}. Der �bliche
Prefix lautet \val{ppd}. Das PPD-XML wird normalerweise mit Prefixes
geschrieben.

\section{Datasheet-Pfade}

Normalerweise werden die Werte, die die personalisierten Daten f�r einen
bestimmten Studenten annehmen, nach ihrer Erzeugung in einem Datasheet
abgelegt. Deshalb m�ssen die Daten im PPD-XML mit Pfaden versehen werden.  S.
hierzu \href{data_sheet_xml\#adressierung}{Adressierung von
  Datasheet-Eintr�gen}.

\section{Elemente}\label{ppd_xml}

Im Folgenden werden alle PPD-XML-Elemente aufgelistet.

\subsection{ppd:copy}\label{element_copy}

Erzeugt eine Kopie eines anderen PPD-Ausdrucks. Prototyp:

\begin{preformatted}[code]%
  <ppd:copy path="\var{path}"/>
\end{preformatted}

\var{path} gibt den Pfad an, unter dem der zu kopierende PPD-Ausdruck im
Datasheet eingeordnet ist.

Beispiel:

\begin{preformatted}[code]%
  <ppd:random_integer path="user/problem/value1" min="1" max="4"/>

  <!-- ... -->

  <ppd:copy path="user/problem/value1"/>
\end{preformatted}

Das erste PPD-Element erzeugt eine zuf�llige ganze Zahl zwischen 1 und 4 (s.
\ref{element_random_integer}), das zweite eine Kopie davon.

\subsection{ppd:random_integer}\label{element_random_integer}

Erzeugt eine zuf�llige ganze Zahl. Prototyp:

\begin{preformatted}[code]%
  <ppd:random_integer path="\var{path}" min="\var{min}" max="\var{max}" \optional{non_zero="\alt{{yes}{no}}"}/>
\end{preformatted}

\var{min} und \var{max} bestimmen den Bereich, in dem die Zufallszahl $x$
liegen soll. Es m�ssen ganze Zahlen mit $\,\mbox{\var{min}} \le
\mbox{\var{max}}$ sein.  Mit \attrib{non_zero} kann festgelegt werden, ob $x$
von Null verschieden sein muss (\val{yes}) oder nicht (\val{no}). Default ist
\val{no}. \var{path} gibt den Pfad an, unter dem $x$ in das Datasheet
eingeordnet wird.

[Genauer gilt: Der Bereich $\Omega$, aus dem $x$ kommt, ist gleich $ \{
\mbox{\var{min}}, \ldots, \mbox{\var{max}} \}\setminus\{0\}$, falls \attrib{non_zero} den
Wert \val{yes} hat, andernfalls $\{ \mbox{\var{min}}, \ldots, \mbox{\var{max}}
\}$. Innerhalb von $\Omega$ sind die Zufallszahlen gleichverteilt.]

\subsection{ppd:random_rational}\label{element_random_rational}

Erzeugt eine zuf�llige rationale Zahl. Prototyp:

\begin{preformatted}[code]%
  <ppd:random_rational path="\var{path}"
                       numerator_min="\var{num_min}" numerator_max="\var{num_max}"
                       denominator_min="\var{den_min}" denominator_max="\var{den_max}"
                       \optional{non_zero="\alt{{yes}{no}}"} \optional{reduce="\alt{{yes}{no}}"}/>
\end{preformatted}

\var{num_min} und \var{num_max} bestimmen den Bereich, in dem der Z�hler der
Zufallszahl $x$ liegen soll, \var{den_min}, \var{den_max} den entsprechenden
Bereich f�r den Nenner. Es m�ssen ganze Zahlen mit $\,\mbox{\var{num_min}} \le
\mbox{\var{num_max}}$ und $\,\mbox{\var{den_min}} \le \mbox{\var{den_max}}$
sein. Mit \attrib{non_zero} kann festgelegt werden, ob $x$ von Null verschieden
sein muss (\val{yes}) oder nicht (\val{no}). Default ist \val{no}. Hat
\attrib{reduce} den Wert \val{yes}, so wird $x$ so weit wie m�glich gek�rzt,
andernfalls nicht gek�rzt.  Default f�r \attrib{reduce} ist \val{yes}.
\var{path} gibt den Pfad an, unter dem $x$ in das Datasheet eingeordnet wird.

[Genauer gilt: Z�hler und Nenner werden zuf�llig und unabh�ngig aus Bereichen
$\Omega_Z$ bzw. $\Omega_N$ gezogen, die wie folgt definiert sind: Hat
\attrib{non_zero} den Wert \val{yes}, so ist $\Omega_Z = \{
\mbox{\var{num_min}}, \ldots, \mbox{\var{num_max}} \}\setminus\{0\}$,
andernfalls $\Omega_Z = \{ \mbox{\var{num_min}}, \ldots, \mbox{\var{num_max}}
\}$. Unabh�ngig von \attrib{non_zero} ist $\Omega_N = \{ \mbox{\var{den_min}},
\ldots, \mbox{\var{den_max}} \}\setminus\{0\}$. Z�hler und Nenner sind in
$\Omega_Z$ bzw. $\Omega_N$ gleichverteilt.]

\subsection{ppd:random_real}\label{element_random_real}

Erzeugt eine zuf�llige reelle Zahl (tats�chlich eine zuf�llige
\code{double}-Zahl im IT-Sinne). Prototyp:

\begin{preformatted}[code]%
  <ppd:random_real path="\var{path}" min="\var{min}" max="\var{max}"/>
\end{preformatted}

\var{min} und \var{max} bestimmen den Bereich, in dem die Zufallszahl $x$
liegen soll. Es m�ssen reelle Zahlen mit $\,\mbox{\var{min}} \le
\mbox{\var{max}}$ sein. \var{path} gibt den Pfad an, unter dem $x$ in das
Datasheet eingeordnet wird.

[Genauer gilt: Der Bereich $\Omega$, aus dem $x$ kommt, ist gleich
$[\mbox{\var{min}}, \mbox{\var{max}}]$. Innerhalb von $\Omega$ sind die
Zufallszahlen gleichverteilt.]



-- ENDE DER DATEI --

\end{document}