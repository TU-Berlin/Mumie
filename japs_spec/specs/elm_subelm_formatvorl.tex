\documentclass{generic}

% Author: Tilman Rassy <rassy@math.tu-berlin.de>
% $Id: macros.tex,v 1.5 2006/10/05 15:58:42 rassy Exp $

% XML code:
\newcommand{\element}[1]{\code[xml-element]{#1}}
\newcommand{\attrib}[1]{\code[xml-attrib]{#1}}

% Database code:
\newcommand{\dbtable}[1]{\code[db-table]{#1}}
\newcommand{\dbcol}[1]{\code[db-column]{#1}}
\newcommand{\sql}[1]{\code[sql]{#1}}

% TeX code:
\newcommand{\texcmd}[1]{\code[tex-cmd]{\backslash#1}}
\newcommand{\texenv}[1]{\code[tex-env]{#1}}

% General
\newcommand{\val}[1]{\code[value]{#1}}
\newcommand{\prm}[1]{\code[param]{#1}}

% Other:
\newcommand{\notimpl}[1]{\emph[not-impl]{#1}}
\newcommand{\new}[1]{\emph[new]{#1}}
\newcommand{\deprecated}[1]{\emph[deprecated]{#1}}
\newcommand{\wvar}[1]{\var[weak]{#1}}


\newcommand{\b}{\backslash}
\newcommand{\c}[1]{\backslash#1}
\newcommand{\altvar}[1]{\var[alt]{#1}}

\begin{document}

\title{Formatvorlagen}

\begin{authors}
  \author[rassy@math.tu-berlin.de]{Tilman Rassy}
\end{authors}

\version{$Id: elm_subelm_formatvorl.tex,v 1.3 2006/09/12 15:35:45 rassy Exp $}

Diese Spezifikation listet alle Formatvorlagen f�r Elemente und Subelemente auf und gibt
f�r jede Formatvorlage eine Skizze, ein XML-Fragment und ein TeX-Fragment an.

\tableofcontents

\section{Definition}

\subsection{Free}

Skizze:

\image{def\_free.png}

TeX-Fragment:

\begin{preformatted}[code]%
\c{documentclass}[free]\{japs.element.definition\}

\c{begin}\{metainfo\}
  \altvar{Meta-Informationen}
\c{end}\{metainfo\}

\c{begin}\{content\}

  \c{defnotion}\{\altvar{Begriff(e)}\}

  \altvar{Freier Text}

\c{end}\{content\}
\end{preformatted}

XML-Fragment:

\begin{preformatted}[code]%
<elm:element category="definition">

  <elm:defnotions>
    \altvar{Begriff(e)}
  </elm:defnotions>

  \altvar{Freier Text}

</elm:element>
\end{preformatted}


\subsection{Simple}

Skizze:

\image{def\_simple.png}

TeX-Fragment:

\begin{preformatted}[code]%
\c{documentclass}\{japs.element.definition\}

\c{begin}\{metainfo\}
  \altvar{Meta-Informationen}
\c{end}\{metainfo\}

\c{begin}\{content\}

  \c{defnotion}\{\altvar{Begriff(e)}\}

  \c{begin}\{suppositions\}
    \altvar{Voraussetzungen}
  \c{end}\{suppositions\}

  \c{begin}\{statement\}
    \altvar{Text der Definition}
  \c{end}\{statement\}

  \c{begin}\{remarks\}
    \altvar{Bemerkungen}
  \c{end}\{remarks\}

\c{end}\{content\}
\end{preformatted}

XML-Fragment:

\begin{preformatted}[code]%
<elm:element category="definition">

  <elm:defnotions>
    \altvar{Begriff(e)}
  </elm:defnotions>

  <elm:suppositions>
    \altvar{Voraussetzungen}
  </elm:suppositions>

  <elm:statement>
    \altvar{Text der Definition}
  </elm:statement>

  <elm:remarks>
    \altvar{Bemerkungen}
  </elm:remarks>

</elm:element>
\end{preformatted}



\subsection{Side-by-side}

Skizze:

\image{def\_side\_by\_side.png}

TeX-Fragment:

\begin{preformatted}[code]%
\c{documentclass}\{japs.element.definition\}

\c{begin}\{metainfo\}
  \altvar{Meta-Informationen}
\c{end}\{metainfo\}

\c{begin}\{content\}

  \c{defnotion}\{\altvar{Begriff(e)}\}

  \c{begin}\{suppositions\}
    \altvar{Voraussetzungen}
  \c{end}\{suppositions\}

  \c{begin}\{defequivalence\}
    \altvar{Definiertes Subjekt}
    \c{isdefinedas}
    \altvar{Definierender Text}
  \c{end}\{defequivalence\}

  \c{begin}\{remarks\}
    \altvar{Bemerkungen}
  \c{end}\{remarks\}

\c{end}\{content\}
\end{preformatted}

XML-Fragment:

\begin{preformatted}[code]%
<elm:element category="definition">

  <elm:defnotions>
    \altvar{Begriff(e)}
  </elm:defnotions>

  <elm:suppositions>
    \altvar{Voraussetzungen}
  </elm:suppositions>

  <elm:defequivalence arrange="{}side-by-side">
    <elm:statement>
      \altvar{Definiertes Subjekt}
    </elm:statement>
    <elm:statement>
      \altvar{Definierender Text}
    </elm:statement>
  </elm:defequivalence>

  <elm:remarks>
    \altvar{Bemerkungen}
  </elm:remarks>

</elm:element>
\end{preformatted}



\subsection{Top-bottom}

Skizze:

\image{def\_top\_bottom.png}

TeX-Fragment:

\begin{preformatted}[code]%
\c{documentclass}\{japs.element.definition\}

\c{begin}\{metainfo\}
  \altvar{Meta-Informationen}
\c{end}\{metainfo\}

\c{begin}\{content\}

  \c{defnotion}\{\altvar{Begriff(e)}\}

  \c{begin}\{suppositions\}
    \altvar{Voraussetzungen}
  \c{end}\{suppositions\}

  \c{begin}\{defequivalence\}[top-bottom]
    \altvar{Definiertes Subjekt}
    \c{isdefinedas}
    \altvar{Definierender Text}
  \c{end}\{defequivalence\}

  \c{begin}\{remarks\}
    \altvar{Bemerkungen}
  \c{end}\{remarks\}

\c{end}\{content\}
\end{preformatted}

XML-Fragment:

\begin{preformatted}[code]%
<elm:element category="definition">

  <elm:defnotions>
    \altvar{Begriff(e)}
  </elm:defnotions>

  <elm:suppositions>
    \altvar{Voraussetzungen}
  </elm:suppositions>

  <elm:defequivalence arrange="{}top-bottom">
    <elm:statement>
      \altvar{Definiertes Subjekt}
    </elm:statement>
    <elm:statement>
      \altvar{Definierender Text}
    </elm:statement>
  </elm:defequivalence>

  <elm:remarks>
    \altvar{Bemerkungen}
  </elm:remarks>

</elm:element>
\end{preformatted}



\subsection{Mehrfach, side-by-side}

Skizze:

\image{def\_mehrf\_side\_by\_side.png}

TeX-Fragment:

\begin{preformatted}[code]%
\c{documentclass}\{japs.element.definition\}

\c{begin}\{metainfo\}
  \altvar{Meta-Informationen}
\c{end}\{metainfo\}

\c{begin}\{content\}

  \c{defnotion}\{\altvar{Begriff(e)}\}

  \c{begin}\{suppositions\}
    \altvar{Voraussetzungen}
  \c{end}\{suppositions\}

  \c{begin}\{defequivalence\}
    \altvar{Definiertes Subjekt}
    \c{isdefinedas}
    \altvar{Definierender Text}
  \c{end}\{defequivalence\}

  \c{begin}\{defequivalence\}
    \altvar{Definiertes Subjekt}
    \c{isdefinedas}
    \altvar{Definierender Text}
  \c{end}\{defequivalence\}

  \c{begin}\{remarks\}
    \altvar{Bemerkungen}
  \c{end}\{remarks\}

\c{end}\{content\}
\end{preformatted}

XML-Fragment:

\begin{preformatted}[code]%
<elm:element category="definition">

  <elm:defnotions>
    \altvar{Begriff(e)}
  </elm:defnotions>

  <elm:suppositions>
    \altvar{Voraussetzungen}
  </elm:suppositions>

  <elm:defequivalence arrange="{}side-by-side">
    <elm:statement>
      \altvar{Definiertes Subjekt}
    </elm:statement>
    <elm:statement>
      \altvar{Definierender Text}
    </elm:statement>
  </elm:defequivalence>

  <elm:defequivalence arrange="{}side-by-side">
    <elm:statement>
      \altvar{Definiertes Subjekt}
    </elm:statement>
    <elm:statement>
      \altvar{Definierender Text}
    </elm:statement>
  </elm:defequivalence>

  <elm:remarks>
    \altvar{Bemerkungen}
  </elm:remarks>

</elm:element>
\end{preformatted}



\subsection{Mehrfach, top-bottom}

Skizze:

\image{def\_mehrf\_top\_bottom.png}

TeX-Fragment:

\begin{preformatted}[code]%
\c{documentclass}\{japs.element.definition\}

\c{begin}\{metainfo\}
  \altvar{Meta-Informationen}
\c{end}\{metainfo\}

\c{begin}\{content\}

  \c{defnotion}\{\altvar{Begriff(e)}\}

  \c{begin}\{suppositions\}
    \altvar{Voraussetzungen}
  \c{end}\{suppositions\}

  \c{begin}\{defequivalence\}[top-bottom]
    \altvar{Definiertes Subjekt}
    \c{isdefinedas}
    \altvar{Definierender Text}
  \c{end}\{defequivalence\}

  \c{begin}\{defequivalence\}[top-bottom]
    \altvar{Definiertes Subjekt}
    \c{isdefinedas}
    \altvar{Definierender Text}
  \c{end}\{defequivalence\}

  \c{begin}\{remarks\}
    \altvar{Bemerkungen}
  \c{end}\{remarks\}

\c{end}\{content\}
\end{preformatted}

XML-Fragment:

\begin{preformatted}[code]%
<elm:element category="definition">

  <elm:defnotions>
    \altvar{Begriff(e)}
  </elm:defnotions>

  <elm:suppositions>
    \altvar{Voraussetzungen}
  </elm:suppositions>

  <elm:defequivalence arrange="{}top-bottom">
    <elm:statement>
      \altvar{Definiertes Subjekt}
    </elm:statement>
    <elm:statement>
      \altvar{Definierender Text}
    </elm:statement>
  </elm:defequivalence>

  <elm:defequivalence arrange="{}top-bottom">
    <elm:statement>
      \altvar{Definiertes Subjekt}
    </elm:statement>
    <elm:statement>
      \altvar{Definierender Text}
    </elm:statement>
  </elm:defequivalence>

  <elm:remarks>
    \altvar{Bemerkungen}
  </elm:remarks>

</elm:element>
\end{preformatted}




\subsection{Mehrfach, gemischt}

Skizze:

\image{def\_mehrf\_gem.png}

TeX-Fragment:

\begin{preformatted}[code]%
\c{documentclass}\{japs.element.definition\}

\c{begin}\{metainfo\}
  \altvar{Meta-Informationen}
\c{end}\{metainfo\}

\c{begin}\{content\}

  \c{defnotion}\{\altvar{Begriff(e)}\}

  \c{begin}\{suppositions\}
    \altvar{Voraussetzungen}
  \c{end}\{suppositions\}

  \c{begin}\{defequivalence\}
    \altvar{Definiertes Subjekt}
    \c{isdefinedas}
    \altvar{Definierender Text}
  \c{end}\{defequivalence\}

  \c{begin}\{defequivalence\}[top-bottom]
    \altvar{Definiertes Subjekt}
    \c{isdefinedas}
    \altvar{Definierender Text}
  \c{end}\{defequivalence\}

  \c{begin}\{remarks\}
    \altvar{Bemerkungen}
  \c{end}\{remarks\}

\c{end}\{content\}
\end{preformatted}

XML-Fragment:

\begin{preformatted}[code]%
<elm:element category="definition">

  <elm:defnotions>
    \altvar{Begriff(e)}
  </elm:defnotions>

  <elm:suppositions>
    \altvar{Voraussetzungen}
  </elm:suppositions>

  <elm:defequivalence arrange="{}side-by-side">
    <elm:statement>
      \altvar{Definiertes Subjekt}
    </elm:statement>
    <elm:statement>
      \altvar{Definierender Text}
    </elm:statement>
  </elm:defequivalence>

  <elm:defequivalence arrange="{}top-bottom">
    <elm:statement>
      \altvar{Definiertes Subjekt}
    </elm:statement>
    <elm:statement>
      \altvar{Definierender Text}
    </elm:statement>
  </elm:defequivalence>

  <elm:remarks>
    \altvar{Bemerkungen}
  </elm:remarks>

</elm:element>
\end{preformatted}

\section{Theorem}


\subsection{Free}

Skizze:

\image{thm\_free.png}

TeX-Fragment:

\begin{preformatted}[code]%
\c{documentclass}[free]\{japs.element.theorem\}

\c{begin}\{metainfo\}
  \altvar{Meta-Informationen}
\c{end}\{metainfo\}

\c{begin}\{content\}

  \c{title}\{\altvar{Titel}\}

  \altvar{Freier Text}

\c{end}\{content\}
\end{preformatted}

XML-Fragment:

\begin{preformatted}[code]%
<elm:element category="theorem">

  <elm:title>
    \altvar{Titel}
  </elm:title>

  \altvar{Freier Text}

</elm:element>
\end{preformatted}


\subsection{Simple}

Skizze:

\image{thm\_simple.png}

TeX-Fragment:

\begin{preformatted}[code]%
\c{documentclass}\{japs.element.theorem\}

\c{begin}\{metainfo\}
  \altvar{Meta-Informationen}
\c{end}\{metainfo\}

\c{begin}\{content\}

  \c{title}\{\altvar{Titel}\}

  \c{begin}\{suppositions\}
    \altvar{Voraussetzungen}
  \c{end}\{suppositions\}

  \c{begin}\{statement\}
    \altvar{Text des Theorems}
  \c{end}\{statement\}

  \c{begin}\{remarks\}
    \altvar{Bemerkungen}
  \c{end}\{remarks\}

\c{end}\{content\}
\end{preformatted}

XML-Fragment:

\begin{preformatted}[code]%
<elm:element category="theorem">

  <elm:title>
    \altvar{Titel}
  </elm:title>

  <elm:suppositions>
    \altvar{Voraussetzungen}
  </elm:suppositions>

  <elm:statement>
    \altvar{Text des Theorems}
  </elm:statement>

  <elm:remarks>
    \altvar{Bemerkungen}
  </elm:remarks>

</elm:element>
\end{preformatted}


\subsection{Implikation, side-by-side}

Skizze:

\image{thm\_impl\_side\_by\_side.png}

TeX-Fragment:

\begin{preformatted}[code]%
\c{documentclass}\{japs.element.theorem\}

\c{begin}\{metainfo\}
  \altvar{Meta-Informationen}
\c{end}\{metainfo\}

\c{begin}\{content\}

  \c{title}\{\altvar{Titel}\}

  \c{begin}\{suppositions\}
    \altvar{Voraussetzungen}
  \c{end}\{suppositions\}

  \c{begin}\{implication\}
    \altvar{Aussage A}
    \c{implies}
    \altvar{Aussage B}
  \c{end}\{implication\}

  \c{begin}\{remarks\}
    \altvar{Bemerkungen}
  \c{end}\{remarks\}

\c{end}\{content\}
\end{preformatted}

XML-Fragment:

\begin{preformatted}[code]%
<elm:element category="theorem">

  <elm:title>
    \altvar{Titel}
  </elm:title>>

  <elm:suppositions>
    \altvar{Voraussetzungen}
  </elm:suppositions>

  <elm:implication arrange="{}side-by-side">
    <elm:statement>
      \altvar{Aussage A}
    </elm:statement>
    <elm:statement>
      \altvar{Aussage B}
    </elm:statement>
  </elm:implication>

  <elm:remarks>
    \altvar{Bemerkungen}
  </elm:remarks>

</elm:element>
\end{preformatted}


\subsection{Implikation, top-bottom}

Skizze:

\image{thm\_impl\_side\_by\_side.png}

TeX-Fragment:

\begin{preformatted}[code]%
\c{documentclass}\{japs.element.theorem\}

\c{begin}\{metainfo\}
  \altvar{Meta-Informationen}
\c{end}\{metainfo\}

\c{begin}\{content\}

  \c{title}\{\altvar{Titel}\}

  \c{begin}\{suppositions\}
    \altvar{Voraussetzungen}
  \c{end}\{suppositions\}

  \c{begin}\{implication\}[top-bottom]
    \altvar{Aussage A}
    \c{implies}
    \altvar{Aussage B}
  \c{end}\{implication\}

  \c{begin}\{remarks\}
    \altvar{Bemerkungen}
  \c{end}\{remarks\}

\c{end}\{content\}
\end{preformatted}

XML-Fragment:

\begin{preformatted}[code]%
<elm:element category="theorem">

  <elm:title>
    \altvar{Titel}
  </elm:title>>

  <elm:suppositions>
    \altvar{Voraussetzungen}
  </elm:suppositions>

  <elm:implication arrange="{}top-bottom">
    <elm:statement>
      \altvar{Aussage A}
    </elm:statement>
    <elm:statement>
      \altvar{Aussage B}
    </elm:statement>
  </elm:implication>

  <elm:remarks>
    \altvar{Bemerkungen}
  </elm:remarks>

</elm:element>
\end{preformatted}


\subsection{�quivalenz, side-by-side}

Skizze:

\image{thm\_equiv\_side\_by\_side.png}

TeX-Fragment:

\begin{preformatted}[code]%
\c{documentclass}\{japs.element.theorem\}

\c{begin}\{metainfo\}
  \altvar{Meta-Informationen}
\c{end}\{metainfo\}

\c{begin}\{content\}

  \c{title}\{\altvar{Titel}\}

  \c{begin}\{suppositions\}
    \altvar{Voraussetzungen}
  \c{end}\{suppositions\}

  \c{begin}\{equivalence\}
    \altvar{Aussage A}
    \c{isequivalentto}
    \altvar{Aussage B}
  \c{end}\{equivalence\}

  \c{begin}\{remarks\}
    \altvar{Bemerkungen}
  \c{end}\{remarks\}

\c{end}\{content\}
\end{preformatted}

XML-Fragment:

\begin{preformatted}[code]%
<elm:element category="theorem">

  <elm:title>
    \altvar{Titel}
  </elm:title>>

  <elm:suppositions>
    \altvar{Voraussetzungen}
  </elm:suppositions>

  <elm:equivalence arrange="{}side-by-side">
    <elm:statement>
      \altvar{Aussage A}
    </elm:statement>
    <elm:statement>
      \altvar{Aussage B}
    </elm:statement>
  </elm:equivalence>

  <elm:remarks>
    \altvar{Bemerkungen}
  </elm:remarks>

</elm:element>
\end{preformatted}


\subsection{�quivalenz, top-bottom}

Skizze:

\image{thm\_equiv\_side\_by\_side.png}

TeX-Fragment:

\begin{preformatted}[code]%
\c{documentclass}\{japs.element.theorem\}

\c{begin}\{metainfo\}
  \altvar{Meta-Informationen}
\c{end}\{metainfo\}

\c{begin}\{content\}

  \c{title}\{\altvar{Titel}\}

  \c{begin}\{suppositions\}
    \altvar{Voraussetzungen}
  \c{end}\{suppositions\}

  \c{begin}\{equivalence\}[top-bottom]
    \altvar{Aussage A}
    \c{isequivalentto}
    \altvar{Aussage B}
  \c{end}\{equivalence\}

  \c{begin}\{remarks\}
    \altvar{Bemerkungen}
  \c{end}\{remarks\}

\c{end}\{content\}
\end{preformatted}

XML-Fragment:

\begin{preformatted}[code]%
<elm:element category="theorem">

  <elm:title>
    \altvar{Titel}
  </elm:title>>

  <elm:suppositions>
    \altvar{Voraussetzungen}
  </elm:suppositions>

  <elm:equivalence arrange="{}top-bottom">
    <elm:statement>
      \altvar{Aussage A}
    </elm:statement>
    <elm:statement>
      \altvar{Aussage B}
    </elm:statement>
  </elm:equivalence>

  <elm:remarks>
    \altvar{Bemerkungen}
  </elm:remarks>

</elm:element>
\end{preformatted}


\subsection{Mehrfach}

- Analog zu Definitionen -


\subsection{Liste von Aussagen}

Skizze:

\image{thm\_prop\_list.png}


TeX-Fragment:

\begin{preformatted}[code]%
\c{documentclass}\{japs.element.theorem\}

\c{begin}\{metainfo\}
  \altvar{Meta-Informationen}
\c{end}\{metainfo\}

\c{begin}\{content\}

  \c{title}\{\altvar{Titel}\}

  \c{begin}\{suppositions\}
    \altvar{Voraussetzungen}
  \c{end}\{suppositions\}

  \c{begin}\{propositionlist\}
    \c{proposition} \altvar{Aussage A}
    \c{proposition} \altvar{Aussage B}
    \c{proposition} \altvar{Aussage C}
  \c{end}\{propositionlist\}

  \c{begin}\{remarks\}
    \altvar{Bemerkungen}
  \c{end}\{remarks\}

\c{end}\{content\}
\end{preformatted}


XML-Fragment:

\begin{preformatted}[code]%
<elm:element category="theorem">

  <elm:title>
    \altvar{Titel}
  </elm:title>>

  <elm:suppositions>
    \altvar{Voraussetzungen}
  </elm:suppositions>

  <elm:propositionlist>
    <elm:proposition>
      \altvar{Aussage A}
    </elm:proposition>
    <elm:proposition>
      \altvar{Aussage B}
    </elm:proposition>
    <elm:proposition>
      \altvar{Aussage C}
    </elm:proposition>
  </elm:propositionlist>>

  <elm:remarks>
    \altvar{Bemerkungen}
  </elm:remarks>

</elm:element>
\end{preformatted}


\subsection{Liste von �quivalenten Aussagen}

Skizze:

\image{thm\_equiv\_prop\_list.png}


TeX-Fragment:

\begin{preformatted}[code]%
\c{documentclass}\{japs.element.theorem\}

\c{begin}\{metainfo\}
  \altvar{Meta-Informationen}
\c{end}\{metainfo\}

\c{begin}\{content\}

  \c{title}\{\altvar{Titel}\}

  \c{begin}\{suppositions\}
    \altvar{Voraussetzungen}
  \c{end}\{suppositions\}

  \c{begin}\{propositionlist\}[equivalent]
    \c{proposition} \altvar{Aussage A}
    \c{proposition} \altvar{Aussage B}
    \c{proposition} \altvar{Aussage C}
  \c{end}\{propositionlist\}

  \c{begin}\{remarks\}
    \altvar{Bemerkungen}
  \c{end}\{remarks\}

\c{end}\{content\}
\end{preformatted}


XML-Fragment:

\begin{preformatted}[code]%
<elm:element category="theorem">

  <elm:title>
    \altvar{Titel}
  </elm:title>>

  <elm:suppositions>
    \altvar{Voraussetzungen}
  </elm:suppositions>

  <elm:propositionlist variant="equivalent">
    <elm:proposition>
      \altvar{Aussage A}
    </elm:proposition>
    <elm:proposition>
      \altvar{Aussage B}
    </elm:proposition>
    <elm:proposition>
      \altvar{Aussage C}
    </elm:proposition>
  </elm:propositionlist>>

  <elm:remarks>
    \altvar{Bemerkungen}
  </elm:remarks>

</elm:element>
\end{preformatted}

\section{Lemma}

- Wie Theorem -

\section{Algorithmus}

\subsection{Side-by-side}

Skizze:

\image{alg\_side\_by\_side.png}


TeX-Fragment:

\begin{preformatted}[code]%
\c{documentclass}\{japs.element.algorithm\}

\c{begin}\{metainfo\}
  \altvar{Meta-Informationen}
\c{end}\{metainfo\}

\c{begin}\{content\}

  \c{title}\{\altvar{Titel}\}

  \c{begin}\{background\}
    \altvar{Hintergrund}
  \c{end}\{background\}

  \c{begin}\{input\}
    \altvar{Input}
  \c{end}\{input\}

  \c{begin}\{output\}
    \altvar{Output}
  \c{end}\{output\}

  \c{begin}\{algsteps\}[side-by-side]
    \c{algstep} \altvar{Erster Schritt}
    \c{algstep} \altvar{Zweiter Schritt}
    \c{algstep} \altvar{Dritter Schritt}
  \c{end}\{algsteps\}

  \c{begin}\{remarks\}
    \altvar{Bemerkungen}
  \c{end}\{remarks\}

\c{end}\{content\}
\end{preformatted}


XML-Fragment:

\begin{preformatted}[code]%
<elm:element category="theorem">

  <elm:title>
    \altvar{Titel}
  </elm:title>>

  <elm:background>
    \altvar{Hintergrund}
  </elm:background>

  <elm:input>
    \altvar{Input}
  </elm:input>

  <elm:output>
    \altvar{Output}
  </elm:output>

  <elm:algsteps arrange="side-by-side">
    <elm:algstep>
      \altvar{Erster Schritt}
    </elm:algstep>
    <elm:algstep>
      \altvar{Zweiter Schritt}
    </elm:algstep>
    <elm:algstep>
      \altvar{Dritter Schritt}
    </elm:algstep>
  </elm:algsteps>>

  <elm:remarks>
    \altvar{Bemerkungen}
  </elm:remarks>

</elm:element>
\end{preformatted}

\subsection{Top-bottom}

Skizze:

\image{alg\_top\_bottom.png}


TeX-Fragment:

\begin{preformatted}[code]%
\c{documentclass}\{japs.element.algorithm\}

\c{begin}\{metainfo\}
  \altvar{Meta-Informationen}
\c{end}\{metainfo\}

\c{begin}\{content\}

  \c{title}\{\altvar{Titel}\}

  \c{begin}\{background\}
    \altvar{Hintergrund}
  \c{end}\{background\}

  \c{begin}\{input\}
    \altvar{Input}
  \c{end}\{input\}

  \c{begin}\{output\}
    \altvar{Output}
  \c{end}\{output\}

  \c{begin}\{algsteps\}
    \c{algstep} \altvar{Erster Schritt}
    \c{algstep} \altvar{Zweiter Schritt}
    \c{algstep} \altvar{Dritter Schritt}
  \c{end}\{algsteps\}

  \c{begin}\{remarks\}
    \altvar{Bemerkungen}
  \c{end}\{remarks\}

\c{end}\{content\}
\end{preformatted}


XML-Fragment:

\begin{preformatted}[code]%
<elm:element category="theorem">

  <elm:title>
    \altvar{Titel}
  </elm:title>>

  <elm:background>
    \altvar{Hintergrund}
  </elm:background>

  <elm:input>
    \altvar{Input}
  </elm:input>

  <elm:output>
    \altvar{Output}
  </elm:output>

  <elm:algsteps arrange="top-bottom">
    <elm:algstep>
      \altvar{Erster Schritt}
    </elm:algstep>
    <elm:algstep>
      \altvar{Zweiter Schritt}
    </elm:algstep>
    <elm:algstep>
      \altvar{Dritter Schritt}
    </elm:algstep>
  </elm:algsteps>>

  <elm:remarks>
    \altvar{Bemerkungen}
  </elm:remarks>

</elm:element>
\end{preformatted}

-- ENDE DER DATEI --

\end{document}