\documentclass{generic}


% ------------------------------------------------------------------------------
% Schriftgroessen
% ------------------------------------------------------------------------------

\renewcommand{\normalsize}{\fontsize{18}{22}\selectfont}
\renewcommand{\large}{\fontsize{20}{26}\selectfont}
\renewcommand{\Large}{\fontsize{22}{30}\selectfont}
\renewcommand{\small}{\fontsize{14}{16}\selectfont}
\newcommand{\smaller}{\fontsize{12}{14}\selectfont}
\newcommand{\verysmall}{\fontsize{6}{6}\selectfont}
\renewcommand{\footnotesize}{\fontsize{12}{14}\selectfont}
\newcommand{\tocsize}{\fontsize{12}{16}\selectfont}
\newcommand{\titlesize}{\fontsize{24}{36}\selectfont}

% ------------------------------------------------------------------------------
% Texthervorhebungen
% ------------------------------------------------------------------------------

%\definecolor{emphcolor}{rgb}{0.77,0.00,0.00}
%\renewcommand{\emph}[1]{\textbf{\textcolor{emphcolor}{#1}}}
\renewcommand{\emph}[1]{\textbf{#1}}

\newcommand{\term}[1]{\textit{#1}}
% \newcommand{\code}[1]{\texttt{#1}}
\newcommand{\var}[1]{\textit{#1}}

%\definecolor{headlinecolor}{rgb}{0.77,0.00, 0.00}
\definecolor{headlinecolor}{rgb}{0.67,0.09, 0.22}
\newcommand{\headline}[1]{\textbf{\textcolor{headlinecolor}{#1}}}

\definecolor{emphboxbordercolor}{rgb}{0.80,0.36,0.36}
\definecolor{emphboxcolor}{rgb}{0.98,0.98,0.77}
\newcommand{\emphbox}[2]{\fcolorbox{emphboxbordercolor}{emphboxcolor}{\parbox{#1}{#2}}}

\definecolor{labelcolor}{rgb}{0.35,0.35, 0.8}


% -------------------------------------------------------------------------------
% Pfeile
% -------------------------------------------------------------------------------

\newcommand{\myarrow}{{\Large\boldmath$\;\rightarrow\;$}}
\newcommand{\mylongarrow}{{\Large\boldmath$\;\longrightarrow\;$}}



% -------------------------------------------------------------------------------
% Befehle f�r Abbildungen 
% -------------------------------------------------------------------------------

\renewcommand{\thefigure}{\thesubsection.\arabic{figure}}

\newcommand{\mycaption}[2]{%
 \centerline{%
  \parbox{#1}{%
   \small \refstepcounter{figure} Figure \thefigure. #2 }}}


%---------------------------------------------------------------------------
% Gliederungsbefehle
%---------------------------------------------------------------------------

\makeatletter

\renewcommand{\section}{\@startsection
 {section}%                                   % Name
 {1}%                                         % Ebene
 {0pt}%                                       % Einzug
 {-0.0\baselineskip}%                        % Vorabstand 
 {0.85\baselineskip}%                         % Nachabstand
 {\fontsize{20}{24}\selectfont\itshape\bfseries}}%    % Stil

\makeatother

\newcommand{\mysection}[1]{%

\vspace*{-1.2\baselineskip}

{\color{headlinecolor}
\section{#1}}}

\newcounter{myappendix}

\newcommand{\myappendix}[2]{%
\hfill\makebox[0cm][l]{\tocref}

\vspace*{-1.2\baselineskip}

\refstepcounter{myappendix}
\section*{A\arabic{myappendix}\hspace{1em}#1}\hypertarget{#2}{}}



% -------------------------------------------------------------------------------
% Inhaltsverzeichnis 
% -------------------------------------------------------------------------------

\newcommand{\mytoctitle}[1]{\begin{center}\textbf{\hypertarget{inhalt}{#1}}\end{center}}

\newenvironment{mytoc}
  {\tocsize\textbf{\hypertarget{inhalt}{Inhalt}}\\\begin{enumerate}}
  {\end{enumerate}}

\newcommand{\mytocitem}[3]{\item[#1]\hyperlink{#3}{#2}}

\newcommand{\tocref}{\verysmall\hyperlink{inhalt}{TOC}\normalsize}

% -------------------------------------------------------------------------------
% Listen 
% -------------------------------------------------------------------------------

%\newcommand{\mylabelitemi}{\labelitemi}
\newcommand{\mylabelitemi}{\raisebox{0.25ex}{{\small $\blacktriangleright$}\hspace{0.1em}}}

%\newcommand{\mylabelitemii}{\labelitemii}
%\newcommand{\mylabelitemii}{\raisebox{0.25ex}{{\small $\triangleright$}\hspace{0.1em}}}
\newcommand{\mylabelitemii}{\raisebox{0.1ex}{\large\bfseries-}\hspace{0.075em}}

\newenvironment{mylist}[1][\mylabelitemi]
  {\begin{list}{{\color{labelcolor}#1}}
    {\setlength{\itemindent}{0cm}
     \setlength{\labelwidth}{1em}
     \setlength{\leftmargin}{0.5cm}
     \setlength{\rightmargin}{1cm}}}
  {\end{list}}

\newenvironment{myenum}
  {\begin{list}{\theenumi}
    {\usecounter{enumi}
     \setlength{\itemindent}{0cm}
     \setlength{\labelwidth}{1em}
     \setlength{\leftmargin}{0.5cm}
     \setlength{\rightmargin}{1cm}}}
  {\end{list}}

\newcommand{\pitem}{\pause\item}
\newcommand{\itemii}{\item[\mylabelitemii]}

\newcounter{romanlistcounter}
\newenvironment{romanlist}{\begin{list}%
 {(\roman{romanlistcounter})\hfill}{\usecounter{romanlistcounter}%
 \topsep0.05\baselineskip% 
 \partopsep0.85\baselineskip%
 \leftmargin2em%
 \labelwidth2em%
 \labelsep0pt%
 \itemindent0pt%
 \parsep0.15\baselineskip%
 \itemsep0.0\baselineskip}}%
 {\end{list}}

\newcounter{alphlistcounter}
\newenvironment{alphlist}{\begin{list}%
 {(\alph{alphlistcounter})\hfill}{\usecounter{alphlistcounter}%
 \topsep0.05\baselineskip% 
 \partopsep0.85\baselineskip%
 \leftmargin1.8em%
 \labelwidth1.8em%
 \labelsep0pt%
 \itemindent0pt%
 \parsep0.15\baselineskip%
 \itemsep0.0\baselineskip}}%
 {\end{list}}

\newcounter{arabiclistcounter}
\newenvironment{arabiclist}{\begin{list}%
 {(\arabic{arabiclistcounter})\hfill}{\usecounter{arabiclistcounter}%
 \topsep0.05\baselineskip% 
 \partopsep0.85\baselineskip%
 \leftmargin1.8em%
 \labelwidth1.8em%
 \labelsep0pt%               
 \itemindent0pt%
 \parsep0.15\baselineskip%
 \itemsep0.0\baselineskip}}%
 {\end{list}}





\newcommand{\b}{\backslash}
\newcommand{\c}[1]{\backslash#1}
\newcommand{\altvar}[1]{\var[alt]{#1}}

\begin{document}

\title{Formatvorlagen}

\begin{authors}
  \author[rassy@math.tu-berlin.de]{Tilman Rassy}
\end{authors}

\version{$Id: elm_subelm_formatvorl.tex,v 1.3 2006/09/12 15:35:45 rassy Exp $}

Diese Spezifikation listet alle Formatvorlagen f�r Elemente und Subelemente auf und gibt
f�r jede Formatvorlage eine Skizze, ein XML-Fragment und ein TeX-Fragment an.

\tableofcontents

\section{Definition}

\subsection{Free}

Skizze:

\image{def\_free.png}

TeX-Fragment:

\begin{preformatted}[code]%
\c{documentclass}[free]\{japs.element.definition\}

\c{begin}\{metainfo\}
  \altvar{Meta-Informationen}
\c{end}\{metainfo\}

\c{begin}\{content\}

  \c{defnotion}\{\altvar{Begriff(e)}\}

  \altvar{Freier Text}

\c{end}\{content\}
\end{preformatted}

XML-Fragment:

\begin{preformatted}[code]%
<elm:element category="definition">

  <elm:defnotions>
    \altvar{Begriff(e)}
  </elm:defnotions>

  \altvar{Freier Text}

</elm:element>
\end{preformatted}


\subsection{Simple}

Skizze:

\image{def\_simple.png}

TeX-Fragment:

\begin{preformatted}[code]%
\c{documentclass}\{japs.element.definition\}

\c{begin}\{metainfo\}
  \altvar{Meta-Informationen}
\c{end}\{metainfo\}

\c{begin}\{content\}

  \c{defnotion}\{\altvar{Begriff(e)}\}

  \c{begin}\{suppositions\}
    \altvar{Voraussetzungen}
  \c{end}\{suppositions\}

  \c{begin}\{statement\}
    \altvar{Text der Definition}
  \c{end}\{statement\}

  \c{begin}\{remarks\}
    \altvar{Bemerkungen}
  \c{end}\{remarks\}

\c{end}\{content\}
\end{preformatted}

XML-Fragment:

\begin{preformatted}[code]%
<elm:element category="definition">

  <elm:defnotions>
    \altvar{Begriff(e)}
  </elm:defnotions>

  <elm:suppositions>
    \altvar{Voraussetzungen}
  </elm:suppositions>

  <elm:statement>
    \altvar{Text der Definition}
  </elm:statement>

  <elm:remarks>
    \altvar{Bemerkungen}
  </elm:remarks>

</elm:element>
\end{preformatted}



\subsection{Side-by-side}

Skizze:

\image{def\_side\_by\_side.png}

TeX-Fragment:

\begin{preformatted}[code]%
\c{documentclass}\{japs.element.definition\}

\c{begin}\{metainfo\}
  \altvar{Meta-Informationen}
\c{end}\{metainfo\}

\c{begin}\{content\}

  \c{defnotion}\{\altvar{Begriff(e)}\}

  \c{begin}\{suppositions\}
    \altvar{Voraussetzungen}
  \c{end}\{suppositions\}

  \c{begin}\{defequivalence\}
    \altvar{Definiertes Subjekt}
    \c{isdefinedas}
    \altvar{Definierender Text}
  \c{end}\{defequivalence\}

  \c{begin}\{remarks\}
    \altvar{Bemerkungen}
  \c{end}\{remarks\}

\c{end}\{content\}
\end{preformatted}

XML-Fragment:

\begin{preformatted}[code]%
<elm:element category="definition">

  <elm:defnotions>
    \altvar{Begriff(e)}
  </elm:defnotions>

  <elm:suppositions>
    \altvar{Voraussetzungen}
  </elm:suppositions>

  <elm:defequivalence arrange="{}side-by-side">
    <elm:statement>
      \altvar{Definiertes Subjekt}
    </elm:statement>
    <elm:statement>
      \altvar{Definierender Text}
    </elm:statement>
  </elm:defequivalence>

  <elm:remarks>
    \altvar{Bemerkungen}
  </elm:remarks>

</elm:element>
\end{preformatted}



\subsection{Top-bottom}

Skizze:

\image{def\_top\_bottom.png}

TeX-Fragment:

\begin{preformatted}[code]%
\c{documentclass}\{japs.element.definition\}

\c{begin}\{metainfo\}
  \altvar{Meta-Informationen}
\c{end}\{metainfo\}

\c{begin}\{content\}

  \c{defnotion}\{\altvar{Begriff(e)}\}

  \c{begin}\{suppositions\}
    \altvar{Voraussetzungen}
  \c{end}\{suppositions\}

  \c{begin}\{defequivalence\}[top-bottom]
    \altvar{Definiertes Subjekt}
    \c{isdefinedas}
    \altvar{Definierender Text}
  \c{end}\{defequivalence\}

  \c{begin}\{remarks\}
    \altvar{Bemerkungen}
  \c{end}\{remarks\}

\c{end}\{content\}
\end{preformatted}

XML-Fragment:

\begin{preformatted}[code]%
<elm:element category="definition">

  <elm:defnotions>
    \altvar{Begriff(e)}
  </elm:defnotions>

  <elm:suppositions>
    \altvar{Voraussetzungen}
  </elm:suppositions>

  <elm:defequivalence arrange="{}top-bottom">
    <elm:statement>
      \altvar{Definiertes Subjekt}
    </elm:statement>
    <elm:statement>
      \altvar{Definierender Text}
    </elm:statement>
  </elm:defequivalence>

  <elm:remarks>
    \altvar{Bemerkungen}
  </elm:remarks>

</elm:element>
\end{preformatted}



\subsection{Mehrfach, side-by-side}

Skizze:

\image{def\_mehrf\_side\_by\_side.png}

TeX-Fragment:

\begin{preformatted}[code]%
\c{documentclass}\{japs.element.definition\}

\c{begin}\{metainfo\}
  \altvar{Meta-Informationen}
\c{end}\{metainfo\}

\c{begin}\{content\}

  \c{defnotion}\{\altvar{Begriff(e)}\}

  \c{begin}\{suppositions\}
    \altvar{Voraussetzungen}
  \c{end}\{suppositions\}

  \c{begin}\{defequivalence\}
    \altvar{Definiertes Subjekt}
    \c{isdefinedas}
    \altvar{Definierender Text}
  \c{end}\{defequivalence\}

  \c{begin}\{defequivalence\}
    \altvar{Definiertes Subjekt}
    \c{isdefinedas}
    \altvar{Definierender Text}
  \c{end}\{defequivalence\}

  \c{begin}\{remarks\}
    \altvar{Bemerkungen}
  \c{end}\{remarks\}

\c{end}\{content\}
\end{preformatted}

XML-Fragment:

\begin{preformatted}[code]%
<elm:element category="definition">

  <elm:defnotions>
    \altvar{Begriff(e)}
  </elm:defnotions>

  <elm:suppositions>
    \altvar{Voraussetzungen}
  </elm:suppositions>

  <elm:defequivalence arrange="{}side-by-side">
    <elm:statement>
      \altvar{Definiertes Subjekt}
    </elm:statement>
    <elm:statement>
      \altvar{Definierender Text}
    </elm:statement>
  </elm:defequivalence>

  <elm:defequivalence arrange="{}side-by-side">
    <elm:statement>
      \altvar{Definiertes Subjekt}
    </elm:statement>
    <elm:statement>
      \altvar{Definierender Text}
    </elm:statement>
  </elm:defequivalence>

  <elm:remarks>
    \altvar{Bemerkungen}
  </elm:remarks>

</elm:element>
\end{preformatted}



\subsection{Mehrfach, top-bottom}

Skizze:

\image{def\_mehrf\_top\_bottom.png}

TeX-Fragment:

\begin{preformatted}[code]%
\c{documentclass}\{japs.element.definition\}

\c{begin}\{metainfo\}
  \altvar{Meta-Informationen}
\c{end}\{metainfo\}

\c{begin}\{content\}

  \c{defnotion}\{\altvar{Begriff(e)}\}

  \c{begin}\{suppositions\}
    \altvar{Voraussetzungen}
  \c{end}\{suppositions\}

  \c{begin}\{defequivalence\}[top-bottom]
    \altvar{Definiertes Subjekt}
    \c{isdefinedas}
    \altvar{Definierender Text}
  \c{end}\{defequivalence\}

  \c{begin}\{defequivalence\}[top-bottom]
    \altvar{Definiertes Subjekt}
    \c{isdefinedas}
    \altvar{Definierender Text}
  \c{end}\{defequivalence\}

  \c{begin}\{remarks\}
    \altvar{Bemerkungen}
  \c{end}\{remarks\}

\c{end}\{content\}
\end{preformatted}

XML-Fragment:

\begin{preformatted}[code]%
<elm:element category="definition">

  <elm:defnotions>
    \altvar{Begriff(e)}
  </elm:defnotions>

  <elm:suppositions>
    \altvar{Voraussetzungen}
  </elm:suppositions>

  <elm:defequivalence arrange="{}top-bottom">
    <elm:statement>
      \altvar{Definiertes Subjekt}
    </elm:statement>
    <elm:statement>
      \altvar{Definierender Text}
    </elm:statement>
  </elm:defequivalence>

  <elm:defequivalence arrange="{}top-bottom">
    <elm:statement>
      \altvar{Definiertes Subjekt}
    </elm:statement>
    <elm:statement>
      \altvar{Definierender Text}
    </elm:statement>
  </elm:defequivalence>

  <elm:remarks>
    \altvar{Bemerkungen}
  </elm:remarks>

</elm:element>
\end{preformatted}




\subsection{Mehrfach, gemischt}

Skizze:

\image{def\_mehrf\_gem.png}

TeX-Fragment:

\begin{preformatted}[code]%
\c{documentclass}\{japs.element.definition\}

\c{begin}\{metainfo\}
  \altvar{Meta-Informationen}
\c{end}\{metainfo\}

\c{begin}\{content\}

  \c{defnotion}\{\altvar{Begriff(e)}\}

  \c{begin}\{suppositions\}
    \altvar{Voraussetzungen}
  \c{end}\{suppositions\}

  \c{begin}\{defequivalence\}
    \altvar{Definiertes Subjekt}
    \c{isdefinedas}
    \altvar{Definierender Text}
  \c{end}\{defequivalence\}

  \c{begin}\{defequivalence\}[top-bottom]
    \altvar{Definiertes Subjekt}
    \c{isdefinedas}
    \altvar{Definierender Text}
  \c{end}\{defequivalence\}

  \c{begin}\{remarks\}
    \altvar{Bemerkungen}
  \c{end}\{remarks\}

\c{end}\{content\}
\end{preformatted}

XML-Fragment:

\begin{preformatted}[code]%
<elm:element category="definition">

  <elm:defnotions>
    \altvar{Begriff(e)}
  </elm:defnotions>

  <elm:suppositions>
    \altvar{Voraussetzungen}
  </elm:suppositions>

  <elm:defequivalence arrange="{}side-by-side">
    <elm:statement>
      \altvar{Definiertes Subjekt}
    </elm:statement>
    <elm:statement>
      \altvar{Definierender Text}
    </elm:statement>
  </elm:defequivalence>

  <elm:defequivalence arrange="{}top-bottom">
    <elm:statement>
      \altvar{Definiertes Subjekt}
    </elm:statement>
    <elm:statement>
      \altvar{Definierender Text}
    </elm:statement>
  </elm:defequivalence>

  <elm:remarks>
    \altvar{Bemerkungen}
  </elm:remarks>

</elm:element>
\end{preformatted}

\section{Theorem}


\subsection{Free}

Skizze:

\image{thm\_free.png}

TeX-Fragment:

\begin{preformatted}[code]%
\c{documentclass}[free]\{japs.element.theorem\}

\c{begin}\{metainfo\}
  \altvar{Meta-Informationen}
\c{end}\{metainfo\}

\c{begin}\{content\}

  \c{title}\{\altvar{Titel}\}

  \altvar{Freier Text}

\c{end}\{content\}
\end{preformatted}

XML-Fragment:

\begin{preformatted}[code]%
<elm:element category="theorem">

  <elm:title>
    \altvar{Titel}
  </elm:title>

  \altvar{Freier Text}

</elm:element>
\end{preformatted}


\subsection{Simple}

Skizze:

\image{thm\_simple.png}

TeX-Fragment:

\begin{preformatted}[code]%
\c{documentclass}\{japs.element.theorem\}

\c{begin}\{metainfo\}
  \altvar{Meta-Informationen}
\c{end}\{metainfo\}

\c{begin}\{content\}

  \c{title}\{\altvar{Titel}\}

  \c{begin}\{suppositions\}
    \altvar{Voraussetzungen}
  \c{end}\{suppositions\}

  \c{begin}\{statement\}
    \altvar{Text des Theorems}
  \c{end}\{statement\}

  \c{begin}\{remarks\}
    \altvar{Bemerkungen}
  \c{end}\{remarks\}

\c{end}\{content\}
\end{preformatted}

XML-Fragment:

\begin{preformatted}[code]%
<elm:element category="theorem">

  <elm:title>
    \altvar{Titel}
  </elm:title>

  <elm:suppositions>
    \altvar{Voraussetzungen}
  </elm:suppositions>

  <elm:statement>
    \altvar{Text des Theorems}
  </elm:statement>

  <elm:remarks>
    \altvar{Bemerkungen}
  </elm:remarks>

</elm:element>
\end{preformatted}


\subsection{Implikation, side-by-side}

Skizze:

\image{thm\_impl\_side\_by\_side.png}

TeX-Fragment:

\begin{preformatted}[code]%
\c{documentclass}\{japs.element.theorem\}

\c{begin}\{metainfo\}
  \altvar{Meta-Informationen}
\c{end}\{metainfo\}

\c{begin}\{content\}

  \c{title}\{\altvar{Titel}\}

  \c{begin}\{suppositions\}
    \altvar{Voraussetzungen}
  \c{end}\{suppositions\}

  \c{begin}\{implication\}
    \altvar{Aussage A}
    \c{implies}
    \altvar{Aussage B}
  \c{end}\{implication\}

  \c{begin}\{remarks\}
    \altvar{Bemerkungen}
  \c{end}\{remarks\}

\c{end}\{content\}
\end{preformatted}

XML-Fragment:

\begin{preformatted}[code]%
<elm:element category="theorem">

  <elm:title>
    \altvar{Titel}
  </elm:title>>

  <elm:suppositions>
    \altvar{Voraussetzungen}
  </elm:suppositions>

  <elm:implication arrange="{}side-by-side">
    <elm:statement>
      \altvar{Aussage A}
    </elm:statement>
    <elm:statement>
      \altvar{Aussage B}
    </elm:statement>
  </elm:implication>

  <elm:remarks>
    \altvar{Bemerkungen}
  </elm:remarks>

</elm:element>
\end{preformatted}


\subsection{Implikation, top-bottom}

Skizze:

\image{thm\_impl\_side\_by\_side.png}

TeX-Fragment:

\begin{preformatted}[code]%
\c{documentclass}\{japs.element.theorem\}

\c{begin}\{metainfo\}
  \altvar{Meta-Informationen}
\c{end}\{metainfo\}

\c{begin}\{content\}

  \c{title}\{\altvar{Titel}\}

  \c{begin}\{suppositions\}
    \altvar{Voraussetzungen}
  \c{end}\{suppositions\}

  \c{begin}\{implication\}[top-bottom]
    \altvar{Aussage A}
    \c{implies}
    \altvar{Aussage B}
  \c{end}\{implication\}

  \c{begin}\{remarks\}
    \altvar{Bemerkungen}
  \c{end}\{remarks\}

\c{end}\{content\}
\end{preformatted}

XML-Fragment:

\begin{preformatted}[code]%
<elm:element category="theorem">

  <elm:title>
    \altvar{Titel}
  </elm:title>>

  <elm:suppositions>
    \altvar{Voraussetzungen}
  </elm:suppositions>

  <elm:implication arrange="{}top-bottom">
    <elm:statement>
      \altvar{Aussage A}
    </elm:statement>
    <elm:statement>
      \altvar{Aussage B}
    </elm:statement>
  </elm:implication>

  <elm:remarks>
    \altvar{Bemerkungen}
  </elm:remarks>

</elm:element>
\end{preformatted}


\subsection{�quivalenz, side-by-side}

Skizze:

\image{thm\_equiv\_side\_by\_side.png}

TeX-Fragment:

\begin{preformatted}[code]%
\c{documentclass}\{japs.element.theorem\}

\c{begin}\{metainfo\}
  \altvar{Meta-Informationen}
\c{end}\{metainfo\}

\c{begin}\{content\}

  \c{title}\{\altvar{Titel}\}

  \c{begin}\{suppositions\}
    \altvar{Voraussetzungen}
  \c{end}\{suppositions\}

  \c{begin}\{equivalence\}
    \altvar{Aussage A}
    \c{isequivalentto}
    \altvar{Aussage B}
  \c{end}\{equivalence\}

  \c{begin}\{remarks\}
    \altvar{Bemerkungen}
  \c{end}\{remarks\}

\c{end}\{content\}
\end{preformatted}

XML-Fragment:

\begin{preformatted}[code]%
<elm:element category="theorem">

  <elm:title>
    \altvar{Titel}
  </elm:title>>

  <elm:suppositions>
    \altvar{Voraussetzungen}
  </elm:suppositions>

  <elm:equivalence arrange="{}side-by-side">
    <elm:statement>
      \altvar{Aussage A}
    </elm:statement>
    <elm:statement>
      \altvar{Aussage B}
    </elm:statement>
  </elm:equivalence>

  <elm:remarks>
    \altvar{Bemerkungen}
  </elm:remarks>

</elm:element>
\end{preformatted}


\subsection{�quivalenz, top-bottom}

Skizze:

\image{thm\_equiv\_side\_by\_side.png}

TeX-Fragment:

\begin{preformatted}[code]%
\c{documentclass}\{japs.element.theorem\}

\c{begin}\{metainfo\}
  \altvar{Meta-Informationen}
\c{end}\{metainfo\}

\c{begin}\{content\}

  \c{title}\{\altvar{Titel}\}

  \c{begin}\{suppositions\}
    \altvar{Voraussetzungen}
  \c{end}\{suppositions\}

  \c{begin}\{equivalence\}[top-bottom]
    \altvar{Aussage A}
    \c{isequivalentto}
    \altvar{Aussage B}
  \c{end}\{equivalence\}

  \c{begin}\{remarks\}
    \altvar{Bemerkungen}
  \c{end}\{remarks\}

\c{end}\{content\}
\end{preformatted}

XML-Fragment:

\begin{preformatted}[code]%
<elm:element category="theorem">

  <elm:title>
    \altvar{Titel}
  </elm:title>>

  <elm:suppositions>
    \altvar{Voraussetzungen}
  </elm:suppositions>

  <elm:equivalence arrange="{}top-bottom">
    <elm:statement>
      \altvar{Aussage A}
    </elm:statement>
    <elm:statement>
      \altvar{Aussage B}
    </elm:statement>
  </elm:equivalence>

  <elm:remarks>
    \altvar{Bemerkungen}
  </elm:remarks>

</elm:element>
\end{preformatted}


\subsection{Mehrfach}

- Analog zu Definitionen -


\subsection{Liste von Aussagen}

Skizze:

\image{thm\_prop\_list.png}


TeX-Fragment:

\begin{preformatted}[code]%
\c{documentclass}\{japs.element.theorem\}

\c{begin}\{metainfo\}
  \altvar{Meta-Informationen}
\c{end}\{metainfo\}

\c{begin}\{content\}

  \c{title}\{\altvar{Titel}\}

  \c{begin}\{suppositions\}
    \altvar{Voraussetzungen}
  \c{end}\{suppositions\}

  \c{begin}\{propositionlist\}
    \c{proposition} \altvar{Aussage A}
    \c{proposition} \altvar{Aussage B}
    \c{proposition} \altvar{Aussage C}
  \c{end}\{propositionlist\}

  \c{begin}\{remarks\}
    \altvar{Bemerkungen}
  \c{end}\{remarks\}

\c{end}\{content\}
\end{preformatted}


XML-Fragment:

\begin{preformatted}[code]%
<elm:element category="theorem">

  <elm:title>
    \altvar{Titel}
  </elm:title>>

  <elm:suppositions>
    \altvar{Voraussetzungen}
  </elm:suppositions>

  <elm:propositionlist>
    <elm:proposition>
      \altvar{Aussage A}
    </elm:proposition>
    <elm:proposition>
      \altvar{Aussage B}
    </elm:proposition>
    <elm:proposition>
      \altvar{Aussage C}
    </elm:proposition>
  </elm:propositionlist>>

  <elm:remarks>
    \altvar{Bemerkungen}
  </elm:remarks>

</elm:element>
\end{preformatted}


\subsection{Liste von �quivalenten Aussagen}

Skizze:

\image{thm\_equiv\_prop\_list.png}


TeX-Fragment:

\begin{preformatted}[code]%
\c{documentclass}\{japs.element.theorem\}

\c{begin}\{metainfo\}
  \altvar{Meta-Informationen}
\c{end}\{metainfo\}

\c{begin}\{content\}

  \c{title}\{\altvar{Titel}\}

  \c{begin}\{suppositions\}
    \altvar{Voraussetzungen}
  \c{end}\{suppositions\}

  \c{begin}\{propositionlist\}[equivalent]
    \c{proposition} \altvar{Aussage A}
    \c{proposition} \altvar{Aussage B}
    \c{proposition} \altvar{Aussage C}
  \c{end}\{propositionlist\}

  \c{begin}\{remarks\}
    \altvar{Bemerkungen}
  \c{end}\{remarks\}

\c{end}\{content\}
\end{preformatted}


XML-Fragment:

\begin{preformatted}[code]%
<elm:element category="theorem">

  <elm:title>
    \altvar{Titel}
  </elm:title>>

  <elm:suppositions>
    \altvar{Voraussetzungen}
  </elm:suppositions>

  <elm:propositionlist variant="equivalent">
    <elm:proposition>
      \altvar{Aussage A}
    </elm:proposition>
    <elm:proposition>
      \altvar{Aussage B}
    </elm:proposition>
    <elm:proposition>
      \altvar{Aussage C}
    </elm:proposition>
  </elm:propositionlist>>

  <elm:remarks>
    \altvar{Bemerkungen}
  </elm:remarks>

</elm:element>
\end{preformatted}

\section{Lemma}

- Wie Theorem -

\section{Algorithmus}

\subsection{Side-by-side}

Skizze:

\image{alg\_side\_by\_side.png}


TeX-Fragment:

\begin{preformatted}[code]%
\c{documentclass}\{japs.element.algorithm\}

\c{begin}\{metainfo\}
  \altvar{Meta-Informationen}
\c{end}\{metainfo\}

\c{begin}\{content\}

  \c{title}\{\altvar{Titel}\}

  \c{begin}\{background\}
    \altvar{Hintergrund}
  \c{end}\{background\}

  \c{begin}\{input\}
    \altvar{Input}
  \c{end}\{input\}

  \c{begin}\{output\}
    \altvar{Output}
  \c{end}\{output\}

  \c{begin}\{algsteps\}[side-by-side]
    \c{algstep} \altvar{Erster Schritt}
    \c{algstep} \altvar{Zweiter Schritt}
    \c{algstep} \altvar{Dritter Schritt}
  \c{end}\{algsteps\}

  \c{begin}\{remarks\}
    \altvar{Bemerkungen}
  \c{end}\{remarks\}

\c{end}\{content\}
\end{preformatted}


XML-Fragment:

\begin{preformatted}[code]%
<elm:element category="theorem">

  <elm:title>
    \altvar{Titel}
  </elm:title>>

  <elm:background>
    \altvar{Hintergrund}
  </elm:background>

  <elm:input>
    \altvar{Input}
  </elm:input>

  <elm:output>
    \altvar{Output}
  </elm:output>

  <elm:algsteps arrange="side-by-side">
    <elm:algstep>
      \altvar{Erster Schritt}
    </elm:algstep>
    <elm:algstep>
      \altvar{Zweiter Schritt}
    </elm:algstep>
    <elm:algstep>
      \altvar{Dritter Schritt}
    </elm:algstep>
  </elm:algsteps>>

  <elm:remarks>
    \altvar{Bemerkungen}
  </elm:remarks>

</elm:element>
\end{preformatted}

\subsection{Top-bottom}

Skizze:

\image{alg\_top\_bottom.png}


TeX-Fragment:

\begin{preformatted}[code]%
\c{documentclass}\{japs.element.algorithm\}

\c{begin}\{metainfo\}
  \altvar{Meta-Informationen}
\c{end}\{metainfo\}

\c{begin}\{content\}

  \c{title}\{\altvar{Titel}\}

  \c{begin}\{background\}
    \altvar{Hintergrund}
  \c{end}\{background\}

  \c{begin}\{input\}
    \altvar{Input}
  \c{end}\{input\}

  \c{begin}\{output\}
    \altvar{Output}
  \c{end}\{output\}

  \c{begin}\{algsteps\}
    \c{algstep} \altvar{Erster Schritt}
    \c{algstep} \altvar{Zweiter Schritt}
    \c{algstep} \altvar{Dritter Schritt}
  \c{end}\{algsteps\}

  \c{begin}\{remarks\}
    \altvar{Bemerkungen}
  \c{end}\{remarks\}

\c{end}\{content\}
\end{preformatted}


XML-Fragment:

\begin{preformatted}[code]%
<elm:element category="theorem">

  <elm:title>
    \altvar{Titel}
  </elm:title>>

  <elm:background>
    \altvar{Hintergrund}
  </elm:background>

  <elm:input>
    \altvar{Input}
  </elm:input>

  <elm:output>
    \altvar{Output}
  </elm:output>

  <elm:algsteps arrange="top-bottom">
    <elm:algstep>
      \altvar{Erster Schritt}
    </elm:algstep>
    <elm:algstep>
      \altvar{Zweiter Schritt}
    </elm:algstep>
    <elm:algstep>
      \altvar{Dritter Schritt}
    </elm:algstep>
  </elm:algsteps>>

  <elm:remarks>
    \altvar{Bemerkungen}
  </elm:remarks>

</elm:element>
\end{preformatted}

-- ENDE DER DATEI --

\end{document}