\documentclass{generic}

% Author: Tilman Rassy <rassy@math.tu-berlin.de>
% $Id: macros.tex,v 1.5 2006/10/05 15:58:42 rassy Exp $

% XML code:
\newcommand{\element}[1]{\code[xml-element]{#1}}
\newcommand{\attrib}[1]{\code[xml-attrib]{#1}}

% Database code:
\newcommand{\dbtable}[1]{\code[db-table]{#1}}
\newcommand{\dbcol}[1]{\code[db-column]{#1}}
\newcommand{\sql}[1]{\code[sql]{#1}}

% TeX code:
\newcommand{\texcmd}[1]{\code[tex-cmd]{\backslash#1}}
\newcommand{\texenv}[1]{\code[tex-env]{#1}}

% General
\newcommand{\val}[1]{\code[value]{#1}}
\newcommand{\prm}[1]{\code[param]{#1}}

% Other:
\newcommand{\notimpl}[1]{\emph[not-impl]{#1}}
\newcommand{\new}[1]{\emph[new]{#1}}
\newcommand{\deprecated}[1]{\emph[deprecated]{#1}}
\newcommand{\wvar}[1]{\var[weak]{#1}}


\begin{document}

\title{Anh�ngbarkeit}
\subtitle{von Dokumenten an andere}

\begin{authors}
  \author[rassy@math.tu-berlin.de]{Tilman Rassy}
  \author[lehmannf@math.tu-berlin.de]{Fritz Lehmann-Grube}
\end{authors}

\version{$Id: attachable.tex,v 1.2 2006/08/11 16:18:22 rassy Exp $}

Die "Anh�ngbarkeit" von Subelementen an Elemente wird im neuen System durch
eine normale Referenz dargestellt. Daf�r wird ein neuer Referenztyp
\code{attachable} eingf�hrt.

Dieser Referenztyp verallgemeinert die bisherige Relation \code{contained_in}.
Die Sonderrolle der Subelement - Element - Beziehung entf�llt also.

Referenzen des Typs \code{attachable} sind prinzipiell auch zwischen anderen
Dokumenttypen erlaubt, insbesondere aber nicht, wenn "uberhaupt keine
Referenzen zwischen den jeweiligen Typen erlaubt sind wie in \file{config/config.xml}
durch das Attribut \attrib{no-refs-to} des referenzierenden Dokumenttyps definiert.

-- ENDE DER DATEI --

\end{document}