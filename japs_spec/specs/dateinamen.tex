\documentclass{generic}

% Author: Tilman Rassy <rassy@math.tu-berlin.de>
% $Id: macros.tex,v 1.5 2006/10/05 15:58:42 rassy Exp $

% XML code:
\newcommand{\element}[1]{\code[xml-element]{#1}}
\newcommand{\attrib}[1]{\code[xml-attrib]{#1}}

% Database code:
\newcommand{\dbtable}[1]{\code[db-table]{#1}}
\newcommand{\dbcol}[1]{\code[db-column]{#1}}
\newcommand{\sql}[1]{\code[sql]{#1}}

% TeX code:
\newcommand{\texcmd}[1]{\code[tex-cmd]{\backslash#1}}
\newcommand{\texenv}[1]{\code[tex-env]{#1}}

% General
\newcommand{\val}[1]{\code[value]{#1}}
\newcommand{\prm}[1]{\code[param]{#1}}

% Other:
\newcommand{\notimpl}[1]{\emph[not-impl]{#1}}
\newcommand{\new}[1]{\emph[new]{#1}}
\newcommand{\deprecated}[1]{\emph[deprecated]{#1}}
\newcommand{\wvar}[1]{\var[weak]{#1}}


\begin{document}

\title{Dateinamen}

\begin{authors}
  \author[rassy@math.tu-berlin.de]{Tilman Rassy}
\end{authors}

\version{$Id: dateinamen.tex,v 1.4 2006/10/05 15:58:43 rassy Exp $}

Diese Spezifikation beschreibt den Standard f�r \emph{Japs-Checkin-Dateien}.
Letztere sind Master-, Content- und Preview-Dateien von Dokumenten und
Pseudo-Dokumenten.

\tableofcontents

\section{Aufbau}\label{aufbau}

Mit Ausnahme von Sections bestehen die Dateinamen von Dokumenten und
Pseudo-Dokumenten aus einem \emph{Rumpfnamen (pure name)}\label{rumpfname} und
einer zweiteiligen \emph{Endung (suffix)}; jeweils durch Punkte getrennt; die
zwei Teile der Endung ebenfalls durch einen Punkt getrennt:

\begin{preformatted}%
    \var{pure_name}.\var{suffix1}.\var{suffix2}
\end{preformatted}

Hierbei bezeichnen \var{pure_name} den Rumpfnamen und \var{suffix1} und
\var{suffix2} die beiden Teile der Endung.

Der Rumpfname ist der inhaltlich motivierte Name der Datei. Er darf nur aus
Buchstaben, Ziffern und dem Unterstrich ("_") bestehen. Der erste Teil der
Endung (\var{suffix1}) klassifiziert die Datei nach ihrer technischen Rolle in
der Mumie, d.h. danach, ob es sich um eine Master-, Content- oder Preview-Datei
handelt. Die entsprechenden Werte f�r \var{suffix1} lauten \file{meta} bzw.
\file{content} bzw. \file{preview}. Der zweite Teil der Endung (\var{suffix2})
klassifiziert die Datei nach ihrem \emph{Media Type}. Die m�glichen Werte sind
in \ref{media_type_suffixes} definiert. (F�r weitere Informationen �ber Media
Types s.
\link{http://www.graphcomp.com/info/specs/mime.html}{www.graphcomp.com/info/specs/mime.html}.)

Da Master-Dateien immer vom Media Type \val{text/xml} sind, lautet ihre Endung
(erster und zweiter Teil) stets \file{meta.xml}.

Es wird empfohlen, dass der Rumpfname immer mit einem sogenannten
\emph{Typ-Indikator} beginnt. Der Typ-Indikator ist ein K�rzel f�r den Typ des
Dokuments bzw. Pseudo-Dokuments. Er sollte durch einen Unterstich ("_") vom
restlichen Runpfnamen getrennt sein. Die m�glichen Typ-Indikatore sind in
\ref{type_indicators} definiert. 

\section{Typ-Indikatoren}\label{type_indicators}

Typ-Indikatoren f�r nicht-generische Dokumenttypen und Pseudo-Dokumenttypen
bestehen aus Kleinbuchstaben und sind in der Regel drei Zeichen lang. In
Ausnahmef�llen sind auch l�ngere Typ-Indikatoren erlaubt. Die Typ-Indikatoren der
generischen Dokumenttypen entstehen aus denjenigen der ensprechenden
nicht-generischen Dokumenttypen durch Voranstellen von \file {g_}.

Die folgende Tabelle fasst alle Typ-Indikatoren zusammen:

\begin{table}
  \head
    Typ-Indikator & (Pseudo-)Dokumenttyp  \\
  \body
    \file{apl} & \val{applet}  \\
    \file{bsr} & \val{binary_source} \\
    \file{cls} & \val{class} \\
    \file{crs} & \val{course}  \\
    \file{csb} & \val{course_subsection}  \\
    \file{csc} & \val{course_section}  \\
    \file{css} & \val{css_stylesheet}  \\
    \file{elm} & \val{element}  \\
    \file{g_css} & \val{generic_css_stylesheet}  \\
    \file{g_img} & \val{generic_image}  \\
    \file{g_mov} & \val{generic_movie}  \\
    \file{g_pge} & \val{generic_page} \\
    \file{g_snd} & \val{generic_sound}  \\
    \file{g_xsl} & \val{generic_xsl_stylesheet}  \\
    \file{img} & \val{image}  \\
    \file{jar} & \val{jar}  \\
    \file{jcl} & \val{java_class}  \\
    \file{jsl} & \val{js_lib}  \\
    \file{lng} & \val{language} \\
    \file{mov} & \val{movie}  \\
    \file{pge} & \val{page}  \\
    \file{prb} & \val{problem}  \\
    \file{sbe} & \val{subelement}  \\
    \file{sem} & \val{semester} \\
    \file{snd} & \val{sound}  \\
    \file{tme} & \val{theme} \\
    \file{tsr} & \val{text_source} \\
    \file{tut} & \val{tutorial} \\
    \file{ugr} & \val{user_group} \\
    \file{usr} & \val{user} \\
    \file{xsl} & \val{xsl_stylesheet}  \\
\end{table}

\section{Media-Type-Endungen}\label{media_type_suffixes}

Die folgende Tabelle gibt die m�glichen Werte von \var{suffix2} (zweiter Teil der
Dateiendung, s. \ref{aufbau}) und die entsprechenden Media Types an.


\begin{table}
  \head
    2. Endung & Media Type  \\
  \body
    \file{class} & \val{application/x-java-vm} \\
    \file{css} & \val{text/css} \\
    \file{gif} & \val{image/gif} \\
    \file{html} & \val{text/html} \\
    \file{jar} & \val{application/x-java-archive} \\
    \file{java} & \val{text/java} \\
    \file{jpg} & \val{image/jpeg} \\
    \file{js} & \val{application/x-javascript} \\
    \file{mpg} & \val{audio/mpeg} \\
    \file{png} & \val{image/png} \\
    \file{swf} & \val{application/x-shockwave-flash} \\
    \file{tex} & \val{text/tex} \\
    \file{tiff} & \val{image/tiff} \\
    \file{txt} & \val{text/plain} \\
    \file{wav} & \val{audio/wav} \\
    \file{xml} & \val{text/xml} \\
    \file{xhtml} & \val{text/xml} \\
    \file{zip} & \val{application/zip} \\
\end{table}

Man beachte, das die Endungen \file{xml} und \file{xhtml} beide zum Media Type
\val{text/xml} geh�ren. \file{xhtml} kennzeichnet eine XHTML-Datei, \file{xml}
eine XML-Datei sonstiger Art.

\section{Ausnahme: Sections}

Master-Dateien von Sections haben immer den Namen 

\begin{preformatted}%
  \file{.meta.xml}
\end{preformatted}

Content-und Preview-Dateien gibt es bei Sections nicht (vgl.
\href{sections.xhtml}{Sections-Spezifikation}).

\section{Beispiele}

\begin{enumerate}
  \item \file{elm_def_erzeugendensystem.meta.xml}

    Master-Datei eines Elements.

  \item \file{elm_def_erzeugendensystem.content.xml}

    Content-Datei von s.o.

  \item \file{elm_def_erzeugendensystem.preview.xhtml}

    Preview-Datei von s.o.

  \item \file{tsr_def_erzeugendensystem.meta.xml}

    Meta-Datei einer Text-Quelle

  \item \file{tsr_def_erzeugendensystem.content.tex}

    Content-Datei einer Text-Quelle

  \item \file{img_lachende_mumie.meta.ml}

    Meta-Datei eines Bildes (Dokumenttyp \val{image})

  \item \file{img_lachende_mumie.content.png}

    Content-Datei von s.o.

  \item \file{img_lachende_mumie.preview.png}

    Preview-Datei von s.o.

\end{enumerate}



-- ENDE DER DATEI --

\end{document}