\documentclass{generic}

% Author: Tilman Rassy <rassy@math.tu-berlin.de>
% $Id: macros.tex,v 1.5 2006/10/05 15:58:42 rassy Exp $

% XML code:
\newcommand{\element}[1]{\code[xml-element]{#1}}
\newcommand{\attrib}[1]{\code[xml-attrib]{#1}}

% Database code:
\newcommand{\dbtable}[1]{\code[db-table]{#1}}
\newcommand{\dbcol}[1]{\code[db-column]{#1}}
\newcommand{\sql}[1]{\code[sql]{#1}}

% TeX code:
\newcommand{\texcmd}[1]{\code[tex-cmd]{\backslash#1}}
\newcommand{\texenv}[1]{\code[tex-env]{#1}}

% General
\newcommand{\val}[1]{\code[value]{#1}}
\newcommand{\prm}[1]{\code[param]{#1}}

% Other:
\newcommand{\notimpl}[1]{\emph[not-impl]{#1}}
\newcommand{\new}[1]{\emph[new]{#1}}
\newcommand{\deprecated}[1]{\emph[deprecated]{#1}}
\newcommand{\wvar}[1]{\var[weak]{#1}}


\begin{document}

\title{Was ist Mumie?}

\begin{authors}
  \author[rassy@math.tu-berlin.de]{Tilman Rassy}
\end{authors}

\version{$Id: was_ist_mumie.tex,v 1.1 2008/12/12 13:27:40 rassy Exp $}

\lang{de}{Mumie ist eine auf Mathematik spezialisierte Lehr- und Lernplattform.
  Sie enth�lt mathematischen Inhalt in feingranularer Form. Die kleinsten
  Einheiten sind Definitionen, S�tze, Beispiele, Visualisierungen usw. Diese
  werden bez�glich einer festen, fachlich motivierten Taxonomie in der
  Datenbank abgelegt. Daraus lassen sich Kurse und kurs�hnliche Dokumente
  zusammensetzen.  Deren Bestandteile k�nnen auf verschiedene Weise angeordnet
  werden; neben einfachen hierarchischen Strukturen sind auch Netzstrukturen
  m�glich, die die logischen Zusammenh�nge der Bestandteile untereinander
  wiederspeigeln (Satz erfordert Definition; aus Satz folgt Satz usw.). Alle
  Komponenten sind auch unabh�ngig von Kursen oder kurs�hnlichen Gebilden
  zug�nglich, da die oben erw�hnte Taxonomie online "`browsebar"' ist.
  
  Mit Mumie lassen sich verschiedene Typen von Aufgaben realisieren;
  insbesondere solche mit personalisierten Daten (jeder Student hat eigene
  Zahlen), elektronischer Abgabe und automatischer Korrektur. Die
  Aufgabenstellung ist nicht auf Multiple-Choice beschr�nkt, sondern umfasst
  wesentlich kompexere M�glichkeiten.
  
  Mumie ist webbasiert. Ein Server stellt die Inhalte zur Verf�gung, bereitet
  sie dynamisch zu Webseiten auf und verabeitet Daten. Der Benutzer ben�tigt
  lediglich einen Webbrowser (welcher allerdings MathML-f�hig sein muss, was
  momantan nur f�r Browser der Mozilla-Familie der Fall ist).}

\lang{en}{Mumie is a teaching and learning platform specialized on mathematics.
  It contains mathematical content in fine-granular form. The smallest entities
  are definitions, theorems, examples, visualizations, etc. They are filed in a
  database according to a fixed, field-specific taxonomy. From these entities,
  courses or course-like documents can be composed. Their constituents may be
  arranged in different ways. Besides simple hierarchical structures, net
  structures are possible which mirror the logical interrelations between the
  constituents (theorem requires theorem; theorem implies theorem; etc.). All
  components are available independently on courses or course-like structures,
  too, since the taxonomy mentioned above is online browsable.
  
  With Mumie, problems of different types can be realized; inparticular those
  with personalized data (each student has particular values), electronic
  editing, and automatic correction. The problem formulation is not restricted
  to multiple choice, but comprises much more complex possibilities.
  
  Mumie is web-based. A server provides the content, dynamically creates wep
  pages from it, and processes data. The user only needs a web browser (which,
  however, must be MathML-enabled, which is currently true only for the
  browsers of the Mozilla familiy).

 }

\end{document}