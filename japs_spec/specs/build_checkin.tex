\documentclass{generic}

% Author: Tilman Rassy <rassy@math.tu-berlin.de>
% $Id: macros.tex,v 1.5 2006/10/05 15:58:42 rassy Exp $

% XML code:
\newcommand{\element}[1]{\code[xml-element]{#1}}
\newcommand{\attrib}[1]{\code[xml-attrib]{#1}}

% Database code:
\newcommand{\dbtable}[1]{\code[db-table]{#1}}
\newcommand{\dbcol}[1]{\code[db-column]{#1}}
\newcommand{\sql}[1]{\code[sql]{#1}}

% TeX code:
\newcommand{\texcmd}[1]{\code[tex-cmd]{\backslash#1}}
\newcommand{\texenv}[1]{\code[tex-env]{#1}}

% General
\newcommand{\val}[1]{\code[value]{#1}}
\newcommand{\prm}[1]{\code[param]{#1}}

% Other:
\newcommand{\notimpl}[1]{\emph[not-impl]{#1}}
\newcommand{\new}[1]{\emph[new]{#1}}
\newcommand{\deprecated}[1]{\emph[deprecated]{#1}}
\newcommand{\wvar}[1]{\var[weak]{#1}}


\begin{document}

\title{Build-Checkin}

\begin{authors}
  \author[lehmannf@math.tu-berlin.de]{Fritz Lehmann-Grube}
\end{authors}

\version{$Id: build_checkin.tex,v 1.1 2006/12/29 16:44:11 lehmannf Exp $}

Einige Pseudo-Dokumente m�ssen in der Datenbank vorhanden sein, bevor der
normale Checkin funktioniert.
Hier wird beschrieben, welche das sind, wie sie und aus welchen Quellen in die
Datenbank geschrieben werden.

\tableofcontents

\section{�berblick}\label{ueberblick}

Es werden in der Datenbanksprache SQL \code{INSERT} Befehle geschrieben. Diese werden
dann auf leeren wie in der \href{mumie_database}{Datenbankstruktur}
festgelegten Tabellen augef�hrt. Das setzt eine laufende Postgres voraus.

Der SQL \code{INSERT} Befehl wird f�r jedes Pseudo-Dokument zusammengesetzt aus
\begin{itemize}
\item einem in XSL geschriebenen Template, das ein Skelett des Befehls enth�lt und 
\item Eintr�gen in der Konfigurationsdatei \file{config.xml}. Diese
  fungiert dabei in gewisser Weise als Masterdatei f�r alle so erzeugten Pseudo-Dokumente.
\end{itemize}

Insbesondere wird (anders als sonst bei Masterdateien!) die Datenbank-Id in
\file{config.xml} festgelegt. Daher muss nachher in den betroffenen
Datenbanktabellen die kleinste automatisch zu vergebende Id initialisiert
werden. F�r die PostgreSQL geschieht das mit dem \code{SELECT setval(.,.)}
Befehl. Als deren erstes Argument wird der Tabellenname und als zweites die
gr�sste festgelegte Id + 1 eingetragen.

Abschnitt \ref{build_docs} enth�lt eine Liste aller 

\section{Liste aller zu erzeugenden Pseudo-Dokumente}\label{build_docs}

\subsection{Root Section}

\subsection{Org Section}

\subsection{Org User-Groups Section}

\subsection{Org Users Section}

\subsection{System Section}

\subsection{System Themes Section}

\subsection{System Languages Section}

\subsection{Admin User-Group}

\subsection{Admin User}

\subsection{Default Theme}

\subsection{Neutral Language}

\subsection{Default Language}

-- WIRD FORTGESETZT --

\end{document}