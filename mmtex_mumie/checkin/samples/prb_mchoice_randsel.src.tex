%$Id: prb_mchoice_randsel.src.tex,v 1.1 2009/10/27 16:00:13 linges Exp $

\documentclass{japs.problem.mchoice}

\begin{metainfo}
  \name{Test}
  \begin{description}
    Ein Test
  \end{description}
  \begin{changelog}
    Started
  \end{changelog}
  \begin{components}

  \end{components}
  \creategeneric
\end{metainfo}

\begin{content}
  \title{Der Titel}

  Einleitender Text

  \begin{ppdrandsel}

    \ppdoption

      \begin{choices}{unique}
        \choice
          \assertion{(1) Erste Behauptung}
          \begin{explanation}
            Erkl�rung
          \end{explanation}
        \choice
          \assertion{(1) Zweite Behauptung}
          \solution{true}
          \begin{explanation}
            Erkl�rung
          \end{explanation}
        \choice
          \assertion{(1) Dritte Behauptung}
          \solution{false}
          \begin{explanation}
            Erkl�rung
          \end{explanation}
      \end{choices}

    \ppdoption

      \begin{choices}{unique}
        \choice
          \assertion{(2) Erste Behauptung}
          \begin{explanation}
            Erkl�rung
          \end{explanation}
        \choice
          \assertion{(2) Zweite Behauptung}
          \solution{true}
          \begin{explanation}
            Erkl�rung
          \end{explanation}
        \choice
          \assertion{(2) Dritte Behauptung}
          \solution{false}
          \begin{explanation}
            Erkl�rung
          \end{explanation}
      \end{choices}

    \ppdoption

      \begin{choices}{unique}
        \choice
          \assertion{(3) Erste Behauptung}
          \begin{explanation}
            Erkl�rung
          \end{explanation}
        \choice
          \assertion{(3) Zweite Behauptung}
          \solution{true}
          \begin{explanation}
            Erkl�rung
          \end{explanation}
        \choice
          \assertion{(3) Dritte Behauptung}
          \solution{false}
          \begin{explanation}
            Erkl�rung
          \end{explanation}
      \end{choices}

    \ppdoption

      \begin{choices}{unique}
        \choice
          \assertion{(4) Erste Behauptung}
          \begin{explanation}
            Erkl�rung
          \end{explanation}
        \choice
          \assertion{(4) Zweite Behauptung}
          \solution{true}
          \begin{explanation}
            Erkl�rung
          \end{explanation}
        \choice
          \assertion{(4) Dritte Behauptung}
          \solution{false}
          \begin{explanation}
            Erkl�rung
          \end{explanation}
      \end{choices}

    \end{ppdrandsel}

\end{content}