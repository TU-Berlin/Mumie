\documentclass{japs.summary}

\begin{metainfo}
  \name{Test}
  \begin{description}
    Ein Test
  \end{description}
  \begin{changelog}
    Started
  \end{changelog}
  \begin{components}

  \end{components}
  \generic{samples/g_sum_free.meta.xml}
\end{metainfo}

\begin{content}

  \section{Erster Abschnitt}

  Erster Absatz des ersten Abschnitts.

  Eine Liste:

  \begin{enumerate}
    \item Eins
    \item Zwei
    \item Drei
  \end{enumerate}

  \section{Zweiter Abschnitt}

  Zweiter Abschnitt.

  \begin{ifstate}{feedback}

    Nur sichtbar, wenn der Zustand des Arbeitsblatts = feedback ist

    \begin{ifpoints}{0 10}
      Zu wenig Punkte
    \end{ifpoints}

    \begin{ifpoints}{10.1 20}
      Ok
    \end{ifpoints}

    \begin{ifpoints}{25 30}
      Sehr gut!
    \end{ifpoints}

    \begin{casepoints}
      \uptopoints{10}
      Bis zu 10
      \uptopoints{20}
      Zwischen 10.1 und 20
      \uptopoints{max}
      20.1 oder mehr
    \end{casepoints}

  \end{ifstate}

  \begin{iftimeframe}{inside}
    Nur waehrend des Bearbeitungszeitraums sichtbar.
  \end{iftimeframe}

  \begin{iftimeframe}{after}
    Nur nach dem Bearbeitungszeitraum sichtbar
  \end{iftimeframe}

\end{content}
