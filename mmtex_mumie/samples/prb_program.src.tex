\documentclass{japs.problem.program}

\begin{metainfo}
  \name{Minimum eines Arrays von Zahlen}
  \begin{description}
    Ein Test
  \end{description}
  \begin{changelog}
    Started
  \end{changelog}
  \corrector{system/problem/GenericProgramCorrector.meta.xml}
  \creategeneric
\end{metainfo}

\begin{content}

\title{Minimum eines Arrays von Zahlen}

Schreiben Sie ein Java-Methode, die das Minimum in einem Array berechnet und zur�ckgibt.
Ist das Array leer, so soll die gr��te int-Zahl zur"uckgegeben werden. Erg�nzen
Sie dazu folgendes Code-Fragment:

\begin{prganswer}{20}{80}
  public static int getMin(int[] array) {
    @USER_ANSWER@
  }
\end{prganswer}

\begin{hidden}

\begin{prgwrapper}
public class Wrapper
{
  public static int getMin(int[] array) {
    @USER_ANSWER@
  }
}
\end{prgwrapper}

\begin{prgevaluator}
public class Evaluator
{
  public static void main (String[] args)
  {
    int[] array1 = {1,2,3};
    int[] array2 = {};
    int fehler = 0;

    try
      {
        if ( Wrapper.getMin(array1) != 1 ) fehler = 2;
        if ( Wrapper.getMin(array2) != Integer.MAX_VALUE ) fehler = 1;
      }
    catch (Throwable throwable)
      {
        fehler = 3;
        throwable.printStackTrace();
      }

    System.exit(fehler);
  }
}
\end{prgevaluator}

\begin{prgsolution}
public static int getMin(int[] array) {
  int length = array.length;
  int min = Integer.MAX_VALUE;
  for ( int i = 0; i < length; ++i )
    min = Math.min(min, array[i]);          
  return min;
}
\end{prgsolution}

\prggrading{0}{1.0}{Herzlichen Glueckwunsch, Sie haben volle Punktzahl}
\prggrading{1}{0.5}{Beachten Sie, dass das Array auch leer sein kann!}
\prggrading{2}{0.0}{Sie berechnen leider nicht das Minimum!}
\prggrading{3}{0.0}{Ihr Code hat leider einen Laufzeitfehler verursacht!}

\end{hidden}

\end{content}
