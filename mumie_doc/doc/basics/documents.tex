\documentclass{generic}

% Author: Tilman Rassy <rassy@math.tu-berlin.de>
% $Id: macros.tex,v 1.5 2006/10/05 15:58:42 rassy Exp $

% XML code:
\newcommand{\element}[1]{\code[xml-element]{#1}}
\newcommand{\attrib}[1]{\code[xml-attrib]{#1}}

% Database code:
\newcommand{\dbtable}[1]{\code[db-table]{#1}}
\newcommand{\dbcol}[1]{\code[db-column]{#1}}
\newcommand{\sql}[1]{\code[sql]{#1}}

% TeX code:
\newcommand{\texcmd}[1]{\code[tex-cmd]{\backslash#1}}
\newcommand{\texenv}[1]{\code[tex-env]{#1}}

% General
\newcommand{\val}[1]{\code[value]{#1}}
\newcommand{\prm}[1]{\code[param]{#1}}

% Other:
\newcommand{\notimpl}[1]{\emph[not-impl]{#1}}
\newcommand{\new}[1]{\emph[new]{#1}}
\newcommand{\deprecated}[1]{\emph[deprecated]{#1}}
\newcommand{\wvar}[1]{\var[weak]{#1}}


\begin{document}

\title{Dokumente}

\begin{authors}
  \author[rassy@math.tu-berlin.de]{Tilman Rassy}
\end{authors}

\version{$Id: documents.tex,v 1.4 2007/07/25 15:40:02 rassy Exp $}

\tableofcontents

\section{\lang{de}{Dokumente und Pseudo-Dokumente}\lang{en}{Documents and pseudo-documents}}

\lang{de}{\emph{Dokumente} sind in Mumie Einheiten mit textuellem oder bin�rem
  Inhalt, z.B. (mathematische) S�tze, Definitionen, Bilder oder Java-Applets.
  Neben Dokumenten gibt es in Mumie auch sogenannte \emph{Pseudo-Dokumente}.
  Das sind Einheiten, die sich formal wie Dokumente behandeln lassen, aber der
  anschaulichen Bedeutung des Begriffes nach keine wirklichen Dokumente sind
  und i.d.R. auch keinen Inhalt haben. Beispiele: Benutzer, Benutzergruppen,
  Semester.

  Jedes Dokument oder Pseudo-Dokument hat einen \emph{Typ}. Abschnitt
  \ref{types} stellt die m�glichen Typen vor. Dar�berhinaus besitzt jedes
  Dokument oder Pseudo-Dokument einen Satz von \emph{Metainformationen}. Welche
  Metainformationen es gibt, h�ngt vom Typ ab. Im Kapitel
  "\href{metainfos.xhtml}{Metainformationen}" wird ausf�hrlich darauf
  eingegenagen.}

\section{\lang{de}{Generische Dokumente}\lang{en}{Generic
    documents}}\label{generic_documents}

\lang{de}{Eine besondere Art von Dokumenten sind die sogenannten
  \emph{generischen Dokumente}. Sie haben keinen eigenen Inhalt, sondern
  stellen gewissermassen Platzhalter f�r "reale" Dokumente dar. Diese
  Platzhalter k�nnen in unterschiedlichen Kontexten durch unterschiedliche
  reale Dokumente implementiert werden. Generische Dokumente dienen in Mumie
  zur Realisierung der folgenden beiden Konzepte:

  \begin{itemize}
    \item \emph{Themes:} Um dem Benutzer denselben Inhalt in verschiedenen
      Layouts anzubieten, sind in Mumie verschiedene "Themes" m�glich, �hnlich
      wie bei modernen Desktops. Umgesetzt wird dies mit Hilfe generischer
      Dokumente, die in verschiedenen Themes durch verschiedene reale
      Dokumente implementiert werden.
    \item \emph{Mehrsprahigkeit:} Mumie ist in der Lage, dem Benutzer denselben
      Inhalt in verschiedenen Sprachen anzubieten. Umgesetzt wird dies mit
      Hilfe generischer Dokumente, die in verschiedenen Sprachen durch
      verschiedene reale Dokumente implementiert werden.
  \end{itemize}
}

\section{\lang{de}{Typen}\lang{en}{Types}}\label{types}

\lang{de}{Jedes Dokument oder Pseudo-Dokument hat einen bestimmten
  \emph{Typ}. Typen werden durch Namen (intern auch durch ganze Zahlen)
  charakterisiert. 

  Die folgende Tabelle listet alle Dokument-Typen auf:

  \begin{table}
    \head
      Typ & Beschreibung \\
    \body
      \code{applet} & Ein Java-Applet \\
      \code{course} & Ein Kurs \\
      \code{course_section} & Ein Kursabschnitt \\
      \code{css_stylesheet} & Ein CSS-Stylesheet \\
      \code{element} & Ein Inhaltsbaustein \\
      \code{flash} & Ein Flash-Movie \\
      \code{image} & Ein Bild \\
      \code{jar} & Ein Java-Arvhiv \\
      \code{java_class} & Eine Java-Klasse\\
      \code{js_lib} & Eine JavaScript-Bibliothek\\
      \code{movie} & Ein Film\\
      \code{page} & Eine Webseite \\
      \code{problem} & Eine Aufgabe \\
      \code{sound} & Ein Audi-Dokument\\
      \code{subelement} & Ein Inhalts-Unterbaustein \\
      \code{summary} & Eine Zusammenfassung \\
      \code{worksheet} & Ein Aufgabenblatt \\
      \code{xsl_stylesheet} & Ein XSL-Stylesheet\\
      \code{generic_css_stylesheet} & Ein generisches CSS-Stylesheet \\
      \code{generic_element} & Ein generischer Inhaltsbaustein \\
      \code{generic_image} & Ein generisches Bild \\
      \code{generic_page} & Eine generische Webseite \\
      \code{generic_problem} & Eine generische Aufgabe\\
      \code{generic_subelement} & Ein generischer Inhalts-Unterbaustein \\
      \code{generic_summary} & Eine generische Zusammenfassung  \\
      \code{generic_xsl_stylesheet} & Ein generisches XSL-Stylesheet
  \end{table}

  Dokumente eines mit \code{generic_} beginnenden Typs sind generische
  Dokumente. Alle anderen sind nicht-generische Dokumente. Wir sprechen auch
  kurz von generischen bzw. nicht-generischen Dokument-Typen.

  Zu jedem generischen Dokument-Typ gibt es einen entsprechenden
  nicht-generischen, der durch Weglassen des \code{generic_} ensteht. Ein
  generisches Dokument kann nur durch Dokumente des entsprechenden
  nicht-generischen Typs implementiert werden.

  Die folgende Tabelle listet alle Pseudo-Dokument-Typen auf:

  \begin{table}
    \head
      Typ & Beschreibung \\
    \body
      \code{class} & Eine Lehrveranstaltung\\
      \code{language} & Eine Sprache \\
      \code{section} & Eine Art Verzeichnis \\
      \code{semester} & Ein Semester \\
      \code{theme} & Ein Theme im Sinne des Theme-Konzepts (s.o.) \\
      \code{tutorial} & Ein Tutorium \\
      \code{user} & Ein Benutzer \\
      \code{user_group} & Eine Benutzer-Gruppe \\
  \end{table}
}

\end{document}