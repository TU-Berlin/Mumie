%This work is licensed under the Creative Commons Attribution-ShareAlike 3.0
%Unported License. To view a copy of this license, see the file LICENSE in the
%root directory of this content package; or, visit
%http://creativecommons.org/licenses/by-sa/3.0/legalcode; or, send a letter to
%Creative Commons, 171 2nd Street, Suite 300, San Francisco, California, 94105,
%USA.
%
%Here is a human-readable summary of the license:
%
%-------------------------------------------------------------------------------
%
%  You are free:
%
%    * to Share -- to copy, distribute and transmit the work
%    * to Remix -- to adapt the work
%
%  Under the following conditions:
%
%    * Attribution. You must attribute the work in the manner specified by the
%      author or licensor (but not in any way that suggests that they endorse
%      you or your use of the work). 
%
%    * Share Alike. If you alter, transform, or build upon this work, you may
%      distribute the resulting work only under the same, similar or a
%      compatible license.
%
%  For any reuse or distribution, you must make clear to others the license
%  terms of this work. The best way to do this is with a text like this one. 
%
%  Any of the above conditions can be waived if you get permission from the
%  copyright holder.
%
%  Nothing in this license impairs or restricts the author's moral rights.
%
%-------------------------------------------------------------------------------
%
%
\documentclass{japs.element.definition}
\begin{metainfo}
  \name{Basis}
  \begin{description}
    Definition der Basis
  \end{description}
  \copyrightinfo{(c) MUMIE-Projekt Technische Universitaet Berlin 2003}
  %\authors{RS,EZ,HV}
  \begin{components}

  \end{components}
  \status{content_complete}
  \begin{changelog}
	Kopie von 1.1.3.1.4
  \end{changelog}
  \creategeneric
\end{metainfo}
\begin{content}
\defnotion{Basis}
\begin{suppositions}
\lang{de}{Sei $B$ eine Teilmenge von $\vectorspace{V}$.}
\lang{en}{Let $B$ be a collection of vectors in  $\vectorspace{V}$.}
\end{suppositions}
\begin{defequivalence}[side-by-side]
\lang{de}{$B$ hei"st \notion{Basis} des Vektorraums $\vectorspace{V}$}
\lang{en}{$B$ is a \notion{Basis} for the Vector space $\vectorspace{V}$}
\isdefinedas
%%\begin{enumerate}
%%\lang{de}{\item Die \notion{lineare H"ulle} von $B$ ist ganz $\vectorspace{V}$.}
%%\lang{de}{\item Die Vektoren in $B$ sind \notion{linear unabh"angig}.}
%%\end{enumerate}
%\begin{enumerate}
%\lang{en}{\item The \notion{Span} of $B$ is $\vectorspace{V}$.}
%\lang{en}{\item The vectors in $B$ are \notion{linearly independent}.}
%\end{enumerate}
\lang{en}{
\begin{enumerate}
\item The \notion{Span} of $B$ is $\vectorspace{V}$.
\item The vectors in $B$ are \notion{linearly independent}.
\end{enumerate}}
\lang{de}{
\begin{enumerate}
\item Die \notion{lineare H"ulle} von $B$ ist ganz $\vectorspace{V}$.
\item Die Vektoren in $B$ sind \notion{linear unabh"angig}.
\end{enumerate}}
\end{defequivalence}
\end{content}



%-----------------------------------------------------------------------%
%---------------------- mehrsprachige Version-------------------%
%-----------------------------------------------------------------------%
%\documentclass{japs.element.definition}
%\begin{metainfo}
%  \name{Basis}
%  \begin{description}
%    \lang{de}{Definition der Basis}
%    \lang{en}{Definition of the basis}
%  \end{description}
%  \copyrightinfo{(c) MUMIE-Projekt Technische Universitaet Berlin 2003}
%  %\authors{RS,EZ,HV}
%  \begin{components}
%
%  \end{components}
%  \status{content_complete}
%  \begin{changelog}
%	Kopie von 1.1.3.1.4
%  \end{changelog}
%  \creategeneric
%\end{metainfo}
%\begin{content}
%\defnotion{Basis}
%\begin{suppositions}
%\lang{de}{Sei $B$ eine Teilmenge von $\vectorspace{V}$.}
%\lang{en}{Let $B$ be a subset of $\vectorspace{V}$.}
%\end{suppositions}
%\begin{defequivalence}[side-by-side]
%\lang{de}{$B$ hei"st \notion{Basis} des Vektorraums $\vectorspace{V}$}
%\lang{en}{$B$ is called a \notion{Basis} for the vector space $\vectorspace{V}$}
%\isdefinedas
%\begin{enumerate}
%\item \lang{de}{Die \notion{lineare H"ulle} von $B$ ist ganz $\vectorspace{V}$.} \lang{en}{The \notion{Span} of $B$ is $\vectorspace{V}$.} 
%\item \lang{de}{Die Vektoren in $B$ sind \notion{linear unabh"angig}.} \lang{en}{The vectors in $B$ are \notion{linearly independent}.}
%\end{enumerate}
%\end{defequivalence}
%\end{content}
