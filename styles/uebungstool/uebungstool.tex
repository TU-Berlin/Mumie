\section{Philosophie und Ziele des "Ubungstools}\label{sec:philosophie}

Die folgenden didaktischen Grund- und Ans"atze sind ein Versuch, die Gedanken
zu dem Thema, die in den Treffen mit den Vertretern der vier beteiligten
Hochschulen und im Gespr"ach mit Frau Keitel (FU Berlin) formuliert wurden,
zusammenzufassen. Sie haben vorl"aufigen Charakter und werden bis Ende
September --- zusammen mit Frau Keitel --- "uberarbeitet. Ziel dieser
Vorabversion ist es, die bereits gesicherten Grunds"atze zu fokussieren und
an unkritischen Orten mit der konkreten Umsetzung beginnen zu k"onnen.

\subsection{Grunds"atzliches}

Im gesamten Multimediaprojekt soll den Lernenden ein tiefgehendes
Verst"andnis der Mathematik vermittelt werden. Gleichzeitig soll der
Anwender in die Lage versetzt werden, das Gelernte zur L"osung seiner
Probleme zu verwenden. Der Wissensdurst soll vom Kreuzwortr�tsels und
Computerspielen auf ``mathematisch interessante Anwendungen'' und auf
``f�r die Anwendung interessante Mathematik'' gelenkt werden. Das
"Ubungstool soll eine Art Implementierung eines virtuellen
Instrumentes sein, zur ``Kunst des Erfindens''.

Die Gesamtkonzeption folgt einem zweistufigem Aufbau:\\
Zur Ausbildung dieses Verst"andnisses und zur Entwicklung von
Probleml"osungsstrategien ist der sichere Umgang mit den
\textit{grundlegenden} Elementen der Mathematik wie Definitionen und S"atzen
und den zugeh"origen prototypischen Anwendungen und Beispielen 
unabdingbar.\\
Darauf aufbauend mu"s eine eingehende Besch"aftigung mit
\textit{komplexeren} Problemen erfolgen, die mit der Entwicklung von
Probleml"osungs\textit{strategien} einhergeht: Der Lernende soll in
die Lage versetzt werden, auch komplizierte Probleme selbst"anding zu
l"osen, f"ur die keine ``Standard''-Methode bekannt ist.

Neben der Sicherheit im Umgang mit den mathematischen Grundlagen ist daher
nicht die Bereitstellung einer m"oglichst umfangreichen Aufgabensammlung mit
Musterl"osungen das Ziel, sondern die systematische Heranf"uhrung des
Lernenden an \text{Strategien} zur Probleml"osung.

\subsection{Didaktische Ans"atze}

Einen grossen Schritt in die Richtung die ``Heuristik der L"osung
mathematischer Problem'' hat G.Polya mit seinem Buch ``Schule des
Denkens (How to solve it)'' getan. In einer bestechend bescheidenen
Art wird dieses schwierige Thema meisterhaft aufgeschl"usselt und
dargestellt. Es enth"alt wesentliche Anregungen f"ur einen interessanten
Unterricht, didaktische Ans�tze und relevante Beispiele.

Darauf aufbauend soll ein didaktisches Konzept erarbeitet werden, das
einerseits den Anspr"uchen der mathematischen Ausbildung f�r Anwender
im ingenieur- und naturwissenschaftlichen Bereich gerecht wird und
andererseits die multimedialen M"oglichkeiten optimal nutzt. Dazu sind
ab September verschiedene Arbeitstreffen mit Frau Keitel geplant.


\clearpage

\section{Aufbau und Struktur des "Ubungstools}\label{sec:aufbau}

\subsection{Zuordnung der Probleme}

Zur Umsetzung der in Kap. \ref{sec:philosophie} beschriebenen Ziele
werden im "Ubungstool Aufgaben und Probleme verschiedener Schwierigkeit und
Komplexit"at zur Verf"ugung gestellt werden. 

Die einfachsten Probleme (\textbf{``Basis-Probleme''}) sind Gesamt-Elementen
(Elemente zusammen mit ihren Subelementen) zugeordnet.\\
Sie dienen
direkt der Ein"ubung des zentralen Gegenstandes eines Elementes
(Definitionen, S"atze \dots) und setzen daher wenig "ubergreifendes
Wissen voraus.\\
Die ``Basis-Probleme'' helfen, die mathematischen Grundlagen zu
festigen und damit die Grundlage f"ur eine eingehendere Besch"aftigung
zu schaffen.

Zur L"osung \textbf{komplexerer Probleme} sind mathematische Konzepte aus
mehreren Elementen oder weiter "ubergeordneten Hierarchien
(Modulgruppen, Module, Submodule etc.) notwendig. \\
Erst in solchen Problemen wird das Zusammenwirken der verschiedenen
mathematischen Ideen deutlich. Gleichzeitig treten erst/gerade diese
"ubergreifenden Probleme im ``Alltag'' des Anwenders
auf. \\ 
"Ubergreifende Probleme werden daher nicht Gesamt-Elementen, sondern
(je noch ``Komplexit"atsgrad'') den zust"andigen "ubergeordneten
Hierarchien wie Modulgruppen, Modulen etc. bzw. Kombinationen aus
diesen zugeordnet.

\clearpage

\subsection{Grundkonzeption}

Den Grund-Baustein des "Ubungstools stellen die
sog. ``Basis-Probleme'' dar.
Sie sind inhaltlich identisch mit den (umformulierten) Beispielen
zu den einzelnen Elementen aus dem Lerntool\footnote{Die Beispiele
werden von den Autoren von Beginn an in zwei Fassungen formuliert: 
in einer ``Ziel--L"osung''-Struktur f"ur das Lerntool,
in einer ``Frage--L"osung''-Struktur f"ur das "Ubungstool.}
(ggf. auch mit weiteren geeigneten Elementen oder Suelementen
des Lerntools).\\ 
Bei der L"osung (komplexer) Probleme kann der Lernende mit Hilfe
dieser Basisprobleme
(kleine) Zwischenschritte nachvollziehen, die (mit Hilfe der
bereitgestellten weiteren Hilfen wie etwa ``kleine Tips'') aufgezeigt
werden. Sie bilden in diesem Sinne eine ``Basis'' des "Ubungstools, da ("uber
verschiedene Hierarchiestufen mit immer weniger komplexen
Teilproblemen) der Lernende bei Bedarf bis zu diesen einfachen
Problemen gef"uhrt wird\footnote{Da es zu komplexen Problemen keinen
eindeutigen, "uber Zwischenschritte definierten L"osungsweg gibt, kann
der Lernende also "uber \textit{unterschiedliche} Basis-Probleme zur
L"osung gef"uhrt werden.}.

Die Basis-Probleme fungieren hinsichtlich der einzelnen Anteile und
Themen der komplexen Probleme als ``Musterl"osung'', und zwar im
Sinne von \textbf{Analog}beispielen\footnote{Allein wegen der sich
ergebenden Komplexit"at k"onnen die Basis-Probleme nicht auf das
einzelne komplexe Problem abgestimmt werden (etwa bei einem
Basis-Problem zur Berechnung einer $3 \times 3$-Determinante w"urden
die konkreten Zahlen nicht an das jeweilige Problem angepasst).}.

Dieser Ansatz ist konsistent mit der zentralen Philosophie dieses
Tools: Ziel des "Ubungstools ist die Vermittlung von
Probleml"osungs\textit{strategien}, nicht die Bereitstellung
m"oglichst ausgefeilter Musterl"osungen komplexer Probleme, die nicht
ausreichend auf die sp"atere \textit{eigenst"andige} L"osung
\text{neuartiger} Probleme vorbereiten w"urden. Auf eine solche
selbst"andige Bearbeitung von Problemen soll das "Ubungstool
vorbereiten.

\vspace{5mm}

Grunds"atzlich zu unterscheiden sind somit die folgenden beiden
Problemtypen:

\begin{list_sabina}
\item
\textbf{Basis-Probleme}:
                 \begin{sub_list_sabina}
                 \item
                 sind "aquivalent zu den Beispielen aus dem Lerntool\footnote{modulo 
                 geeigneter Umformulierung...}
                 \item
                 sind direkt einem Gesamt-Element zugeordnet 
                 \item
                 haben eine Musterl"osung 
                 \end{sub_list_sabina}
\item
\textbf{komplexe Probleme}: 
                 \begin{sub_list_sabina}
                 \item
                 kommen i.allg. nicht im Lerntool vor
                 \item
                 sind "ubergeordneten Hierarchien zugeordnet
                 \item
                 haben \textit{keine} Musterl"osung
                 \item
                 werden stattdessen stufenweise als Teilprobleme formuliert, die
                 schlie"slich bis zu den einfachsten Aufgaben f"uhren,
		 den o.g. Basis-Problemen
                 \item
                 haben zus"atzliche Hilfen zur \textit{selbst"andigen} 
                 Bearbeitung (etwa allgemeine ``Strategie-Elemente'', 
                 Tips, Zwischenergebnisse, s. Kap. \ref{subsec:hilfen})
                 \item
                 leiten den Lernenden mit Hilfe der o.g. Teilprobleme
		 und den zus"atzlichen Hilfen zu
		 einer selbst"andigen L"osung an
                 \end{sub_list_sabina}
\end{list_sabina}

\clearpage

\subsection{Zugang zum "Ubungstool}\label{subsec:zugang}

Die Benutzung des "Ubungstools ist grunds"atzlich auf zwei verschiedene Arten
m"oglich: 

\begin{list_sabina}
\item
"uber die Hierarchie des Lerntools
\item
direkt ("uber eine eigene Einstiegsseite des "Ubungstools)
\end{list_sabina}

Der dem Benutzer ideal angepa"ste Zugang erfolgt \textbf{aus dem Lerntool}
bzw. \textbf{aus dessen Hierarchie} 
heraus: der Lernende kann zu den einzelnen Elementen die Bearbeitung
von Aufgaben anw"ahlen und dort durch einige (wenige) Einstellungen,
etwa ein `Heraus/Hineinzoomen'', zu Problemen immer gr"o"serer/kleinerer
Komplexit"at und anderen Schwierigkeitsgraden gef"uhrt werden. Da bei
einem Einstieg "uber das Lerntool die komplette inhaltliche
``Vorgeschichte'' des Benutzers bekannt ist, k"onnen Aufgaben ideal
ausgew"ahlt und die angebotenen Hilfestellungen f"ur den Benutzer
optimiert werden. \\

Entscheidende Voraussetzung f"ur diesen Zugang ist lediglich, da"s
vorher ein Kurs durch den Lernenden ausgew"ahlt wurde, der i.w. seinen
Vorkenntnissen entspricht\footnote{Der ``"Ubungswillige'' kann also
durchaus den Eindruck bekommen, er habe es mit einem reinen
"Ubungstool zu tun, er sieht die Lernelemente des Kurses nicht
notwendigerweise im Vordergrund.}; es ist jedoch nicht zwingend
notwendig, da"s zum Einstieg zu den "Ubungsaufgaben einzelne Elemente
(erneut) angew"ahlt/bearbeitet werden: auch das Abrufen eines
``Satzes'' kursbegleitender Aufgaben (wieder mit Anpassung an
verschiedene Schwierigkeitsgrade und Komplexit"atsniveaus) ist
m"oglich, bei dem der Lernende nur dann zu den Elementen des Lerntools
zur"uckkehrt, wenn er sie zur Bearbeitung der Aufgaben als
Hilfestellung (erneut) betrachten m"ochte.

%%Beim Zugang "uber das Lerntool kann
%%jedoch auch allein die Struktur des bearbeiteten Kurses genutzt
%%werden, um zu den gew"unschten Aufgaben zu gelangen (ohne die
%%einzelnen Elemente erneut zu bearbeiten).

Beim reinen Zugang \textbf{"uber das "Ubungstool} (wobei die ``Vorgeschichte''
des Lernenden unbekannt ist) k"onnen die Hilfestellungen nur wie f"ur
einen ``Standard-Benutzer'' erfolgen, der einem ``Standard-Kurs''
gefolgt ist.  Die sich daraus ergebenden Schwierigkeiten liegen auf
der Hand.\\
Eine ``vern"unftige'' Eingangsabfrage insbesondere nach den
Vorkenntnissen des Lernenden w"are enorm aufwendig und damit
abschreckend f"ur den Benutzer:
ist eine genaue Benutzeranpassung gew"unscht, so ist es wesentlich
effizienter und gleichzeitig pr"aziser, wenn der Lernende unter den zu
einem Gebiet vorhandenen Standardkursen denjenigen ausw"ahlt, der
seiner Vorgeschichte so nahe wie m"oglich kommt.\\
Trotz dieser Einschr"ankung bietet sich ein eigenst"andiger Zugang
als sinnvoll an, dann n"amlich, wenn etwa nach speziellen
Aufgaben/bestimmten Stichw"ortern gesucht werden soll: hierf"ur mu"s
eine detaillierte Profisuche bereitgestellt werden.

\clearpage

\subsection{Zus"atzliche Hilfen}\label{subsec:hilfen}

Als Hilfestellungen f"ur die Aufgaben werden angeboten:

\begin{list_sabina}
\item
(Verweise auf) \textbf{Basis-Probleme}:
                 \begin{sub_list_sabina}
                 \item erscheinen mit vollst"andigen Musterl"osungen
                 \item bilden die Basis, auf die alle komplexeren Aufgaben
		 zur"uckgef"uhrt werden k"onnen sollen.
                 \end{sub_list_sabina}
\item
(Verweise auf) \textbf{Strategie-Elemente\footnote{Ein zus"atzliches, 
bisher noch nicht diskutiertes Tool ``der Mumie'' sollte ein ``Stratego-Tool''
sein, das allgemeine Probleml"osungsstrategien vermittelt. Die o.g.
Strategie-Elemente sind Elemente dieses Stratego-Tools. Details folgen.
Idee: Polya.}}: 
                 \begin{sub_list_sabina}
                 \item dienen der Entwicklung von Probleml"osungsstrategien
zu wichtigen Aufgabentypen
                 \item vermitteln allgemeine Heransgehensweisen an
unbekannte Probleme
                 \end{sub_list_sabina}
\item
\textbf{``kleine Tips''}: 
                 \begin{sub_list_sabina}
                 \item helfen, komplexere Probleme in handlichere Teilprobleme zu zerlegen
                 \item f"uhren bei Bedarf bis hin zur Zerlegung in Basis-Probleme
		 \item k"onnen auch auf Elemente des Lerntools verweisen
                 \end{sub_list_sabina}
\item
\textbf{Zwischenergebnisse}: 
                 \begin{sub_list_sabina}
                 \item numerische/boolsche Zwischenergebnisse zur
"Uberpr"ufung von L"osungen der Teilprobleme
                 \item dienen der "Uberpr"ufung von Zwischenergebnissen, um
notfalls eine Zerlegung in weitere Teilprobleme anzuraten
                 \end{sub_list_sabina}
\end{list_sabina}

Autoren von "Ubungsaufgaben m"ussen zu jeder "Ubungsaufgabe den
entsprechenden Satz an Tips, Zwischenergebnissen, Verweisen von
Basiselemente etc. mitliefern.\\
Das "Ubungstool sollte daher sinnvollerweise ``von innen nach au"sen''
entwickelt werden, also von den Aufgaben geringer Komplexit"at hin zu
den Aufgaben hoher Komplexit"at.


\clearpage

\section{Metadaten}\label{sec:metadaten}

Zu beachten ist zun"achst, da"s tats"achlich nicht die einzelnen
\textit{Probleme}, sondern vielmehr deren \textit{L"osungen} den
einzelnen Gesamtelementen bzw. Submodulen etc. zugeordnet werden, da
mehrere L"osungen existieren k"onnen, die \textit{verschiedenen}
Gesamtelementen bzw. Submodulen etc. zugeordnet werden m"ussen (etwa
die Berechnung der L"osung eines linearen Gleichungssystemes mit
verschiedenen Methoden).

Weiter gilt, da"s die Basis-Probleme ebenso wie die komplexeren
Probleme jeweils die Metadaten der Gesamt-Elemente
bzw. "ubergeordneten Hierarchien ``erben'', denen sie zugeordnet
werden. Dadurch wird ihre eigene Metastruktur wesentlich vereinfacht,
Redundanzen (und Widerspr"uchlichkeiten) vermieden.\\

Als ``eigene'' Metadaten erhalten sie diejenigen zus"atzlichen
Informationen, die sich auf ``die Aufgabe als solche'', nicht mehr
aber auf ihre inhaltlichen Abh"angigkeiten beziehen:

\begin{list_sabina}
\item
Schl"usselw"orter zur Stichwortsuche
\item
Typenbezeichnung wie Rechenaufgabe, Verst"andnisaufgabe, ...
\item
Schwierigkeitsgrad, normiert auf ``Erstkontakt''
\item
``mathematisches Label'' etwa ``algebraisch'', ``analytisch'', ...
\item
...
\end{list_sabina}

\clearpage

\section{Umsetzung}\label{sec:umsetzung}

Die enge Verkn"upfung vom "Ubungstool mit dem Lerntool erm"oglicht eine
ideale und f"ur den einzelnen Lernenden optimierte Unterst"utzung des
Lernprozesses f"ur verschiedenste Nutzergruppen:

\begin{list_sabina}
\item
\textbf{komplette ``beginner''} lernen mit Hilfe des Lerntools
die wesentlichen Begriffe und Konzepte (besser) kennen
und vertiefen ihre Kenntnisse dann mithilfe der zugeordneten
"Ubungsaufgaben, die ggf. ihrerseits als Hilfestellung
wieder auf andere Teile des Lerntools zur"uckverweisen
\item
\textbf{Wiederholer, Studierende in der Pr"ufungsvorbereitung}
spezifizieren mithilfe des Lerntools ihre Vorkenntnisse, 
erhalten darauf abgestimmte "Ubungsaufgaben und verwenden 
die Elemente des Lerntools ``zum Nachlesen'', als Hilfestellung etc.
\end{list_sabina}
 

Diese engen Verkn"upfung bedeutet aber, da"s die Realisierung des
"Ubungstools zur Entwicklung der Inhalte des Lerntools zeitlich
versetzt geschehen mu"s, da sonst keine Abgleichung hinsichtlich der
Hierarchie, somit keine inhaltliche Anbindung der Aufgaben an ihre
Thematik, und keine Spezifikation der Hilfen, Tips und R"uckgriffe auf
die Basis-Probleme m"oglich sind.\\
Da die Basis-Probleme selbst (als (umformulierte) Beispiele) bereits
Bestandteil des Lerntools sind, alle komplexeren Aufgaben aber
letztlich auf diese zur"uckverweisen, kann mit der Erstellung von
komplexeren Problemen des "Ubungstools erst begonnen werden, sobald
die Struktur des Lerntools in den entsprechenden Gebiet weitgehend
entworfen wurde. 

Das bedeutet aber nicht, da"s nicht bereits mit dem \textit{Sammeln}
geeigneter Aufgaben schon begonnen werden sollte: in Berlin steht
das Archivierungsprogramm ``Klaunix'' daf"ur bereit.\\
Von zentraler Bedeutung ist jedoch \textit{vor} einer Realisierung des
"Ubungstools eine umfassende Analyse des mathematischen
Verst"andnisprozesses im allgemeinen mit seinen Auswirkungen auf das
Design ``vern"unftiger Aufgabentypen und Aufgabenstellungen'' im
besonderen. Hier erhoffen wir uns intensive Beratung durch Frau
Prof. Keitel.






