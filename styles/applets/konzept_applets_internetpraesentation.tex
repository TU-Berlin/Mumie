\documentclass{article}
\usepackage{german}
\begin{document}
\section*{Konzeptpapier zur Appletpr"asentation}
\textbf{Beteiligte:} Markus Gronau, Tilman Rassy, Helmut Vieritz\\
\textbf{Datum:} 10.8.2004

\subsection*{Ziele} \begin{itemize} \item schneller "Uberblick "uber den aktuellen
Entwicklungsstand der Applets \item umfassende Pr"asentation der vorhandenen Applets \item
unabh"angige Pr"asentation der Applets im Internet \item Sicherstellung der redaktionellen
Kontrolle \end{itemize} \subsection*{Konzept} Die AppletFactory entwickelt eine eigene
L"osung zur Pr"asenation der Applets im Internet, die den jeweils aktuellen Stand
widerspiegelt. Dazu werden die beiden vorhandenen Ans"atze (Webseite bei Frank und Listing
von Tim) zusammengef"uhrt. Details dazu werden von Markus Gronau und Tim P"ahler
festgelegt. (Genaueres von Markus?) Mumiespezifische Anforderungen wie Statusanzeige und
redaktionelle Kontrolle werden nur soweit derzeit n"otig umgesetzt. Die L"osung
gew"ahrleistet, dass eine umfassende inhaltliche Darstellung der Applets erfolgt.

Es muss davon ausgegangen werden, dass der JAPS auf mittlere Sicht die Verwaltung der
Applets "ubernimmt. Um diesen Zeitraum zu verk"urzen, k"onnte der Plan zur
Realisierung eines eigenst"andigen nicht"offentlichen JAPS-Servers f"ur die
Content-Erstellung forciert werden.  Die Funktionalit"at des mmcdk wird separat soweit
ausgebaut, dass ein schneller "Uberblick "uber die vorhandenen Applets m"oglich ist
und die Integration von Applets in Bausteine erleichtert wird. Redaktionelle Kontrolle wird
hier nicht umgesetzt.

\subsection*{Begr"undung}
Die formulierten Ziele rechtfertigen eine separate Pr"asentation der Appletentwicklung, die
unabh"angig vom JAPS erfolgt. Die Einschr"ankung besteht darin, dass diese Pr"asentation
keine Mumie-spezifischen Anforderungen umsetzt bzw. nur soweit, wie absolut notwendig. Das
hei"st, dass die Pr"asentation des Status und vergleichbarer Informationen nur
provisorischen Charakter hat.
Um die Umsetzung einer umfassenden redaktionellen Kontrolle der Appletentwicklung zu
beschleunigen, w"are ein nicht"offentlicher JAPS geeignet, wie er ab Ende September f"ur die
Content-Erstellung geplant ist. Die "Ubernahme der redaktionellen Kontrolle durch diesen
Server w"are dann naheliegend und auch zeitlich entsprechend zu realisieren.
Die Viewer-Funktionalit"aten des mmcdk sollen vor allem die Integration von Applets in
Bausteine unterst"utzen. Hier wird also eine "Uberblicksfunktionalit"at
integriert. Technisch fehlerhafte und inhaltlich grob unvollst"andige Applets sind hier fehl
am Platz und sollen nicht angezeigt werden.
\end{document}