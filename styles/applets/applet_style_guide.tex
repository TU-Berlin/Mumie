\documentclass[a4paper,12pt]{article}
%
%$Id: applet_style_guide.tex,v 1.2 2004/08/25 18:45:53 vieritz Exp $
%

\usepackage{ngerman}
\usepackage{graphicx}

\parindent0em
\parskip0.75\baselineskip

\title{Applet-Styleguide \\ -- Vorschlag --}
\author{Frank Klich, Ester Manya Ndjeka, Helmut Vieritz}
\date{August 2004}
\DeclareSymbolFont{AMSb}{U}{msb}{m}{n}
\DeclareMathSymbol{\R}{\mathbin}{AMSb}{"52}


\begin{document}

\maketitle

\section{Grundidee}
Der Applet-Styleguide soll formale Aspekte festlegen, die eine einheitliche
Darstellung der Applets erm"oglichen sollen. Zus"atzlich soll der Redaktionsprozess f"ur
Applets vereinfacht werden, indem sichergestellt wird, dass Applets mit dem Status
"`devel ok"' den Richtlinien dieses Standards entsprechen.

\section{Einzelheiten}
\subsection{Allgemeines}
\begin{itemize}
\item \textit{Mehrsprachigkeit} - F"ur die Appletprogrammierung und den internen Gebrauch
  ist Englisch die Referenzsprache. F"ur alle Metainformationen und Bezeichnungen, die f"ur
  die Pr"asentation der Applets relevant sind, ist Deutsch die Referenzsprache. In diesem
  Fall sollten jeweils eine deutsche und eine englische Version vorliegen, sodass das Applet
  je nach Einstellung das Parameters "`lang"' komplett deutsch oder englisch erscheint. In
  diesem Styleguide wird "ublicherweise der deutsche Bezeichner gebraucht.

Ausnahme!? Die inhaltliche Hilfe wird in deutsch geschrieben und sp"ater extern "ubersetzt. 

\end{itemize}

\subsection{Funktionalit"aten}
\begin{itemize}
\item \textit{Animationen} - Animationen bleiben am Ende stehen und springen nicht zur"uck.
  Sie k"onnen editiert werden und zeigen die resultierenden Ver"anderungen sofort
  entsprechend dem in der Animation visualisierten Modell.
\end{itemize}
\subsection{Layout}

\subsection{Metainformationen}
\begin{itemize}
\item \textit{title} -- Der Titel ("Uberschrift) des Applets soll kurz und konkret sein, das
  hei"st da"s ein Spezialfall eines allgemeinen Begriffs mit enthalten ist. Ebenso wird die
  Kategorie benannt, z.B. Visualisierung eines Vektorraums - Pfeile im $\R^2$. Der Titel
  enth"alt keine Formeln.
\item \textit{description} -- Beschreibt das Applet so, dass man sich an das Applet wieder
  erinnern kann, wenn man es schon einmal gesehen hat. Enthalten sein sollte die Kategorie
  z.B. Visualisierung, "Ubung, Beispiel. "Ublicherweise ist sie eine Zeile lang (max .zwei
  Zeilen).
\item \textit{file name} -- (nur englisch) Entspricht dem \verb|title|, gegebenenfalls mit
  Abk"urzungen. Es gilt die Java-Konvention f"ur Klassennamen, also erster Buchstabe gro"s,
  keine Unterstriche als Worttrenner und neue W"orter werden mit einem Gro"sbuchstaben
  eingeleitet.
\end{itemize}
\subsection{Hilfeseite}
Die Hilfeseite besteht aus drei Komponenten und wird mit \verb|mmtex| erzeugt, d.h. sie wird
als LaTeX-Datei geschrieben und mit \verb|mmcdk| "ubersetzt und eingebunden. Die
Komponenten sind:
\begin{itemize}
\item \textit{Bedienungsteil}
\item \textit{technischer Teil}
\item \textit{inhaltlicher Teil} -- Hier soll die didaktische Idee, das Lehrziel, kurz
  erl"autert werden. Zus"atzlich wird auf Spezialf"alle und "`Pathologien"' zum Ausprobieren
  hingewiesen - es muss nat"urlich sichergestellt sein, dass das Applet damit auch
  zurechtkommt.
\end{itemize}
\subsection{Default-Einstellungen}
Die Default-Einstellungen betreffen den Aufruf des Applets ohne weitere Parameter.
\begin{itemize}
\item \textit{Button oder Canvas} -- Per Default erscheint das Applet als Canvas etc. also
  nicht als Button.
\item \textit{Gr"o"se des Applets} -- Das Applet bringt in den Metainformationen auf sich
  selbst passend zugeschnittene Gr"o"senangaben mit. Diese gehen nach dem Prinzip "`Sowenig
  wie m"oglich - soviel wie n"otig"'. N"otig ist eine Gr"o"se, bei der man im Default-Applet
  nicht scrollen muss, um alle Elemente zu sehen oder zu erreichen. Zus"atzlich soll eine
  User-Interaktion mit dem Applet im begrenzten Rahmen bequem m"oglich sein. Richtwert ist
  hier: Ausgangsabmessung + 50\%, das hei"st z.B. in einer 2D-Canvas: wenn die dargestellten
  Pfeile in der H"ohe 200 Pixel einnehmen, wird die H"ohe der Canvas mit 300 Pixeln
  eingestellt.
\item \textit{Hilfe-Button} -- Der Hilfe-Button ist immer vorhanden und befindet sich links
  unten.
\item \textit{Reset-Button} -- Der Reset-Button ist immer vorhanden und befindet sich rechts
  unten.
\item \textit{Start Animation-Button} -- Insofern eine einf"uhrende Animation angeboten wird,
  ist ein "`Start Animation"'-Button vorhanden, der sich wo??? befindet. Er ist nur einfach
  konfiguriert, d.h. es muss lediglich noch zwischen automatischer und manueller
  Schrittsteuerung der Animation ausgew"ahlt werden.
\item \textit{Toolbar} - Das ist eine offene Frage!!! Toolsbars sind momentan mal vorhanden
  und mal nicht. 
1. Die Toolbar ist immer vorhanden und bietet einfach ihre Funktionalit"aten an.
2. Die Toolbar ist nicht vorhanden und kann per Button aktiviert werden. Dafuer spricht,
dass Toolbars sehr verschieden sein koennen und 
\end{itemize}
\subsection{Applet-Parameter}
Autoren stehen bestimmte Parameter bei jedem Applet zur Verf"ugung, mit denen sie allgemeine
Aspekte steuern k"onnen. Diese sind im \verb|mmtex|-Styleguide aufgelistet.
\begin{itemize}
\item \textit{lang - de/en/auto} -- Steuert die Spracheinstellung des Applets. "`auto"'
  bedeutet, dass die Systemeinstellungen des Users ausgelesen werden und das Applet in
  Abh"angigkeit von der verwendeten Sprache erscheint.
\end{itemize}
\end{document}
