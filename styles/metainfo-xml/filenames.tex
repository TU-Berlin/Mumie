%Autoren: Tilman Rasy, Fritz Lehmann-Grube
%$Id: filenames.tex,v 1.1 2005/01/04 18:33:59 lehmannf Exp $


%                 MIGRATION VON CVS NACH JAPS -- �BERSICHT


\subsection{Sections-Konzept}
   ================

  F�r den Japs wird eine abstrakte hierarchische Gliederungsstruktur
  definiert. Diese hat eine Implementation in der Japs-DB einerseits und im
  lokalen Dateisystem des Benutzers andererseits.

  Die Gliederungsstruktur wird durch sog. ''Sections'' gebildet. Es gelten
  folgende Prinzipien:
\begin{enumerate}
\item    Sections sind Pseudo-Dokumente.

\item    Jedes Dokument, Pseudo-Dokument und jede Dokumenten-Quelle ist in
         genau einer Section enthalten.

\item    Damit ist insbesondere jede Section in genau einer Section
         enthalten. Dadurch entsteht die hierarchische Struktur.

\item    Die Struktur hat eine eindeutige Wurzel, die sog. ''Rootsection''. Die
         Rootsection ist in sich selbst enthalten. Ansonsten sind Zyklen
         verboten.

\item    Jede Section darf Dokumente, Pseudo-Dokumente und Dokumenten-Quellen
         enthalten. 
\end{enumerate}
\label{sec_sections}

%%------------- AKUT ! ------------------
%%Dez. 04 
%%Kleine "Anderung des Konzepts:\\
%%Alle (bisher nur <mumie:element>s, siehe \ref{sec_elements}) Dokumente sind in sections enthalten
%%(<mumie:contained_in> , siehe \ref{sec_contained_in}).\\
%%Daher brauchen Sections jetzt endlich mal eine korrekte Spezifikation.

siehe auch cvs/japs/docs/notes/section\_concept.txt\\

\emph{Knoten-Name:} \texttt{<mumie:section> ... </mumie:section>}

\emph{Attribute:} \verb+xmlns:mumie, use_mode, id, url+ wie oben in \emph{Abschnitt
  \ref{sec_common_attrs}} beschrieben, und im Wesentlichen mit den gleichen Bedeutungen wie dort.\\

Mindestens die Usemodi info, checkout und checkin sollen definiert sein.


Es k"onnen f"ur Sections eventuell andere Usemodi verwendet werden, sie beschreiben
aber jedenfalls einen Kontext innerhalb des JAPS.\\

\emph{Kinder:}\\ 

\texttt{<mumie:name></mumie:name>} siehe \emph{\ref{sec_name}}\\
\texttt{<mumie:description></mumie:description>} siehe \emph{\ref{sec_description}}\\
\texttt{<mumie:filename></mumie:filename>} \emph{nicht beim checkin!} siehe \emph{\ref{sec_filename}}\\
\texttt{<mumie:contained\_in></mumie:contained\_in>} \emph{nicht beim checkin!} siehe \emph{\ref{sec_contained_in_section}}\\

\texttt{<mumie:permission></mumie:permission>} siehe \emph{\ref{sec_document_permission}}\\

\subsubsection{Beispiel}
\begin{verbatim}
<?xml version="1.0" encoding="UTF-8"?>
<mumie:section xmlns:mumie="http://www.mumie.net/xml-namespace/document/metainfo" use_mode="checkin">
  <mumie:name>System</mumie:name>
  <mumie:description>Technical Stuff</mumie:description>

<!--  filename und contained_in werden dem Filesystem bzw. dem zipfile entnommen. Also:
          So nicht beim checkin !!
  <mumie:filename>system</mumie:filename>
  <mumie:contained_in id="0"/>
-->

  <mumie:permission type="read">
    <mumie:user_group id="0"/>
    <mumie:user_group url="japs/pseudo_document/user_group/authors.meta.xml"/>
  </mumie:permission>
  <mumie:permission type="write">
    <mumie:user_group id="0"/>
  </mumie:permission>


<!--    gibt's nicht mehr !!
  <mumie:level level="1" sublevel="0"/>
  <mumie:final value="false"/>
-->
</mumie:section>
\end{verbatim}


\subsection{Implementation in der DB}
   ========================

  Sections werden durch die Tabelle ''sections'' dargestellt. Diese hat
  mindestens die folgenden Spalten:
\begin{enumerate}
\item    id :  Prim�rschl�ssel

\item    filename :  Verzeichnisname bei Implemetierung im Dateisystem.

\item    contained\_in :  id der Section, in der diese Section enthalten ist.
\end{enumerate}
  Die Spalte 2.3 (contained\_in) ist auch in den Tabellen der �brigen
  Pseudo-Dokumente, der Dokumente und Quellen enthalten.


\subsection{Implementation auf dem lokalen Dateisystem}
   ==========================================

  Die Gliederungsstruktur wird auf eine Verzeichnisstruktur abgebildet. Dabei
  gelten folgende Prinzipien:

\begin{enumerate}
\item    Jede Section entspricht genau einem Verzeichnis. Solche Verzeichnisse
         sollen ''Section-Verzeichnisse'' genannt werden. Die Master-Datei einer
         Section liegt in ihrem Section-Verzeichnis.

\item    Die ''ist-enthalten-in''-Beziehung der Sections wird durch die
         ''ist-Unterverzeichnis-von''-Beziehung der Section-Verzeichnisse
         dargestellt.

\item    F�r alle (Pseudo-)Dokumenttypen <typ> ausser ''section'' gilt: Enth"alt
         eine Section (Pseudo-)Dokumente des Typs <typ>, so enth�lt das 
         entsprechende Section-Verzeichnis ein Unterverzeichnis mit dem Namen
         <typ>. Hierin befinden sich die Master- und ggf. Content-Dateien der
         entsprechenden (Pseudo-)Dokumente. Solche Verzeichnisse sollen
         ''(Pseudo-)Dokumenten-Verzeichnisse'' genannt werden.

\item    Enth�lt eine Section Text- und/oder Bin"ar-Quellen, so enth�lt das
         entsprechende Section-Verzeichnis ein Unterverzeichnis ''src\_txt''
         und/oder ''src\_bin'', in denen sich die Quellen befinden. Solche
         Verzeichnisse sollen ''Quellen-Verzeichnisse'' genannt werden.
\end{enumerate}


\subsection{Dateien}
   =======

  Zu einem nicht-generischen Dokument geh"oren folgende Dateien:

\begin{enumerate}
\item    Master-Datei :\\
           XML-Datei mit den Meta-Informationen.

\item    Content-Datei :\\
           Datei mit dem eigentlichen Inhalt; bei Text-Dokumenten: XML; bei
           Bin"ar-Dokumenten: entsprechendes Bin�r-Format.

\item    Keine, eine oder mehrere Quelldateien :\\
           Datei(en) mit den Quellen des Dokuments; bei Text-Quellen: XML,
           ggf. Wrapper-XML f�r das eigentliche Format; bei Bin�r-Quellen:
           entsprechendes Bin�r-Format.
\end{enumerate}

  Master- und Content-Datei geh�ren eindeutig zu einem Dokument. Quelldateien
  k�nnen auch zu mehreren Dokumenten geh�ren.

  Master- und Content-Datei eines Dokuments m�ssen immer im selben Verzeichnis
  liegen.

  Text-Quellen m�ssen grunds�tzlich als XML vorliegen. Ist das eigentliche
  Format nicht XML (was meistens der Fall ist), so existiert zu der
  eigentlichen Quelldatei eine sog. ''XML-Wrapper-Datei''.

  Generische Dokumente und Pseudo-Dokumente haben lediglich eine Master-Datei.


\subsection{Dateinamen}
   ==========

  Dateinamen bestehen aus einem sog. ''Rumpf'' (''body''), ggf. einem sog. ''Zusatz''
  (''addition'') und einer zweiteiligen Endung (suffix); jeweils durch Punkte
  getrennt; die zwei Teile der Endung ebenfalls durch einen Punkt getrennt:

    <body>[.<addition>].<suffix1>.<suffix2>

  (<suffix1> und <suffix2> bezeichnen die beiden Teile der Endung.)

  Der Rumpf ist der inhaltlich motivierte Name der Datei. Der Zusatz, wenn
  vorhanden, bezeichnet eine inhaltliche Variante. Der erste Teil der Endung
  klassifiziert die Datei nach ihrer technischen Rolle in der Mumie, der zweite
  nach ihrem Format.

  Rumpf und Zusatz sollten nur aus Kleinbuchstaben und Unterstrich (''\_'')
  bestehen. Ausser bei Sections darf der Rumpf nicht leer sein. Bei Sections
  muss der Rumpf leer sein; der Dateiname beginnt in diesem Fall mit einem
  Punkt (vgl. Beispiel (f) unten).

  F�r die Endung gilt:

    Master-Dateien:   meta.xml

    Content-Dateien:  content.<type>

    Quelldateien:     src.<type>

  Hierbei ist <type> eine Endung in Kleinbuchstaben, die dem Dateityp
  entspricht , z.B. ''xml'' f�r XML-Dateien, ''png'' f�r PNG-Dateien usw.

  Beispiele:
\begin{enumerate}
\item[a]    thm\_satz\_ueber\_irgendwas.nach\_mueller.meta.xml
          Master-Datei eines Dokuments �ber den ''Satz von Irgendwas'' in der
          Variante von ''M�ller''.

\item[b]    thm\_satz\_ueber\_irgendwas.nach\_schmidt.meta.xml
          Master-Datei eines Dokuments �ber den ''Satz von Irgendwas'' in der
          Variante von ''Schmidt''.

\item[c]    thm\_satz\_ueber\_irgendwas.nach\_schmidt.content.xml
          Content-Datei von s.o.

\item[d]    thm\_satz\_ueber\_irgendwas.nach\_schmidt.src.tex
          Quelldatei von s.o.

\item[e]    thm\_satz\_ueber\_irgendwas.nach\_schmidt.src.xml
          XML-Wrapper-Datei von s.o.

\item[f]    .meta.xml
          Master-Datei einer Section.        
\end{enumerate}

\subsection{Anh�ngbarkeit von Dokumenten an andere}
   ========================================

  Die ''Anh�ngbarkeit'' von Subelementen an Elemente wird im neuen System durch
  eine normale Referenz dargestellt. Daf�r wird ein neuer Referenztyp
  ''attachable'' eingf�hrt.

  Dieser Referenztyp verallgemeinert diese bisherige Relation ''contained\_in''.
  Die Sonderrolle der Subelement-Element-Beziehung entf�llt also.

  Referenzen des Typs ''attachable'' sind prinzipiell auch zwischen anderen
  Dokumenttypen erlaubt, insbesondere aber nicht, wenn "uberhaupt keine Referenzen
  zwischen den jeweiligen Typen erlaubt sind wie in config/config.xml durch das
  Attribut "no-refs-to" des referenzierenden Dokumenttyps definiert.



