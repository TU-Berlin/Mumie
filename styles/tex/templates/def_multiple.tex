\documentclass{japs.element.definition}
%$Id: def_multiple.tex,v 1.4 2005/02/04 10:53:12 vieritz Exp $
\begin{metainfo}
  \name{}
  \begin{description}

  \end{description}
  \copyrightinfo{(c) MUMIE-Projekt Technische Universitaet Berlin 2003}
  %\authors{}
% Hier kann die Masterdatei der Section angegeben werden.
% Fuer Default muss das Kommando nicht gesetzt sein.
  %\containedin{}
  \begin{components}

  \end{components}
% Bitte loeschen Sie alle redaktionellen Status bis auf den gewuenschten.
% Details finden Sie im mmtex Style Guide.
  \status{pre|devel_ok|content_ok|content_complete|ok_for_publication|final}
  \begin{changelog}

  \end{changelog}
\end{metainfo}
\begin{content}
\defnotion{}
% Leere Umgebungen muessen entfernt werden!
\begin{suppositions}

\end{suppositions}
% Die Layoutangaben koennen beliebig kombiniert werden.
\begin{defequivalence}[side-by-side|top-bottom]

\isdefinedas

\end{defequivalence}
\begin{defequivalence}[side-by-side|top-bottom]

\isdefinedas

\end{defequivalence}
\begin{remarks}

\end{remarks}
\end{content}
