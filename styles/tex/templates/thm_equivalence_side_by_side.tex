\documentclass{japs.element.theorem}
%$Id: thm_equivalence_side_by_side.tex,v 1.5 2005/02/04 10:53:13 vieritz Exp $
\begin{metainfo}
  \name{}
  \begin{description}

  \end{description}
  \copyrightinfo{(c) MUMIE-Projekt Technische Universitaet Berlin 2003}
  %\authors{}
% Hier kann die Masterdatei der Section angegeben werden.
% Fuer Default muss das Kommando nicht gesetzt sein.
  %\containedin{}
  \begin{components}

  \end{components}
% Bitte loeschen Sie alle redaktionellen Status bis auf den gewuenschten.
% Details finden Sie im mmtex Style Guide.
  \status{pre|devel_ok|content_ok|content_complete|ok_for_publication|final}
  \begin{changelog}

  \end{changelog}
\end{metainfo}
\begin{content}
\title{}
% Leere Umgebungen muessen entfernt werden!
\begin{suppositions}

\end{suppositions}
\begin{equivalence}[side-by-side]

\isequivalentto

\end{equivalence}
\begin{remarks}

\end{remarks}
\end{content}
