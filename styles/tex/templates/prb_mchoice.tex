\documentclass[mchoice]{japs.problem.mchoice}
%$Id: prb_mchoice.tex,v 1.2 2006/04/03 19:45:38 vieritz Exp $
%Template fuer ein Multiple-Choice-Arbeitsblatt

\begin{metainfo}
  \name{}
  \begin{description}

  \end{description}
  \copyrightinfo{(c) MUMIE-Projekt Technische Universitaet Berlin 2003}
  %\authors{}
% Hier kann die Masterdatei der Section angegeben werden.
% Fuer Default muss das Kommando nicht gesetzt sein.
  %\containedin{}
  \begin{components}

  \end{components}
% Bitte loeschen Sie alle redaktionellen Status bis auf den gewuenschten.
% Details finden Sie im mmtex Style Guide.
  \status{pre|devel_ok|content_ok|content_complete|ok_for_publication|final}
  \begin{changelog}

  \end{changelog}
\end{metainfo}

\begin{content}
\title{}
% Leere Umgebungen muessen entfernt werden!

\begin{mchoiceproblems}
  \mchoiceproblem
%Einleitender Text

%Ja-Nein-Auswahl
  \begin{choices}{yesno}

    \choice
    \assertion{}
    \solution{no}
    \explanation

    \choice
    \assertion{}
    \solution{yes}
    \explanation

    \commonexpl{}

  \end{choices}

%Eins-Aus-Mehreren-Auswahl
  \begin{choices}{unique}

    \choice
    \assertion{}
    \solution{yes}
    \explanation{}

    \choice
    \assertion{}
    \solution{no}
    \explanation{}

    \commonexpl{}

  \end{choices}

%Mehrere-Aus-Mehreren-Auswahl
  \begin{choices}{multiple}

    \choice
    \assertion{}
    \solution{yes}
    \explanation{}

    \choice
    \assertion{}
    \solution{no}
    \explanation{}

    \commonexpl{}

  \end{choices}

\end{mchoiceproblems}
\end{content}
