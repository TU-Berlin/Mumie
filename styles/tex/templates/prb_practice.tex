\documentclass[practice]{japs.problem.applet}
%$Id: prb_practice.tex,v 1.2 2006/04/03 19:45:38 vieritz Exp $
%Template fuer ein Arbeitsblatt mit Applet

\begin{metainfo}
  \name{}
  \begin{description}

  \end{description}
  \copyrightinfo{(c) MUMIE-Projekt Technische Universitaet Berlin 2003}
  %\authors{}
% Hier kann die Masterdatei der Section angegeben werden.
% Fuer Default muss das Kommando nicht gesetzt sein.
  %\containedin{}
  \begin{components}
    \defapplet{}{applet1}
  \end{components}
% Bitte loeschen Sie alle redaktionellen Status bis auf den gewuenschten.
% Details finden Sie im mmtex Style Guide.
  \status{pre|devel_ok|content_ok|content_complete|ok_for_publication|final}
  \begin{changelog}

  \end{changelog}
\end{metainfo}

\begin{content}
\title{}
% Leere Umgebungen muessen entfernt werden!

%Umgebung fuer Aufgabentext
\begin{task}

  \begin{subtasks}

    \subtask
%Teilaufgabe

   \subtask
%Teilaufgabe

 \end{subtasks}
\end{task}

%Umgebung mit Angaben fuer das Datasheet, Mit Stern bedeutet anklickbar.
\begin{input}

%Eigene Data-Umgebung
    \begin{data}{Path}

    \end{data}

    \begin{data*}{Path}

    \end{data*}

%Eigener Data-Befehl
    \data{DATA}{Path}
    \data*{DATA}{Path}

%Bezieht sich auf die laufende Umgebung
    \datalabel{Path}
    \datalabel*{Path}

\end{input}

%Referenziert genau ein Applet
\begin{execute}

%Applet in Dokument eingebettet
  \begin{applet}[Breite des Applets][Hoehe des Applets]{applet1}
    \param{lang}{de|en}           %Spracheinstellung fuer das Applet
  \end{applet}
  
%Applet in eigenem Fenster
  \begin{applet}[Breite des Buttons][Hoehe des Buttons]{applet1}
    \param{buttonText}{}          %Text fuer den Button
    \param{separateWindow}{true}  %eigenes Fenster
    \param{appletWidth}{}         %Breite des Fensters
    \param{appletHeight}{}        %Hoehe des Fensters
    \param{lang}{de|en}           %Spracheinstellung fuer das Applet
  \end{applet}

\end{execute}
\end{content}
