\section{Mathematische Benennungen und Notationen}\label{benennung_notation_math_style}


\subsection{Benennungen mathematischer Begriffe}\label{benennung_style}

"Ubergeordnete Metas:

\begin{list_sabina}
\item
\textbf{``richtig'', viel Inhalt vermittelnd}
\item
\textbf{mathematischer Standard}
\end{list_sabina}


\subsubsection{Grundlagen}\label{benennung_style_grundlagen}


Es werden ausschlie"slich die u.s. Hauptvarianten im Text verwendet
(die Nebenvarianten werden im Lexikon und beim ``ersten Vorkommen'',
in der betreffenden Definition, genannt):


\begin{tabular}{!{\vrule width 1pt} >{\bfseries}m{7.5cm} | >{\itshape}m{7.5cm} !{\vrule width 1pt}}
\hline
\hline
 & \\
Hauptvariante & Nebenvariante(n)\\
& \\
\hline
\hline
bijektiv & eineindeutig \\
\hline
injektiv & eindeutig \\
\hline
surjektiv & (Abb.) auf ...\\
\hline
& ...\\
\hline
& ...\\
\hline
\hline
\end{tabular}

t.b.c.!!


\clearpage


\subsubsection{Lineare Algebra}\label{benennung_style_lina}


Es werden ausschlie"slich die u.s. Hauptvarianten im Text verwendet
(die Nebenvarianten werden im Lexikon und beim ``ersten Vorkommen'',
in der betreffenden Definition, genannt):


\begin{minipage}{\linewidth}
\renewcommand{\thefootnote}{\thempfootnote}
\begin{tabular}{!{\vrule width 1pt} >{\bfseries}m{7.5cm} | >{\itshape}m{7.5cm} !{\vrule width 1pt}}
\hline
\hline
 & \\
Hauptvariante & Nebenvariante(n)\\
 & \\
\hline
\hline
antisymmetrisch & schiefsymmetrisch\footnote{Hier bezieht sich der Begriff
  schiefsymmetrisch auf Matrizen und lineare Abbildungen.}\\
\hline
antisymmetrische Multilinearform & alternierende Multilinearform\\
\hline
Bild \newline(einer Abb. $f:V\rightarrow W$, also $f(V)$) & Wertebereich/Wertevorrat
\footnote{Die Begriffe Wertevorrat und Wertebereich sollen nicht benutzt werden, weil
  sie in der Literatur keine einheitliche Bedeutung haben und schnell zu Verwirrung f"uhren
  k"onnen.} \\
\hline
Bildbereich \newline(einer Abb. $f:V\rightarrow W$, also $W$)& Wertebereich/Wertevorrat \\
\hline
darstellende Matrix (einer lin. Abb. ) & Darstellungsmatrix, Matrixdarstellung \\ 
\hline
Defektraum \newline(einer lin. Abb. $f:V\rightarrow W$, also \newline$W\ominus f(V))$& Differenz zwischen Bildbereich und Bild\\
\hline
Dualraum\footnote{Da wir in der Linearen Algebra lediglich endlichdimensionale Lineare R"aume
  betrachten, fallen algebraischer Dualraum und topologischer Dualraum zusammen.} & dualer Raum, topol. Dualraum, topol. Dual, Dual\\
\hline
elementare Umformungen (Gau"s) & Elementarumformungen%\footnote{Der Terminus
  %``Elementarmatrizen'' soll ebenfalls nicht verwenden werden, da dieser
  %autorabh"angig einmal die beim Gaussalgorithmus ben"otigten darstellenden Matrizen der
  %elementaren Umformungen bezeichnet und andererseits die Basis des linearen Raums der
  %Matrizen.}
\\
\hline
Elementarmatrizen & elementare Matrizen\\
\hline
Kreuzprodukt & Vektorprodukt \\
\hline
Minoren  & Streichungsmatrix\\
\hline
Nullraum & Kern \\
\hline
orthogonal, -normal & rechtwinklig\\
\hline
Parallelogrammgleichung & Parallelogrammidentit"at\\
\hline
partikul"are L"osung & Partikul"arl"osung \\
\hline
Quotientenraum & Quotientenvektorraum \\
\hline
Rechte-Hand-Regel & Korkenzieherregel\\
\hline
Skalarprodukt & inneres Produkt\\
\hline
Span, span(M) & lineare H"ulle\\
\hline
Spektraldarstellung & spektrale Darstellung\\
\hline
Spur, spur(A) & Trace, tr(A)\\
\hline
Unterraum & Untervektorraum, Teilraum\\
\hline
Vektorraum & Linearer Raum\\
\hline
\hline
\end{tabular}
\end{minipage}


\subsubsection{Analysis}\label{benennung_style_ana}

t.b.c.



%======================================================================================================
\clearpage
%======================================================================================================

\subsection{Benennung mathematischer Theoreme, Algorithmen etc.}\label{theoremname_style}

"Ubergeordnete Metas:

\begin{list_sabina}
\item
\textbf{mathematischer Standard}
\item
\textbf{historisch korrekt}
\item
\textbf{(kurz)}
\end{list_sabina}

\vspace{5mm}

Bei Theoremen werden die durch ``grammatikalisches Umdrehen''
entstehende Variationen zugelassen; es kann also durchaus einmal vom
``Austauschsatz von Steinitz'', ein anderes Mal vom ``Steinitz'schen
Austauschsatz'' gesprochen werden (wenn die verschiedenen Ergebnisse
nicht gar zu grauslig klingen...).

Auch andere ``Nebenvarianten'' zugelassen, wenn dadurch keinerlei
Verwechslungsgefahren bestehen. Unten sind einige solcher F"alle
exemplarisch angedeutet, \textbf{zul"assige} Nebenvarianten sind
``italic-fett'' gekennzeichnet.


\subsubsection{Lineare Algebra}\label{theoremname_style_lina}

\begin{tabular}{| >{\bfseries}p{65mm} | >{\itshape}p{65mm} |}
\hline
\hline
 & \\
Hauptvariante & Nebenvariante(n)\\
 & \\
\hline
\hline
Austauschsatz von Steinitz & \textbf{Steinitz'scher Austauschsatz}\\
\hline
Cauchy-Schwarz'sche Ungleichung & Schwarz'sche Ungleichung\\
\hline
Gram-Schmidt'sches Orthonormierungsverfahren & Schmidt'sches Orthonormierungsverfahren \\
\hline
Laplace'scher Entwicklungssatz & \textbf{Entwicklungssatz von Laplace}\\
\hline
Satz von Jordan & \textbf{Satz von der Jordan'schen Normalform} \\
\hline
\hline
\end{tabular}

t.b.c.!!

\subsubsection{Analysis}\label{theoremname_style_ana}

\begin{tabular}{| >{\bfseries}p{65mm} | >{\itshape}p{65mm} |}
\hline
\hline
 & \\
Hauptvariante & Nebenvariante(n)\\
 & \\
\hline
\hline
Satz von Bolzano-Weierstra"s & Satz von Bolzano\\
\hline
\hline
\end{tabular}

t.b.c.!!


%======================================================================================================
\clearpage
%======================================================================================================

\subsection{Bedeutung mehrdeutiger mathematischer Begriffe}\label{benennung_mehrdeutig_style}

In diesem Abschnitt finden sich die Regelungen zur Verwendung
mathematischer Begriffe, deren Bedeutung nicht einheitlich in der
Standardliteratur verwendet wird.

\subsubsection{Lineare Algebra}

\begin{tabular}{| >{\bfseries}p{45mm} | >{\bfseries}p{40mm} | >{\itshape}p{40mm} |}
\hline
\hline
 & & \\
 & ist/sind\dots & ist/sind nicht\dots \\
 & & \\
\hline
\hline
Adjungierte (Matrix) & an 45-Grad-Achse gespiegelte Matrix & die Adjunkten \\
\hline
Elementarmatrizen & Matrizen zur Realisierung der elementaren Zeilen- und Spaltenumformungen, 
etwa beim Gau"salgorithmus & die ``kanonische Basis'' im Raum der Matrizen\\
\hline
Sesquilinearform & die math. Version (linear im 1. Eingang) & die 
physik. Version (linear im 2. Eingang)\footnote{Im entsprechenden
Modul werden beide unterschiedlichen Notationen (und ihr Hintergrund) erl"autert.}\\
\hline
$\mathbb{N}$ & Nat"urliche Zahlen incl. der Null & nicht die nat"urlichen Zahlen ohne die Null\\
\hline
\hline
\end{tabular}

t.b.c.!!

\subsubsection{Analysis}

t.b.c.!!

%======================================================================================================
\clearpage
%======================================================================================================

\subsection{Defaults zur mathematischen Pr"azision und Notation}\label{math_praezise_style}\label{notation_mathsym_style}

Folgende Metas sind zur Einigung vorgeschlagen und w"aren dann grunds"atzlich zu beachten:

\begin{list_sabina}
\item
\textbf{kurz}
\item
\textbf{mathematisch pr"azise}
\item
\textbf{``standalone'' verstehbar}\footnote{Da die einzelnen Elemente 
auch immer als Referenzen f"ur andere fachliche Kapitel dienen, kann praktisch nie
von ``Kontextwissen'' ausgegangen werden: ohne Kontextwissen kann der ``quer'' auf dieses
Element Geleitete aber nicht wissen, ob $f$ nach $\mathbb{R}$
oder $\mathbb{C}$ abbildet...}
\item
\textbf{"ubersichtlich} 
\item
\textbf{mathematischer Standard} 
\end{list_sabina}

\vspace{5mm}

\textbf{Bemerkungen:} 

\begin{enumerate}
\item
ACHTUNG: Die u.s. Angaben sind die \textit{inhaltlichen} Vorgaben an
die Pr"azision mathematischer Objekte (und \textbf{nicht} die einzigen
Darstellungen); f"ur viele der u.g. Objekte wird eine weitere Textform
(etwa ``stetig differenzierbar'' anstelle von ``$C^{1}$'')
vorgehalten.
\item
Alle mathematischen Bezeichnungen und Symbole
(Vektoren, Matrizen etc. sowie All-, Existenzquantoren usw.) werden so
programmiert, da"s mittelfristig die M"oglichkeit der
Benutzeranpassung besteht.\\
Im u.s. handelt es sich um die \textit{Default-}Einstellung, die
verbindlich verwendet wird, solange diese Anpassung an den Benutzer
noch nicht m"oglich ist, und die au"serdem als echte 
``Defaulteinstellung'' (f"ur dem System nicht bekannte User) verwendet wird.
\item
Die u.s. Formulierungen ``entscheidend'', ``relevant'' sind immer unter
den ``standalone''-Anspruch zu betrachten!\\
Kontextwissen gibt es nur \textit{innerhalb} eines Elementes, 
h"ochstens noch beim "Ubergang zu gewissen abh"angigen Subelementen.
\item
Die u.s. Regelungen beziehen sich "uberwiegend auf das Gebiet der
Linearen Algebra, die "ubrigen Regelungen m"ussen noch folgen.
\end{enumerate}



%\begin{tabular}{|l|l|p{65mm}|}
%\hline
%\hline
%\myraise{Sei} & $f \in C^{1}$ & \myraise{\footnotesize{Def.- und Werteber. nicht relevant}}\\
%\cline{2-2}
%&$f$ stetig differenzierbar & \\
%\hline
%\myraise{Sei} & $f \in C^{1} ([a,b])$ & \myraise{\footnotesize{Def.-Ber. relelvant, Werteber. nicht}}\\
%\cline{2-2}
%&$f$ stetig differenzierbar in $[a,b]$& \\
%\hline
%\myraise{Sei} & $f \in C^{1} ([a,b]), \mathbb{C})$ & \myraise{\footnotesize{Def.- und Werteber. relevant}}\\
%\cline{2-2}
%&$f$ stetig differenzierbar in $[a,b]$ nach $\mathbb{C}$& \\
%\hline
%\hline
%\end{tabular}

\clearpage
\vspace{5mm}

\textbf{Die Regelungen:} 

\begin{list_sabina}
\item
\textbf{Quantoren:} $\forall$ und $\exists$ (alte Notation)
\item
\textbf{Skalare\footnote{Skalare kommen in verschiedensten Kontexten vor 
und es haben sich vielfach bestimmte Buchstaben f"ur bestimmte Sachverhalte 
etabiliert: diesen ``Standardkonventionen soll stets so weit wie m"oglich 
gefolgt werden.}}: kleiner nicht-fetter Buchstabe, Buchstabenwahl nach 
Bedarf, griechische Buchstaben grunds"atzlich zul"assig
\item
\textbf{Vektorr"aume}:
        \begin{sub_list_sabina}
        \item
        $V$: korrekt, wenn der K"orper nicht entscheidend ist
        \item
        $V [\mathbb{C}]$: korrekt, wenn der K"orper relevant ist
        \item
        Notation/Name Vektorr"aume: $V, W$ oder $V_{1},V_{2},...,V_{n}$ 
        \item
        Notation/Name Vektoren: $\mathbf{v}, \mathbf{w}$ oder 
        $\mathbf{v_{1}},\mathbf{v_{2}},...,\mathbf{v_{n}}$ (klein und fett)
        \item
        Notation/Name Vektorr"aume mit Zusatzstruktur: einfach $V$ und
        nicht $<V, \langle \cdot, \cdot \rangle>$
        \end{sub_list_sabina}
\item
\textbf{Basen}:
        \begin{sub_list_sabina}
        \item
        allg. Basis: $\{\mathbf{b_{1}}, \mathbf{b_{2}}, ..., \mathbf{b_{n}}\}$ 
        (bzw. $\{\tilde{\mathbf{b_{1}}}, \tilde{\mathbf{b_{2}}}, ..., \tilde{\mathbf{b_{n}}}\}$, falls eine zweite ben"otigt wird)
        \item
        kanon. Basis des $\mathbb{R}^{n}$: $\{\mathbf{e_{1}}, \mathbf{e_{2}}, ..., \mathbf{e_{n}}\}$
        \end{sub_list_sabina}
\item
\textbf{Normen}:
        \begin{sub_list_sabina}
        \item
        Standardnorm: $\| \cdot \|$
        \item
        Betrag: $| \cdot |$
        \item
        allg. Norm: mit Index, wenn nicht die Standardnorm gemeint ist, z.B. 
        $\| \cdot \|_{\infty}$
        \end{sub_list_sabina}
\item
\textbf{Funktionenr"aume}:
        \begin{sub_list_sabina}
        \item
        $C^{1}$: korrekt, wenn sowohl Werte- als auch Definitionsbereich 
        nicht entscheidend sind
        \item
        $C^{1}((a,b)), C^{1}(O)$\footnote{$O\subset\mathbb{R}^n$ offen.}: korrekt, wenn der Definitionsbereich (und dessen ``Struktur'') 
        relevant ist, der Wertebereich aber nicht
        \item
        $C^{1}((a,b), \mathbb{C})$: korrekt, wenn sowohl Definitions- als 
        auch Wertebereich relevant sind
        \item
        Notation/Name Funktionenr"aume: entsprechend mathematischem Standard
        \item
        Notation/Name Funktionen: $f, g$ oder $f_{1},f_{2},...,f_{n}$ 
        \end{sub_list_sabina}
\item
\textbf{R"aume linearer Abbildungen}:
        \begin{sub_list_sabina}
        \item
        $\mathcal{L}(V,W)$: (einzige Form\footnote{Das zwar ist nicht konsistent
        mit der Festlegung f"ur allg. Funktionenr"aume, aber 
        $L \in \mathcal{L}$ sagt einfach niemand...}) 
        \item
        Notation/Name von R"aumen linearer Abbildungen: $\mathcal{L}$ oder $\mathcal{L}_{1}, \mathcal{L}_{2}, ..., \mathcal{L}_{n}$
        \item
        Notation/Name von linearen Abbildungen: $L$ oder $L_{1},L_{2},...,L_{n}$ (gro"s)
        \end{sub_list_sabina}
\item
\textbf{Matrizenr"aume}:
        \begin{sub_list_sabina}
        \item
        $\mathcal{M}$: korrekt, wenn sowohl die Dimension als auch der zugrundeliegende
        K"orper nicht relevant sind
        \item
        $\mathcal{M}(m \times n)$: korrekt, wenn die Dimension relevant ist, 
        der K"orper aber nicht
        \item
        $\mathcal{M}(m \times n, \mathbb{R})$: korrekt, wenn sowohl die Dimension als 
        auch der K"orper relevant sind
        \item
        Notation/Name von Matrizenr"aumen: s.o.
        \item
        Notation/Name von Matrizen: $\mathbf{A}, \mathbf{B}$ oder 
        $\mathbf{A}_{1},\mathbf{A}_{2},...,\mathbf{A}_{n}$ (gro"s und fett)
        \item
        Notation Einheitsmatrix: $\mathbf{E}$ (gro"s und fett)\footnote{Wenn zus"atzlich die Dimension
          gekennzeichnet werden muss, dann schreiben wir $E_{n\times n}$.}
        \end{sub_list_sabina}
\item
\textbf{Dualr"aume}:
        \begin{sub_list_sabina}
        \item
        $V^{\ast}$: Dualraum eines allg. Vektorraumes
        \item
        $(\mathbb{R}^n)^{\ast}$: Dualraum des $\mathbb{R}^n$
        \end{sub_list_sabina}
\item
\textbf{R"aume der Bilinearformen}:
        \begin{sub_list_sabina}
        \item
        $\mathbf{B}(V,W)$: einzige Form
        \item
        Notation/Name von R"aumen von Bilinearformen: s.o.
        \item
        Notation/Name einer allg. Bilinearform: ???\footnote{Vorschl"age???}% $F, G$ oder $F_{1}, F_{2}, ..., F_{n}$
        \item
        Standardskalarprodukt: $\langle \cdot, \cdot \rangle$ (eckige Klammer)
        \item
        allg. Skalarprodukt: mit Index, wenn nicht das Standardskalarprodukt gemeint ist, z.B.
        $\langle \cdot, \cdot \rangle_{L^2}$
        \item
        Pseudo-Skalarprodukt\footnote{Gemeint ist das ``Skalarprodukt'' 
        der Minkowski-Metrik. Physiker kennzeichnen das Ding h"aufig 
        "uberhaupt nicht...}:  $``\langle \cdot, \cdot \rangle$'' 
        (eckige Klammern in Anf"uhrungszeichen)
        \end{sub_list_sabina}
\item
\textbf{R"aume der Multilinearformen}:
        \begin{sub_list_sabina}
        \item
        $\mathbf{\Lambda}{(V_{1}, \dots, V_{n})}$: einzige Form
        \item
        Notation/Name von R"aumen von Multilinearformen: s.o.
        \item
        \addtocounter{footnote}{-1}
        Notation/Name einer Multilinearform: ???\footnote{Vorschl"age???}% $F, G$ oder $F_{1}, F_{2}, ..., F_{n}$
        \item
        Notation/Name von Tensoren: $T$ und weitere geeignete Buchstaben\footnote{Hier 
        verh"alt es sich "ahnlich wie bei den Skalaren: bestimmte Buchstaben werden
        f"ur bestimmte Tensoren verwendet, etwa $G$ f"ur den metrischen Tensor der
        ART usw.}, 
        kovariante Indices unten, kontravariante Indeces oben, Einsteinkonvention
        \end{sub_list_sabina}
\end{list_sabina}


t.b.c.!!






























