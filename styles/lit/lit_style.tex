\section{Skizzen zum ``Literarischen'' Style}\label{lit_style}

\begin{list_sabina}

\item
\textbf{Rechtschreibung:}\\
Es gelten die Regeln der \textit{neuen} Rechtschreibung.

\item
\textbf{Anglizismen:}\\
Anglizismen sind nach M"oglichkeit zu vermeiden.

\item
\textbf{Abk"urzungen:}\\
Abk"urzungen sind nach M"oglichkeit zu vermeiden.\\
Sie sind notfalls zul"assig, wenn sie der besseren "Ubersicht dienen
und in formelartigen Zusammenh"angen verwendet werden,
m"ussen dann aber im standalone-Sinne immer beim ihrem ersten
Vorkommen innerhalb desselben Elementes oder Subelementes 
ausgeschrieben sowie der Abk"urzungsk"urzel erkl"art werden:

\begin{center}
\fbox{\parbox{110mm}{
Der zugeh"orige Eigenraum (ER) ist hier ...\\
Es gilt also:\\
\mbox{ }\hspace{40mm}ER$(\lambda)= ...$
}}
\end{center}

\item
\textbf{Satzbau:}\\
Texte werden grunds"atzlich im \emph{vollst"andigen Satz} geschrieben.
Anstelle reiner Stichwortansammlungen werden kurze S"atze mit
listenartiger Aufz"ahlung verwendet.\\
Eine Ausnahme stellen dabei ``Textchen'' dar, die eher einen 
Beschriftungscharakter haben (auf Buttons, an Bildern, 
in Instruktionen etc.).\\
Formeln, die einen vollst"andigen Satz ersetzen und dabei
einfach und pr"agnant sind, sind zul"assig
\footnote{Die Boxenumrandungen auf dieser Seite dienen nur zur Hervorhebung und geh"oren
\textbf{nicht} zum eigentlichen Style.}:

\begin{center}
\fbox{\parbox{110mm}{
Voraussetzungen: ...\\
\begin{center}
\fbox{$\det(A) \neq 0$} $\Longleftrightarrow$ \fbox{A ist invertierbar.} 
\end{center}
}}
\end{center}


\item
\textbf{Pers"onliche Anrede:}\\
Pers"onliche Anrede ist nicht zul"assig in allen geschriebenen 
(Sub-)Elementen des Lerntools sowie in den Aufgaben des "Ubungstools.\\
Sie ist dagegen zul"assig in angeleiteten Musterl"osungen.\\
Sie ist weiter zul"assig in den Audio-Teilen durch den ``Begleiter'' 
und in dessen visuellen Kurzanweisungen bei stark moderierten 
und/oder spielerisch gestalteten Teilen (``Folge mir'').\\
Wird pers"onliche Anrede eingesetzt, so wird stets \textbf{geduzt}.

\item
\textbf{``man-Stil'':}\\
In allen Teilen, in denen pers"onliche Anrede n"otig/sinnvoll w"are, 
aber nicht zugelassen ist (in Motivationen, Aufgaben
\footnote{Alternativ kann es auch hei"sen: ``zu berechnen ist...''} etc.), soll der
``man-Stil'' verwendet werden (``man sieht also'', ``man beachte'' etc.).

\clearpage

\item
\textbf{logische Symbole:}\\
Logische Symbole sollen verwendet werden, wenn sie eher ein Graphikelement, 
nicht ein Textbaustein sind, zum Beispiel innerhalb eines Theorems oder
einer Definition
\footnote{Grundsa"atzlich gilt das Prinzip, dass im Zweifelsfall
die k\"urzere Version (hier: Verwendung logischer Symbole versus ausgeschriebener Text)
zu bevorzugen ist.}:\\

\begin{center}\label{logische_symbole_bsp}
\fbox{\parbox{110mm}{
Gegeben sei $f:[a,b]\rightarrow\mathbb{R}$.\\
Dann gilt:\\
\begin{center}
\fbox{\parbox{55mm}{Sei 
\begin{itemize}
\item[i)]
$f$ stetig in $[a,b]$,
\item[ii)]
$f$ differenzierbar in $(a,b)$,
\item[iii)]
$f(a)=f(b)=0$.
\end{itemize}
}}
$\Longrightarrow$ 
\fbox{\parbox{35mm}{
Es existiert $c\in(a,b)$ mit $f'(c)=0$.}}
\end{center}
}}
\end{center}

F"ur jedes logische Symbol wird bei "Uberfahren durch die Maus ein
Text-Minihilfefenster sichtbar mit der verbalen "Ubersetzung.

\item
\textbf{``Konjunktiv-Sprache'':}\\
Es wird in den inhaltlich ``harten'' Teilen (insbesondere bei Definitionen,
Theoremen etc.)  die f"ur die Mathematik typische
``Konjunktiv-Sprache'' (``Sei $V$ Vektorraum, ...'') verwendet. In diesem Zusammenhang
ist auch eine Abschw"achung vom Prinzip des ``vollst"andigen Satzes'' erlaubt
(``Seien $V$ ein Vektorraum, $v,w\in V$.'' \emph{anstatt} ``Sei $V$ ein Vektorraum und seien
$v,w\in V$.'').

\item
\textbf{mathematische Voraussetzungen:}\\
Jedes (Sub-)Element/jede Aufgabe mu"s \emph{``standalone''} vollst"andig
verst"andlich sein; alle mathematischen Voraussetzungen 
(sofern vorhanden) m"ussen immer genannt werden.\\
Ausnahmen d"urfen lediglich bei denjenigen \textbf{Sub}elementen
auftreten, die ohne ihr zugeh"origes ``Papi-Element'' ohnehin
zusammenhangslos und somit nicht verst"andlich w"aren, 
etwa bei Beweisen eines Theorems etc.

\item \textbf{mathematische Sprache/mathematisches Sprachniveau:}\\ 
Es wird mathematisch pr"azise und i.a. mithilfe der "ublichen
symbolischen Fachnotation formuliert:``$f \in C^{1}(...)$'' 
(f"ur genauere Regelungen, hier etwa hinsichtlich der Angabe von
Definitions- und Wertebereich, siehe Kap. \ref{math_praezise_style}). \\
F"ur jede solche Formulierung (hier f"ur $C^{1}$) wird bei mouse-over ein verbales
Minihilfefenster, hier mit dem Inhalt ``stetig differenzierbar''
vorgehalten\footnote{Die Existenz
eines solchen Hilfefensters ist f"ur den Benutzer
transparent.}. Alternativ k"onnte dieser Text mittelfristig auch
h"orbar gemacht werden.\\
Die ``Symbolschreibweise'', etwa f"ur Funktionenr"aume, Quantoren etc.
ist, wann immer m"oglich, \footnote{Man erkennt an o.s. Beispiel zur
Verwendung logischer Symbole, da"s die Verwendung der Kurznotation
problematisch ist, wenn etwa zwei strukturell gleichartige Forderungen
gestellt werden, von denen die eine in einer Symbolschreibweise
notierbar ist ($f$ stetig), die andere aber nicht, weil kein
entsprechendes Symbol zur Verf"ugung steht ($f$ diffbar): In diesen
F"allen sollte f"ur beide Forderungen die ``Langversion'' verwendet
werden, um den Vergleich der Forderungen zu erleichtern.} zu
bevorzugen.\\
In einer sp"ateren "Uberarbeitung, wenn also die inhaltliche Seite
grunds"atzlich abgeschlossen ist, wird aus ihnen eine zweite
Version erstellt, die dann nicht die Kurznotationen, sondern
stattdessen ausgeschriebenen Text verwendet.

%%\item \textbf{mathematisches Sprachniveau:}\\ Es wird mathematisch
%%pr"azise und mithilfe der "ublichen Fachnotation formuliert (``$f \in
%%C^{1}(...)$'' bzw. "aquivalent ``$f$ stetig differenzierbar auf
%%...''\footnote{Die genauen Regel, in diesem Beispiel etwa hinsichtlich
%%der Angabe von Definitions- und Wertebereich, finden sich in
%%Kap. \ref{math_praezise_style}.} etc.)\footnote{F"ur jede solche
%%Formulierung wird bei mouse-over ein verbales Minihilfefenster mit dem
%%jeweils ``entgegengesetzten Inhalt'' ``stetig differenzierbar''
%%bzw. ``$C^{1}$'' vorgehalten, wenn man mit der Maus dar"uberf"ahrt.
%%(Die Existenz eines solchen Hilfefensters ist f"ur den Benutzer
%%transparent.). Alternativ k"onnte dieser Text mittelfristig auch
%%h"orbar gemacht werden.}.

\clearpage

\item
\textbf{Sloppy-Varianten:}\\
Mathematisch ``harte Elemente'' (``Definition'', ``Theorem'',
``Lemma'' und ``Algorithmus'' etc.) k"onnen zus"atzlich als
``Sloppy-Variante'' angelegt werden.\footnote{Beispiel: ``Jede an einer
Stelle $x_0$ differenzierbare Funktion ist dort auch stetig.''}
Sloppy-Varianten sind i.a. mehr text- als formel/symbolorientiert, ggf.
auch ein Bild.\\
Strukturell werden sie als Subelement ``Bemerkung'' behandelt.

\end{list_sabina}


\vspace{30mm}


\textbf{Anmerkungen:}


\begin{enumerate}

\item
Einige der oben getroffenen Regelungen sind mathematisch konservativ
gepr"agt; sie sind aber unbedingt notwendig, wenn mit vertretbarem
Aufwand Teile dieses Projektes auch f"ur andere Zielgruppen verwendbar
sein sollen.

\item
Auf die Pr"azision aller oben erw"ahnten mathematischen Konventionen\\
(etwa $f\in C^1$ versus $f\in C^1([a,b])$ versus $f \in C^{1}([a,b],\mathbb{R})$)
wird in Kapitel \ref{math_praezise_style} (``Defaults zur mathematischen Pr"azision'')
eingegangen.


\end{enumerate}




















