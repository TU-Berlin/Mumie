% Author: Tilman Rassy <rassy@math.tu-berlin.de>
% $Id: test.tex,v 1.9 2007/06/15 11:34:17 rassy Exp $
% Copyright (C) 2003, Berlin University of Technology

\documentclass[lang=de]{generic}

%\newcommand{\mycssloc}{file:///home/foo/bar/css}

%\rmstyles{*}
%\setstyles{\mycssloc/mystyle1.css, \mycssloc/mystyle2.css}
%\setstyles{foo/mystyle1.css, foo/bar/mystyle2.css}

\newcommand{\mycmd}[2][1][{Hello}]{#1 \emph{#2}}
\newcommand{\mumie}{MUMIE}
\newcommand{\helloworld}{\mycmd{world}}


\begin{document}

  \title{Test}
  \subtitle{Eine Testseite}

  Hier etwas einleitender Text

  \tableofcontents

  \section[foo]{\"{U}berschrift}

    Dies ist ein Test-Text. Er dient nur zum Testen, sein Inhalt ist egal. Der Text braucht
    auch gar keinen Sinn zu machen, denn er dient ja nur zum Ausprobieren. M\"ussen noch
    nicht einmal ganze S\"atze drinstehen, ja, noch nicht einmal richtige W\"orter.
    Dies ist ein Test-Text. Er dient nur zum Testen, sein Inhalt ist egal. Der Text braucht
    auch gar keinen Sinn zu machen, denn er dient ja nur zum Ausprobieren. M\"ussen noch
    nicht einmal ganze S\"atze drinstehen, ja, noch nicht einmal richtige W\"orter.
    Dies ist ein Test-Text. Er dient nur zum Testen, sein Inhalt ist egal. Der Text braucht
    auch gar keinen Sinn zu machen, denn er dient ja nur zum Ausprobieren. M\"ussen noch
    nicht einmal ganze S\"atze drinstehen, ja, noch nicht einmal richtige W\"orter.

    Etwas \emph{hervorgehobener} Text.

    Hier ein benutzerdefinierter Befehl: \mycmd{Welt!}

    % Auskommentiert

     
     Hier nochmal \emph{hervorgehobener % Auskommentiertes mit }-Klammer
     Text.} Hier ein paar Sonderzeichen, direkt eingegeben: � � � � � � � ����. Ein @.
     Normale "Anf\"{u}hrungszeichen". Ein paar andere 
     Soderzeichen: \$\&\%\#\_. In \{geschweiften Klammern\}. {\AA}ngstr{\o}m.
     Test: \"A, \"u. Im Wort: xx\"oyy.

     Und hier ein Bild: \image[][c][Mumie-Bild]{resources/mumie.png\balloon{Ein Bild!}\id{bild-id}}

     Noch etwas Text, mit < und > Zeichen. Interaktivit\"at:
     \emph{Action\onclick{alert('Geklickt!')}}.
     In Anf\"uhrungszeichen: \quoted{Hallo}.

     Ein \balloon{Balloon help} balloon help. Noch ein \emph{Test daf\"{u}r\balloon{Help 2}}.

     Eine Liste:

     \begin{itemize}[circle]
     \item Eins

       Neuer Absatz

     \item Zwei
     \item drei
     \end{itemize}

     Eine nummerierte Liste:

     \begin{enumerate}
       \item Eins
         \begin{enumerate}[alph]
           \item Eins A
           \item Eins B
         \end{enumerate}
       \item Zwei
       \item Drei
     \end{enumerate}

     Eine griechisch numerierte Liste:

     \begin{enumerate}[greek]
       \item Eins
       \item Zwei
       \item Drei
     \end{enumerate}

     Eine "description"-Liste:

     \begin{description}
       \item[Erstens] Die ist der erste Punkt der Liste
       \item[Zweitens] Die ist der zweite Punkt der Liste
       \item[Drittens] Die ist der erste Punkt der Liste
     \end{description}

     Etwas Mathematik: $-3xy+5a^{x^2+1}_z+9(u+w)$.

   \section*{Gesternter Abschnitt}

     Eine gesternte Section. 

   \section{N\"{a}chster Abschnitt}

     Und jetzt Tests mit \emph{counters}.

    \newcounter{mycounter}\setcounter{mycounter}{7}

    Wert: \value{mycounter}. Als Buchstabe: \alph{mycounter}. Als kleine r\"{o}mische Zahl:
    \roman{mycounter}. Als Fu\"{s}notensymbol: \fnsymbol{mycounter}.

    \newcounter{mycounterii}[mycounter]Noch ein counter. \setcounter{mycounterii}{4} Er hat
    jetzt den Wert \value{mycounterii}.
    \stepcounter{mycounter}
    Neue Werte: Erster Z\"{a}hler: \value{mycounter}, zweiter Z\"{a}hler: \value{mycounterii}.
    \setcounter{mycounterii}{\value{mycounter}}
    Wieder neue Werte: Erster Z\"{a}hler: \value{mycounter}, zweiter Z\"{a}hler: \value{mycounterii}.

    \subsection{Unterabschnitt}

    Innerhalb Unterabschnitt 1.1.

    Dies ist ein Test-Text. Er dient nur zum Testen, sein Inhalt ist egal. Der Text braucht
    auch gar keinen Sinn zu machen, denn er dient ja nur zum Ausprobieren. M\"ussen noch
    nicht einmal ganze S\"atze drinstehen, ja, noch nicht einmal richtige W\"orter.
    Dies ist ein Test-Text. Er dient nur zum Testen, sein Inhalt ist egal. Der Text braucht
    auch gar keinen Sinn zu machen, denn er dient ja nur zum Ausprobieren. M\"ussen noch
    nicht einmal ganze S\"atze drinstehen, ja, noch nicht einmal richtige W\"orter.
    Dies ist ein Test-Text. Er dient nur zum Testen, sein Inhalt ist egal. Der Text braucht
    auch gar keinen Sinn zu machen, denn er dient ja nur zum Ausprobieren. M\"ussen noch
    nicht einmal ganze S\"atze drinstehen, ja, noch nicht einmal richtige W\"orter.

    \begin{equation}
      a^2 + b^2 = c^2
    \end{equation}

    Etwas \code{Quelltext}. Eine \var{Variable}. Ein \plhld{Platzhalter}. Eingabe von der 
    \keyb{Tastatur}. - Noch mehr Mathe:

    \begin{equation}
      \frac{1}{1+x}
    \end{equation}

    Noch eine Formel:
    
    \begin{equation}
      \sqrt{\frac{3x - 8}{x^2 + y^2}}
    \end{equation}
    
    Eine Formel mit griechischen Buchstaben: $\alpha \beta \Gamma$. Und noch eine Formel,
    diesmal mit Integral:
    \begin{equation}
      \int_0^1\frac{x}{1+x^2}dx
    \end{equation}
    Ein Integral $\int_a^{y^2} f(x) dx$ im Fliesstext. Eine Summe:
    
    \begin{equation}
      \sum_{n=0}^\infty\frac{a_n}{n^2}
    \end{equation}
    
    Weitere Mathe-Tests:
    
    \begin{equation}
      \nolimits\int_0^1\frac{x}{1+x^2}\,d\!x
    \end{equation}
    
    Noch einer
    
    \begin{equation}
      \nolimits\int_{-\pi}^\pi f(x) \, \sin(x) d\!x
    \end{equation}

    Und noch einer
    
    \begin{equation}
      \lim_{N\to\infty}\sum_{n=0}^N\frac{x^n}{n!} \; = \; e^x
    \end{equation}
    
    Noch eine Formel:
    
    \begin{equation}
      \{1+\frac{1}{x}\}^4
    \end{equation}
    
    und noch eine:

    \begin{equation}
      \left.1+\frac{1}{x}\right|_{-1}^{X^2} \qquad \left.1+\frac{1}{x}\right|_1^2
    \end{equation}

    Abst\"{a}nde: 

    $xxx$ 

    $x\,xx$

    $x\:xx$
        
    $x\;xx$
    
    $x\!xx$

    $x\quad xx$ 

    $x\qquad xx$ 

    Eine Matrix:
    
    \begin{equation}
      A \; = \;
      \left(
        \begin{mtable}
          a_{11} & a_{12} \\
          a_{21} & a_{22}
        \end{mtable}
      \right)
    \end{equation}
    
    Vektoren:
    \begin{equation}
      \vec{a} \; = \;
      \left(
        \begin{mtable}
          -4 \\ 5 \\ 0
        \end{mtable}
      \right)
      \; , \qquad
      \vec{b} \; = \;
      \left(
        \begin{mtable}
          4x \\ 3y \\ x+y
        \end{mtable}
      \right)
    \end{equation}
    
    Vektoren mit Indizes:
    \begin{equation}
      \vec{F_{1+}}(x, y, z) \; = \;
      \left(
        \begin{mtable}
          3y \\ 2x \\ 5z
        \end{mtable}
      \right)
    \end{equation}
    
    Ein
    
    \subsubsection{Unter-Unterabschnitt}
    
    Innerhalb des Unter-Unterabschnitts.

    
  \section{Zwei}

  Hier innerhalb des zweiten Abschnitts.
  Innerhalb Unterabschnitt 1.1.

  \begin{table}
    \head
    eins & zwei & drei
    \body
    \rowspan[plhld]{2} 11 & 12 & 13 \\
    22 & \anchor{test} \\
    31 & 32 & 33\anchor[verankert]{foo} \\
    41 & \colspan{2} 42 43 \\
    \colspan{2}
    \begin{table}
      a & b \\
      c & d 
    \end{table} & 0
  \end{table}
  
  \begin{table}
    aa & ab & ac \\
    ba & bb & bc \\
    ca & cb & cc
  \end{table}
  
  Jetzt soll etwas praeformatierter Text ausgegeben werden:
  
  \begin{preformatted}[code]%
Ganz links            grosser Zwischenraum

         \emph{hervorgehoben}
 \alt{{eins}{zwei}{\var{drei}}}
         untereinander
         untereinander%
    \end{preformatted}

    Und jetzt ein Link zum \link{http://www.xemacs.org}{XEmacs}, dem besten Editor der Welt.
    
    Und jetzt verbatim Text:

    \verbatim[code]
 Dies ist verbatim Text. Befehle, 
 
   \section{Name}
     \foo[bar]{ein Argment}

 sollten nicht
             interpretiert werden
    \endverbatim

    Dasselbe nochmal, jetzt mit anderer Endmarke.

    \verbatim[code][stop]
 Dies ist verbatim Text. Befehle, 
 
   \section{Name}
     \foo[bar]{ein Argment}

 sollten nicht
             interpretiert werden
\endverbatim
\stop

    Verbatim im Fliesstext \verb$\foo{bar}$. Der Code einer Tabelle:

\verbatim[code]
 \begin{table}
   aa & ab & ac \\
   ba & bb & bc \\
   ca & cb & cc
 \end{table}
\endverbatim

    Test \emph[foo]{Geht das?}.

    Benutzerdefinierte L\"angen.
    Definition einer L\"ange \verb'\mylength'. \newlength{\mylength}[12em]
    Ausgabe: \writelength{\mylength}.
    Ausgabe in mm: \writelength[mm]{\mylength}.
    Formatiert: \writelength[mm][.2]{\mylength}.
    
    \mylength{10em}
    Ausgabe: \writelength{\mylength}.
    Ausgabe in mm: \writelength[mm]{\mylength}.
    
    Definition einer weiteren L\"ange \verb'\mylengthii'. \newlength{\mylengthii}[0.5\mylength]
    Ausgabe: \writelength{\mylengthii}.
    Ausgabe in mm: \writelength[mm]{\mylengthii}.
    
    Setzen auf einen neuen Wert mit \verb'\setlength': \setlength{\mylengthii}{8ex}
    Ausgabe: \writelength{\mylengthii}.
    
    Ende des Tests.
\end{document}