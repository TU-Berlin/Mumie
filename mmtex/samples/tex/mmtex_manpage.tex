% -*- mmtex -*-
% Author: Tilman Rassy <rassy@math.tu-berlin.de>
% $Id: mmtex_manpage.tex,v 1.1 2003/05/08 15:56:42 ruppert Exp $
% Copyright (C) 2003, Berlin University of Technology

\documentclass{webpage}

\addstyles{manpage.css}

\newcommand{\convertoptions}{\plhld{convert\_options}}
\newcommand{\queryoptions}{\plhld{query\_options}}
\newcommand{\MmTeX}{MmTeX}
\newcommand{\Mumie}{Mumie}
\newcommand{\LaTeX}{LaTeX}

\begin{document}

\title{MmTeX Manual Page}
\subtitle{Converting \LaTeX-like code to XML}

Tilman Rassy \link{mailto:rassy@math.tu-berlin.de}{<rassy@math.tu-berlin.de>}

Copyright \copyright 2003, Berlin University of Technology

\verb'$Id: mmtex_manpage.tex,v 1.1 2003/05/08 15:56:42 ruppert Exp $'

\tableofcontents


%\section{Name}
%
%\code{mmtex} -- Convert \LaTeX-like code to XML.


\section[synopsis]{Synopsis}

\begin{preformatted}[code]%
mmtex [\convertoptions] [--time | --notime] \var{source}
mmtex [\queryoptions] [--time | --notime]
mmtex [--help | --version] [--time | --notime]
\end{preformatted}


\section{Description}

This manual describes the usage of the \code{mmtex} executable. For the \MmTeX \LaTeX
dialect or the Perl modules and libraries, see the respective documentations. 

\code{mmtex} converts the \MmTeX source file \var{source} to XML. By default, the output is
written to a file with the same name but the suffix replaced by "xml". This can be changed
using the \code{--output} option. If logging is enabled, \code{mmtex} prints log messages to
a file. Logging can be switched on and off by the \code{--log} and \code{--nolog} option,
respectively. Default is off.  By default, the name of the log file is \var{source} with the
suffix replaced by "log". This can be changed using the \code{--log-file} option. For more
information, see the \link{\#convert\_options}{convert options} and the
\link{\#converting\_examples}{converting examples}.

With query options, \code{mmtex} prints help on \MmTeX commands and environments; either as
plain text to stdout, or as XML to a file. For details, see the \link{\#query\_options}{query
options} and \link{\#query\_examples}{query examples}.  


\section{Options}

\subsection[convert\_options]{Convert Options}

(Abbreviated with \convertoptions in the \link{\#synopsis}{synopsis}.)

\begin{description}[code-doc]
  \item[--columns=\var{number}]
    Set text width to \var{number} columns.
  \item[--dcl-opts=\var{options}]
    Set additional document class options \var{options}.
  \item[--dcl-path=\var{path}]
    Set the search path for document classes to \var{path}.
  \item[--inc-path=\var{path}]
    Set the search path for libraries to \var{path}.
  \item[--indent=\var{number}]
    Indent each level by \var{number} columns.
  \item[--log]
    Print log messages.
  \item[--log-exclude=\var{list}]
    Suppress logging of data specified by \var{list}. \var{list} must be a comma-separated
    list of keywords. Each keyword specifies a group of logging data. The following keywords
    are recognized:

    \begin{table}[plain]
      \head
        Keyword & Meaning & Default
      \body[code,normal,normal]
        cmds & Log command table & off \\
        envs & Log environment table & off \\
        tokens & Log token table & off\\
        allowed-tokens & Log list of alowed tokens & off \\
        output-functions & Log output function table & off\\
        output-list & Log output list & off
    \end{table}

  \item[--log-expand-depth=\var{number}]
    Set maximal depth of expanding references in logged data to \var{number}. By default,
    references are not expanded, i.e., they are printed as Perl converts them to
    strings. But with this option, references are resolved recursively up to a depth of
    \var{number}. If you changed the default, you can use this option to suppress reference
    expansion by setting \var{number} to -1.
  \item[--log-include=\var{list}]
    Enable logging of data specified by \var{list}. \var{list} must be a comma-separated
    list of keywords. Each keyword specifies a group of logging data. See option
    \code{--log-exclude} for the allowed keywords and their meanings.

    \emph{Note:} Sice each of these groups of data is represented by a reference, you should
    consider the \code{--log-expand-depth} option.
  \item[--log-file=\var{file}]
    Set the file where the log messages go to \var{file}.
  \item[ --log-root=\var{path}]
    Set the root directory of the log files to \var{path}.
  \item[--nolog]
    Don't print log messages.
  \item[ --output=\var{file}]
    Set the name of the output file to \var{file}. If omitted, the name is constructed from
    \var{source} by replacing the suffix by \code{xml}.
 \item[--source-root=\var{path}]
    Set the root directory of the sources to \var{path}. Noramlly, sources (either
    \var{source} or files imported from \var{source} by means of the \code{\verb'\insert'}
    command) are searched with the current directory as root. This can by changed using this
    option.
 \item[--special-char-mode=\var{mode}]
    Set special character mode to \var{mode}. This controles how special characters are
    represented in the XML output. The following modes are defined:

    \begin{table}[plain]
      \head
        Mode & Meaning
      \body[code,normal]
        as-entities & As entities (e.g., \code{\&uuml;}) \\
        as-numcodes & As numerical codes (e.g., \code{\&\#252;}) \\
        literal & Literally (e.g., \code{\"{u}}), except \code{<},\code{>}, and \code{\&} \\
        strictly-literal & Literally including  \code{<},\code{>}, and \code{\&}
    \end{table}

    Default is \code{as-numcodes}.
 \item[--src-root=\var{path}]
    Same as --source-root=\var{path}.
 \item[--target-root=\var{path}]
    Set the root directory of the "final" targets to \var{path}. Final targets are the
    documents made from the XML; usually HTML or XHTML documents. This option may be
    necessary for \code{mmtex} to be able to set links correctly.

    \emph{Note: This option may become deprecated soon.}
 \item[--trg-root=\var{path}]
    Same as --target-root=\var{path}.
 \item[--xml-root=\var{path}]
    Set the root directory of the XML output files to \var{path}. Noramlly, XML outputs are
    written to locations with the current directory as root. This can by changed using this
    option.
 \item[--xsl-stylesheet=\var{uri}]
    Set the default XSL stylesheet to \var{uri}. This is the stylesheet entered in the
    \code{xml-stylesheet} processing instruction.
\end{description}

\subsection[query\_options]{Query Options}

(Abbreviated with \queryoptions in the \link{\#synopsis}{synopsis}.)

\newcommand{\HintDclNeeded}{Use this together with the \code{--dcl} or \code{--query-dcl} option
 to specify a documentclass to which the list referres.}

\begin{description}[code-doc]
  \item[--dcl=\var{documentclass}]
    Same as \code{--query-dcl=\var{documentclass}}.
  \item[--describe-cmd=\var{cmd}]
    Describe command \var{cmd}.
  \item[--describe-env=\var{env}]
    Describe environment \var{env}.
  \item[--help-cmd=\var{cmd}]
    Same as \code{--describe-cmd=\var{cmd}}.
  \item[--help-env=\var{env}]
    Same as \code{--describe-env=\var{env}}.
  \item[--lib=\var{library}]
    Same as \code{--query-lib=\var{library}}.
  \item[--list-cmd-libs]
    List all libraries that may export commands.
  \item[--list-env-libs]
    List all libraries that may export environments.
  \item[--list-cmds]
    List commands.
  \item[--list-envs]
    List environments.
  \item[--output=\var{file}]
    Set the name of the file where the documentation produced by the
    \code{--xdoc-\plhld{xxx}} options should go to. If omitted, a default is used depending
    on the particular \code{--xdoc-\plhld{xxx}} option.
  \item[--query-dcl=\var{documentclass}]
    Set the documentclass to which the queries refer.
  \item[--query-dcl=\var{library}]
    Set the library to which the queries refer.
  \item[--xdoc-cmd-libs]
    Make a list of all command-exporting libraries, and write it in an XML format to the
    file \file{libraries.xml}. The latter filename can by changed using the \code{--output}
    option. 
  \item[--xdoc-env-libs]
    Make a list of all environment-exporting libraries, and write it in an XML format to the
    file  \file{libraries.xml}. The latter filename can by changed using the \code{--output}
    option. 
  \item[--xdoc-cmds]
    Make a documentation of all commands, and write it in an XML format to the file
    \file{commands.xml}. The latter filename can by changed using the \code{--output}
    option. 
  \item[--xdoc-envs]
    Make a documentation of all environments, and write it in an XML format to the file
    \file{environments.xml}. The latter filename can by changed using the \code{--output}
    option. 
\end{description}

\subsection{Other Options}

\begin{description}[code-doc]
  \item[--help]
    Print a help message and exit.
  \item[--notime]
    Don't print elapsed CPU time (default).
  \item[--time]
    Print elapsed CPU time
  \item[--version]
    Print version information and quit
\end{description}


\section{Examples}

\subsection{Converting Examples}

\begin{preformatted}[code]%
mmtex \file{foo.tex}
\end{preformatted}

Converts \file{foo.tex} and writes the result to \file{foo.xml}.

\begin{preformatted}[code]%
mmtex \file{foo.tex} --output=\file{bar.xml}
\end{preformatted}

Converts \file{foo.tex} and writes the result to \file{bar.xml}.

\begin{preformatted}[code]%
mmtex \file{foo.tex} --log
\end{preformatted}

Converts \file{foo.tex}, writes the result to \file{foo.xml}, and writes log messages to
\file{foo.log}.

\begin{preformatted}[code]%
mmtex \file{foo.tex} --log --log-include=cmds,envs --log-expand-depth=2
\end{preformatted}

Converts \file{foo.tex}, writes the result to \file{foo.xml}, and writes log messages to
\file{foo.log}. The log messages include the command and environment tables. References in
the logged data are resolved up to a depth of 2.

\subsection{Query Examples}

To optain a list of all commands defined in the library \code{counter}, type

\begin{preformatted}[code]%
mmtex --list-cmds --lib=counter
\end{preformatted}

This should print the following to stdout:

\verbatim[code]
\Alph                Prints the value of COUNTER as a upper case latin character
\Roman               Prints the value of COUNTER as a upper case roman number
\addtocounter        Adds NUMBER to the value of COUNTER
\alph                Prints the value of COUNTER as a lower case latin character
\arabic              Prints the value of COUNTER as an arabic number
\fnsymbol            Prints the value of COUNTER as a footnote symbol
\newcounter          Defines COUNTER as a new counter
\roman               Prints the value of COUNTER as a lower case roman number
\setcounter          Sets the value of COUNTER to VALUE
\stepcounter         Increases the value of COUNTER by 1
\value               Returns the value of COUNTER
\endverbatim

To get more information about the \verb'\newcounter' command, type

\begin{preformatted}[code]%
mmtex --describe-cmd=newcounter --lib=counter
\end{preformatted}

This should print the following to stdout:

\verbatim[code]
\newcounter{COUNTER}[MASTER]

  Defines COUNTER as a new counter. If MASTER is set, it must be the name of
  another counter. Each time MASTER is incremented, COUNTER is reset to 0.

  COUNTER   Name of the new counter

  MASTER    Counter that resets COUNTER

  Defined in: counter
\endverbatim



\end{document}