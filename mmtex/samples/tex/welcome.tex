\documentclass{webpage}

\begin{document}
Dies ist ein Test. Nach dem Punkt.

\title{Welcome}

\section{What is Mumie?}
\emph{ An e-learning platform, developed to provide a student and teacher friendly environment for 
learning and teaching.}

If that's enough for you: alright. If not: read on and you will learn more
about the features Mumie comes with, how to use it, the development of this project, 
our intentions, our standards, the technology we use, why we use it and a lot more.

\subsection{Features}
Content is of course the big advantage Mumie comes with. We provide you with tons of
applets, pictures, texts and \quoted{ready-to-learn} courses, that thus provide the default courses
you can offer to your students, if you don't want to design your own courses.

If you like the content but not the road the course suggests, you are free to rearrange the
content in any manner you prefer.

The technical side of Mumie basically consists of several \quoted{tools}, as we call them, which
are in detail:

\begin{enumerate}
\item The Course Creator

For teachers and their assistants. A Java[tm]-based software, designed to, well, create courses by 
basically \quoted{drag and drop}. Here you can choose, what students should learn in your course, 
set a path through the subjects, offer additional information or exercises.
\item The Learning Tool

As you guess: for students. After they selected a course they will be sent on \quoted{the journey}.
On that journey they are free to just travel the path their teacher set for them, or perhaps, 
\quoted{leave the road} to take a closer look at something they are interested in.

\item The Training Tool

Also for students. Exercises for everything they saw in the Learning Tool. Where possible with
a practical background. These exercises can be designed by the teacher of the course, but it is
also possible for the student to simply search for exercises on, e.g. Eigenvalues.
\item The Lexicon Tool (Encyclopedia?)

A place where students (or of course teachers ;-)) can look up a short but 
precise definition of a mathematical term.
For each definition they will find links that lead them
to more information related to this subject.

The second purpose of this tool is to offer interesting articles on modern mathematics, anecdotes,
history of Math, etc.
\item Communication

Mumie provides lots of possibilities for communication. Once a student or a member of teaching 
personel has created an account, they have full access on the forums, private messaging, email and
chat.

\item The MumieTeX to MumieXML Converter

TeX is widely spread among people working with Mathematics so we had the idea that an author 
shouldn't need to learn another language (like XML) to write a text on a topic. Therefore we developed 
our \quoted{own} TeX-style, including tags for plugging in all types of pictures, animations and of 
course interactive applets that could help explaining the mathematical content.

It is easy to learn, and an author actually just has to press a button in order to convert it to 
valid XML. %erwaehnen, dass man auch andere Konverter benutzen kann???
\item Upload

....t.b.c.....

\item The Applet Factory

....t.b.c.....
\end{enumerate}

\subsection{Requirements for using MuMIE}

\subsubsection{Accessing the courses}
We would love to say: a browser. But we don't spin the web and we have to deal with some 
restrictions. Math in the web still is a problem for HTML has't been invented for more complex
things than just plain text and math is more than text... We want to display formulas, equations,
and everything else related to math. Also we want to access these mathematical objects with 
JavaScript or Java, in order to get an \emph{ interactive} learning platform. So, what you need 
for \emph{ optimal} access on all the possibilities of Mumie:

\begin{enumerate}
\item a MathML- capable Browser

We recommend Mozilla 1.1 or above, as it is \quoted{ready to go} with its support for MathML integrated in the
standard-version. Downloadable at: 

www.mozilla.org

If you insist on using IE you have to download a plug-in, ActiveX... that is able to display
MathML. At the moment there are two InternetExplorer extensions available:

Techexplorer (from IBM) for IE all Versions from 4.0 UP TO 5.1 (!!!). All higher Versions are not 
yet supported, for Techexplorer is a classical plug-in which are no longer supported since IE 5.5!

MathPlayer (by DesignScience) for all IE Version from 6.0 and higher.
\item Browser needs Java 1.4 -plug-in (or higher)

This is necessary for the applets embedded in the courses and the navigation we use.

(Of course Java also has to be enabled in the browser...)You can download Java from

www.java.sun.com

for your specific Browser and operating system.
\item JavaScript and Cookies enabled

% THIS SENTENCE IS UTTER NONSENSE.  AT BEST, IT'S DANGEROUS.
We know that enabling these by default is a security risk, but we garantee, that we will do no
harm to your computer...:-)

\end{enumerate}
Remember: these are the OPTIMAL settings that we recommend. You can have a look at our site without 
e.g. Java and MathML and yes, you will see something, but there are some features you will not be 
able to use. Applets and formulas will be nothing more than pictures, and thus your possibilities of 
interacting with these are restricted. (Examples???)

\subsubsection{Setting up an own platform}
Of course you need a computer that can work as a server and it should't be a 386 with 10 MB RAM...

You have to consider, that this computer has to manage a complex database, do a lot \quoted{server-side}
stuff (see next chapter on that) and of course has to deal with perhaps thousands of requests at a
time.

The operating system we used and thus tested thoroughly for our purposes is Suse Linux 7.(1??).
Actually all the components we use don't \emph{ require} Linux, you can use any operating system you like
but we can not guarantee it will work the way it should and can not fully support OS related problems
that might occur (ARRGHS NEED HELP ON THAT!!!!).

What else? Mumie, of course: go here 

....t.b.c.....

If you need help 

.....t.b.c.......

for full installation instructions go here:

.....t.b.c.......

for a detailed FAQ:

.....t.b.c.......


\subsection{Components and standards we use}

\subsection{Getting started}
.....
\subsection{Why we developed a new platform - a history of Mumie}
The idea of the project has been developed at the Technical University of Berlin, Department of 
Mathematics, and in the beginning it wasn't intended to become such a \quoted{technical} thing.
The question at hand was: How can we use the WWW and the new technologies coming along with it,
for education? Or better: Does it make sense to use them in education and if yes, where would a 
webbased tool help improve the learning process for students.

Quite naturally we focussed on the Math training for engineers, since a lot of the people
developing the original idea of Mumie were involved in this part of education.
For most students, Math is one of the most challenging courses they have to pass. Whoever wants 
to work in a technical or scientific job has to do at least a year of math (for Science two years).
Among the students the interest for mathematics differs from: \quoted{Let me pass quickly, I don't care
about the details!} to \quoted{Hey, what's that? Give me more!}. Along with the interest comes of course
the fact that there are students/people who are definitely more talented or simply had a better 
training before they went to university than others. Getting them all together is the challenge 
of the teaching personel, from Professor to tutor, each semester. And ... it works. But, of
course, you have to compromise. Some people are disappointed because they thought each step would 
be proven, others need more training or simply more time to get comfortable with the content
presented during lectures. Then, there is another, take it literally, big problem: communicating
with every single student might work if you have a course where 30 people greet you in the morning.
Having around 1200 students yelling their \quoted{Good Morning}, it could be difficult to shake each's 
hand, ask if they have problems, or offer help.

So: what would help?

\begin{enumerate}
\item An online learning course, reflecting what has been done during lectures

(This of course shouldn't mean that the content of the lectures is put online and that's it.)
\item An online teaching course, offering exercises
\item Some sort of \quoted{glossary} where students can look up terms they didn't understand

(The explanation should be more then a \quoted{translation}, it should lead to places of the platform
where these terms are explained and trained in depth)
\item A communication platform

This includes everything from Chat, Forum to an easy \quoted{Messenger} sort of thing, as well as
good old email (and, even older, telephone numbers...)
\end{enumerate}
Alright, you might say; where is the problem? The answer is; if we had found a (technical) platform 
that offered everything we needed, we would now be producing content all day... But we aren't.
We didn't find a technical solution for our needs, so we had to do it ourselves. To understand
why we didn't take an existing platform, we have to examine our demands more thoroughly.

\subsubsection{What we want}
As mentioned before, our first intention was to create content. Create courses, images, texts, applets,
sounds, whatever you might use to teach math. But very soon we realized that providing content
means you have a technology that is able to serve your needs. 

So, what \emph{ did} we need?
\begin{enumerate}
\item Reliability/Stability

% Course, that isn't the real reason.  The real reason is that this is a
% production system.  People work providing content (this costs money), people
% use the platform to learn (this is what the money is spent on, so to speak).
% Downtime therefore costs money.
Learning Math with a server crashing constantly isn't that much fun.
\item Many supported formats

Might sound strange, but if you wish to create courses that consist of different \quoted{multimedia} 
formats like applets, animation, normal pictures, sounds, you can not use a platform that does not
allow, e.g. flash-movies to be integrated into a course. 
\item Open to changing technology

Can be understood combined with point 2. A new, better, faster format should not crash the system.
Anyway, this \quoted{being open to new technology} can not be guaranteed, but should be an outline of 
developent.
\item Flexibility

% TERRIBLE
Of course, can not be missing.  Nowadays everybody has to be flexible.... 
But that's not the way we mean it:

We don't want the students to adapt to the platform, we want the platform to adapt to their demands.

The teachers shouldn't be forced into a given, unbreakable structure for creating courses. 
If they don't like the exsiting texts on differentiation, they can check in their own.
\item Easy to use

Doesn't need an explanation
\item Seperation of style and content

Makes life easier. This means that authors can write text and don't have to care how it looks in
the end unless they really want to. 
\end{enumerate}
\end{document}

