\documentclass{generic}

\begin{document}

\title{Lists}

\tableofcontents



\section{The result}

\subsection{Normal lists}

\subsubsection{Enumerate list}
\begin{enumerate}
\item[First item]
  This is the first item of an enumerated list. 
  You may label the item or not, just as you prefer.
\item[Second item]
  Another labeled item, perhaps the next one should 
  be without a label, to get an impression
  how it might look
\item
  This is not labeled, so the text appears right 
  next to the number..
\end{enumerate} 

\subsubsection{Description list}
\begin{description}
\item[Item1]
  The same with a description list, 
  the first two are labeled...
\item[Item2]
  Some more test text, nothing really creative.
\item
  And as always, the last unlabeled item.
\end{description}

\subsubsection{Itemize list}
\begin{itemize}
\item[Item1]
  And, last but not least, a list created with \verb/\begin{itemize}/
\item[Item2]
  Some more test text, nothing really creative.
\item
  And as always, the last unlabeled item.
\end{itemize}

\subsection{Lists inside a table}

There is a slight difference if these lists are inside a table cell. This is caused by the
stylesheet, and may be changed according to your preferences. I just thought a short
demostration would be nice..


\begin{table}
\begin{enumerate}
\item[First item]
  This is the first item of an enumerated list. 
\item[Second item]
  Another labeled item, perhaps the ...
\item
  This is not labeled, so the text appears right 
  next to the number..
\end{enumerate} 
\\

\begin{description}
\item[Item1]
  The same with a description list..
\item[Item2]
  Some more test text, nothing really creative.
\item
  And as always, the last unlabeled item.
\end{description}
\\

\begin{itemize}
\item[Item1]
  And, last but not least, a list created with \verb/\begin{itemize}/
\item[Item2]
  Some more test text, nothing really creative.
\item
  And as always, the last unlabeled item.
\end{itemize}

\end{table}

\section{The source code}
\subsection{Normal lists}

\verbatim
\subsubsection{Enumerate list}
\begin{enumerate}
\item[First item]
  This is the first item of an enumerated list. 
  You may label the item or not, just as you prefer.
\item[Second item]
  Another labeled item, perhaps the next one should 
  be without a label, to get an impression
  how it might look
\item
  This is not labeled, so the text appears right 
  next to the number..
\end{enumerate} 

\subsubsection{Description list}
\begin{description}
\item[Item1]
  The same with a description list, 
  the first two are labeled...
\item[Item2]
  Some more test text, nothing really creative.
\item
  And as always, the last unlabeled item.
\end{description}

\subsubsection{Itemize list}
\begin{itemize}
\item[Item1]
  And, last but not least, a list created with \verb/\begin{itemize}/
\item[Item2]
  Some more test text, nothing really creative.
\item
  And as always, the last unlabeled item.
\end{itemize}
\endverbatim

\subsection{Lists inside a table}

\verbatim

\begin{table}
\begin{enumerate}
\item[First item]
  This is the first item of an enumerated list. 
\item[Second item]
  Another labeled item, perhaps the ...
\item
  This is not labeled, so the text appears right 
  next to the number..
\end{enumerate} 
\\

\begin{description}
\item[Item1]
  The same with a description list..
\item[Item2]
  Some more test text, nothing really creative.
\item
  And as always, the last unlabeled item.
\end{description}
\\

\begin{itemize}
\item[Item1]
  And, last but not least, a list created with \verb/\begin{itemize}/
\item[Item2]
  Some more test text, nothing really creative.
\item
  And as always, the last unlabeled item.
\end{itemize}

\end{table}
\endverbatim



\end{document}

