\emph{Test}:
\begin{itemize}
\item Die Ecken erscheinen an der x- bzw. y-Position wo geklickt wurde.
\item Es werden abwechselnd die horizontale und vertikale Position der
  Eckpunkte festgelegt.
\item Mit der rechten Maustaste wird das letzte Kantenst�ck entfernt.
\item Bereits "feste" Kantenst�cke sind grau, das letzte Kantenst�ck ist blau.
\item Die Kante wird gel�scht, wenn man wieder auf die Startkomponente klickt.
\item Das Kantenziehen wird beendet, indem auf eine von der Startkompontente
  verschiedene Komponente geklickt wird, dabei wird gegebenenfalls der letzte
  Punkt angepasst, oder ein weiteres waagerechte Kantenst�ck erzeugt.
\item Es k�nnen keine Kanten zu Subkomponenten erzeugt werden.
\item Wird von der Startkomponente gleich auf die Endkomponente geklickt, wird
  die ganze Kante mit zwei Eckpunkten und dem waagerechten Teilst�ck in der
  Mitte der beiden Komponenten erzeugt.
\item Das Erstellen der Kante geht auch wenn der Graph gezoomt ist.
\end{itemize}