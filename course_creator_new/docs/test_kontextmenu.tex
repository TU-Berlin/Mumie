Beim Klicken mit der rechten Maustaste im Graphframe erscheint das Kontextmen�,
welches abh�ngig von dem Typ der markierten Komponenten ist.

Beim Aufruf des Kontextmen�s wird unterschieden, ob sich dabei eine
Komponente bzw. Kante "unter" dem Mauszeiger befindet, oder ob der Bereich
darunter leer ist. Da die rechte Maustaste ebenso zum Markieren verwendet wird,
bezieht sich im ersten Fall das Kontextmen� nur auf die markiere eine
Komponente bzw. Kante, im letzten Fall auf alle markierten Komponenten
bzw. Kanten.

Soll das Kontextmen� f�r mehrere Komponenten und Kanten aufgerufen werden, so muss
an eine freie Stelle geklickt werden.

Im Folgenden sind die Eintr�ge der Kontextmen�s tabellarisch in Abh�ngigkeit
zum Typ der markierten Komponenten aufgef�hrt.

\emph{Mainkomponenten}
\begin{table}
  \head
  Text
  \body
"L�schen"\\
"Schwarze Kante ziehen"\\
"rote Kante ziehen"\\
"Label setzen"
\end{table}

\emph{Branchkomponenten}
\begin{table}
  \head
  Text
  \body
"L�schen"\\
"Schwarze Kante ziehen"
\end{table}

\emph{Subkomponente}
\begin{table}
  \head
  Text
  \body
"L�schen"\\
"Move In"\\
"Move Out"\\
"Label setzen"
\end{table}

\emph{Kanten}
\begin{table}
  \head
  Text
  \body
"L�schen"\\
 "rote Kante setzen" bzw. "rote Kante entfernen"\\
"Kante in Netzwerk nehmen" oder "aus Netzwerk entfernen"\\
''Richtung �ndern"
\end{table}

Die vorherigen Tabellen beschrieben das Kontextmen�, wenn nur eine Komponente
markiert ist. Sind beliebige Komponenten und/oder Kanten ausgew�hlt, so enth�lt
das Kontextmen� nur den Punkt "L�schen".

Sind Komponente ausgew�hlt, die einen Roten Faden erzeugen k�nnen, so erscheint
dazu auch der Men�punkt "Roten Faden erzeugen".

Wenn genau zwei Komponenten vom Typ Graphkomponente ausgew�hlt sind, dann
enth�lt das Kontextmen� neben dem Punkt "L�schen" noch den Punkt "Verbinden".
