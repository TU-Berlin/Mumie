\emph{Howto}:

Um das * zu testen, ist folgendes zyklisch zu tun:
\begin{enumerate}
\item Den Graph speichern.
\item Pr�fen, dass das * nicht angezeigt wird.
\item Eine Ver�nderung durchf�hren.
\item Pr�fen, dass das * angezeigt wird.
\end{enumerate}


Zu Testen ist:
\begin{itemize}
\item Einf�gen von Komponenten
\item Verschieben von Komponenten 
\begin{itemize}
  \item mit Maus und
  \item mit Tastatur in alle vier Richtungen
\end{itemize}
\item Einf�gen von Subkomponenten
\item Verbinden von Komponenten
\begin{itemize}
  \item automatisch
  \begin{itemize}
    \item mit der entsprechenden Taste
    \item mit dem Button in der Toolbar
    \item mit dem Kontextmenu
  \end{itemize}
  \item von Hand
\end{itemize}
\item �ndern der Richtung einer Kante
\begin{itemize}
  \item mit der entsprechenden Taste
  \item Mit dem Button aus der Toolbar
  \item mit dem Kontextmen�
\end{itemize}
\item Ver�ndern von Kanten
\begin{itemize}
  \item Ver�ndern von Kantenteilst�cken
  \item Ver�ndern von ganzen Kanten
\end{itemize}
\item L�schen von Komponenten
\begin{itemize}
  \item mit der entsprechenden Taste
  \item Mit dem Button aus der Toolbar
  \item mit dem Kontextmen�
\end{itemize}
\item Erstellen des Roten Faden 
\begin{itemize}
  \item Ziehen einer roten Kante mit dem Kontextmen�
  \item �ndern der Farbe einer Kante
  \begin{itemize}
    \item mit den entsprechenden Tasten
    \item Mit den Buttons aus der Toolbar
    \item mit dem Kontextmen�
  \end{itemize}
  \item Erzeugen eines generischen Roten Faden
\end{itemize}
\item Zuweisen eines Dokumentes
\item Zuweisen einer Summary
\item Setzen oder l�schen von Punkten
\item Setzen eines Labels
\item �ndern der Metainformationen
\begin{itemize}
  \item der Name
  \item die Beschreibung
  \item der Status
  \item die Lehrveranstaltung
  \item die Kategorie
  \item der Bearbeitungszeitraum
\end{itemize}
\end{itemize}
