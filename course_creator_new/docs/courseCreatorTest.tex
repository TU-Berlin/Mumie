\documentclass{generic}

\begin{document}
\title{CourseCreator Handbuch}
\tableofcontents

\section*{Einf�hrung}

Dies ist die Testanleitung f�r die Benutzung des CourseCreators. Die einzelnen
Abschnitte sind identisch mit der Bedienanleitung, das hei�t, sie sind
gr��tenteils aus dem Tutorial �bernommen. Weggelassen sind die Beschreibungen
der einzelnen Funktionalit�ten, da davon ausgegangen wird, dass der Tester mit
der Benutzung des CourseCreators vertraut ist.

%\section{Testen}

\section{Allgemeine Programmbenutzung}

\subsection{neuer Graph}

\emph{Tutorial}:

Erstellen eines neuen Graphen.

\emph{Howto}:

\begin{itemize}
\item Klicken des Button \image{howtopics/newNavGraph_norm.png}, dann
  erscheinen die Buttons \image{howtopics/newCourse_norm.png},
  \image{howtopics/newCourseSection_norm.png} und
  \image{howtopics/newProblem_norm.png}
\item Klicken des Button \image{howtopics/tool_only_new.png} aus der Toolbar
  und anschliessende Wahl von "NavGraph".
\item Wahl von Datei--> neuer NavGraph.
\end{itemize}

\emph{Test}:
\begin{itemize}
\item Es wird ein neuer (leerer) Graph erzeugt und alle Eintr�ge sind leer,
  oder enthalten den Defaultwert.
\item Der neue Graph erscheint ganz oben im Buffermen�.
\item Das Klicken der Buttons "Neuer Kurs", "Neuer Kursabschnitt" und "neues
Worksheet" f�hrt zu den entsprechenden Graphen.
\end{itemize}

\subsection{Beenden des Programms}

\emph{Tutorial}:

Beenden des Programms.

\emph{Howto}:

Das Programm wird mit ALT+F4 oder dem Beendenmen�punkt im Dateimen�
beendet. Bei noch nicht gespeicherten Graphen wird nachgefragt, ob diese
gespeichert werden sollen.

\emph{Test}:
\begin{itemize}
\item Alle ungespeichterten Graphen werden angezeigt, und abgefragt, ob diese
  gespeichert werden sollen.
\item Es werden alle Graphen, bei denen dies verlangt wird gespeichert.
\item Der "Beenden abbrechen" beendet das Abbrechen und das Programm wird fortgesetzt.
\end{itemize}

\subsection{Konfiguration der Tasten}

\emph{Tutorial}:

Im Folgenden werden die Tastatureinstellungen gesetzt.

\emph{Howto}:

Um Die ShortCuts zu �ndern wird das Fenster "Tastenk�rzel" im Men�punkt
"Einstellungen" ge�ffnet.
Es �ffnet sich folgendes Fenster:

\image{howtopics/gui_keys.png}

in die einzelnen Eingabefelder sind die ShortCuts (dopplungsfrei) einzugeben, diese werden dann
sofort in die lokale Konfigurationsdatei geschrieben.

Folgende Tasten sind m�glich:
a,..,z
0,..,9
F1,..,F12
Delete, Insert, Pause, ENTER, BACKSPACE


Zus�tzlich die Modifier Shift, Ctrl und Alt

\emph{Vorsicht bei Funktionstasten, diese sind -je nach Arbeitsumgebung-
eventuell schon vergeben und k�nnen daher nicht f�r den CourseCreator benutzt
werden.}

\emph{Au�erdem ist zu beachten, das sich die Tasteneingabe auf den gesamten
CourseCreator bezieht. Die Eingabe eines einzelnen Buchstaben, oder eines
Gro�buchstaben ist daher zu vermeiden, da es zu Problemen bei der Verwendung
dieser Tasten im Metainformationsfenster kommen kann.}

\emph{Test}:
\begin{itemize}
\item Auf die eingetragenen Zeichen reagiert das Programm in der gew�nschten
  Weise. (eine Taste muss nur f�r eine Funktion getestet werden, insgesamt
  sollen aber alle Funktionen wenigstens einmal getestet werden.)
\item Zeichen d�rfen sich nicht doppeln. Gegebenenfalls erscheint eine
  Mitteilung und man kann ausw�hlen, ob man dieses Tastaturk�rzel doch benutzen
  m�chte und das andere l�scht, oder ein neues Zeichen eingeben m�chte.
\item Die Modifier Shift, Ctrl und Alt sind m�glich und werden in das
  Eingabefenster geschrieben.
\item Der Defaultbutton setzt nach einer "wollen Sie wirklich?"-Abfrage die Eingaben zur�ck auf die Defaultwerte.
\item Die Eingaben sind sofort verwendbar.
\item Ist eine Taste nicht m�glich, erscheint eine Fehlermeldung.
\end{itemize}

\section{Erstellen des Graphen}

\subsection{Einf�gen von Komponenten}

\emph{Tutorial}: 

Einf�gen von Mainkomponenten, um die drei verschiedenen
Graphtypen zu erzeugen: erzeuge drei Graphen (siehe
\href{\#neuerGraph}{Erstellen eines neuen Graphen}, je einen vom Typ Kurs,
Kursabschnitt und Worksheet und f�ge beliebig viele Mainkomponenten und
Branchkomponenten hinzu. Mit Hilfe des Buffermen� kann zwischen den einzelnen
Graphen gewechselt werden.

\emph{Howto} :

Durch Klicken auf ein Symbol in der Komponentenleiste wird ein neuer Knoten eingef�gt.


\emph{Test}: 
\begin{itemize}
\item Die Kategorie, die auf den Buttons links angegeben ist, stimmt mit der
  Kategorie der eingef�gten Komponente �berein.
\item Es lassen sich in jeden Graphen Branchkomponenten einf�gen.
\end{itemize}

\label{verschiebenVonKomponenten}\subsection{Verschieben von Komponenten}

\emph{Tutorial}: 

Verschieben der Komponenten mittels Maus und Tastatur in beliebiger also auch
negativer Richtung.


\emph{Howto}:

Eine oder mehrere markierte Komponenten werden mit der Maus per Drag'n'Drop
oder mit der Tastatur durch die entsprechenden Tasten verschoben.


\emph{Test}:
\begin{itemize}
\item Die Komponenten k�nnen beliebig mit der Maus oder der Tastatur
  verschoben werden
\item Wird eine Komponente nach oben oder links ''aus dem Fenster raus''
  geschoben, so rutschen die restlichen Komponenten nach unten, bzw. rechts und
  die verschobene Komponente ist wieder im Graphen zu sehen. Dies geht auch mit
  mehreren markierten Komponenten.
\item �ndert man die angabe von "MoveStep" im Men� Ansicht, so werden markierte
  Komponenten entsprechend der neuen Schrittgr��e verschoben.
\end{itemize}

\subsection{Einf�gen von Subkomponenten}

\emph{Tutorial}:

Hinzuf�gen von Subkomponenten zu beliebigen Mainkomponenten in
beliebigen Graphtypen und Vertauschen der Subelemente.


\emph{Howto}: 

Eine Subkomponente wird durch Anklicken des Symbols in der
Komponentenleiste eingef�gt. Mit den Men�punkten ''Move In'' bzw. ''Move Out''
sowie den Tasten, die daf�r in der Tastenkonfiguration eingetragen sind, kann
eine Subkomponente nach au�en bzw. innen getauscht werden.

\emph{Test}:
\begin{itemize}
\item Die Subkomponenten werden an die richtige (die markierte) Mainkomponente
  angef�gt.
\item Die Kategorie, die auf den Buttons links angegeben ist, stimmt mit der
  Kategorie der eingef�gten Komponente �berein.
\item Die Subkomponenten k�nnen nicht einzeln verschoben werden.
\item Ist keine oder mehr als eine Mainkomponente, eine Branchkomponente oder
  eine Subkomponente ausgew�hlt, kann keine Subkomponenten zugef�gt werden.
\item Beim "Move in" bzw. "Move out" wird die markierte Subkomponente in die
  gew�nschte Richtung getauscht.
\item Ist ein "Move in" oder ein "Move out" nicht m�glich, da die maximale
  bzw. minimale Zoomstufe schon erreicht ist, so erscheint der entsprechende
  Punkt nicht im Kontextmen�.
\end{itemize}

\subsection{Verbinden von Komponenten}

\subsubsection{Automatisches Verbinden von Komponenten}

\emph{Tutorial}:

Automatisches Verbinden von verschiedenen Graphkomponenten.


\emph{Howto}:

Um die zwei markierten Knoten zu verbinden gibt es folgende M�glichkeiten:
\begin{itemize}
\item Mit der Tastatur verbunden werden k�nnen zwei ausgew�hlte Knoten, wenn die
  Taste, die in der Tastenkonfiguration daf�r eingetragen ist, gedr�ckt wird.
\item Zwei ausgew�hlte Komponenten werden mit dem Verbinden-Button
  \image{howtopics/tool_edge_connect.png} verbunden.
\item Sind zwei Komponenten ausgew�hlt, so erscheint im Kontextmen� (rechte
  Maustaste) der Punkt ''Verbinden''.
\end{itemize}

\emph{Test}:
\begin{itemize}
\item Es werden die beiden ausgew�hlten Komponenten verbunden.
\item Es kann nur verbunden werden, wenn genau zwei Komponenten markiert sind.
\item Es k�nnen keine Kanten zu Subkomponenten erzeugt werden.
\item Werden die beiden Komponenten gleichzeitig markiert, ist die
  Kantenrichtung von oben nach unten, bzw. falls beide Knoten gleich hoch sind,
  von links nach rechts.
\item Werden die Knoten nacheinander markiert, so ist der Knoten, der zuerst
  markiert wurde, der Startknoten.
\item Es werden zwei Eckpunkte in der horizontalen Mitte der beiden Komponenten
  erzeugt.
\end{itemize}

\subsubsection{Verbinden von Komponenten von Hand} 

\emph{Tutorial}:

Verbinden von zwei Graphkomponenten mittels Maus.


\emph{Howto}:

Es wird eine Graphkomponente mit der rechten Mausetaste
angeklickt und im erscheinenden Kontextmen� der Unterpunkt ''Schwarze Kante
ziehen'' gew�hlt. Dann werden mit der linken Maustaste beliebige Eckpunkte der
Kante festgelegt. Das jeweils zuletzt erzeugte Kantenst�ck kann mit der rechten
Maustaste wieder gel�scht werden. Ganz gel�scht wird die Kante, wenn wieder auf
die Startkomponente geklickt wird. Beendet wird die Kante durch Klicken auf
eine zweite Graphkomponente.



\emph{Test}:
\begin{itemize}
\item Die Ecken erscheinen an der x- bzw. y-Position wo geklickt wurde.
\item Es werden abwechselnd die horizontale und vertikale Position der
  Eckpunkte festgelegt.
\item Mit der rechten Maustaste wird das letzte Kantenst�ck entfernt.
\item Bereits "feste" Kantenst�cke sind grau, das letzte Kantenst�ck ist blau.
\item Die Kante wird gel�scht, wenn man wieder auf die Startkomponente klickt.
\item Das Kantenziehen wird beendet, indem auf eine von der Startkompontente
  verschiedene Komponente geklickt wird, dabei wird gegebenenfalls der letzte
  Punkt angepasst, oder ein weiteres waagerechte Kantenst�ck erzeugt.
\item Es k�nnen keine Kanten zu Subkomponenten erzeugt werden.
\item Wird von der Startkomponente gleich auf die Endkomponente geklickt, wird
  die ganze Kante mit zwei Eckpunkten und dem waagerechten Teilst�ck in der
  Mitte der beiden Komponenten erzeugt.
\item Das Erstellen der Kante geht auch wenn der Graph gezoomt ist.
\end{itemize}

\subsection{ Richtung einer Kante}
\subsubsection{ �ndern der Richtung einer Kante}

\emph{Tutorial}:
�ndern der Richtung von Kanten.


\emph{Howto}:

Um die Richtung einer Kante zu �ndern, muss die gew�nschte Kante
markiert sein und die Richtung kann dann ge�ndert werden durch:

\begin{itemize}
\item Dr�cken der Taste, die in der Tastenkonfiguration f�r die Orientierung
  eingetragen ist.
\item Aufruf des Punktes "Richtung �ndern" des Kontextmen�s der Kante.
\item Anw�hlen des Buttons \image{howtopics/tool_edge_orientation.png} aus der
  Toolbar.
\end{itemize}

\emph{Test}: 
\begin{itemize}
\item Beim Ausf�hren der oben genannten Aktionen �ndert sich die Richtung der
  markierten Kante.
\end{itemize}

\subsubsection{ �ndernAnzeigen der Richtungspfeile}

\emph{Tutorial}:

Anzeigen der Richtungpfeile der Kanten.

\emph{Howto}:

Die Anzeige der Kantenrichtung kann im Men�punkt "Richtungspfeile" im Ansichtsmen�
festgelegt werden.

\emph{Test}:
\begin{itemize}
\item Die Kantenrichtungspfeile werden entsprechend der Einstellung im
  Ansichtsmen� angezeigt.
\end{itemize}

\subsection{Komponenten vertauschen}

\emph{Tutorial}:

Vertauschen zweier Graphkomponenten.

\emph{Howto}:

Zwei Komponenten werden wie folgt vertauscht:
\begin{itemize}
\item Mit der Tastatur vertauscht werden k�nnen zwei ausgew�hlte Knoten wenn
  die Taste, die daf�r in der Tastenkonfiguration eingetragen ist, gedr�ckt
  wird.
\item Zwei ausgew�hlte Komponenten werden mit dem Vertauschen-Button
  \image{howtopics/tool_change.png} verbunden.
\item Sind zwei Komponenten ausgew�hlt, so erscheint im Kontextmen� (rechte
Maustaste) der Punkt ''Vertauschen''.
\end{itemize}

\emph{Test}:
\begin{itemize}
\item Es werden genau die beiden Komponenten (ggf. mit Subkomponenten)
  getauscht.
\item Es k�nnen nur Graphkomponenten getauscht werden.
\end{itemize}

\subsection{Ver�ndern von Kanten}

\subsubsection{Ver�ndern von Kantenteilst�cken}

\emph{Tutorial}:

Ver�ndern von Kanten in der Position ihrer Eckpunkte.

\emph{Howto}:

Eine Kante wird markiert, so dass die St�tzstellen der Kante
angezeigt werden. Dann kann eine St�tzstelle per drag'n'drop an eine beliebige
Stelle gezogen werden.

\emph{Test}:

\begin{itemize}
\item Nach beliebigen Modifizieren der Kante ist diese immer noch orthogonal. 
\item Die Eck- bzw. Kantenst�tzstellen sind dort, wo sie hingezogen werden.
\end{itemize}


\subsubsection{Verschieben von ganzen Kanten}

\emph{Tutorial}:

Verschieben von ganzen Kanten.

\emph{Howto}:

Eine Kante wird markiert, so dass die St�tzstellen der Kante
angezeigt werden. Dann wird die Kante zwischen zwei St�tzstellen "angefasst"
und per Drag'n'Drop an eine beliebige Stelle gezogen.

\emph{Test}:
\begin{itemize}
\item Die Kanten sind komplett (bis auf Anfangs- und Endkantenst�ck) in
  beliebigen Richtungen verschiebbar.
\item Das erste und letzte St�ck der Kante werden angepasst.
\end{itemize}

\subsubsection{Verschieben von Komponenten mit  Kanten}

\emph{Tutorial}:

Verschieben von Komponenten, die Kanten haben.

\emph{Howto}:

Eine oder mehrere markierte Komponenten werden mit der Maus per Drag'n'Drop
oder mit der Tastatur durch die daf�r in der Tastenkonfiguration eingetragenen Tasten verschoben.


\emph{Test}:
\begin{itemize}
\item Wird eine Komponente mit Kanten verschoben, so werden alle zugeh�rigen Kanten 
  angepasst.
\end{itemize}

\subsubsection{Verschieben von Teilgraphen}

\emph{Tutorial}:

Verschieben von Teilen des Graphes

\emph{Howto}:

Teilgraphen werden verschoben, indem ein Teil des Graphen markiert wird und
dieser Teil dann mit den Tasten, die daf�r in der Tastenkonfiguration
eingetragen sind, verschoben wird, oder indem eine
der markierten Komponenten "angefasst wird" und per Drag'n'Drop verschoben wird.

\emph{Test}
\begin{itemize}
\item Es werden nur die Teile des Graphen verschoben, die auch markiert sind,
  alle anderen bleiben an ihrer Position.
\item Kanten werden gegebenenfalls angepasst.
\end{itemize}

\subsubsection{Ausrichten am Gitter}

\emph{Tutorial}:

Ausrichten von Komponenten am Gitter.

\emph{Howto}:

Am Gitter ausgerichtet werden Komponenten, wenn die Taste, die in der
Tastenkonfiguration daf�r eingetragen ist, gedr�ckt wird.

                


\emph{Test}:
\begin{itemize}
\item Die Komponenten werden am Gitter ausgerichtet.
\item Kanten werden entsprechend angepasst.
\item Kanten werden nur dann ebenfalls am Gitter ausgerichtet, wenn sie auch
  markiert waren.
\end{itemize}

\subsection{L�schen von Komponenten}

\emph{Tutorial}:

L�schen von verschiedenen Komponenten.

\emph{Howto}:

Das L�schen geschieht auf unterschiedliche Art:

\begin{itemize}
\item Durch Ausw�hlen des Punktes "L�schen" im Kontextmen� der markierten Komponente[n].
\item Durch Anklicken des L�schenbuttons
  \image{howtopics/tool_delete_komp.png} in der Toolbar.
\item Durch Dr�cken der Taste, die daf�r in der Tastenkonfiguration eingetragen
  ist.
\end{itemize}

\emph{Test}:
\begin{itemize}
\item Beim L�schen einer Mainkomponente mit Subkomponenten m�ssen die
  Subkomponenten ebenfalls gel�scht werden (ggf. Pr�fen durch Speichern).
\item Beim L�schen einer Mainkomponente oder Branchkomponente mit Kanten werden
  auch die Kanten gel�scht (ggf. Pr�fen durch Speichern).
\end{itemize}

\subsection{�berpr�fen des Graphen}

\emph{Tutorial}:

Testen des Graphen auf gerichtete Kreise, Quellen und Senken.

\emph{Howto}:

Der Graph l�sst sich �berpr�fen, indem man im Men� den Punkt ''Bearbeiten''
''Graph pr�fen'' und dann den entsprechenden gew�nschten Unterpunkt ausw�hlt.


\emph{Test}:

Anzeigen gerichteter Kreise:
\begin{itemize}
\item Existiert mindestens ein gerichteter Kreis, so wird ein beliebiger
  gerichteter Kreis angezeigt, indem alle zugeh�rigen Komponente und Kanten
  markiert werden. 
\item Existiert kein Kreis, so erscheint ein Fenster, welches dar�ber
Auskunft gibt.
\item Enth�lt der Graph keine Kanten oder ist er leer so erscheint ein Fenster,
  welches dar�ber Auskunft gibt.
\end{itemize}

Anzeigen von Quellen und Senken:
\begin{itemize}
\item Enth�lt der Graph keine Quellen bzw. Senken, weil er z.B. leer ist, oder
  jeder Knoten auf einem gerichteten Kreis liegt, erscheint ein Fenster mit
  einer Meldung, dass es keine Quellen bzw. Senken gibt.
\item Enth�lt der Graph Quellen bzw. Senken, so werden sie angezeigt, indem sie
  markiert werden.
\end{itemize}

\subsection{Der Rote Faden}

\subsubsection{Erstellen des Roten Faden}

\emph{Tutorial}:

F�rben von Kanten.

\emph{Howto}:

Zum Wechsel der Farbe einer Kante zwischen zwei Mainkomponenten wird eine
schwarze Kante markiert und 
\begin{itemize}
\item im Kontextmen� dieser Kante der Punkt "rote Kante setzen" gew�hlt.
\item der Button \image{howtopics/tool_edge_red.png} aus der Toolbar
  angeklickt.
\item die Taste, die daf�r in der Tastenkonfiguration eingetragen ist,
  gedr�ckt.
\end{itemize}

Auf die gleiche Weise kann eine Kante wieder vom Roten Faden herunter genommen
werden (mit dem Unterschied, dass der Kontextmen�eintrag dann "rote Kante
entfernen" hei�t). Dabei ist zu beachten, dass eine rote Kante die nicht in
Netzwerk ist, bei der Aktion "rote Kante entfernen" gel�scht werden wird. Der
CourseCreator zeigt dann einen Auswahldialog an, um darauf hinzuweisen.

\image{howtopics/dialog_kanteLoeschen.png}

Es k�nnen auch Kanten des Roten Faden ins Netzwerk genommen werden, bzw. auch
wieder entfernt werden. Dazu wird eine Kante markiert,:
\begin{itemize}
\item im Kontextmen� dieser Kante der Punkt "aus Netzwerk entfernen" bzw. "ins
  Netzwerk nehmen" gew�hlt.
\item der Button \image{howtopics/tool_edge_exists.png} aus der Toolbar
  angeklickt.
\item die Taste, die daf�r in der Tastenkonfiguration eingetragen ist,
  gedr�ckt.
\end{itemize}

Eine rote Kante zwischen zwei Mainkomponenten kann auch erstellt werden, indem
im Kontextmen� einer Komponente der Punkt "rote Kante ziehen" angew�hlt wird
und dann verfahren wird wie beim
\href{\#verbindenVonKomponentenVonHand}{Verbinden von Komponenten von
  Hand}. Dieses geht aber nur, wenn Start- und Endknoten keine
Branchkomponenten sind.


\emph{Test}:
\begin{itemize}
\item Die Kanten werden wie gew�nscht gef�rbt.
\item Es ist nicht m�glich, rote Kanten zu oder von Branchkomponenten zu
  erzeugen.
\item Soll eine Kante, die nur rot ist, vom Roten Faden genommen werden oder
  eine Kante, die nur schwarz ist, aus dem Netzwerk genommen werden, so
  erscheint ein Auswahldialog, um darauf hinzuweisen, dass diese Kante im
  Folgenden gel�scht wird. W�hlt man dann ''Ja'', so wird die Kante gel�scht,
  w�hlt man ''Nein'', so wird keine Aktion durchgef�hrt.
\item Im Kontextmen� erscheinen die korrekten Men�eintr�ge gezeigt, also "rote
  Kante setzen", falls die Kante nicht auf dem Roten Faden ist und beide
  Endpunkte Dokumentkomponenten sind, bzw. "rote Kante entfernen", falls die
  Kante auf dem Roten Faden ist, sowie "aus Netzwerk entfernen" falls sich die
  Kante im Netzwerk befindet, bzw. "Kante in Netzwerk nehmen", falls sich die
  Kante nicht im Netzwerk befindet.
\end{itemize}

\subsubsection{Generischer Roter Faden}

\emph{Tutorial}:

Generieren einer roten Kante.

\emph{Howto}:

Um eine rote Kante zu generieren, werden die schwarzen Kanten
markiert und dann
\begin{itemize}
\item im Kontextmen� dieser Kanten der Punkt "erzeuge Roten Faden" gew�hlt
\item der Button \image{howtopics/tool_edge_red.png} aus der Toolbar
  angeklickt
\item die Taste, die in der Tastenkonfiguration daf�r eingetragen ist, dr�ckt.
\end{itemize}
wird.

\emph{Test}

\begin{itemize}
\item Wird aus einem Kantenzug ''Mainkomponente - beliebig viele
  Branchkomponenten - Mainkomponente'' eine rote Kante erzeugt, dann verl�uft
  diese zwischen den Dokumentkomponenten. Befinden sich mehrere solche Wege in
  markierten Teilgraph, so werden auch mehrere rote Kanten erzeugt. 
\item Werden unterschiedliche Teile des Graphen markiert, so ist nur m�glich,
  Kanten des Roten Faden zu erzeugen, wenn sich im markierten Teilgraph
  mindestens ein gerichteter Weg ''Mainkomponente - beliebig viele
  Branchkomponenten - Mainkomponente'' befindet.
\end{itemize}


\subsubsection{�berpr�fen des Roten Faden}

\emph{Tutorial}:

�berpr�fen des Roten Fadens.

\emph{Howto}:

Um den Roten Faden zu pr�fen w�hle im Men�punkt
"Bearbeiten" den Unterpunkt "rote Linie �berpr�fen".

Folgende (Fehler-)Meldungen sind m�glich:
\begin{itemize}
\item Eine Komponente hat zu viele eingehende roten Kanten (gibt es eine
  Komponente mit zu vielen ausgehenden Kanten, so gibt es ebenfalls eine mit zu
  vielen eingehenden Kanten, dieser Fall wird zuerst �berpr�ft).
\item Es gibt mehr als einen Anfang des Roten Faden.
\item Der Rote Faden ist ein Kreis.
\item Es sind keine roten Kanten vorhanden.
\item Der Rote Faden ist korrekt.
\end{itemize}

\section{Elementzuweisungsfenster}

\subsection{Funktionen des Elementzuweisungsfensters}

\emph{Tutorial}:

Testen der Funktionsweise des Elementzuweusungsfensters.

\emph{Howto}:

Das Elementzuweisungsfenster wird mit dem Untermen�punkt
"Elementzuweisungsfenster" des Men�punktes "Ansicht", oder den "Change"-Button
neben der Summary ge�ffnet.

\emph{Test}:
\begin{itemize}
\item Der Button "Set" ist nur w�hlbar, wenn eine Komponente im Graph markiert
  ist und das Element des Baumes zu dieser Komponente passt (au�er Summary).
\item Der Button "Preview" �ffnet ein Firefox-Fenster, welches das im Baum
  markierte Dokument anzeigt (geht nur f�r generische Elemente).
\item Der Button "MetaInfo" �ffnet ein Firefox-Fenster, welches die
  Metainformationen des im Baum markierten Dokumentes anzeigt (geht nur f�r
  generische Elemente).
\item Cancel-Button schliesst das Fenster.
\item Die Metainformationen im oberen Fensterteil lassen sich mit dem "show
  MetaInfos"-Button an- und ausschalten.
\item Generische Elemente werden nur angezeigt, wenn die Checkbox "show
  generic" angew�hlt ist, nicht generische Dokumente werden nur angezeigt, wenn
  die Checkbox "show non-generic" angew�hlt ist. Verzeichnisse werden immer
  angezeigt.
\end{itemize}

\subsection{Metainformationen des Elementzuweisungsfenster}

\emph{Tutorial}:

Testen der Metainformationen des Elementzuweisungsfensters.

\emph{Howto}:

Die Metainformationen werden jeweils zu dem Element des Baumes angezeigt,
welches gerade markiert ist.

\emph{Test}:
\begin{itemize}
\item Der Metainformationsteil wird jedes Mal angepasst, wenn eine andere
  Komponente des Baumes gew�hlt wird.
\item Der Tooltip �ber dem Baum zeigt die Beschreibung des Dokuments an.
\item Der Tooltip �ber den Ausgaben des Metainformationsteils enth�lt jeweils
  den ungek�rzten Inhalt der Ausgaben.
\item Die Metainformationen lassen sich mit dem "show MetaInfos"-Button an- und
  ausschalten.
\end{itemize}

\subsection{Dokumente zuweisen}

\emph{Tutorial}:

Weise den verschiedenen Komponenten der verschiedenen Graphen
Dokumente zu.

\emph{Howto}:

Um ein Dokument zuzuweisen wird eine Dokumentkomponente im
Graphframe markiert und das gew�nschte Dokument im Baum des
Elementzuweisungsfensters angeklickt und der Set-Button gedr�ckt.

\emph{Test}:
\begin{itemize}
\item Es k�nnen nur generische Dokumente zugewiesen werden.
\item Summarys k�nnen immer zugewiesen werden.
\item Es k�nnen nur Dokumente zugewiesen werden, die dem Typ des Graphen und
  dem Typ der Komponente entsprechen.
\item Es k�nnen nur Kursabschnitte zugewiesen werden, die auch eingecheckt sind.
\end{itemize}

\section{Punkte, Labels, Metainformationen}

\subsection{Punkte f�r Aufgaben}

\emph{Tutorial}:

Problemkomponenten k�nnen ganzzahlige nicht-negative Punkte zugewiesen werden.

\emph{Howto}:

Punkte werden zugewiesen, indem im Kontextmen� einer Problemkomponente der
Unterpunkt "Punkte vergeben" gew�hlt wird. Dann �ffnet sich ein Dialogfenster,
in das die Punkte eingegeben werden k�nnen.

\emph{Test}:
\begin{itemize}
\item Das Dialogfenster �ffnet sich, wenn der Untermen�punkt "Punkte vergeben"
  aus dem Kontextmen� gew�hlt wird.
\item Beim �ffnen des Dialogfensters steht dort die zuletzt angegebene
  Punktzahl.
\item Beim Beenden des Dialogs mit "Cancel" wird der angegebene Wert nicht
  �bernommen.
\item Beim Beenden des Dialogs mit "Ok" wird der angegebene Wert �bernommen.
\item Es ist m�glich keine Punkte zu vergeben. (ggf pr�fen mit speichern)
\item Es ist nicht m�glich negative Zahlen, Kommazahlen oder Buchstaben und
  Sonderzeichen.. einzugeben. Wird dies doch gemacht, erscheint eine
  Fehlermeldung und der eingegebene Wert wird nicht �bernommen.
\item Nur Problemkomponenten k�nnen Punkte zugewiesen werden. Allen anderen
  Komponenten nicht.
\end{itemize}

\subsection{Labels}

\emph{Tutorial}:

Setzen von Labels.

\emph{Howto}:

Labels werden Dokumentkomponenten zugewiesen, indem im Kontextmen� einer
Dokumentkomponente der Unterpunkt "Label setzen" gew�hlt wird. Dann �ffnet sich
ein Dialogfenster, in das das Label eingegeben werden kann.

\emph{Test}:
\begin{itemize}
\item Das Dialogfenster �ffnet sich, wenn der Untermen�punkt "Label setzen"
  aus dem Kontextmen� gew�hlt wird.
\item Beim �ffnen des Dialogfensters steht dort das zuletzt angegebene Label.
\item beim Beenden des Dialogs mit "Cancel" wird der angegebene String nicht
  �bernommen.
\item beim Beenden des Dialogs mit "Ok" wird der angegebene String �bernommen.
\item Es ist nicht m�glich, das Label leer zu lassen.
\item Das Defaultlabel ist "- no label -" (ggf pr�fen durch Speichern).
\item Nur Dokumentkomponenten k�nnen Label zugewiesen werden. Branchkomponenten
  und Kanten haben keine.
\end{itemize}


\subsection{Metainformationen des Graphen}
In diesem Kapitel werden die Funktionen des Metainformationsfensters (unten
rechts) getestet.

\subsubsection{Name und Beschreibung}

\emph{Tutorial}:

Ver�ndern des Names und der Beschreibung eines Graphen.

\emph{Howto}:

Der Name und die Beschreibung eines Graphen werden im Metainformationsfenster
durch die Eingabezeilen ver�ndert.

\emph{Test}:
\begin{itemize}
\item Der Name des Graphen erscheint in der Kopfzeile des Graphfensters und als
  Eintrag im "Buffers"-Men�.
\end{itemize}

\subsubsection{Summary}

\emph{Tutorial}:

�ndern der Summary f�r einen Graphen.

\emph{Howto}:

Um dIE Summary zu �ndern muss das Elementzuweisungsfenster ge�ffnet
werden.Siehe \href{\#elementzuweisungsfenster}{Elementzuweisungsfenster}. Im
Elementzuweisungsfenster wird dann eine Summary aus dem Baum ausgew�hlt und
der "Set"-Button gedr�ckt. Der Dateiname der Summary wird angezeigt. Ist dieser
mit Pfad zu lang f�r das Feld, so wird der Pfad abgek�rzt. Der vollst�ndige
Pfad erscheint als Tooltip.

\emph{Test}:
\begin{itemize}
\item Das Elementzuweisungsfenster �ffnet sich nur, wenn man eingeloggt ist. 
\item Ein zu langer Pfad wird mit "../verzeichnis/verzeichnis/foo.xml"
  abgek�rzt.
\item Der Dateiname mit dem vollst�ndigen Pfad erscheint als Tooltip.
\item Eine Summary kann immer zugewiesen werden.
\item Ist keine Summary gesetzt, so steht dort "keine Summary gesetzt"
\end{itemize}

\subsubsection{Dateiname}
Es ist sinnvoll, diesen Abschnitt nur kurz zu lesen und im Zusammenhang mit dem
Abschnitt \href{\#ladenUndSpeichern}{Laden und Speichern} zu testen.

\emph{Tutorial}:

Testen der Datenamenausgabe.

\emph{Howto}:

Der Eintrag f�r den Dateinamen wird gesetzt, sobald ein Graph gespeichert oder
geladen wird, also ist das Laden und Speichern auszutesten und zu pr�fen, ob
der erwartete Dateiname angezeigt wird.

\emph{Test}:
\begin{itemize}
\item Es wird nur der Dateiname angezeigt, der Pfad erscheint als Tooltip.
\item Ist der Graph noch nicht gespeichert, erscheint "Datei nicht gespeichert" und der Tooltip ist leer.
\end{itemize}

\subsubsection{Status}

\emph{Tutorial}:

Setzen des Statuses eines Graphen.

\emph{Howto}:

Der Status eines Graph wird mit den sechs Radiobuttons ver�ndert.

\emph{Test}:
\begin{itemize}
\item Es kann jeweils nur genau ein Status aktiv sein.
\item Der Defaultstatus ist "vorl�ufig".
\end{itemize}

\subsubsection{Lehrveranstaltung}

\emph{Tutorial}:

�ndern der Lehrveranstaltung eines Kurses.

\emph{Howto}:

Eine Lehrveranstaltung wird ge�ndert indem der Button "Change" gedr�ckt
wird. Dann �ffnet sich ein Fenster, in dem alle Lehrveranstaltungen des Servers
auf dem man eingeloggt ist, aufgef�hrt sind. 

\emph{Test}:
\begin{itemize}
\item Nur ein Kurs hat eine Lehrveranstaltung.
\item Ist man nicht auf einem Server eingeloggt, ist der "Change"-Button ausgeschaltet.
\item Ist die vorherige Lehrveranstaltung auf dem Server nicht vorhanden,
  erscheint ein Warnhinweis.
\item Die neu gew�hlte Lehrveranstaltung wird angezeigt, als Tooltip erscheint
  der Pfad.
\end{itemize}

\subsubsection{Kategorie}

\emph{Tutorial}:

Setzen der Kategorie eine Problemgraphen.

\emph{Howto}:

Die Kategorie eines Problemgraph wird mit den drei Radiobuttons ver�ndert.

\emph{Test}:
\begin{itemize}
\item Nur Problemgraphen haben eine Kategorie.
\item Es kann jeweils nur genau eine Kategorie aktiv sein.
\item der Defaultwert ist "Homework".
\end{itemize}

\subsubsection{Bearbeitungszeitraum}

\emph{Tutorial}:

Wird gerade �berarbeitet

\emph{Howto}:

Wird gerade �berarbeitet

\emph{Test}:
Wird gerade �berarbeitet


\section{Laden und Speichern}
\label{ladenUndSpeichern}
\subsection{Laden}
\subsubsection{"einfaches" Laden}

\emph{Tutorial}:

Laden eines beliebigen Graphen.



\emph{Howto}:

Um einen Graphen zu laden verwendet man den Untermen�punkt "Laden" aus dem Men�
"Datei" oder man dr�ckt die Laden-Taste aus der Tastenkonfiguration. Es �ffnet sich ein Dateichooser, in dem
man eine Datei ausw�hlen kann. Es ist ebenso m�glich, eine Datei schon bei
Programmstart zu laden, indem der CourseCreator mit \\
\code{mmcc file.meta.xml} bzw\\
\code{mmcc file} gestartet wird.



\emph{Test}:

\begin{itemize}
\item Die Datei wird so geladen, wie sie gespeichert wurde.
\item Ein Graph, der beim Programmstart angegeben wird, wird wie erwartet
  geladen.
\item Beim Programmstart wird der Graph geladen, unabh�ngig davon, ob die
  Endung ".meta.xml" angegeben wird, oder nicht.
\item Wird versucht eine Datei zu laden, die nicht existiert, erscheint eine
  Fehlermeldung dar�ber.
\end{itemize}

\subsubsection{Laden eines bereits ge�ffneten Graphen}
\emph{Tutorial}:

Laden eines Graphen, der bereits ge�ffnet ist.

\emph{Howto}:

Es wird mit dem  Untermen�punkt "Laden" aus dem Men� "Datei" oder durch Dr�cken
der entsprechenden Taste eine Datei geladen, die bereits ge�ffnet ist.

Es erscheint folgender Dialog:


\image{howtopics/dialog_graphExists.png}

Der Benutzer kann entscheiden, ob er
\begin{itemize}
\item (linker Button) die Datei �ffnen m�chte und deren Dateiname l�scht. Der Versuch, die
  Datei zu speichern, f�hrt dann automatisch zu einem "Speichern unter".
\item (mittlerer Button) die Datei �ffnen m�chte und den bereits ge�ffneten Graphen l�scht (alle
  �nderungen gehen verloren) oder
\item (rechter Button) das Laden abbricht.
\end{itemize}

\emph{Test}:

\begin{itemize}
\item Es erscheint der Dialog wie erwartet.
\item Bei Auswahl von "Datei �ffnen, Dateiname l�schen", bleibt die alte Datei
  unver�ndert, die neue verliert ihren Dateinamen.
\item Bei Auswahl von "Datei �ffnen, alte verwerfen", wird die alte Datei
  geschlossen und die neue Datei ge�ffnet.
\end{itemize}

\subsection{Speichern}
\label{speichern}
\subsubsection{"einfaches" Speichern}
\emph{Tutorial}:

Speichern eines erstellten beliebigen Graphen unter einen Namen, der bisher
noch nicht existiert.



\emph{Howto}:

Ein Graph wird gespeichert indem die Tast "s" gedr�ckt wird, oder der
Untermen�punkt "Speichern" bzw "Speichern unter" gew�hlt wird. Bei der Aktion
"Speichern unter" wird ein Dateiauswahlfenster ge�ffnet.

\emph{Test}:

\begin{itemize}
\item Die Datei wird gespeichert, und kann danach ebenso geladen werden.
\item Wurde der Graph nicht ver�ndert, erscheint ein Hinweis, dass der Graph
  unver�ndert ist und der Autor hat die M�glichkeit trotzdem zu speichern.
\item Hat ein Graph bisher keinen Dateinamen (zb, weil er nicht geladen oder
  bisher nicht gespeichert wurde, oder seinen Namen durch erneutes �ffnen
  verloren hat) wird automatisch die Aktion "Speichern unter" statt "Speichern"
  ausgef�hrt.
\item Hat der Graph einen Dateinamen, so wird der Graph bei der Aktion
  "Speichern" ohne Dateiauswahlfenster gespeichert.
\item Es werden eine Datei "file.meta.xml" und eine Datei "file.content.xml"
  erzeugt.
\item Der Graph wird gespeichert, unabh�ngig davon, ob der Name mit
  "file.meta.xml" oder nur mit "file" angegeben wird.
\item Im Dateiauswahlfenster werden per Default nur Verzeichnisse und die
  ".meta.xml"-Dateien angezeigt.
\item Beim erfolgreichen oder missgl�ckten Speichern erscheint jeweils eine
  Nachricht dar�ber.
\item Es werden Warnungen ausgegeben wenn mindestens einer Komponente kein
  Dokument zugewiesen wurde oder der Rote Faden nicht korrekt ist.
\item Der gespeicherte Name erscheint im Metainformationsfenster auf der
  rechten Seite.
\end{itemize}

\subsubsection{Speichern unter einem Dateinamen einer Datei, die bereits
  existiert, aber nicht ge�ffnet ist}

\emph{Tutorial}:

Speichern eines Graphen unter einem Dateinamen einer Datei, die bereits
existiert, jedoch nicht ge�ffnet ist.



\emph{Howto}:

Es wird die Aktion "Speichern unter" gew�hlt, und eine Datei ausgesucht, die
bereits existiert, jedoch nicht ge�ffnet ist. Es erscheint ein Dialog, der
darauf hinweist, dass die Datei bereits existiert und fragt, ob die
existierende Datei �berschrieben werden soll. Der Dialog hat die Optionen
"Ja", "Nein" und "Speichern unter".



\emph{Test}:

\begin{itemize}
\item Der Dialog erscheint wie erwartet.
\item Das Dialogfenster erscheint auch, wenn man nur "file" statt
  "file.mata.mxl" im Dateiauswahlfenster angibt.
\item Wird die Option "Nein" gew�hlt, wird der Graph nicht gespeichert und der
  gespeicherte Graph unbelassen.
\item Wird die Option "Ja" gew�hlt, wird der Graph gespeichert und �berschreibt
  den anderen Graph.
\item Wird die Option "Speichern unter" gew�hlt, wird der vorhandene Graph
  nicht ver�ndert und es �ffnet sich ein neues Dateiauswahlfenster.
\end{itemize}

\subsubsection{Speichern unter einem Dateinamen einer Datei, die bereits
  existiert und ge�ffnet ist}

\emph{Tutorial}:

Speichern eines Graphen unter einem Dateinamen einer Datei, die bereits
existiert und ge�ffnet ist.



\emph{Howto}:

Es wird die Aktion "Speichern unter" gew�hlt, und eine Datei ausgesucht, die
bereits existiert, und ge�ffnet ist. Es erscheint ein Dialog, der
darauf hinweist, dass die Datei bereits existiert und ge�ffnet ist, und hat die
Optionen "Speichern wiederholen", "Abbrechen" und "Speichern und Dateinamen des
anderen Kurses verwerfen".



\emph{Test}:

\begin{itemize}
\item Der Dialog erscheint wie erwartet.
\item Das Dialogfenster erscheint auch, wenn man nur "file" statt
  "file.mata.mxl" im Dateiauswahlfenster angibt.
\item Wird die Option "Speichern wiederholen" gew�hlt, wird nur ein neues
  Dateiauswahlfenster ge�ffnet.
\item Wird die Option "Abbrechen" gew�hlt, wird der Graph nicht gespeichert und der
  andere Graph nicht ver�ndert.
\item Wird die Option "Speichern und Dateinamen des anderen Kurses verwerfen"
  gew�hlt, so wird der Graph gespeichert und der Dateiname des anderen Graphen
  wird gel�scht.
\end{itemize}

\section{Programmfunktionen}

\subsection{Das Buffermen�}

\emph{Tutorial}:

Testen des Buffermen�s.

\emph{Howto}:

Das Buffermen� enth�lt eine Liste aller ge�ffneten Graphen. Diese sind nach dem
Schema "Name des Graphen" - "Dateiname" in der Reihenfolge der zuletzt
angezeigten Graphen aufgelistet. Ist ein Graph ver�ndert worden, so erscheint
ein * hinter dem Dateinamen.

\emph{Test}:
\begin{itemize}
\item Die Buffereintr�ge haben die Form "Name des Graphen" - "Dateiname"
  gegebenenfalls gefolgt von einem *.
\item Ist ein Graph noch nicht gespeichert, so erscheint statt des Dateinamens
  die Auskunft "kein Dateiname".
\item Der Name wird korrekt angezeigt und wird ge�ndert, wenn er im Eingabefeld
  des Metainformationsfensters ge�ndert wird.
\item Die Eintr�ge habe die Reihenfolge in der sie zuletzt angezeigt wurden.
\item Wird mit Hilfe des Buffermen�s ein anderer Graph angezeigt, so �ndert
  sich der angezeigte Graph entsprechend, aber es werden auch die
  entsprechenden Metainformationen im Metainformationsfenster angezeigt.
\end{itemize}

\emph{Howto}:

Um das * zu testen, ist folgendes zyklisch zu tun:
\begin{enumerate}
\item Den Graph speichern.
\item Pr�fen, dass das * nicht angezeigt wird.
\item Eine Ver�nderung durchf�hren.
\item Pr�fen, dass das * angezeigt wird.
\end{enumerate}


Zu Testen ist:
\begin{itemize}
\item Einf�gen von Komponenten
\item Verschieben von Komponenten 
\begin{itemize}
  \item mit Maus und
  \item mit Tastatur in alle vier Richtungen
\end{itemize}
\item Einf�gen von Subkomponenten
\item Verbinden von Komponenten
\begin{itemize}
  \item automatisch
  \begin{itemize}
    \item mit der entsprechenden Taste
    \item mit dem Button in der Toolbar
    \item mit dem Kontextmenu
  \end{itemize}
  \item von Hand
\end{itemize}
\item �ndern der Richtung einer Kante
\begin{itemize}
  \item mit der entsprechenden Taste
  \item Mit dem Button aus der Toolbar
  \item mit dem Kontextmen�
\end{itemize}
\item Ver�ndern von Kanten
\begin{itemize}
  \item Ver�ndern von Kantenteilst�cken
  \item Ver�ndern von ganzen Kanten
\end{itemize}
\item L�schen von Komponenten
\begin{itemize}
  \item mit der entsprechenden Taste
  \item Mit dem Button aus der Toolbar
  \item mit dem Kontextmen�
\end{itemize}
\item Erstellen des Roten Faden 
\begin{itemize}
  \item Ziehen einer roten Kante mit dem Kontextmen�
  \item �ndern der Farbe einer Kante
  \begin{itemize}
    \item mit den entsprechenden Tasten
    \item Mit den Buttons aus der Toolbar
    \item mit dem Kontextmen�
  \end{itemize}
  \item Erzeugen eines generischen Roten Faden
\end{itemize}
\item Zuweisen eines Dokumentes
\item Zuweisen einer Summary
\item Setzen oder l�schen von Punkten
\item Setzen eines Labels
\item �ndern der Metainformationen
\begin{itemize}
  \item der Name
  \item die Beschreibung
  \item der Status
  \item die Lehrveranstaltung
  \item die Kategorie
  \item der Bearbeitungszeitraum
\end{itemize}
\end{itemize}


\emph{Test}:
\begin{itemize}
\item Nach dem Speichern wird das * nicht angezeigt.
\item Nach dem Ver�ndern des Graphen wird das * angezeigt.
\item Ist ein Graph nicht mit einem * markiert, so erscheint beim Versuch ihn
  zu Speichern ein Hinweis, dass dieser Graph sich seit dem letzten Speichern
  nicht ver�ndert hat.
\end{itemize}

\subsection{Kontextmen�}

Beim Klicken mit der rechten Maustaste im Graphframe erscheint das Kontextmen�,
welches abh�ngig von dem Typ der markierten Komponenten ist.

Beim Aufruf des Kontextmen�s wird unterschieden, ob sich dabei eine
Komponente bzw. Kante "unter" dem Mauszeiger befindet, oder ob der Bereich
darunter leer ist. Da die rechte Maustaste ebenso zum Markieren verwendet wird,
bezieht sich im ersten Fall das Kontextmen� nur auf die markiere eine
Komponente bzw. Kante, im letzten Fall auf alle markierten Komponenten
bzw. Kanten.

Soll das Kontextmen� f�r mehrere Komponenten und Kanten aufgerufen werden, so muss
an eine freie Stelle geklickt werden.

Im Folgenden sind die Eintr�ge der Kontextmen�s tabellarisch in Abh�ngigkeit
zum Typ der markierten Komponenten aufgef�hrt.

\emph{Mainkomponenten}
\begin{table}
  \head
  Text
  \body
"L�schen"\\
"Schwarze Kante ziehen"\\
"rote Kante ziehen"\\
"Label setzen"
\end{table}

\emph{Branchkomponenten}
\begin{table}
  \head
  Text
  \body
"L�schen"\\
"Schwarze Kante ziehen"
\end{table}

\emph{Subkomponente}
\begin{table}
  \head
  Text
  \body
"L�schen"\\
"Move In"\\
"Move Out"\\
"Label setzen"
\end{table}

\emph{Kanten}
\begin{table}
  \head
  Text
  \body
"L�schen"\\
 "rote Kante setzen" bzw. "rote Kante entfernen"\\
"Kante in Netzwerk nehmen" oder "aus Netzwerk entfernen"\\
''Richtung �ndern"
\end{table}

Die vorherigen Tabellen beschrieben das Kontextmen�, wenn nur eine Komponente
markiert ist. Sind beliebige Komponenten und/oder Kanten ausgew�hlt, so enth�lt
das Kontextmen� nur den Punkt "L�schen".

Sind Komponente ausgew�hlt, die einen Roten Faden erzeugen k�nnen, so erscheint
dazu auch der Men�punkt "Roten Faden erzeugen".

Wenn genau zwei Komponenten vom Typ Graphkomponente ausgew�hlt sind, dann
enth�lt das Kontextmen� neben dem Punkt "L�schen" noch den Punkt "Verbinden".


\subsection{Statusleiste}

\emph{Tutorial}:

Testen der Ausgaben verschiedener Komponenten in der Statuszeile.

\emph{Howto}:

Ausgaben in der Statuszeile werden erzeugt, indem einzelne Komponenten und Kanten
markiert werden.

\emph{Test}:
\begin{itemize}
\item Ist eine Kante markiert, so wird eine Aussage �ber die Farbe der Kante
  ausgegeben, also ''Kante schwarz'', ''Kante rot im Netzwerk'' oder ''Kante
rot''.
\item Ist eine Branchkomponente markiert, wird die Kategorie ausgegeben, also
  ''or'' oder ''and''.
\item Ist eine Dokumentkomponenten markiert, wird das Label und der
  gegebenenfalls gek�rzte Pfad des zugewiesenen Dokuments angezeigt. Der Pfad
  erscheint vollst�ndig, wenn man das Fenster verbreitert. Ist kein Label
  gesetzt, wird nur der Pfad ausgegeben. Ist noch kein Dokument zugewiesen, so
  steht in der Statusleiste ''kein Dokument zugewiesen''.
\item Es erscheint nur eine Ausgabe wenn genau eine Komponente oder Kante
  markiert ist.
\end{itemize}

\subsection{undo}

R�ckg�ngig machen des letzten Schrittes.

\emph{Howto}:

Um den letzten Schritt r�ckg�ngig zu machen, wird der Button
\image{howtopics/tool_undo.png} bet�tigt.

\emph{Test}:
\begin{itemize}
\item Der letzte Schritt wird r�ckg�ngig gemacht.
\item Es k�nnen alle Schritte, die den Graphen ver�ndert haben, r�ckg�ngig
  gemacht werden.
\end{itemize}

\subsection{HTTPS-Server Zertifikat}
\subsubsection{Setzen des Zertifikates}

\emph{Tutorial}:

Testen der Zertifikateinstellung.

\emph{Howto}:

Um das Zertifikat zu �ndern, wird der Untermen�punkt "Zertifikat �ndern" des
Men�s "Einstellungen" gew�hlt. Wird versucht, sich auf einem https-Server
einzuloggen, ohne das dieser Server dem System bekannt ist, erscheint ein
Dialog, der ebenfalls erm�glicht, dass eine neue Zertifikatdatei gesetzt
wird.

Es �ffnet sich ein Dateiauswahlfenster in dem die entsprechende Datei
ausgew�hlt werden kann.

\emph{Test}:

\begin{itemize}
\item Es kann nur eine Datei, die ein korrektes Keystore-Format hat, ausgew�hlt
  werden, ansonsten erscheint eine Fehlermeldung.
\item Ist ein Https-Server nicht bekannt, erscheint eine Fehlermeldung.
\item Nachdem die korrekt Datei als Zertifikat gesetzt ist, kann man sich auf
allen zertifizierten https-Servern einloggen.
\end{itemize}

\subsubsection{Anzeigen der bekannten Server}

\emph{Tutorial}:

Anzeigen der bekannten Aliase f�r das Zertifikat.

\emph{Howto}:

Um sich die bekannten Aliase anzeigen zu lassen wird der Untermen�punkt
"Aliase" des Men�s "Ansicht" gew�hlt.


\emph{Test}:

\begin{itemize}
\item Es erscheint ein Fenster, in dem alle bekannten Aliase aufgef�hrt
  sind. Verifiziert werden kann die Angabe mit dem Programm keytool.
\end{itemize}

\label{serverLogin}\subsection{Serverlogin}

\emph{Tutorial}:

Einloggen auf einem Server.


\emph{Howto}:

Um sich auf einem Server einzuloggen, muss das Login-Fenster ge�ffnet
werden. Dies �ffnet sich mit dem  Untermen�punkt "neuer Server-Login" des Men�s
"Einstellungen", und beim Starten des Programms. Ebenso, wenn man offline ist
und das Elementzuweisungsfenster �ffnen m�chte, oder die Summary �ndern will.

\input{test_serverLogin.tex}

\section{Konfiguration}

\emph{Tutorial}:

Die Konfiguration teilt sich in die lokale Konfigurationsdatei \code
{.coursecreator.xml} im \code{\$HOME} Verzeichnis und die globale
Konfigurationsdatei \code{config.xml} im Verzeichnis \code
{\$MM_BUILD_PREFIX/etc/coursecreator/}.

Alle Einstellungen die in der Konfiguration ge�ndert werden,  werden sofort in
die entsprechende Konfigurationsdatei geschrieben, dass hei�t, die Ver�nderung
kann mit Hilfe eines Editors oder �hnlich sofort kontrolliert werden.

Die Konfiguration wird nur einmalig zu Programmbeginn gelesen, eine Ver�nderung
der Konfigurationsdatei ben�tigt folglich einen Neustart des Programms.

\subsection{Schrittweite}

\emph{Tutorial}:

\code{movestep} beschreibt die Schrittweite beim Verschieben einer Komponente
mit der Tastatur. Siehe \href{\#verschiebenVonKomponenten}{verschieben von
Komponenten}.


\emph{Howto}:

Es gibt noch keine Eingabe f�r diese Angabe, daher ist nur das Auslesen der
Angabe zu testen. Dazu wird die Variable mit Hilfe eines Editors ver�ndert und
nach dem (Neu-)start des Programms die Schrittweite �berpr�ft.

\emph{Test}:
\begin{itemize}
\item Die Schrittweite �ndert sich, wenn diese in der Konfigurationsdatei
  ver�ndert wird (nach Neustart des Programms).
\end{itemize}

\subsection{showGeneric - showNonGeneric - showMetainfopane}

\emph{Tutorial}:

�ndern der ShowGeneric ShowNonGeneric und ShowMetainfopane-Variable.

\emph{Howto}:

Alle drei Variablen werden im Elementzuweisungsfenster durch die entsprechenden
Checkboxes gesetzt. "true" bedeutet dabei, dass die Variable angeklickt ist.

\emph{Test}:
\begin{itemize}
\item Wird eine Variable im Elementzuweisungsfenster ver�ndert, �ndert sie sich
  entsprechend in der lokalen Konfigurationsdatei.
\item Wird eine Variable  in der lokalen Konfigurationsdatei ver�ndert, �ndert
  sie sich im Elementzuweisungsfenster.
\end{itemize}

\subsection{Serverliste}

\emph{Tutorial}:

Ver�ndern der Serverliste.

\emph{Howto}:

Bisher k�nnen Server nur hinzugef�gt werden. Dies geschieht, indem im
Server-Login-Fenster (siehe \href{\#serverLogin}{Serverlogin}) eine neuer
Server eingetragen wird und man sich erfolgreich an diesem Server einloggen kann.

\emph{Test}:
\begin{itemize}
\item Es werden nur die Server angezeigt, die auch in der Konfiguration stehen
  (au�er es wurden bereits neue Server im Server-Login-Fenster hinzugef�gt).
\item Durch das erfolgreiche Einloggen auf einem Server wird dieser an den
  Anfang der Liste geschoben. (Falls er bereits weiter unten in der Liste
  stand, wird er dort gel�scht.)
\item Ein Server erscheint nur in der Serverliste, wenn man sich
  \emph{erfolgreich} anmelden konnte.
\item Die Server erscheinen im ServerLogin in der Reihenfolge, in der sie in
  der Konfigurationsdatei stehen.
\end{itemize}

\subsection{Laden und Speichern Verzeichnisse}

\emph{Tutorial}:

Ver�ndern der Laden und Speichern Verzeichnisse.

\emph{Howto}:

Die \code{storedir}-Variable enth�lt den Pfad in dem zuletzt gespeichert wurde,
die \code{loaddir}-Variable ananlog das Verzeichnis aus dem zuletzt geladen
wurde. Dies dient dazu, beim �ffnen der Verzeichnisfenster "Speichern unter",
bzw "Laden" im zuletzt verwendeten Verzeichnis zu sein.

\emph{Test}:
\begin{itemize}
\item Der Storedir-Verzeichnispfad in der Konfigurationsdatei �ndern sich, sobald eine
  Datei mit dem "Speichern unter"-Befehl (siehe \href{\#speichern}{Speichern})
  in einem anderen (als die vorherige Datei) gespeichert wird.
\item Der Loaddir-Verzeichnispfad in der Konfigurationsdatei �ndern sich, wenn
  eine Datei aus einem anderen Verzeichnis (als die zuletzt geladene Datei)
  geladen wird.
\item Die Verzeichnisdialoge "Laden" und "Speichern unter" �ffnen sich in dem
  Verzeichnispfad, welcher in der Konfigurationsdatei stehen.
\end{itemize}

\subsection{Zertifikatsverzeichnis}

\emph{Tutorial}:

�ndern des Zertifikates f�r https-Server.

\emph{Howto}:

Das Zertifikat wird ver�ndert, indem im Men� "Einstellungen" der Unterpunkt
"Zertifikat �ndern" gew�hlt wird.

\emph{Test}:
\begin{itemize}
\item Es wird die Datei benutzt, die in der \code{certdir}-Variable angegeben ist.
\item Ist die Datei, die in der \code{certdir}-Variable angegeben ist keine korrekte
  Datei, oder existiert der Tag nicht, kann sich nicht an https-Servern
  angemeldet werden.
\item Es k�nnen keinen Dateien gew�hlt werden, die keine g�ltige
  Zertifikatsdateistruktur haben.
\end{itemize}

\subsection{Gr��e und Position der Fenster}

\emph{Tutorial}:

Ver�ndern der Position und der Gr��e von Fenstern.

\emph{Howto}:

Die Variablen der Position und der Gr��e in der Konfigurationsdatei werden
ver�ndert, indem die Fenstergr��en, bzw die Position eines Fensters ver�ndert
werden.

\emph{Test}:
\begin{itemize}
\item Wird ein Fester in seiner Gr��e, oder seiner Position ver�ndert, so wird
  dies in die lokalen Konfigurationsdatei geschrieben.
\item Die Fenster �ffnen sich an der Position (Ecke oben Links) und mit der
  Gr��e, die in der lokalen Konfigurationsdatei geschrieben steht.
\end{itemize}

\subsection{Debugging}

\emph{Tutorial}:

�ndern der Debugeinstellungen.

\emph{Howto}:

Die Debugeinstellungen werden ge�ndert, indem das Debugfenster mit dem
Untermen�punkt "Debugeinstellungen" des Men�punktes "Einstellungen" ge�ffnet
wird.

\emph{Test}:
\begin{itemize}
\item Das �ndern der Debugeinstellungen im Debugfenster wird in die
  Konfigurationsdatei �bernommen.
\item Die "H�kchen" im Debugfenster sind so gesetzt, wie sie in der
  Konfigurationsdatei stehen.
\item Nur wenn eine Variable "true" ist und die "DEBUG"-Variable ebenfalls
  "true" ist, werden Debugausgaben get�tigt.
\item Ist die Exception-Variable "true", so werden auftretende Exceptions als
  Dialog ausgegeben (ansonsten gar nicht).
\end{itemize}

\section*{Organisatorisches}

Alle Fehler sollen in unser Bug-Tracking-System eingetragen werden.

URL:
\begin{preformatted}%
  \link{http://www3.math.tu-berlin.de/rt}{www3.math.tu-berlin.de/rt}
\end{preformatted}

Dabei bitte folgendes beachten:

\begin{itemize}
\item Fehler, die bereits gemeldet wurden und noch nicht behoben sind, nicht
  noch einmal melden. Falls sich durch diesem Test neue Erkenntnisse ergeben,
  diese bei dem alten Bug eintragen (als Comment oder Reply).
\item Als Kategorie (Queue) CourseCreator angeben.
\item \emph{Owner} nicht setzen
\end{itemize}
\end{document}
