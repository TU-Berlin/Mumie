\documentclass[a4paper,11pt]{article}
\usepackage{graphicx,german}
\usepackage[latin1]{inputenc}

\textwidth15.5cm \textheight23cm
\oddsidemargin2mm \evensidemargin2mm \topmargin-10mm
\nonfrenchspacing
\parskip0.15\baselineskip

\title{Der CourseCreator}
\author{Jens Binder $<$binder@math.tu-berlin.de$>$}

\newcommand{\minorheadline}[1]{\textbf{#1:}}

\begin{document}

\maketitle


\begin{center}
{\verb'$Id: CC_spec.tex,v 1.1 2006/10/05 09:20:50 binder Exp $'}
\end{center}

\section{Pr�ambel}

Der \emph{CourseCreator} ist ein Autorenwerkzeug f�r die multimediale
Lernplattform ''Mumie''. Der Benutzer(Autor) ist mit dem CourseCreator in der
Lage, die innerhalb der Mumie vorhandenen Inhalte miteinander zu verkn�pfen um
sie in einer f�r den Studenten logischen, nichtliniearen Reihenfolge
darzustellen. Gerade bei mathematischen Inhalten ist es zumeist sehr hilfreich
auf eine nichtlineare, parallele Darstellung von Inhalten zur�ckzugreifen
um Gemeinsamkeiten, Unterschiede sowie Abh�ngikeiten von Elementen
aufzuzeigen. Das wird im CourseCreator mittels ungerichteten Graphen
erreicht deren Knoten den jeweiligen Inhaltselementen und die
Kanten logische Verkn�pfungen zwischen diesen Elementen entsprechen.
Wir erreichen damit eine Darstellungsform, welche �ber
die pure Abbildung der Reihenfolge von Elementen hinausgeht, aber auch f�r
Nichtinformatiker bearbeitbar bleibt. 
Vor diesem Hintergrund wurde der CourseCreator als ein grafisches Interface f�r
die Erstellung und Verwaltung von Kursen entwickelt. Er setzt indirekt auf dem
Appliaktion-Server der Mumie - dem \emph{Japs} auf

\section{Das Grunddesign}

Der CourseCreator ist eine Java-Applikation mit grafischer Benutzeroberfl�che
auf Basis der Java-Komponente \emph{Swing}. 
Die Hauptaufgaben sind:
\begin{itemize}
\item \minorheadline{Darstellung von Kursen} Der (nichtlinearen) Darstellung
von Kursen kommt eine besondere Bedeutung zu. Nach au�en hin wird dies mit
\emph{JGraph}, einer Unterkomponente von Swing realisiert, die es erlaubt
komplexe Graphen grafisch darzustellen. Der Benutzer arbeitet also einzig
auf einem bildlich dargestellten Graphen. Intern (innerhalb von
Japs und CourseCreator) werden die Graphen im XML-Format dargestellt. F�r diese
Darstellung existiert eine eigene Spezifikation (''KursXML'').
\item \minorheadline{Laden von Kursen aus der Datenbank bzw. aus dem
Dateisystem} Der CourseCreator soll in der Lage sein Kurse zu laden -
unabh�ngig vom Speicherort. Im Falle des Ladens aus der Datenbank soll die
Japs-Komponente MMCDK benutzt werden um die XML codierten Kursdaten in den
CourseCreator zu laden, welcher diese anschlie�end grafisch aufbereitet (siehe
oben).
Die Aufgabe des MMCDK ist das Verwalten der Kurs-Inhalte einschliesslich deren
Checkout aus dem Japs. Er soll vom CourseCreator ''fernsteuerbar'' sein.
\item \minorheadline{Zuweisen von Inhalten} Dies setzt eine http(s)-Verbindung
zur Mumie vorraus, um konsitente Inhaltsverkn�pfungen zu garantieren. Das
bedeutet das die Kurselemente die im CourseCreator zugeordnet werden und die
sp�ter vom Studenten verwendet werden sollen auch wirklich in der Datenbank
vorhanden sein m�ssen. Die Zuweisung direkt geschieht �ber ein weiteres Fenster
in welchem die aktuelle Pfadstruktur innerhalb des Japs abgebildet ist. In
dieser Struktur kann sich der Benutzer die gew�nschten Elemente ausw�hlen und in
seiner Kurstruktur zuweisen.
\item \minorheadline{Speichern von Kursen - lokal oder in die Datenbank}
Beim Speichern der Kurse wird der erstellte Graph auf das interne XML
abgebildet und anschlie�end in die Datenbank mittels MMCDK eingecheckt oder
einfach lokal auf der Festplatte gespeichert - wenn zum Beispiel ein Kurs noch
unvollst�ndig ist.

\end{itemize}



\end{document}
