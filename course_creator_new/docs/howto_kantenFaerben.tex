\emph{Howto}:

Zum Wechsel der Farbe einer Kante zwischen zwei Mainkomponenten wird eine
schwarze Kante markiert und 
\begin{itemize}
\item im Kontextmen� dieser Kante der Punkt "rote Kante setzen" gew�hlt.
\item der Button \image{howtopics/tool_edge_red.png} aus der Toolbar
  angeklickt.
\item die Taste, die daf�r in der Tastenkonfiguration eingetragen ist,
  gedr�ckt.
\end{itemize}

Auf die gleiche Weise kann eine Kante wieder vom Roten Faden herunter genommen
werden (mit dem Unterschied, dass der Kontextmen�eintrag dann "rote Kante
entfernen" hei�t). Dabei ist zu beachten, dass eine rote Kante die nicht in
Netzwerk ist, bei der Aktion "rote Kante entfernen" gel�scht werden wird. Der
CourseCreator zeigt dann einen Auswahldialog an, um darauf hinzuweisen.

\image{howtopics/dialog_kanteLoeschen.png}

Es k�nnen auch Kanten des Roten Faden ins Netzwerk genommen werden, bzw. auch
wieder entfernt werden. Dazu wird eine Kante markiert,:
\begin{itemize}
\item im Kontextmen� dieser Kante der Punkt "aus Netzwerk entfernen" bzw. "ins
  Netzwerk nehmen" gew�hlt.
\item der Button \image{howtopics/tool_edge_exists.png} aus der Toolbar
  angeklickt.
\item die Taste, die daf�r in der Tastenkonfiguration eingetragen ist,
  gedr�ckt.
\end{itemize}

Eine rote Kante zwischen zwei Mainkomponenten kann auch erstellt werden, indem
im Kontextmen� einer Komponente der Punkt "rote Kante ziehen" angew�hlt wird
und dann verfahren wird wie beim
\href{\#verbindenVonKomponentenVonHand}{Verbinden von Komponenten von
  Hand}. Dieses geht aber nur, wenn Start- und Endknoten keine
Branchkomponenten sind.
