\documentclass{generic}

% ------------------------------------------------------------------------------
% Schriftgroessen
% ------------------------------------------------------------------------------

\renewcommand{\normalsize}{\fontsize{18}{22}\selectfont}
\renewcommand{\large}{\fontsize{20}{26}\selectfont}
\renewcommand{\Large}{\fontsize{22}{30}\selectfont}
\renewcommand{\small}{\fontsize{14}{16}\selectfont}
\newcommand{\smaller}{\fontsize{12}{14}\selectfont}
\newcommand{\verysmall}{\fontsize{6}{6}\selectfont}
\renewcommand{\footnotesize}{\fontsize{12}{14}\selectfont}
\newcommand{\tocsize}{\fontsize{12}{16}\selectfont}
\newcommand{\titlesize}{\fontsize{24}{36}\selectfont}

% ------------------------------------------------------------------------------
% Texthervorhebungen
% ------------------------------------------------------------------------------

%\definecolor{emphcolor}{rgb}{0.77,0.00,0.00}
%\renewcommand{\emph}[1]{\textbf{\textcolor{emphcolor}{#1}}}
\renewcommand{\emph}[1]{\textbf{#1}}

\newcommand{\term}[1]{\textit{#1}}
% \newcommand{\code}[1]{\texttt{#1}}
\newcommand{\var}[1]{\textit{#1}}

%\definecolor{headlinecolor}{rgb}{0.77,0.00, 0.00}
\definecolor{headlinecolor}{rgb}{0.67,0.09, 0.22}
\newcommand{\headline}[1]{\textbf{\textcolor{headlinecolor}{#1}}}

\definecolor{emphboxbordercolor}{rgb}{0.80,0.36,0.36}
\definecolor{emphboxcolor}{rgb}{0.98,0.98,0.77}
\newcommand{\emphbox}[2]{\fcolorbox{emphboxbordercolor}{emphboxcolor}{\parbox{#1}{#2}}}

\definecolor{labelcolor}{rgb}{0.35,0.35, 0.8}


% -------------------------------------------------------------------------------
% Pfeile
% -------------------------------------------------------------------------------

\newcommand{\myarrow}{{\Large\boldmath$\;\rightarrow\;$}}
\newcommand{\mylongarrow}{{\Large\boldmath$\;\longrightarrow\;$}}



% -------------------------------------------------------------------------------
% Befehle f�r Abbildungen 
% -------------------------------------------------------------------------------

\renewcommand{\thefigure}{\thesubsection.\arabic{figure}}

\newcommand{\mycaption}[2]{%
 \centerline{%
  \parbox{#1}{%
   \small \refstepcounter{figure} Figure \thefigure. #2 }}}


%---------------------------------------------------------------------------
% Gliederungsbefehle
%---------------------------------------------------------------------------

\makeatletter

\renewcommand{\section}{\@startsection
 {section}%                                   % Name
 {1}%                                         % Ebene
 {0pt}%                                       % Einzug
 {-0.0\baselineskip}%                        % Vorabstand 
 {0.85\baselineskip}%                         % Nachabstand
 {\fontsize{20}{24}\selectfont\itshape\bfseries}}%    % Stil

\makeatother

\newcommand{\mysection}[1]{%

\vspace*{-1.2\baselineskip}

{\color{headlinecolor}
\section{#1}}}

\newcounter{myappendix}

\newcommand{\myappendix}[2]{%
\hfill\makebox[0cm][l]{\tocref}

\vspace*{-1.2\baselineskip}

\refstepcounter{myappendix}
\section*{A\arabic{myappendix}\hspace{1em}#1}\hypertarget{#2}{}}



% -------------------------------------------------------------------------------
% Inhaltsverzeichnis 
% -------------------------------------------------------------------------------

\newcommand{\mytoctitle}[1]{\begin{center}\textbf{\hypertarget{inhalt}{#1}}\end{center}}

\newenvironment{mytoc}
  {\tocsize\textbf{\hypertarget{inhalt}{Inhalt}}\\\begin{enumerate}}
  {\end{enumerate}}

\newcommand{\mytocitem}[3]{\item[#1]\hyperlink{#3}{#2}}

\newcommand{\tocref}{\verysmall\hyperlink{inhalt}{TOC}\normalsize}

% -------------------------------------------------------------------------------
% Listen 
% -------------------------------------------------------------------------------

%\newcommand{\mylabelitemi}{\labelitemi}
\newcommand{\mylabelitemi}{\raisebox{0.25ex}{{\small $\blacktriangleright$}\hspace{0.1em}}}

%\newcommand{\mylabelitemii}{\labelitemii}
%\newcommand{\mylabelitemii}{\raisebox{0.25ex}{{\small $\triangleright$}\hspace{0.1em}}}
\newcommand{\mylabelitemii}{\raisebox{0.1ex}{\large\bfseries-}\hspace{0.075em}}

\newenvironment{mylist}[1][\mylabelitemi]
  {\begin{list}{{\color{labelcolor}#1}}
    {\setlength{\itemindent}{0cm}
     \setlength{\labelwidth}{1em}
     \setlength{\leftmargin}{0.5cm}
     \setlength{\rightmargin}{1cm}}}
  {\end{list}}

\newenvironment{myenum}
  {\begin{list}{\theenumi}
    {\usecounter{enumi}
     \setlength{\itemindent}{0cm}
     \setlength{\labelwidth}{1em}
     \setlength{\leftmargin}{0.5cm}
     \setlength{\rightmargin}{1cm}}}
  {\end{list}}

\newcommand{\pitem}{\pause\item}
\newcommand{\itemii}{\item[\mylabelitemii]}

\newcounter{romanlistcounter}
\newenvironment{romanlist}{\begin{list}%
 {(\roman{romanlistcounter})\hfill}{\usecounter{romanlistcounter}%
 \topsep0.05\baselineskip% 
 \partopsep0.85\baselineskip%
 \leftmargin2em%
 \labelwidth2em%
 \labelsep0pt%
 \itemindent0pt%
 \parsep0.15\baselineskip%
 \itemsep0.0\baselineskip}}%
 {\end{list}}

\newcounter{alphlistcounter}
\newenvironment{alphlist}{\begin{list}%
 {(\alph{alphlistcounter})\hfill}{\usecounter{alphlistcounter}%
 \topsep0.05\baselineskip% 
 \partopsep0.85\baselineskip%
 \leftmargin1.8em%
 \labelwidth1.8em%
 \labelsep0pt%
 \itemindent0pt%
 \parsep0.15\baselineskip%
 \itemsep0.0\baselineskip}}%
 {\end{list}}

\newcounter{arabiclistcounter}
\newenvironment{arabiclist}{\begin{list}%
 {(\arabic{arabiclistcounter})\hfill}{\usecounter{arabiclistcounter}%
 \topsep0.05\baselineskip% 
 \partopsep0.85\baselineskip%
 \leftmargin1.8em%
 \labelwidth1.8em%
 \labelsep0pt%               
 \itemindent0pt%
 \parsep0.15\baselineskip%
 \itemsep0.0\baselineskip}}%
 {\end{list}}




\begin{document}

\title{Klassenstruktur}
\tableofcontents
\section{CellKlassen}
\image{cells.png}
\subsection{GraphCell}\label{graphcell}
GraphCellKlassen sind die, die zu einem Graphen geh�ren, das heisst, sie treten
als Knoten im Netzwerk auf. GraphCell ist abgeleitet von der JGraph-Klasse
DefaultGraphCell und somit abstrakt. Sie verf\"ugen �ber eine 
\begin{itemize}
 \item \element{NodeID}, welche dazu verwendet wird, festzulegen, zwischen
  welchen zwei Knoten eine Kante besteht (siehe kurs\_xml), und �ber
 \item \element{posx,posy}, Koordinaten die die (x,y)-Position im GraphView festlegen.
\end{itemize}
Von der Klasse GraphCell sind abgeleitet BranchCell und MainComponentCell.

\subsection{das Interface ComponentCell}
ComponentCells sind die Klassen, die ''\"uber Inhalte verf\"ugen'', das heisst,
Instanzen dieser Klassen wird ein Document aus dem Japs
zugeordnet. ComponentCells haben folgende Variablen:

\begin{itemize}
 \item \val{doctype:} bestimmt die Art des Documentes, m�gliche Werte sind Section,
  Element, Problem, SubSection und SubElement
 \item \element{category:} diese kommt vom Japs, und ist abh�ngig vom docType, m�gliche
  Werte sind definition, lemma, theorem, proof, pretest...
 \item \element{lid:} die lokale ID
 \item \element{id:} DatenbankID
 \item \element{label:} Label
\end{itemize}
Zudem besitzt die Klasse eine \texenv{toXMLMeta} Methode, die das XML-Tag f�r
die einzelnen Komponenten erzeugt.
Die Klassen MainComponentCell und SubComponentCell impementieren ComponentCell.

\section{Cells}
Es gibt drei Klassen von Cells:
Branchcells, MainCompomentCells und SubComponentCells.

\subsection{BranchCell}
Die Branchcell entspricht einem Verzweigungsknoten vom Typ ''Und'' oder
''Oder''. Die Klasse ist abgeleitet von% \href{\#graphcell}
 GraphCell. Die
Klasse hat eine Variable
\begin{itemize}
\item \element{category:} bestimmt die Art des Knotens, m\"ogliche Werte sind
  ''Und'' und ''Oder'' (ggf sinnvollerweise als Integer zu verwalten)
\end{itemize}
und folgende Methoden:
\begin{itemize}
\item \texenv{paint()}: die Paintmethode
\item \texenv{toXMLContent}: diese Methode erzeugt einen String, der dem
  XML-String nach der kurs_xml-spezifikation entspricht.
\end{itemize}


\subsection{MainComponentCell}
Diese Klasse ist abgeleitet von ComponentCell und GraphCell. Sie hat 
\begin{itemize}
 \item \element{SubComponentCell[]}: eine List der SubComponentcells die zu
   dieser Instanz geh\"oren.
\end{itemize}

und die Methoden:
\begin{itemize}
 \item \texenv{paint()}: die Paintmethode
 \item \texenv{toXMLContent()}: diese Methode erzeugt einen String, der dem
  XML-String nach der kurs_xml-spezifikation entspricht.
\end{itemize}
\subsection{SubComponentCell}
Diese Klasse ist abgeleitet von ComponentCell. Sie hat 
\begin{itemize}
 \item \element{parent}: Die MainComponentcell die zu dieser Instanz geh\"ort.
 \item \element{posx,posy}: Obwohl die Koordinaten von der zugeh\"origen
   MainComponentCell abh\"angen, ben\"otigt die SubComponentCell zum Zeichnen
   ebenfalls die Variablen.
 
 \item \element{count}: gibt an, die wievielte SubKomponente diese ist. Z\"ahlung
   von innen nach aussen, beginnend bei eins.
\end{itemize}
FIXME: align, count und parent erzeugen schon posx und posy..
%TODO: 
Sollte es m\"oglich sein, die Reihenfolge der SubKomponenten zu veraendern, so
sind align und count sinnvoll. ist die Reihenfolge aber nicht relevant, so
werden diese nicht notwendig ben\"otigt, die SubKomponenten werden einfach in
beliebiger Reihenfolge verwaltet.
\\Frage: sollte die Reihenfolge von SubKomponenten ver\"anderbar sein?

und die Methoden:
\begin{itemize}
 \item \texenv{getGlobalKoordinates()}erzeugt aus parent, count und align die
   globalen Koordinaten, die zum Zeichnen ben\"otigt werden. (siehe FIXME)
 \item \element{getAlign()}: die Ausrichtung, ist abh\"angig von der
   \element{category} dieser Instanz. m\"ogliche Werte sind topleft, topright,
   bottomleft, bottomright //siehe FIXME
 \item \texenv{paint()}: die Paintmethode
 \item \texenv{toXMLContent}: diese Methode erzeugt einen String, der dem
  XML-String nach der kurs_xml-Spezifikation entspricht.
\end{itemize}

\section{EdgeCell}
Diese Klasse ist abgeleitet von DefaultEdge. Einen Kante ist orthogonal und
gerichtet, sie ist also spezifiziert durch Start- und End-Graphcell, eine Liste
von St\"utzpunkten und einer Farbe (eine Kante kann rot oder nicht rot sein,
''existieren'' oder ''nicht existieren'' Kanten des Netzwerkes und der rote
Faden unterscheiden sich nicht, nur in diesen beiden Variablen)\\

Eine Kante wird per Mausklick auf der Oberfl\"ache erzeugt. Daher wird eine
Kante erzeugt, indem zuerst der StartKnoten festgelegt wird, dann eine reihe
von (orthogonalen) EckSt\"utzPunkten hinzugef\"ugt wird, dann ein EndKnoten
festgelegt wird. Abschliessend wird ein eventuell letzter EckSt\"utzPunkt
angehangen und die KantenSt\"utzPunkte werden berechnet.

Variablen:
\begin{itemize}
 \item \element{startCell} vom Type GraphCell
 \item \element{endCell} vom Type GraphCell
 \item \element{points[]} Liste der St\"utzpunkte vom Typ Point
 \item \element{isRed} gibt an, ob diese Kante rot ist.
 \item \element{exists} gibt an, ob eine Kante im Netzwerk ist.
 \item \element{isForward} gibt an, ob diese Kante vorw\"arts (also bei
   StartCell beginnend) ist (sonst beginnend bei Endcell)
\end{itemize}

Methoden:
\begin{itemize}
\item \texenv{addPoint(Point)} f\"ugt einen St\"utzpunkt hinzu
\item \texenv{getOrthogonalPoint(Point p)} berechnet zu p den orthogonalen
  Punkt in Abh\"angigkeit zu dem letzten Punkt in der Liste.
\item \texenv{concludeAdding()} f\"ugt, falls die bisherige Anzahl ungerade,
  noch einen letzten EckSt\"utzPunkt mit der y-Koordinate des bisher lezten und
  der x-Koordinate des Endvertexes hinzu (Orthogonale Kanten haben gerade Anzahl
  von EckSt\"utzPunkten), und berechnet die KantenSt\"utzPunkte (Mittel des
  vorhergehenden und nachfolgenden EckSt\"utzPunktes).
\item \texenv{toggleDirection()} Dreht die Kante um. Switcht isForward
  -moeglich w\"are auch tauschen von Start- und EndCell, aber dann muesste die
  St\"utzPunkteListe auch umgedreht werden.
\end{itemize}

\section{NavGraph}
Diese Klasse ist abgeleitet von der JGraph-Klasse und verwaltet einen Graphen
(das Model vom Typ NavModel extends DefaultGraphModel), d.h eine Liste der
Komponenten (Cellen) und Kanten, die zu diesem Graphen geh\"oren. Desweiteren von
welchen Typ der Graph ist (Section vs Element), ..
FIXME: Element ist kein sch\"oner Name f\"ur einen Graphen..

Variablen:
\begin{itemize}
 \item \element{isSection}: gibt an, ob ein Graph eine Section ist, falls
   nicht, ist der Graph ein ElementGraph. \\M\"ogliche Werte: \\
   \val{true}: Section\\
   \val{false}: Element
 \item \element{Cells[]}: Liste aller Cellen in diesem Graphen (auch branchCells)
 \item \element{Edges[]}: Liste aller Kanten in diesem Graphen

\end{itemize}

Zudem ben\"otigt die Klasse Methoden zur Manipulation des Graphen:

\begin{itemize}
 \item \texenv{insertComponent(cellType, category)}: erzeugt eine neue (leere)
  Celle, wobei \element{cellType} angibt, ob es sich um eine Branchcell, eine
  MainComponentCell oder eine SubComponentCell handelt, bei der ComponentCell
  ist der docType eindeutig durch die Variable \element{isSection}
  beschrieben. \element{category} ist kontextabh\"angig von \element{cellType}
  und \element{isSection}
 \item \texenv{insert(EdgeCell)}
 \item \texenv{delete(selection)} loescht die Selection und, falls die
   Selection vom Typ GraphCell ist, auch die anliegenden Kanten.
 \item \texenv{connect(GraphCell1, GraphCell2,isRed,exists)} erzeugt Defaultkante und
   verbindet zwei Zellen mit dieser Defaultkante
 \item \texenv{connect(EdgeCell)} verbindet zwei Cellen mit dieser Kante
\end{itemize}

Ausserdem verf\"ugt der NavGraph \"uber den roten Faden\\
Variablen:
\begin{itemize}
 \item \element{redLineCells[]} Liste der GraphCells auf dem roten Faden
 \item \element{redLineEdges[]} Liste der Kanten auf dem roten Faden
\end{itemize}
und Methoden zur Manipulation:
\begin{itemize}
 \item \texenv{addToRedLine(Object)} f\"ugt GraphCell oder Edge zur redLineCells
  oder redLineEdges hinzu
 \item \texenv{deleteFromRedLine(Object)} l\"oscht GraphCell oder Edge aus
  redLineCells oder redLineEdges
 \item \texenv{clearRedLine()} l\"oescht redLineCells und redLineEdges
\end{itemize}

zudem gibt es noch Korrektheitsmethoden:
\begin{itemize}
 \item \texenv{checkRedLine()} \"uberpr\"uft, ob der Rote Faden korrekt
   ist. \\
   (1.) pr\"ufe, dass GraphCellen, die _nicht_ auf dem roten Faden liegen keine
   roten Kanten haben\\
   (2.) pr\"ufe, ob GraphCellen, die auf dem roten Faden liegen genau zwei rote
   Kanten haben, bzw. genau zwei GraphCellen genau eine rote Kante
   haben. letztere sind start und endCelle (merken)\\
   (3.) Laufe von StartCelle aus \"uber alle Kanten und z\"ahle mit. entspricht
   Anzahl der gr\"osse von redLineCell --> korrekt
 \item \texenv{checkNet} pr\"uft, ob Netzwerk korrekt. \\ pr\"ufe, ob
   gerichtete Kreise

\end{itemize}

%Zum Speichern des Graphen besitzt der NavGraph noch toXML-Methoden:
\end{document}
