\emph{Test}:

\begin{itemize}
\item Die Datei wird gespeichert, und kann danach ebenso geladen werden.
\item Wurde der Graph nicht ver�ndert, erscheint ein Hinweis, dass der Graph
  unver�ndert ist und der Autor hat die M�glichkeit trotzdem zu speichern.
\item Hat ein Graph bisher keinen Dateinamen (zb, weil er nicht geladen oder
  bisher nicht gespeichert wurde, oder seinen Namen durch erneutes �ffnen
  verloren hat) wird automatisch die Aktion "Speichern unter" statt "Speichern"
  ausgef�hrt.
\item Hat der Graph einen Dateinamen, so wird der Graph bei der Aktion
  "Speichern" ohne Dateiauswahlfenster gespeichert.
\item Es werden eine Datei "file.meta.xml" und eine Datei "file.content.xml"
  erzeugt.
\item Der Graph wird gespeichert, unabh�ngig davon, ob der Name mit
  "file.meta.xml" oder nur mit "file" angegeben wird.
\item Im Dateiauswahlfenster werden per Default nur Verzeichnisse und die
  ".meta.xml"-Dateien angezeigt.
\item Beim erfolgreichen oder missgl�ckten Speichern erscheint jeweils eine
  Nachricht dar�ber.
\item Es werden Warnungen ausgegeben wenn mindestens einer Komponente kein
  Dokument zugewiesen wurde oder der Rote Faden nicht korrekt ist.
\item Der gespeicherte Name erscheint im Metainformationsfenster auf der
  rechten Seite.
\end{itemize}