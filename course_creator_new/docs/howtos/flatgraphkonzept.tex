\documentclass{generic}
%this.filename = flatgraphkonzept.tex
%usage:
% mmtex flatgraphkonzept.tex
% mmxalan flatgraphkonzept.xml
% firefox flatgraphkonzep.xhtml

%author: vrichter

\begin{document}

Alle Graphen, die erzeugt werden, sind in der Klasse CCModel in der Liste bufferedGraphs. jeder neue Graph wird als mit bufferedGraphs.addFirst(graph) zugef�gt.
Verwaltet werden die einzelnen Srukturen  (zb NavGraph und sp�ter FlatGraph) in den Klassen NavModel-bzw FlatModel)



\section{flatgraphpackage}
\subsection{FlatCell} (enth�lt beschreibung der einzelnen Cellen - sind die Komponenten von FlatGraph unterscheide zwischen Main und SubKomponenten)

ben�tigt Funktionen rund um Label, points, category..
	toMetaXML und toContentXML
	

\subsection{FlatGraph}
abgeleitet von Graph ben�tigt daher  die Funktionen:\\
void setChanged(boolean c);\\
	int getUndoNumber();\\
	void setUndoNumber(int n);\\
	String getUndoName();\\
	void setUndoName(String s);\\
	\\
	boolean isProblemGraph();\\
	boolean isSectionGraph();\\
	boolean isElementGraph();\\

	void setLid_counter(int i);\\
	MetaInfos getMetaInfos();\\
	void setMetaInfos(MetaInfos meta);\\
	void setGraphType(int type);\\
	Object createCell(int cellType, String path, int cat, int nid, String lid,
			Object parent, int x, int y,String label);\\

	Object insertCell(Object o);\\
	void setSummary(String s);\\
	void repaint();\\
	void delete();\\
	boolean getChanged();\\
	
	Object[] getSelectionCells();\\
	int getSelectionCount();\\
	Object getSelectionCell();\\
	boolean getButtonEnable(String s);\\
	public String toXMLContent();\\
	public String toXMLMeta();\\

ausserdem eigene Funktionalit�ten..

\subsection{FlatGraphController}
Controller f�r FlatGraph, analog NavGraphController 

\subsection{FlatMarqueeHandler}
Kontextmenue fuer Flatgraph. 
\subsection{FlatModel}
Diese Klasse h�lt alle FlatGraphen und kommuniziert zwischen den einzelnen FlatGraphen und der CCModelklasse.

es m�ssen Funktionen die an den Flatgraphen weitergegeben werden implementiert werden zB.:\\
 public void showCircles(){\\
    	this.getMainGraph().showCircles();\\
    } \\
ausserdem das Speichern des ersten Graphen.

\section{CCModel}
diese Klasse ist Verbindung zwischen der Oberfl�che und den Graphen. 

"Angesprochen" wird immer nur der erste Graph (da alle Ver�nderungen der Strukturen durch Benutzerinteraktionen geschehen und dieser nur den ersten Graph sieht.) angesprochen wird der erste Graph mit CCModel.getFirstGraph.

- zuerst sollte die Funktion newFlatCourse bearbeitet werden (entfernen der if(false)-zeile) ganz wichtig ist das setzten der Variable CCModel.setKursTyp(KURS_FLATGRAPH);
- CourseCreator.createFlatGraphPanel muss sinnvollen Inhalt bekommen (erzeugen eines leeren Baums)
- mehrere Funktionen leiten Befehle weiter an den Flatgraph. diese Funktionen m�ssen im Flatgraph geschrieben werden (dabei handelt es sich um die Liste der Funktionen, die unter dem Kapitel Flatgraph beschrieben sind.) im CCModel sind diese zu erkennen an der Struktur:\\
	 public void setLIDcount(int cnt) {\\
   		this.getFirstGraph().setLid_counter(cnt);\\
    	}\\
desweiteren gibt es spezielle Funktionen, die nicht jeder Graph haben muss, hier muss �berlegt werden, ob die der FlatGraph ebenso ben�tig. gekennzeichnet sind diese im Quellcode durch: FIXME anpassen an andere Graphtypen oder aehnlich.
passiert dies, sollte der erste FlatGraph nie direkt angesprochen werden, sondern �ber die Klasse FlatModel (noch zu implementieren)also zum Beispiel:
    
   public boolean checkSummaryExists(){\\
    	if (this.getKursTyp()==KURS_NAVGRAPH) {\\
    		return this.navModel.getMainGraph().checkSummaryExists();\\
    	}else if (this.getKursTyp()==KURS_FLATGRAPH) {\\
    		return this.flatModel.getMainGraph().checkSummaryExists();\\
    	}\\
    	return false;\\
    }\\

\section{bearbeiten der Struktur}
- es gibt noch kein Konzept, wie die Struktur genau bearbeitet werden soll: 
Vorschlag: 
- Einf�gen von Mainkomponenten, Einf�gen von SubKomponenten an MainKomponenten, Einf�gen von Verzeichnissen (�hnlich jetzt den Branchknoten). MainKomponenten nur als Bl�tter Verzeichnisse nur als "nicht-Bl�tter) - m�ssen Bl�tter haben, sonst ist die Struktur "nicht richtig". Beachte: durch die undo Fuktion wird jeweils die Struktur gespeichert, es darf also keinen Moment geben, wo der Graph nicht speicherbar ist. - wo werden die neuen Komponenten eingef�gt zB vor, nach .. der markierten Komponente..



\section{Laden}
bisher gibt es keine XML-Darstellung flacher Kurse. 
Am XML sollte erkennen zu sein, welcher Kurstyp es ist.

Angepasst an das neue XML ist eine Klasse FlatGraphLoader analog zu NavGraphLoader zu schreiben.

\section{sonstiges}
die Buttons in der Toolbar m�ssen angepasst werden (die auf der linken seite.)
Vorbereitet ist daf�r die Funktion getButtonEnable in der Klasse FlatGraph. f�r
jeden Button muss gepr�ft werden, ob dieser enable oder disabled sein soll (die
Funktion ist eigentlich schon fertig.)
Sinnvoll w�re noch eine Funktion getButtonVisible, da der Connect-Button f�r
Graphen ohne Kanten noch sinnvoll ist.

\end{document}