% Author: Tilman Rassy <rassy@math.tu-berlin.de>
% $id$

% $Id: praes.tex,v 1.2 2007/10/14 01:03:00 rassy Exp $

\documentclass{article}

\usepackage{ngerman}
\usepackage{amssymb,amsbsy}
\usepackage[pdftex,pdfpagemode=None,bookmarks=false,colorlinks=true,linkcolor=blue]{hyperref}
\usepackage{epsfig}
\usepackage[display]{texpower}
\usepackage{color}


% ------------------------------------------------------------------------------
% Schriftgroessen
% ------------------------------------------------------------------------------

\renewcommand{\normalsize}{\fontsize{18}{22}\selectfont}
\renewcommand{\large}{\fontsize{20}{26}\selectfont}
\renewcommand{\Large}{\fontsize{22}{30}\selectfont}
\renewcommand{\small}{\fontsize{14}{16}\selectfont}
\newcommand{\smaller}{\fontsize{12}{14}\selectfont}
\newcommand{\verysmall}{\fontsize{6}{6}\selectfont}
\renewcommand{\footnotesize}{\fontsize{12}{14}\selectfont}
\newcommand{\tocsize}{\fontsize{12}{16}\selectfont}
\newcommand{\titlesize}{\fontsize{24}{36}\selectfont}

% ------------------------------------------------------------------------------
% Texthervorhebungen
% ------------------------------------------------------------------------------

%\definecolor{emphcolor}{rgb}{0.77,0.00,0.00}
%\renewcommand{\emph}[1]{\textbf{\textcolor{emphcolor}{#1}}}
\renewcommand{\emph}[1]{\textbf{#1}}

\newcommand{\term}[1]{\textit{#1}}
% \newcommand{\code}[1]{\texttt{#1}}
\newcommand{\var}[1]{\textit{#1}}

%\definecolor{headlinecolor}{rgb}{0.77,0.00, 0.00}
\definecolor{headlinecolor}{rgb}{0.67,0.09, 0.22}
\newcommand{\headline}[1]{\textbf{\textcolor{headlinecolor}{#1}}}

\definecolor{emphboxbordercolor}{rgb}{0.80,0.36,0.36}
\definecolor{emphboxcolor}{rgb}{0.98,0.98,0.77}
\newcommand{\emphbox}[2]{\fcolorbox{emphboxbordercolor}{emphboxcolor}{\parbox{#1}{#2}}}

\definecolor{labelcolor}{rgb}{0.35,0.35, 0.8}


% -------------------------------------------------------------------------------
% Pfeile
% -------------------------------------------------------------------------------

\newcommand{\myarrow}{{\Large\boldmath$\;\rightarrow\;$}}
\newcommand{\mylongarrow}{{\Large\boldmath$\;\longrightarrow\;$}}



% -------------------------------------------------------------------------------
% Befehle f�r Abbildungen 
% -------------------------------------------------------------------------------

\renewcommand{\thefigure}{\thesubsection.\arabic{figure}}

\newcommand{\mycaption}[2]{%
 \centerline{%
  \parbox{#1}{%
   \small \refstepcounter{figure} Figure \thefigure. #2 }}}


%---------------------------------------------------------------------------
% Gliederungsbefehle
%---------------------------------------------------------------------------

\makeatletter

\renewcommand{\section}{\@startsection
 {section}%                                   % Name
 {1}%                                         % Ebene
 {0pt}%                                       % Einzug
 {-0.0\baselineskip}%                        % Vorabstand 
 {0.85\baselineskip}%                         % Nachabstand
 {\fontsize{20}{24}\selectfont\itshape\bfseries}}%    % Stil

\makeatother

\newcommand{\mysection}[1]{%

\vspace*{-1.2\baselineskip}

{\color{headlinecolor}
\section{#1}}}

\newcounter{myappendix}

\newcommand{\myappendix}[2]{%
\hfill\makebox[0cm][l]{\tocref}

\vspace*{-1.2\baselineskip}

\refstepcounter{myappendix}
\section*{A\arabic{myappendix}\hspace{1em}#1}\hypertarget{#2}{}}



% -------------------------------------------------------------------------------
% Inhaltsverzeichnis 
% -------------------------------------------------------------------------------

\newcommand{\mytoctitle}[1]{\begin{center}\textbf{\hypertarget{inhalt}{#1}}\end{center}}

\newenvironment{mytoc}
  {\tocsize\textbf{\hypertarget{inhalt}{Inhalt}}\\\begin{enumerate}}
  {\end{enumerate}}

\newcommand{\mytocitem}[3]{\item[#1]\hyperlink{#3}{#2}}

\newcommand{\tocref}{\verysmall\hyperlink{inhalt}{TOC}\normalsize}

% -------------------------------------------------------------------------------
% Listen 
% -------------------------------------------------------------------------------

%\newcommand{\mylabelitemi}{\labelitemi}
\newcommand{\mylabelitemi}{\raisebox{0.25ex}{{\small $\blacktriangleright$}\hspace{0.1em}}}

%\newcommand{\mylabelitemii}{\labelitemii}
%\newcommand{\mylabelitemii}{\raisebox{0.25ex}{{\small $\triangleright$}\hspace{0.1em}}}
\newcommand{\mylabelitemii}{\raisebox{0.1ex}{\large\bfseries-}\hspace{0.075em}}

\newenvironment{mylist}[1][\mylabelitemi]
  {\begin{list}{{\color{labelcolor}#1}}
    {\setlength{\itemindent}{0cm}
     \setlength{\labelwidth}{1em}
     \setlength{\leftmargin}{0.5cm}
     \setlength{\rightmargin}{1cm}}}
  {\end{list}}

\newenvironment{myenum}
  {\begin{list}{\theenumi}
    {\usecounter{enumi}
     \setlength{\itemindent}{0cm}
     \setlength{\labelwidth}{1em}
     \setlength{\leftmargin}{0.5cm}
     \setlength{\rightmargin}{1cm}}}
  {\end{list}}

\newcommand{\pitem}{\pause\item}
\newcommand{\itemii}{\item[\mylabelitemii]}

\newcounter{romanlistcounter}
\newenvironment{romanlist}{\begin{list}%
 {(\roman{romanlistcounter})\hfill}{\usecounter{romanlistcounter}%
 \topsep0.05\baselineskip% 
 \partopsep0.85\baselineskip%
 \leftmargin2em%
 \labelwidth2em%
 \labelsep0pt%
 \itemindent0pt%
 \parsep0.15\baselineskip%
 \itemsep0.0\baselineskip}}%
 {\end{list}}

\newcounter{alphlistcounter}
\newenvironment{alphlist}{\begin{list}%
 {(\alph{alphlistcounter})\hfill}{\usecounter{alphlistcounter}%
 \topsep0.05\baselineskip% 
 \partopsep0.85\baselineskip%
 \leftmargin1.8em%
 \labelwidth1.8em%
 \labelsep0pt%
 \itemindent0pt%
 \parsep0.15\baselineskip%
 \itemsep0.0\baselineskip}}%
 {\end{list}}

\newcounter{arabiclistcounter}
\newenvironment{arabiclist}{\begin{list}%
 {(\arabic{arabiclistcounter})\hfill}{\usecounter{arabiclistcounter}%
 \topsep0.05\baselineskip% 
 \partopsep0.85\baselineskip%
 \leftmargin1.8em%
 \labelwidth1.8em%
 \labelsep0pt%               
 \itemindent0pt%
 \parsep0.15\baselineskip%
 \itemsep0.0\baselineskip}}%
 {\end{list}}





\hoffset0cm
\voffset0cm
\topmargin0cm
\headheight0cm
\headsep0cm
\topskip0cm
\marginparwidth0cm
\marginparsep0cm

\paperheight16cm
\paperwidth20cm
\textheight14cm
\textwidth16cm

\oddsidemargin0mm
\evensidemargin0mm

\nonfrenchspacing
\parindent0em
\parskip0.2\baselineskip

\pagestyle{empty}

\sloppy

\begin{document}

\sffamily

\pagebreak

\begin{center}

\textbf{\color{headlinecolor}\titlesize 
Web-2-Technologien fuer eine multimediale Lernplatform}

\vspace{1.0cm}

\normalsize
Tilman Rassy

\vspace{1.0cm}

Technische Universit"at Berlin \\
Fakult"at II -- Mathematik und Naturwissenschaften \\
Institut f"ur Mathematik

\end{center}

\pagebreak

\mysection{Was ist MUMIE?}

\begin{mylist}

\pitem E-Learning-Plattform f"ur Mathematik

\pitem Funktionalit"aten:
\begin{mylist}[\mylabelitemii]
  \pitem Darstellung mathematischer Inhalte: Wissensbausteine,
         erg"anzt durch Bemerkungen, Beispiele, Visualisierungen (Multimedia)
  \pitem Aufgaben: verschiedene Typen; individualisiert; automatisch
         korrigiert und bewertet
  \pitem Strukturiert in Kursen
  \pitem Autorentools
\end{mylist}

\pitem Open-Source-Projekt
\begin{mylist}[\mylabelitemii]
  \pitem TU Berlin, Institut f"ur Mathematik
\end{mylist}

\end{mylist}

\pagebreak

\mysection{Was ist MUMIE?}

\begin{center}
\hspace*{-0.5cm} \includegraphics[width=16cm]{mumie_screenshot_01}
\end{center}

\pagebreak

\mysection{Was ist MUMIE?}

\begin{center}
\hspace*{-0.5cm} \includegraphics[width=16cm]{mumie_screenshot_02}
\end{center}

\pagebreak

\mysection{MUMIE -- Technisches}

\begin{mylist}

\pitem Technologie:
\begin{mylist}[\mylabelitemii]
  \pitem Java-Servlet-Technologie
  \pitem XML-Technologie
  \pitem Dynamische Seitenerzeugung
\end{mylist}

\pitem Architektur:
\begin{mylist}[\mylabelitemii]
  \pitem Apache (Webserver)
  \pitem Tomcat (Servlet-Container)
  \pitem Cocoon + MUMIE-eigene Komponenten (Servlet)
  \pitem PostgreSQL (Datenbank) 
\end{mylist}

\end{mylist}

\pagebreak

\mysection{MUMIE -- Technisches}

\vspace{2cm}

\hspace*{-1.5cm}\includegraphics{arch_mumie}

\pagebreak

\mysection{MUMIE -- Geschichte}

\begin{mylist}

\pitem 2001 - 2004: Kooperationsprojekt der TU Berlin, Uni Potsdam, RWTH
        Aachen und TU M"unchen, gef"ordert durch das BMWF
\pitem Seit 2004: Fortgef"uhrt an der TU Berlin in loser Zusammenarbeit mit
       der RWTH Aachen und TU M"unchen
\pitem Ab Sommersemester 2005 Testeins"atze an der TU Berlin; ab Wintersemester
       2006/2007 regul"arer Einsatz (Lineare Algebra f"ur Ingenieure, 2000 H"orer)
\pitem Einsatz an der TU M"unchen
\pitem Ab Herbstsemester 2007 Einsatz an der ETH Z"urich

\end{mylist}

\pagebreak

\mysection{Was ist Web 2.0?}

\begin{mylist}

\pitem \glqq 2.0\grqq\ ist keine Software-Versionsnummer

\pitem Vager Begriff; keine pr"azise Definition

% \pitem Popul"ar gemacht von Tim O'Reilly

\pitem Bisher: Feste, wenig interaktive Webseiten;\pause Web 2.0: stark
interaktive Webseiten, Inhalt von Benutzern mitgestaltet

\pitem Neue Technologie: Ajax \pause (Asynchronous JavaScript and XML)

\pitem Erlaubt es, Webseiten mit Eigenschaften von Desktop-GUIs auszustatten

\pitem Neuere Entwicklungen: Java FX, Adobe AIR, Silverlight

\end{mylist}

\pagebreak

\mysection{MUMIE und Web 2.0}

\begin{mylist}
\pitem Bisher keine Web-2.0-Technologie in der MUMIE
\pitem Projektaufgaben:
\begin{mylist}[\mylabelitemii]
  \pitem Erarbeitung von Vorschl"agen f"ur den sinnvollen Einsatz von Web 2.0
         in der MUMIE
  \pitem Konzeption und Implementation von einem oder mehreren Beispielen
\end{mylist}
\end{mylist}

\end{document}
