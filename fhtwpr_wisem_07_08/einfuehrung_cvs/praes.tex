% Author: Tilman Rassy <rassy@math.tu-berlin.de>
% $id$

% $Id: praes.tex,v 1.2 2007/10/14 01:03:00 rassy Exp $

\documentclass{article}

\usepackage{ngerman}
\usepackage{amssymb,amsbsy}
\usepackage[pdftex,pdfpagemode=None,bookmarks=false,colorlinks=true,linkcolor=blue]{hyperref}
\usepackage{epsfig}
\usepackage[display]{texpower}
\usepackage{color}

% Author: Tilman Rassy <rassy@math.tu-berlin.de>
% $Id: macros.tex,v 1.5 2006/10/05 15:58:42 rassy Exp $

% XML code:
\newcommand{\element}[1]{\code[xml-element]{#1}}
\newcommand{\attrib}[1]{\code[xml-attrib]{#1}}

% Database code:
\newcommand{\dbtable}[1]{\code[db-table]{#1}}
\newcommand{\dbcol}[1]{\code[db-column]{#1}}
\newcommand{\sql}[1]{\code[sql]{#1}}

% TeX code:
\newcommand{\texcmd}[1]{\code[tex-cmd]{\backslash#1}}
\newcommand{\texenv}[1]{\code[tex-env]{#1}}

% General
\newcommand{\val}[1]{\code[value]{#1}}
\newcommand{\prm}[1]{\code[param]{#1}}

% Other:
\newcommand{\notimpl}[1]{\emph[not-impl]{#1}}
\newcommand{\new}[1]{\emph[new]{#1}}
\newcommand{\deprecated}[1]{\emph[deprecated]{#1}}
\newcommand{\wvar}[1]{\var[weak]{#1}}


\hoffset0cm
\voffset0cm
\topmargin0cm
\headheight0cm
\headsep0cm
\topskip0cm
\marginparwidth0cm
\marginparsep0cm

\paperheight16cm
\paperwidth20cm
\textheight14cm
\textwidth16cm

\oddsidemargin0mm
\evensidemargin0mm

\nonfrenchspacing
\parindent0em
\parskip0.2\baselineskip

\pagestyle{empty}

\sloppy

\begin{document}

\sffamily

\pagebreak


\begin{center}

\textbf{\color{headlinecolor}\titlesize 
CVS}

\vspace{1.0cm}

\normalsize
Tilman Rassy

\vspace{1.0cm}

Technische Universit"at Berlin \\
Fakult"at II -- Mathematik und Naturwissenschaften \\
Institut f"ur Mathematik

\end{center}

\pagebreak

\mysection{Was ist CVS?}

\begin{mylist}

  \pitem CVS : Cuncurrent Versions System

  \pitem Fr"uhere Versionen einer Datei k"onnen rekonstruiert werden

  \pitem Mehrere Entwickler k"onnen gleichzeitig an einer Datei\\ arbeiten
     (Konflikt-Aufl"osung, Merging)

  \pitem \term{Repository:} Zentraler Aufbewahrungsort f"ur alle Dateien eines Projekts
     unter Versionsverwaltung, einschl. alter Versionen

  \pitem \term{Working directory:} Lokale Version der Dateien eines\\ Entwicklers

  \pitem Repository und Working Directory m"ussen nicht auf demselben Rechner liegen
     (Client-Server-Architektur)

\end{mylist}

\pagebreak

\mysection{Benutzung}

\begin{mylist}

  \pitem Aufruf:

    \verb'  '\code{cvs [-d \var{cvsroot}] \var{cmd} \var{params}}

     \var{cvsroot} = Wurzelverzeichnis der Repositories (s.u.)

     \var{cmd}     = CVS-Commando

     \var{params}  = Parameter f"ur \var{cmd}

  \pitem Format von \var{cvsroot}:

    \begin{mylist}[\mylabelitemii]

       \pitem Repository auf selben Rechner: Absoluter Pfad

       \pitem Repository auf anderem Rechner:

           \verb'  '\code{:ext:\var{user}}\verb'@'\code{\var{host}:\var{absoluter\_pfad}}

         \var{user} = Benutzername, \var{host} = Rechner-Adresse.

    \end{mylist}

\end{mylist}

\pagebreak

\mysection{Umgebungsvariablen}

\begin{mylist}
  \pitem \code{CVSROOT} --  Default f"ur \var{cvsroot}

  \pitem\code{CVS\_RSH} -- Muss auf \code{ssh} gesetzt sein, damit Remote-Zugriff "uber ssh
                  erfolgt
\end{mylist}

\pagebreak


\mysection{Wichtigste Befehle}

\begin{mylist}

  \pitem \code{cvs checkout \var{name}}

       Checkt das Repository \var{name} aus

  \pitem \code{cvs update [\var{dat1} \var{dat2} ...]}

       Aktualisiert Dateien im Working Directory

  \pitem \code{cvs commit -m '\var{logmeldung}' [\var{dat1} \var{dat2} ...]}

       "Ubertr"agt "Anderungen vom Working Directory ins Repository

  \pitem \code{cvs add [\var{dat1} \var{dat2} ...]}

       Meldet neue Dateien f"ur Versionskontrolle an

  \pitem \code{cvs status \var{dat}}

       Gibt den Status einer Datei an


  \pitem \code{cvs log \var{dat}}

       Gibt die Logmeldungen zu einer Datei aus

\end{mylist}

\end{document}