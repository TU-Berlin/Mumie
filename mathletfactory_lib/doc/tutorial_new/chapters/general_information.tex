
\author{Markus Gronau}

\section{About this document}
  This document adresses to mathlet developers who wants to get familiar with the creation
  of mathlets using the MathletFactory's library. It does not aim to describe each
  single class but to illustrate the necessary techniques and philosophies for the 
  construction of a mathlet. This technical stuff will be accompanied by 
  comprehensive examples and hints. A complete overview of all classes can be found
  in the APIDOC. The use of the APIDOC will be assumed.

\section{About the Mumie MathletFactory}
The \mumie is an e-learning platform specialized in mathematics and mathematical sciences.
It is a fully web-based learning- and teaching environment using standard techniques
(such as XHTML, MathML and Java) for deploying mathematical content to the end-user inside 
a standard compliant internet browser.\\
\\
The \mf is a Java library and part of the \mumie in which it is used to produce and to visualize
dynamic, interactive mathematical content with Java applets (so called ``mathlets''). It allows
the rapid development of such mathlets, containing complex mathematical algorithms and scenes
along with a common generic behaviour and appearance.\\
The \mf library contains a large collection of mathematical objects (so called \mmos), which can 
be used both in calculations and presentations of dynamic problems.
Their visualizations may be both symbolic and graphical (both 2D and 3D).
Their state may be interactively changeable by the user and may cause further interaction between them.\\
\\
The \mf is developed since 2001 and released in 2007 the milestone 2.0, which has been used since then by
several thousands of students at different international universities. The main development
effort is actually done for the milestone 2.1 which will be released in spring 2008.\\
\\
The \mf library is open source and provided under the MIT license. It is compliant with 
all Java versions of SUN Microsystems and Apple Inc. starting with Version 1.4.2\footnote{Some 
additional extensions may require a newer Java version.}. Further information and documentation 
as well as source code is available under \url{http://www.mumie.net}.

\section{Fundamental Concepts}

\subsubsection*{Generic programming of mathlets}
The programming of mathlets is generic and easy to use but does not restrict the applet developer
in his creativity. By automatically adding standard features to new mathlets and providing a
flexible and generic applet ``skeleton'' the developer can concentrate on the mathematical content.\\
The \mf library acts hereby as a reusable component system.

\subsubsection*{Separation of logic and representation}
The \mf follows the philosophy to separate the (abstract) mathematical object from its further
representation(s) on the screen, defining a veritable Model-View-Controller architecture (MVC).
By handling interactive actions (e.g. user interaction) through generic but specific \textit{events}, 
changes are automatically reflected to the mathematical model and also propagated to any dependant 
objects, allowing even complex dependency trees.

\subsubsection*{Abstract number fields for arbitrary-precise calculations}
Most mathematical objects are based on an abstract number class which makes it possible to perform
calculations on a particular number field with its own arbitrary precision. While some operations
need a complex number field, other situations can be more satisfied with an answer in whole numbers.
Furthermore while e.g. floating point operations are executed faster than rational ones, the latter 
are more precise and user friendly.

\subsubsection*{Open extensible framework}
The \mf library offers a wide spectrum of objects for the most needed mathematical entities and 
applet developer's concerns but also can be extended in almost every included technology\footnote{
Some extension features are only available in the up-coming milestone 2.1}. By
providing an open framework and both a complete API documentation and tutorials for its techniques, 
the \mf truly underlines its open source idea.

\section{Available resources}
  The MathletFactory library as well as the APIDOC, examples, sources and other
  documentation can be downloaded from the internet site:\\
  \indent\indent\indent\indent\indent\indent \textbf{http://www.mumie.net}
