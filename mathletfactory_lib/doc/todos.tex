\documentclass[a4paper,12pt]{article}
\usepackage{german}

\newcommand{\code}[1]{\texttt{#1}}

%%%%%%%%%%%
%%%
                                %Hier gehts los
%%%
\begin{document}

\section{TODO's f"ur die AppletFactory}

\subsection{Allgemeine Konzepte}

\begin{description}

\item [Transformer]
Die Schnittstelle innerhalb der \code{MMObjekt}e zum Setzen des ``transform
types'' ist etwas unsauber: F"ur die Darstellung in einem \code{MMCanvas}
muss man die Methode \code{setCanvasTransformer(trafoType,screenType)}
aufrufen, f"ur die Darstellung als JComponent ruft man
\code{getAsContainerContent(int trafoType)} auf $\ldots$.

\item [Renderkette]
Kaum getestet wurde der Einbau neuer \code{MMObject}e ``externer''
Entwickler. Diese m"ussen neben dem eigenen \code{MMObject} in der Regel
zus"atzlich einen \code{Transformer} und gegebenenfalls ein \code{Drawable}
hierf"ur schreiben.
Die M"oglichkeiten, einem Objekt neue \code{Transformer} zuzuweisen ist noch
sehr ``krude'' und schlecht strukturiert. Hier sollten die Methoden, mit denen
die Objekte auf den \code{GeneralTransformer} zugreifen, "uberarbeitet werden.

\item [MMEvents]
Unsere \code{MMEvent}s stammen noch aus der ersten Implementation und sind
nicht besonders geschickt aufgebaut: Eigentlich sollen sie lediglich als
Argument-Protokoll f"ur die \code{doAction}-Methode der \code{MMHandler}
dienen und m"ussen die Java Keyboard- und MouseEvents beinhalten. "Anderungen
am \code{MMEvent} f"uhren allerdings auch zu Anpassungen in den Handlern und
im Controller.

\item [Handler/Events]
Jeder \code{MMHandler} sagt "uber die Methode \code{canDealWith(MMEvent e)},
ob er mit einem \code{MMEvent} etwas anfangen kann. Momentan kontrollieren wir
jedoch nicht, ob es gegebenenfalls mehrere Handler gibt, die f"ur den gleichen
Event zust"andig sind.
Die urspr"ungliche Philosophie war hier, da"s jeder Handler auch nur f"ur
genau einen Event zust"andig ist -- dies ist jetzt nicht mehr so. Hier sollte
man einen Mechanismus entwickeln, der ``doppelt'' vergebene Zust"andigkeiten
"uberwacht.
Ein weiterer Aspekt ist, da"s ein \code{MMHandler} eigentlich ``nur'' ein
Handler ist, der f"ur eine Darstellung eines Objektes in einem \code{MMCanvas}
arbeiten kann. Bei den sogenannten Containerdarstellungen "ubernimmt das
Drawable als Listener selbst die Funktion, das \code{MMObjekt} zu ver"andern/
anzupassen.

\item [Update-Mechanismus]
Der Update-Mechanismus arbeitet zwar stabil und garantiert, da"s keine
Endlosschleifen bei zirkul"aren Abh"angigkeiten auftreten. Er kann aber nicht
sicherstellen, da"s das Ergebnis alle Abh"angigkeiten richtig aufl"ost!
Grunds"atzlich w"are es sch"on, wenn man wenigstens den Ablauf der Update-Kette
herausschreiben k"onnte, um eventuelle Fehler besser nachvollziehen zu
k"onnen.\\
Weiterhin fehlt aktuell die M"oglichkeit eine Objekt-Abh"angigkeit und die
damit verbundenen \code{Updater} wieder sauber zu entfernen, d.h. eine einmal
festgelegte Abh"angigkeit kann nicht wieder r"uckg"angig gemacht werden.

\item [DisplayProperties]
Hier haben wir noch kein richtiges Konzept entwickelt --
die \code{DisplayProperties} sind derzeit auch kaum strukturiert und
teilweise auch an den Renderer (etwa Graphics2D) gebunden. Weiter trat immer
die Frage auf, inwieweit man gewisse Eigenschaften (etwa die Farbe) direkt beim
\code{MMObject} zuweist. Generell m"ussen die DisplayProperties
verwendet werden, andere Eigenschaften wie Font werden zur Zeit direkt
zugewiesen.

\item [MMZahlen]
Unsere bisherigen Zahlen sind keine \code{MMObject}e und k"onnen daher nicht
einfach im Applet dargestellt werden. Hier hat Markus angefangen, neue
MMZahlen zu basteln, die dann auch in einer Applikation einfach dargstellt
werden k"onnen.
Hier muss man bei der Darstellung eigentlich erst spezifizieren,
wie man mit den
verschiedenen Typen umgehen m"ochte, etwa inwieweit man Darstellungsoptionen
(etwa wissenschaftliche Notation, Dezimalnotation) zulassen m"ochte.

\item [Java native versus MMZahlen]
Ich habe den Eindruck, da"s bei Operationen mit den Zahlen teilweise zuviel
"uber \code{setDouble(double)} und \code{getDouble()} gearbeitet wird. Dies
r"uhrt wohl noch vor unserer Angst her, dass die Arbeit mit MMZahlen wegen
vermeintlicher ``news'' die Applikationen bremsen werden.
Diesbez"uglich sollten wir unsere Klassen wieder einmal durchschauen.

\item [MMCompound]
Hier haben wir ebenfalls noch wenig Arbeit investiert und daher noch kein
wirklich ausgereiftes Konzept: Aktuell kann man zwar ein ``compound object''
anlegen, in einem Canvas werden die einzelnen Objekte dann allerdings doch
wieder solo behandelt $\ldots$.

\item [Unit-Tests]
In diesen Bereich haben wir ebenfalls noch zu wenig investiert. Zumindest
die wesentlichen Methoden der ``tiefen'' Klassen (\code{NumberMatrix} etc.)
sollten besser getestet sein.

\item [Logging]
Derzeit verwenden wir die \textbf{log4j}-Bibliothek f"ur das Logging. Diese
hat eine Gr"o"se von etwa 350k. F"ur unsere Belange ist die Logging-API des
\textbf{J2DK1.4} ausreichend. Vorteil: Die Ladezeiten der Applets k"onnten
reduziert werden. (Tim:) Ist am 27.06.03 umgestellt worden.

\item [Deployment-Strategien]
Die Gr"o"se der Bibliothek k"onnte mittelfristig f"ur die Einbindung von
Applets in HTML-Seiten ein Problem werden (Stand 20.Juni):
\begin{itemize}
\item appfac-base.jar  -- 580k
\item appfac-g2d.jar -- 95k
\item appfac-j3d.jar -- 41k
\item appfac-noc.jar -- 142k
\item appfac-sample.jar -- 134k
\end{itemize}
Es ist allerdings nicht ganz trivial, das Paket appfac-base.jar weiter
aufzusplitten. Denkbar w"are gegebenenfalls eine Zerlegung in die
mathematischen Themenbereiche (\code{algebra},\code{analysis},$\ldots$), doch
ist auch das nicht unproblemeatisch! (Beispiel: \code{Polynomial} liegt im
Paket \code{analysis.function} hat aber auch algebraische Abh"angigkeiten.)
\end{description}


\subsection{Mathematische Objekte}
Generell gilt, dass unsere mathematischen Klassen bislang nicht sehr
viele spezifisch-mathematische Methoden enthalten. Tendentiell sind solche
Methoden implementiert, die zur Darstellung unerl"asslich sind und die wir
in den aktuellen Beispiel-Applets ben"otigt haben.

\begin{description}

\item [Interval/FiniteBorelSet]
Die Implementation ist noch sehr einfach. Im Interval ist die interne
Datendarstellung sogar noch in den projektiven Punkten (von Thorsten) gegeben.

\item [Funktionen/Polynome]
Generell fehlt bei den Funktionen noch der ``algebraische'' Teil, d.h.
man kann unsere Funktionen aktuell nicht addieren, subtrahieren $\ldots$

\item [Vektorwertige Funktionen]
Hier ist die Struktur noch sehr unfertig. Insbesondere haben wir noch kein
wirklich gutes Konzept, die vektorwertigen Funktionen aus den
Komponentenfunktionen zusammenzusetzen beziehungsweise umgekehrt,
Komponentenanteile zur"uckzukriegen.

\item [Vektorfelder]
Die Klassen zu den Vektorfeldern (etwa \code{MMVectorField2DOverR2}) sind
bislang lediglich ``R"umpfe''. Einzig die zur Darstellung notwendige
evaluate-Methode (basierend auf \code{double}) ist hier implementiert.

\item [Projektive Geometrie]
Unser affinen Objekte bauen der Struktur nach auf den projektiven auf. Diese
M"oglichkeiten haben wir bislang noch nicht ausprobiert und getestet.
(Beispiel: Punkt als Schnitt zweier Geraden...)

\end{description}


\subsection{Fehlende und erg"anzende Features}
\begin{description}

\item [Renderkette]
Derzeit kann ein \code{MMObject} nur in genau einem \code{MMCanvas}
dargestellt werden, lediglich das Rendern in mehreren Containerdarstellungen
(z.B. die Komponentendarstellung von Vektoren) ist m"oglich. Hier w"are eine
Erweiterung w"unschenswert, die allerdings sehr aufw"andig zu implementieren
sein w"urde, da insbesondere \code{MMHandler}- und \code{Transformer}-Klassen
momentan auf \textbf{den} \code{MMCanvas} eines \code{MMObject}s zugreifen.

\item [Grafik-Primitive]
Es fehlt derzeit ein Konzept, um primitive Grafikelemente, wie etwa
"Uberschriften, Bezeichner, Bilder im \code{MMCanvas} unterzubringen.
Momentan m"ussen alle Objekte, die in einem \code{MMCanvas} dargestellt
werden sollen, ``komplette'' \code{MMObjekt}e sein.

\item [Kontext-Menu]
Zumindest f"ur Objekte, die in einem Canvas dargestellt werden, w"are es
reizvoll, wenn man mit Klick der rechten Maustaste f"ur dieses Objekt ein
Kontextmenu erhalten k"onnte, in welchem diverse Informationen (Lage,
Darstellungseigenschaften) abgefragt und gegebenenfalls neu gesetzt werden
d"urften.

\item [Canvi]
Man sollte zwei \code{MMCanvi} ``koppeln'' k"onnen, d.h. etwa, "andere ich in
einem die affine Transformation, so wird sie im anderen Canvas automatisch
mitangepasst. Hierzu w"aren wohl nur wenige zus"atliche Methoden im
\code{MMCanvas} notwendig.\\
Weiterhin sollte die M"oglichkeit bestehen, einen \code{MMCanvas} zu
deaktivieren, so da"s er auf keine Mouse- und KeyEvents mehr h"ort. (Dieses
Feature ist beispielsweise w"ahrend einer laufenden Animation w"unschenswert.)

\end{description}






%%%%%%%%%%%
%%%
                                %Ende des Dokuments
%%%
\end{document}

